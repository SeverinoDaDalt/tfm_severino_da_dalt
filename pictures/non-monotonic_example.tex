\begin{figure}[h]
    \centering

    \begin{tikzpicture}[
        vertex/.style={circle, draw, fill=gray!20, minimum size=0.3 cm, inner sep=1pt},
        label_a/.style={above=2pt, font=\small},
        label_b/.style={below=2pt, font=\small},
        node distance=1.5cm,
        solid edge/.style={draw, thick, black!60},
        dashed edge/.style={draw, dashed, thick, black!40},
        matrix cell/.style={draw, minimum size=0.5 cm, inner sep=0pt},
        matrix label/.style={font=\small, anchor=center}
    ]
	\node[vertex] (a_1) at (1,0) {};
	\node[vertex] (b_1) at (1,-4) {};
	\node[vertex] (a_2) at (2,0) {};
	\node[vertex] (b_2) at (2,-4) {};
	\node[vertex] (a_3) at (3,0) {};
	\node[vertex] (b_3) at (3,-4) {};
	\node[vertex] (a_4) at (4,0) {};
	\node[vertex] (b_4) at (4,-4) {};
	\node[vertex] (a_5) at (5,0) {};
	\node[vertex] (b_5) at (5,-4) {};
	\node[vertex] (a_6) at (6,0) {};
	\node[vertex] (b_6) at (6,-4) {};
	\node[draw, rounded corners, fit=(a_1) (a_2) (a_3) (a_4) (a_5) (a_6)] (box) {};
	\node[left of=box, node distance=3cm] {$A$};
	\node[draw, rounded corners, fit=(b_1) (b_2) (b_3) (b_4) (b_5) (b_6)] (box) {};
	\node[left of=box, node distance=3cm] {$B$};
	\draw[solid edge] (a_1) -- (b_5);
	\draw[solid edge] (a_1) -- (b_6);
	\draw[solid edge] (a_2) -- (b_4);
	\draw[solid edge] (a_2) -- (b_6);
	\draw[solid edge] (a_3) -- (b_4);
	\draw[solid edge] (a_3) -- (b_5);
	\draw[solid edge] (a_4) -- (b_3);
	\draw[solid edge] (a_4) -- (b_4);
	\draw[solid edge] (a_4) -- (b_5);
	\draw[solid edge] (a_4) -- (b_6);
	\draw[solid edge] (a_5) -- (b_2);
	\draw[solid edge] (a_5) -- (b_4);
	\draw[solid edge] (a_5) -- (b_5);
	\draw[solid edge] (a_5) -- (b_6);
	\draw[solid edge] (a_6) -- (b_1);
	\draw[solid edge] (a_6) -- (b_4);
	\draw[solid edge] (a_6) -- (b_5);
	\draw[solid edge] (a_6) -- (b_6);
	\node[label_a] at (a_1.north) {$a_{1}$};
	\node[label_b] at (b_1.south) {$b_{1}$};
	\node[label_a] at (a_2.north) {$a_{2}$};
	\node[label_b] at (b_2.south) {$b_{2}$};
	\node[label_a] at (a_3.north) {$a_{3}$};
	\node[label_b] at (b_3.south) {$b_{3}$};
	\node[label_a] at (a_4.north) {$a_{4}$};
	\node[label_b] at (b_4.south) {$b_{4}$};
	\node[label_a] at (a_5.north) {$a_{5}$};
	\node[label_b] at (b_5.south) {$b_{5}$};
	\node[label_a] at (a_6.north) {$a_{6}$};
	\node[label_b] at (b_6.south) {$b_{6}$};

\begin{scope}[xshift=8 cm]
		\node[matrix cell, fill=gray!20] at (-0.5, -0.8) {0};
		\node[matrix cell, fill=gray!20] at (-0.5, -1.3) {0};
		\node[matrix cell, fill=gray!20] at (-0.5, -1.8) {0};
		\node[matrix cell, fill=gray!20] at (-0.5, -2.3) {0};
		\node[matrix cell, fill=gray!20] at (-0.5, -2.8) {0};
		\node[matrix cell] at (-0.5, -3.3) {1};
		\node[matrix cell, fill=gray!20] at (0.0, -0.8) {0};
		\node[matrix cell, fill=gray!20] at (0.0, -1.3) {0};
		\node[matrix cell, fill=gray!20] at (0.0, -1.8) {0};
		\node[matrix cell, fill=gray!20] at (0.0, -2.3) {0};
		\node[matrix cell] at (0.0, -2.8) {1};
		\node[matrix cell, fill=gray!20] at (0.0, -3.3) {0};
		\node[matrix cell, fill=gray!20] at (0.5, -0.8) {0};
		\node[matrix cell, fill=gray!20] at (0.5, -1.3) {0};
		\node[matrix cell, fill=gray!20] at (0.5, -1.8) {0};
		\node[matrix cell] at (0.5, -2.3) {1};
		\node[matrix cell, fill=gray!20] at (0.5, -2.8) {0};
		\node[matrix cell, fill=gray!20] at (0.5, -3.3) {0};
		\node[matrix cell, fill=gray!20] at (1.0, -0.8) {0};
		\node[matrix cell] at (1.0, -1.3) {1};
		\node[matrix cell] at (1.0, -1.8) {1};
		\node[matrix cell] at (1.0, -2.3) {1};
		\node[matrix cell] at (1.0, -2.8) {1};
		\node[matrix cell] at (1.0, -3.3) {1};
		\node[matrix cell] at (1.5, -0.8) {1};
		\node[matrix cell, fill=gray!20] at (1.5, -1.3) {0};
		\node[matrix cell] at (1.5, -1.8) {1};
		\node[matrix cell] at (1.5, -2.3) {1};
		\node[matrix cell] at (1.5, -2.8) {1};
		\node[matrix cell] at (1.5, -3.3) {1};
		\node[matrix cell] at (2.0, -0.8) {1};
		\node[matrix cell] at (2.0, -1.3) {1};
		\node[matrix cell, fill=gray!20] at (2.0, -1.8) {0};
		\node[matrix cell] at (2.0, -2.3) {1};
		\node[matrix cell] at (2.0, -2.8) {1};
		\node[matrix cell] at (2.0, -3.3) {1};
		\node[matrix label] at (-0.5, -0.30000000000000004) {$b_{1}$};
		\node[matrix label] at (-1, -0.8) {$a_{1}$};
		\node[matrix label] at (0.0, -0.30000000000000004) {$b_{2}$};
		\node[matrix label] at (-1, -1.3) {$a_{2}$};
		\node[matrix label] at (0.5, -0.30000000000000004) {$b_{3}$};
		\node[matrix label] at (-1, -1.8) {$a_{3}$};
		\node[matrix label] at (1.0, -0.30000000000000004) {$b_{4}$};
		\node[matrix label] at (-1, -2.3) {$a_{4}$};
		\node[matrix label] at (1.5, -0.30000000000000004) {$b_{5}$};
		\node[matrix label] at (-1, -2.8) {$a_{5}$};
		\node[matrix label] at (2.0, -0.30000000000000004) {$b_{6}$};
		\node[matrix label] at (-1, -3.3) {$a_{6}$};
    \end{scope}

    \end{tikzpicture}
    \caption{Example of the $\epsilon$-excellence property not being monotonic. 
\emph{On the left}, a bipartite graph with two independent sets $A$ and $B$. 
A simple exhaustive check shows that $A$ is $\frac{1}{5}$-excellent. 
On the other, raising the $\epsilon$-value up to $\frac{2}{5}$ introduces a new 
$\frac{2}{5}$-good set $B$ witnessing that $A$ is not excellent, as half of the vertices of $A$ have one 
truth value, and half the other. 
\emph{On the right} is the corresponding adjacency matrix.}
    \label{fig:non-monotonic_example}
\end{figure}
