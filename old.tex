
In~\cite{regularity_lemmas_for_stable_graphs}\todo{Is there a better prior cite?}, the authors show that a critical
cause of the irregularity of pairs is the presence of a particular structure in the graph, called a \emph{half-graph}.
In fact, in the paper, they prove multiple versions of the regularity lemma which, in a context of graphs without (large)
half-graphs, not only improve drastically the bound on the number of parts, but also completely removes any irregular pair.

    %%%%%%%%%%%%%%%%%%%%%%%%%%%%%%%%%%%%%%%%%%%%%%%%%%%%%%%%%%%%%%%%%%%%%%%%%%%%%%%%%%%%%%%%%%%%%%%%%%%%%%%%%%%%%%%%%%%%

        A couple of of notes need to be made about what is purposely not specified in the previous definition.
        First, vertices from the same sequence are clearly different, since they are uniquely described by the
        connectivity with the other set.
        On the other hand, the two sequences of vertices need not to be disjoint, allowing any pair $a_i, b_j$ with
        $\neg a_i R b_j$ to represent the same vertex.
        Also, the definition does not specify whether vertices from the same sequence are connected or not.
        This results, in the property not only excluding just half-graphs, but a much larger class of subgraphs.

    %%%%%%%%%%%%%%%%%%%%%%%%%%%%%%%%%%%%%%%%%%%%%%%%%%%%%%%%%%%%%%%%%%%%%%%%%%%%%%%%%%%%%%%%%%%%%%%%%%%%%%%%%%%%%%%%%%%%

    \remark
        Notice that the previous two definitions become meaningful only if $|A|^{\epsilon} \leq \frac{\parstraight{A}}{2}$ and
        $f(|A|) \leq \frac{\parstraight{A}}{2}$ respectively.
        In this context, the definition of the truth value $t$ is compatible with that of Definition~\ref{def:truth_value}.
        In particular, Definition~\ref{def:epsilon_indivisible} would require that $\parstraight{A} \geq \parround{2}^{\frac{1}{1-\epsilon}}$.

    %%%%%%%%%%%%%%%%%%%%%%%%%%%%%%%%%%%%%%%%%%%%%%%%%%%%%%%%%%%%%%%%%%%%%%%%%%%%%%%%%%%%%%%%%%%%%%%%%%%%%%%%%%%%%%%%%%%%

    \lemma[Claim 4.5]\label{existance_of_ordered_epsilon_indivisible_partitions}
    Let $G$ be a graph with the non-$k_{*}$-order property.
    Assume $n \geq m_0 > \dots > m_{k_{**}}$ is a sequence of non-zero natural numbers satisfying that for all $l \in [k_{**}]$
        $\lfloor (m_{l-1})^\epsilon \rfloor = m_l$, for $\epsilon \in (0, \frac{1}{2})$.
    If $A \subseteq G$, $|A| = n$, then we can find $\overline{A} = \left< A_i \mid i \in [i(*)] \right>$ with remainder
        $B = A \setminus \bigcup \overline{A}$ such that:
    \begin{enumerate}  % If you do any changes here, update 4.10 too...
        \item \label{itm:4.5.1} For each $j \in [j(*)]$, $A_j$ is $\epsilon$-indivisible
        \item \label{itm:4.5.2} For each $j \in [j(*)]$, $|A_j| \in \left\{ m_0, \dots, m_{k_{**}-1} \right\}$
        \item \label{itm:4.5.3} $A_i \cap A_j = \emptyset$ $\forall i \neq j$
        \item \label{itm:4.5.4} $|B| < m_0$
        \item \label{itm:4.5.5} $\overline{A}$ is $\leq$-increasing
    \end{enumerate}
        \begin{proof}
            The first four clauses are direct consequence of applying Claim~\ref{existance_of_f_indivisible_partitions}
                with $f(n) = n^\epsilon$.
            By renaming the $A_i$'s in ascending-size order, we get (\ref{itm:4.5.5}).
        \end{proof}

    %%%%%%%%%%%%%%%%%%%%%%%%%%%%%%%%%%%%%%%%%%%%%%%%%%%%%%%%%%%%%%%%%%%%%%%%%%%%%%%%%%%%%%%%%%%%%%%%%%%%%%%%%%%%%%%%%%%%

    \remark
    In this context, if $m_{k_{**}} > k_{**}$
    \[
        n^{\epsilon^{k_{**}}} \geq m_0^{\epsilon^{k_{**}}} \geq m_1^{\epsilon^{k_{**}-1}} \geq \dots \geq m_{k_{**}} > k_{**}
    \]
    So, $n^{\epsilon^{k_{**}}} > k_{**}$.

    \lemma[Fact 5.9]\label{fact_5.9}
        Let $p,q \in \left( 0,1 \right)$.
        Let $A$ be a set of size $n$, $B \subseteq A$ a subset of size $p|A|$, and $A' \subseteq A$ a random subset
        of size $\geq q|A|$. % TODO: check conditions
        Then, for $\zeta > 0$,
        $$
            P\left( \frac{\left| A' \cap B \right|}{\left| A' \right|} \in
                 \left( \frac{\left| B \right|}{\left| A \right|} -
                 \zeta, \frac{\left| B \right|}{\left| A \right|} + \zeta \right) \right)
        $$
        can be modeled by a random variable which is asymptotically normally distributed when $n \to +\infty$.

    \lemma[Fact 5.10]\label{fact_5.10}
        Let $A$ be a set of events measured with a probability $P_A$.
        Let $S$ a family of subsets of $A$, which are measurable with $P_A$.
        Let $A_r = \left\{ a_1, \dots, a_r \right\} \subseteq A$ be a random sample of size $r$.
        For each $B \in S$, we may define $v_B^{A_r}$ the relative frequency of events of $B$ in $A_r$, i.e.,
        $$
            v_B^{A_r} = \frac{P_A(A_r \cap B)}{P_A(A_r)}
        $$
        Let
        $$
            \pi ^{A_r} = \sup_{B \in S} \left| v_B^{A_r} - P_A(B) \right|
        $$
        i.e. the upperbound of error of $v_B^{A_r}$ as an approximation of $P_A(B)$.
        Also let $\Delta^s(A_r)$ be the number of subsets of $A_r$ induced by sets of $S$ ($B\in S$ induces
        $B \cap A_r \subseteq A_r$), i.e.
        $$
            \Delta^s(A_r) = \left| \left\{ B \cap A_r \mid B \in S \right\} \right|
        $$
        Finally, let
        $$
            m^S(r) = \max_{C \in {A \choose r}} \Delta^s(C)
        $$
        Then, if there exists a finite $k > 0$ such that $m^s(r) \leq r^k +1$ for all $r > 0$, we have that,
        for all $\epsilon > 0$,
        $$
            \lim_{r \to +\infty} P_A\left( \pi^{A_r} > \epsilon \right) = 0
        $$

    \begin{remark}[Fact 5.12]\label{fact_5.12}
        If there exists $k > 0$ such that $m^s(r) \leq r^k + 1$ and $r$ satisfies:
        $$
            r \geq \frac{16}{\zeta^2} \left( k \log \frac{16 k}{\zeta^2} - \log \frac{\eta}{4} \right)^{k+1}
        $$
        for some $\eta > 0$, then
        $$
            P\left( \pi^{A_r} < \zeta \right) \geq 1 - \eta
        $$
        In particular, if we suppose that all events in $A$ are equiprobable and sampled without replacement, then
        $$
            P\left( \forall B \in S, \; \frac{\left| A_r \cap B \right|}{\left| A_r \right|} \in
                 \left( \frac{\left| B \right|}{\left| A \right|} -
                 \zeta, \frac{\left| B \right|}{\left| A \right|} + \zeta \right) \right) \geq 1 - \eta
        $$
        or in other words, for all but a fraction $\eta$ of all possible choices of $A_r$, we have that
        $$
            \forall B \in S, \; \frac{\left| A_r \cap B \right|}{\left| A_r \right|} \in
                 \left( \frac{\left| B \right|}{\left| A \right|} -
                 \zeta, \frac{\left| B \right|}{\left| A \right|} + \zeta \right)
        $$
    \end{remark}

    %%%%%%%%%%%%%%%%%%%%%%%%%%%%%%%%%%%%%%%%%%%%%%%%%%%%%%%%%%%%%%%%%%%%%%%%%%%%%%%%%%%%%%%%%%%%%%%%%%%%%%%%%%%%%%%%%%%%

            % TODO: add some intuition about the proof.
            \item For each $b \in G$ we say that $\overline{B}_{A,b}$ is \emph{exceptional} if
                $\left| \overline{B}_{A,b} \right| \geq \epsilon \left| A' \right|$.
                Notice that, if we prove that, with probability $1-\xi$, $A'$ satisfies that for all exceptional $\overline{B}_{A,b}$:
                $$
                    \frac{\left| A' \cap \overline{B}_{A,b} \right|}{\left| A' \right|} \in
                         \left( \frac{\left| \overline{B}_{A,b} \right|}{\left| A \right|} -
                         \zeta, \frac{\left| \overline{B}_{A,b} \right|}{\left| A \right|} + \zeta \right)
                $$
                then, with the same probability:
                \begin{equation}\label{eq:equation1}
                    \left| A' \cap \overline{B}_{A,b} \right| < \left( \frac{\left| \overline{B}_{A,b} \right|}{|A|} + \zeta \right) |A'|
                        < \left( \epsilon + \zeta \right) |A'|
                \end{equation}
                and we are done.

                By Lemma~\ref{fact_5.9}, for $n = |A|$ large enough, we can approximate sampling a set of size $m$ from $A$,
                with $m$ i.i.d. random variables $x_1, \dots, x_m$, where each $x_i$ picks a vertex uniformly at random from $A$.
                % TODO: here we ignored the fact that we need to fix p and q in Fact 5.9.
                Let $S \coloneq \left\{ \text{Exceptional } \overline{B}_{A,b} \right\}$.
                Since $G$ has the non-$k_{*}$-order property, we can apply Lemma~\ref{itm:2.6.1} to $G_{\ref{itm:2.6.1}} = A$
                and $A_{\ref{itm:2.6.1}} = A'$, which gets us that:
                $$
                    |S| \leq \left|\left\{ \left\{ a \in A' \mid a R b \not\equiv t(A', b) \right\} \mid b \in G \right\} \right|
                    \leq |A'|^{k_*}
                $$
                Then,
                $$
                    m^s(\ell) \leq \left| S \right| \leq |A'|^{k_*} \leq \ell^{k_*} \leq \ell^{k_{*}} + 1 \quad \forall \ell \geq |A'|
                $$
                Notice that this is enough to satisfy the conditions of Lemma~\ref{fact_5.10}:
                For each $\ell < |A'|$, let $k_l$ be the smallest integer such that $m^s(\ell) \leq \ell^{k_l} + 1$.
                Since there are finitely many of them, we can take the maximum
                $k_{\max} = \max \left\{ k_1, \dots, k_{|A'|-1}, k_* \right\}$, which satisfies
                $$
                    m^s(\ell) \leq \ell^{k_{\max}} + 1 \quad \forall \ell
                $$
                % TODO: all of this may not be necessary, since I think that if \ell' < \ell, then k_l' < k_l
                % TODO: in \ref{fact_5.12}, the bound on m grows with k_max, and here k_max grows with m. Need to solve this
                So we conclude equation (\ref{eq:equation1}), which by itself is sufficient to prove $A'$ is
                $(\epsilon + \zeta)$-good.

    %%%%%%%%%%%%%%%%%%%%%%%%%%%%%%%%%%%%%%%%%%%%%%%%%%%%%%%%%%%%%%%%%%%%%%%%%%%%%%%%%%%%%%%%%%%%%%%%%%%%%%%%%%%%%%%%%%%%

            \[
                M \leq \ceil{\frac{12}{\min\parround{\frac{\sqrt{\gamma}}{2}, \frac{\epsilon_{\ref{lem:H_like_partition_implies_H_abundance}}
                    \parround{1 - \frac{\sqrt{\gamma}}{2}, \ell}}{\ell}}}}^{2^{k_*+1}-1}
            \]

    %%%%%%%%%%%%%%%%%%%%%%%%%%%%%%%%%%%%%%%%%%%%%%%%%%%%%%%%%%%%%%%%%%%%%%%%%%%%%%%%%%%%%%%%%%%%%%%%%%%%%%%%%%%%%%%%%%%%

    \begin{lemma}[Claim 4.14] \label{lem:existance_of_equitative_partition_with_bound_exceptional_pairs}
        Let $G$ be a finite graph with the non-$k_{*}$-order property.
        Let $\Partriangle{m_\ell \mid \ell \in \parcurly{0, \dots, k_{**}}}$ be a sequence of non-zero natural numbers such that
        for all $\ell \in \parcurly{0, \dots, k_{**}-1}$, $\floor{m_\ell^\epsilon} = m_{\ell+1}$,
        for some $\epsilon \in (0, \frac{1}{2})$ such that $2 < (m_{k_{**}-1})^{1-2\epsilon}$.
        Also, suppose $m_0$ satisfies $m_0 < \frac{n}{n^{(1 - 2\epsilon)\epsilon^{k_{**}}}}$ and
        $n^\epsilon \leq m_0 \leq \frac{\sqrt{2}-1}{\sqrt{2}} n$.
        \todo{Probably put both conditions together with a max()}
        Finally, let $m_{**}$ be a divisor of $m_\ell$ for all $\ell \in \parcurly{0, \dots, k_{**}-1}$ and
        $m_{**} \leq n^{\frac{\epsilon^{k_{**}+1}}{3}}$.
        If $A \subseteq G$ with $|A| = n$, then we can find a partition $\overline{A} = \Partriangle{A_i \mid i \in \parcurly{1, \dots, r}}$
        with reminder $B = A \setminus \bigcup \overline{A}$ such that:
        \begin{enumerate}
            \item \label{itm:existance_of_equitative_partition_with_bound_exceptional_pairs.1} $|A_i| = m_{**}$ for all $i \in \parcurly{1, \dots, r}$.
            \item \label{itm:existance_of_equitative_partition_with_bound_exceptional_pairs.2} For all but $\frac{2}{n^{(1-2\epsilon)\epsilon^{k_{**}}}}r^2$ of the pairs
                $(A_i, A_j)$ with $i<j$ there are no exceptional edges, i.e.
                \[
                    \parcurly{(a,b) \in A_i \times A_j \mid a R b \not\equiv t(A_i, A_j)} = \emptyset
                \]
            \item \label{itm:existance_of_equitative_partition_with_bound_exceptional_pairs.3} $|B| < m_0$
        \end{enumerate}
        \begin{proof}
            We can use \Cref{lem:existance_of_ordered_f_indivisible_partitions} to get a partition
            $\overline{A'} = \Partriangle{A'_i \mid i \in \parcurly{1, \dots, i(*)}}$ and remainder $B' = A \setminus \bigcup A'$.
            We can refine the partition by randomly splitting each $A'_i$ into pieces of size $m_{**}$ (\dref{itm:existance_of_equitative_partition_with_bound_exceptional_pairs.1}).
            Consider the resulting partition $\overline{A} = \Partriangle{A_i \mid i \in \parcurly{1, \dots, r}}$ with remainder $B = B'$
            (\dref{itm:existance_of_equitative_partition_with_bound_exceptional_pairs.3}).
            First of all, notice that for each pair $(A_i, A_j)$ such that $A_i \subseteq A'_{i_1}$ and
            $A_j \subseteq A'_{j_1}$ with $i_1 \neq j_1$, the probability of the pair having exceptional edges is
            upper bounded by $\frac{2}{n^{(1-2\epsilon)\epsilon^{k_{**}}}}$.
            This follows \Cref{lem:bound_on_the_probability_of_a_subpair_having_no_exceptions} in the context of
            \Cref{rmk:subpair_bound_specification}.
            Thus, given $X$ the random variable counting the number of exceptional pairs of this kind, we have
            \[
                E(X) = \sum_{\substack{A_i,A_j \text{ s.t.}\\A_i\subseteq A'_{i_1},A_j\subseteq A'_{j_1}\\i_1\neq j_1}} E(X_{A_i, A_j})
                     = \sum_{\substack{A_i,A_j \text{ s.t.}\\A_i\subseteq A'_{i_1},A_j\subseteq A'_{j_1}\\i_1\neq j_1}} P(\varepsilon_{A_i, A_j,m_{**}})
                     \leq \frac{r^2}{2} \frac{2}{n^{(1-2\epsilon)\epsilon^{k_{**}}}}
            \]
            where $X_{A_i,A_j}$ is the random variable giving $1$ if $(A_i, A_j)$ is exceptional, and $0$ otherwise.
            Since the expectation is an average, for some refinement $\overline{A}$ of $\overline{A'}$ we have that
            the number of exceptional pairs when $i_1 \neq j_1$ is at most $\frac{r^2}{2} \frac{2}{n^{(1-2\epsilon)\epsilon^{k_{**}}}}$.
            Now, we have no control if $i_1 = j_1$, so let's bound how many of these we have:
            \[
                \begin{split}
                    |\parcurly{\text{Exceptional } (A_i, A_j) \mid A_i, A_j \subseteq A'_{i_1}, i_1 \in \parcurly{1, \dots, i(*)}}|
                        & \leq {\frac{m_0}{m_{**}} \choose 2} \frac{n}{m_0} \\
                        & \leq \frac{\parround{\frac{m_0}{m_{**}}}^2}{2} \frac{n}{m_0}
                            = \frac{m_0 n}{2 m_{**}^2}
                            = \frac{m_0}{n} \parround{\frac{n}{\sqrt{2}m_{**}}}^2 \\
                        & \leq \frac{m_0}{n} \parround{\frac{n - m_0}{m_{**}}}^2
                            \leq \frac{m_0}{n} r^2
                            < \frac{r^2}{n^{(1-2\epsilon) \epsilon^{k_{**}}}}
                \end{split}
            \]
            Notice that the third inequality comes after the condition $m_0 \leq \frac{\sqrt{2}-1}{\sqrt{2}} n$.
            Putting it all together, we see that the number of exceptional pairs is upper bounded by
                $\frac{2r^2}{n^{(1-2\epsilon)\epsilon^{k_{**}}}}$ satisfying \dref{itm:existance_of_equitative_partition_with_bound_exceptional_pairs.2}.
        \end{proof}
    \end{lemma}

    %%%%%%%%%%%%%%%%%%%%%%%%%%%%%%%%%%%%%%%%%%%%%%%%%%%%%%%%%%%%%%%%%%%%%%%%%%%%%%%%%%%%%%%%%%%%%%%%%%%%%%%%%%%%%%%%%%%%

    \begin{theorem}[Theorem 4.16] \label{thm:existance_of_equitative_partition_with_perfect_pairs_but_with_bound_exceptional_pairs}
        Let $\epsilon = \frac{1}{r} \in \parround{0, \frac{1}{2}}$ with $r \in \mathbb{N}$ (this avoids rounding error)
        and $k_*$ be given.
        Let $G$ be a finite graph with the non-$k_*$-order property.
        Let $A \subseteq G$ with $|A| = n$, and $n > 2^{\frac{r^{k_**}}{1-2\epsilon}}$.
        Then, for any $m_{**} \in \parsquared{n^{\frac{\epsilon^{k_{**}+2}}{3}},
        \parround{\frac{\sqrt{2}-1}{\sqrt{2}}}^{\frac{1}{3}\epsilon^{k_{**}+1}} n^{\frac{\epsilon^{k_{**}+1}}{3} -
        \frac{1-2\epsilon}{3}\epsilon^{2k_{**} + 1}}}$, there is a partition
        $\overline{A} = \Partriangle{A_i \mid i \in \parcurly{1, \dots, m}}$ of $A$ with remainder
        $B = A \setminus \bigcup \overline{A}$ such that:
        \begin{enumerate}
            \item\label{itm:existance_of_equitative_partition_with_perfect_pairs_but_with_bound_exceptional_pairs.1}
                $|A_i| = m_{**}$ for all $i \in \parcurly{1, \dots, m}$.
            \item\label{itm:existance_of_equitative_partition_with_perfect_pairs_but_with_bound_exceptional_pairs.2}
                $|B| < m_{**}^{3r^{k_{**}+1}}$.
            \item\label{itm:existance_of_equitative_partition_with_perfect_pairs_but_with_bound_exceptional_pairs.3}
                $|\parcurly{ (i,j) \mid i,j \in \parcurly{1, \dots, m}, i < j \text{ and }
                \parcurly{(a,b) \in A_i \times A_j \mid a R b} \notin
                \parcurly{A_i \times A_j, \emptyset}}|
                \leq \frac{2}{n^{(1-2\epsilon)\epsilon^{k_{**}}}} m^2$
        \end{enumerate}
        \begin{proof}
            Let $m_{k_{**}} = m_{**}^{3r}$, and consider the sequence
            \[
                m_{**} \leq m_{k_{**}} < \dots < m_0
            \]
            such that for all $\ell \in \parcurly{1, \dots, k_{**}}$ we have that $m_{\ell-1} = m_\ell^r$.
            Notice that:
            \begin{enumerate}
                \item $m_{**}$ divides $m_\ell$ for all $\ell \in \parcurly{0, \dots, k_{**}}$ since the $m_\ell$'s are powers of $m_{k_{**}}$
                    and $m_{**}$ divides $m_{k_{**}}$ by construction.
                \todo{Probably it is not needed that $m_{**}$ divides $m_{k_{**}}$, with $m_{k_{**}-1}$ is enough}
                \item $(m_{\ell-1})^\epsilon = m_\ell$ for all $\ell \in \parcurly{1, \dots, k_{**}}$.
                \item $m_{**} \leq n^{\frac{1}{3}\epsilon^{k_{**}+1}}$.
                \item $m_0 = m_{**}^{3r^{k_{**}+1}}$, so on one hand
                    \[
                        m_0 = m_{**}^{3r^{k_{**}+1}} \geq n^{\frac{1}{3}\epsilon^{k_{**}+2} 3r^{k_{**}+1}}
                            \geq n^{\epsilon}
                    \]
                    and on the other hand,
                    \[
                        m_0 = m_{**}^{3r^{k_{**}+1}} \leq \parround{\frac{\sqrt{2}-1}{\sqrt{2}}} n^{1 - \parround{1 - 2\epsilon} \epsilon^{k_{**}}}
                    \]
                    and thus $n$ is both smaller than $\parround{\frac{\sqrt{2}-1}{\sqrt{2}}} n$ and
                    smaller than $n^{1 - \parround{1 - 2\epsilon} \epsilon^{k_{**}}}$.
                \item $m_{k_{**}-1} = m_{**}^{3r^2} \geq n^{\epsilon^{k_{**}}} > 2^{\frac{1}{1-2\epsilon}}$.
            \end{enumerate}
            So, all the conditions of \Cref{lem:existance_of_equitative_partition_with_bound_exceptional_pairs} are satisfied,
            and we can use it to get a partition $\overline{A}$ with remainder $B$ satisfying the statement.
            Notice that \dref{itm:existance_of_equitative_partition_with_perfect_pairs_but_with_bound_exceptional_pairs.2}
            is satisfied by the fact that $|B| < m_0 \leq m_{**}^{3r^{k_{**}+1}}$.
        \end{proof}
    \end{theorem}

    %%%%%%%%%%%%%%%%%%%%%%%%%%%%%%%%%%%%%%%%%%%%%%%%%%%%%%%%%%%%%%%%%%%%%%%%%%%%%%%%%%%%%%%%%%%%%%%%%%%%%%%%%%%%%%%%%%%%

    \begin{remark} \label{rmk:subpair_bound_specification}
        Since $\epsilon < \frac{1}{2}$, we can take $c = 1 - 2\epsilon$.
        In this context, $\zeta \leq \frac{\epsilon^{k_{**}+1}}{3}$.
    \end{remark} \todo{Move remark at the end, and add other values.}

    %%%%%%%%%%%%%%%%%%%%%%%%%%%%%%%%%%%%%%%%%%%%%%%%%%%%%%%%%%%%%%%%%%%%%%%%%%%%%%%%%%%%%%%%%%%%%%%%%%%%%%%%%%%%%%%%%%%%

    % old bibliography.tex
    \begin{thebibliography}{99}

        \printbibliography[heading=none]

    \end{thebibliography}

    %%%%%%%%%%%%%%%%%%%%%%%%%%%%%%%%%%%%%%%%%%%%%%%%%%%%%%%%%%%%%%%%%%%%%%%%%%%%%%%%%%%%%%%%%%%%%%%%%%%%%%%%%%%%%%%%%%%%

    \begin{theorem}[Theorem 4.23] \label{thm:equitative_partition_high_regularity_parts_grow_with_n}
        Let $G$ be a graph with the non-$k_*$-property.
        For any $c \in \mathbb{N}$, $\epsilon, \xi \in \mathbb{R}$ satisfying the hypothesis of \Cref{lem:many_values_to_equitative_partition_with_bound_exceptional_pairs}
        (with $k = k_*$ and $\zeta = \frac{1}{k_*}$), any $\theta \in (0,1)$ and $A \subseteq G$ large enough
        $\parstraight{A} = n > N\parround{c, \epsilon, \zeta, \xi, \theta}$,
        there is a partition $\overline{A} = \Partriangle{A_i \mid i \in \parcurly{1, \dots, i(*)}}$ of $A$ with remainder
        $B = A \setminus \bigcup_{i \in \parcurly{1, \dots, i(*)}} A_i$ satisfying:
        \begin{itemize}
            \item $|A_i| = \lfloor \lfloor n^\theta \rfloor ^\zeta \rfloor$ for all $i \in \parcurly{1, \dots, i(*)}$.
            \item $A_i$ is $c$-indivisible for all $i \in \parcurly{1, \dots, i(*)}$ where $c$ is the constant function $f(x) = c$.
            \item $|B| < m_0 \approx n^{\frac{\theta}{\epsilon^{k_{**}}}}$. \todo{$m_0$ cannot be the bound as it is not defined in the statement. There is no clear
                to write the real bound in a clear way without error or absurdly worse bounds. I think the better
                solution is to force $\epsilon$ (and possibly other parameters) to be a fraction $\frac{1}{r}$.
                If so, make a remark saying that this condition is not necessary but makes bounds cleaner.}
        \end{itemize}
        \begin{proof}
            Let $n > N\parround{c, \epsilon, \zeta, \xi, \theta} \coloneqq \bparround{g_\epsilon^{k_{**}}\bparround{N_{\ref{lem:n_large_enough_valid_values}}
                \parround{ \epsilon, \frac{1}{k_*}, \frac{\xi}{\epsilon^{k_{**}}}, c}} + 1 }^{\frac{1}{\theta}}$,
            so that $\lfloor n^\theta \rfloor$ satisfies the large enough condition of
            \Cref{lem:many_values_to_equitative_partition_with_bound_exceptional_pairs}:
            \[
                \lfloor n^\theta \rfloor
                    > g_\epsilon^{k_{**}}\bparround{N_{\ref{lem:n_large_enough_valid_values}}
                        \parround{\epsilon, \frac{1}{k_*}, \frac{\xi}{\epsilon^{k_{**}}}, c}}
            \]
            Now, we define a decreasing sequence $m_0 > m_1 > \dots > m_{k_{**}}$ with $m_{k_{**}} = \lfloor n^\theta \rfloor$
            and $m_{\ell} = \lceil \parround{m_{\ell+1}}^{\frac{1}{\epsilon}} \rceil$ for all $\ell \in \parcurly{0, \dots, k_{**}-1}$.
            This sequence satisfies the condition of \Cref{lem:existance_of_indivisible_sets} for $f(n) = n^\epsilon$.
            We will build a sequence of disjoint $c$-indivisible subsets $A_i$ by induction on $i$ as follows.
            Let $R_i = A \setminus \bigcup_{j<i} A_j$ (so $R_1 = A$).
            If $R_i < m_0$, then
            $\overline{A} = \Partriangle{A_j \mid j < i = i(*)}$ and $B = R_i$, and we are done.
            Otherwise, we can apply \Cref{lem:existance_of_indivisible_sets} to $R_i$ with the sequence
            $\Partriangle{m_\ell}_{\ell \leq k_{**}}$, to obtain an $\epsilon$-indivisible subset $B_i \subseteq R_i$ of
            size $m_{k_{**}-\ell}$.
            \todo{WHY IN THE HELL DO WE NEED TO HAVE EPSILON-INDIVISIBLE SETS??}
            Then, since $|B_i| = m_{k_{**}-\ell} \geq m_{k_{**}} = \lfloor n^\theta \rfloor$ by the $n$-large-enough assumption,
            we can apply \Cref{lem:many_values_to_equitative_partition_with_bound_exceptional_pairs} and get a
            $c$-indivisible subset $Z_i$ of size $|Z_i| = \lfloor m_{k_{**}-\ell}^\zeta \rfloor
            \geq \lfloor \lfloor n^{\frac{\theta}{\epsilon^\ell}} \rfloor ^\zeta \rfloor
            \geq \lfloor \lfloor n^{\theta} \rfloor ^\zeta \rfloor$.
            Since $c$-indivisibility is preserved when taking subsets,
            we can choose $A_i \subseteq Z_i$ to be a $c$-indivisible subset of size $\lfloor \lfloor n^{\theta} \rfloor ^\zeta \rfloor$.
        \end{proof}
    \end{theorem}

    %%%%%%%%%%%%%%%%%%%%%%%%%%%%%%%%%%%%%%%%%%%%%%%%%%%%%%%%%%%%%%%%%%%%%%%%%%%%%%%%%%%%%%%%%%%%%%%%%%%%%%%%%%%%%%%%%%%%

    \begin{algorithm}[H]
        \caption{$\epsilon$-test $\mathcal{A}$ for deciding $H$-freeness for a given graph $H$ of size $\ell$}
        \label{alg:h-freeness_tester}
        \begin{algorithmic}[1]
            \Require a graph $G$ of size $n$ with non-$k_*$-order property
            \State $t \leftarrow \frac{2}{3}\frac{1}{\eta_{\ref{thm:property_testing_with_stable_partitions}}(k_*, \epsilon, \ell)}$
            \If{$n < \ell$}
                \State return 0 \label{line:G_smaller_then_H}
            \ElsIf{$n^\ell < t$}
                \State return 0 \label{line:G_and_H_too_small_for_t}
            \Else
                \State $S \leftarrow \emptyset$
                \While{$i \leq t$}
                    \State $S_i = \Partriangle{s_{i,1}, \dots, s_{i,\ell}}$ such that $s_{i,j} \sim G$
                    \While{$S_i \in S$}
                        \State $S_i = \parcurly{s_{i,1}, \dots, s_{i,\ell}}$ such that $s_{i,j} \sim G$
                    \EndWhile
                    \State $S \leftarrow S \cup \parcurly{S_i}$
                \EndWhile
                \State query all edges induced by the vertex set $\overline{S} = \cup_{i \leq t} S_i$
                \If{$\exists v_{i_1}, \dots, v_{i_\ell} \in \overline{S}$ such that
                        $\parcurly{v_{i_1}, \dots, v_{i_\ell}}$ induces a copy of $H$ in $G$}
                    \State return 1 \label{line:found_H}
                \Else
                    \State return 0 \label{line:not_found_H}
                \EndIf
            \EndIf
        \end{algorithmic}
    \end{algorithm}

    %%%%%%%%%%%%%%%%%%%%%%%%%%%%%%%%%%%%%%%%%%%%%%%%%%%%%%%%%%%%%%%%%%%%%%%%%%%%%%%%%%%%%%%%%%%%%%%%%%%%%%%%%%%%%%%%%%%%

    \item \textbf{A rigorous reformulation and correction of the central proofs}
        in~\cite{regularity_lemmas_for_stable_graphs}.
        Our contribution provides a self-contained, combinatorial framework for these results, systematically
        resolving foundational gaps and inaccuracies in the original arguments to ensure their validity.
        This reworking also makes the associated combinatorial bounds fully explicit for the first time.

    %%%%%%%%%%%%%%%%%%%%%%%%%%%%%%%%%%%%%%%%%%%%%%%%%%%%%%%%%%%%%%%%%%%%%%%%%%%%%%%%%%%%%%%%%%%%%%%%%%%%%%%%%%%%%%%%%%%%

    In the context of graphs with no bi-induced large half-graph, a class known as \emph{stable graphs},
    the authors of~\cite{regularity_lemmas_for_stable_graphs} show that a much stronger form
    of regularity is achievable.
    Their \emph{stable regularity lemma} not only guarantees a decomposition entirely free of irregular pairs, but also yields
    vastly improved bounds on the partition size.

    %%%%%%%%%%%%%%%%%%%%%%%%%%%%%%%%%%%%%%%%%%%%%%%%%%%%%%%%%%%%%%%%%%%%%%%%%%%%%%%%%%%%%%%%%%%%%%%%%%%%%%%%%%%%%%%%%%%%



    %%%%%%%%%%%%%%%%%%%%%%%%%%%%%%%%%%%%%%%%%%%%%%%%%%%%%%%%%%%%%%%%%%%%%%%%%%%%%%%%%%%%%%%%%%%%%%%%%%%%%%%%%%%%%%%%%%%%