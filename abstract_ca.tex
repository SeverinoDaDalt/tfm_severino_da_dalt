\documentclass[11pt]{article}

\usepackage{amsmath}

\begin{document}

    El lema de regularitat de Szemerédi és una pedra angular de la teoria moderna de grafs, afirmant que qualsevol graf
    es pot particionar en un nombre acotat de conjunts de vèrtexs, on les connexions entre la majoria de parelles de
    conjunts es comporten de manera quasi-aleatòria.
    Malgrat les seves àmplies aplicacions en àrees com la teoria de nombres, la combinatòria i la informàtica,
    el lema pateix de dues limitacions principals: una mida de la partició acotada per una torre d'exponencials,
    i la presència de parelles irregulars, totes dues inevitables en el cas general.

    Aquest treball se centra en una subclasse específica de grafs, els \emph{grafs estables}, on aquestes limitacions
    es poden superar.
    En evitar una subestructura bipartida coneguda com a half-graph, els grafs estables admeten un lema de regularitat
    molt més fort.
    Aquest lema especialitzat, desenvolupat originalment per Malliaris i Shelah, garanteix una partició on totes les
    parelles són regulars i el nombre de parts està acotat per un polinomi, una millora significativa respecte a la cota
    general de tipus torre.

    Aquesta tesi primer presenta una exposició autocontinguda, combinatòria i completa de la prova del lema de
    regularitat estable, desenvolupant un marc notacional unificat per connectar conceptes de la teoria de grafs
    extremals, l'estabilitat i el property testing.
    Basant-nos en aquesta base teòrica, després construïm un algorisme eficient per comprovar l'\emph{$H$-llibertat}
    (la propietat de no contenir una còpia induïda d'un graf fixat $H$) per a grafs estables.
    Aquesta aplicació aprofita les propietats superiors del lema per aconseguir una complexitat de consulta amb cotes
    significativament millorades en comparació amb els testers per a grafs generals.

\end{document}