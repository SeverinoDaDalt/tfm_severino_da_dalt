\abstracteng{
    Szemerédi's Regularity Lemma is a cornerstone of modern graph theory, asserting that any graph can be partitioned
    into a bounded number of vertex sets, where the connections between most pairs of sets behave quasi-randomly.
    Despite its wide-ranging applications in areas like number theory, combinatorics and computer science, the lemma suffers from
    two major limitations:
    a partition size bounded by a tower of exponentials,
    and the presence of irregular pairs, both unavoidable in the general case.

    This work focuses on a specific subclass of graphs, the \emph{stable graphs}, where these limitations can be overcome.
    By avoiding a bipartite substructure known as the half-graph, stable graphs admit a much stronger regularity lemma.
    This specialized lemma, originally developed by Malliaris and Shelah, guarantees a partition where all pairs are
    regular and the number of parts is bounded by a single exponential, a significant improvement over the general
    tower-type bound.

    This thesis first presents a self-contained, combinatorial, and complete presentation of the proof of the stable
    regularity lemma, developing a unified notational framework to bridge concepts from extremal graph theory, stability,
    and property testing.
    Building on this theoretical foundation, we then construct an efficient algorithm for testing \emph{$H$-freeness}
    (the property of not containing an induced copy of a fixed graph $H$) for stable graphs.
    This application leverages the lemma's superior properties to achieve a query complexity with significantly
    improved bounds compared to testers for general graphs.}