\section{Section 2} \label{sec:section_2}

    All this work holds on the idea that a sufficient condition for a graph to be stable is the absence of
    a certain kind of structure called \emph{half-graph}.
    We now proceed to formalize this property using model theory notation: the \emph{order property}.

    \definition\label{def:k_order_property}
        Let $G$ be a graph.
        We say that $G$ has the \emph{$k$-order property} if there exist two sequences of vertices
        $\partriangle{a_i \mid i \in \parsquared{1, k}}$ and $\partriangle{b_i \mid i \in \parsquared{1, k}}$ such that
        for all $i,j \leq k$, $a_i R b_j$ if and only if $i \neq j$.
        Otherwise, we say that $G$ has the \emph{non-$k$-order property}.
    \todo{Add visual example of a half-graph}

    \remark
    Notice that $G$ having $k$-order property implies $G$ having $k'$-order property for all $k' \leq k$.
    Conversely, $G$ having the non-$k$-order property implies $G$ having non-$k'$-order property for all $k' \geq k$.

    \definition[Truth value]\label{def:truth_value}
        Let $G$ be a graph.
        For any $A, B \subseteq G$, we say that
        $$
            t(A,B) =
            \begin{cases}
                0 & \text{if } \left| \left\{ \parround{a, b} \in A \times B \mid \neg a R b \right\} \right| <
                    \left| \left\{ \parround{a, b} \in A \times B \mid a R b \right\} \right| \\
                1 & \text{otherwise}
            \end{cases}
        $$
        is the \emph{truth value} of the pair $\parround{A,B}$.

        When $B = \parcurly{b}$, we write $t(A,b)$ instead of $t(A,\parcurly{b})$, and we say that it is the truth value of $A$
        with respect to $b$.
    \newline

    \todo{I don't like how this looks}
    Extra notation:
    \begin{itemize}
        \item $B_{A,b} = \left\{ a \in A \mid a R b \equiv t(A,b) \right\}$.
        \item $\overline{B}_{A,b} = \left\{ a \in A \mid a R b \not\equiv t(A,b) \right\}$.
        \item $B^+_{A,b} = \left\{ a \in A \mid a R b \right\}$.
        \item $B^-_{A,b} = \left\{ a \in A \mid \neg a R b \right\}$.
    \end{itemize}
    With this notation, notice that either $t(A,b) = 1$ and thus $B_{A,b} = B^+_{A,b}$, or $t(A,b) = 0$ and $B_{A,b} = B^-_{A,b}$.

    Large sets $B_{A,b}$, as we will see in the next sections, are closely related with lack of regularity in the graph.
    A useful tool to deal with them is Claim~\ref{claim_2.6}, which gives a bound on the number of such sets under the
    non-$k$-order property.
    In order to prove it, we first need to introduce the \emph{VC dimension} of a family of sets, and relate it to the
    $k$-order property.
    This, together with Lemma~\ref{lemma:sauer-shelah}, will give us the desired result.

    \definition\label{def:shattered}
        Let $S = \parcurly{S_i \mid i \in I}$ be a family of sets.
        A set $A$ is said to be \emph{shattered} by $S$ (and $S$ is said to \emph{shatter} $A$) if
        for every $B \subseteq A$, there exists $S_i \in S$ such that $S_i \cap A = B$.

    \definition\label{def:VC_dimension}
        Let $S = \parcurly{S_i \mid i \in I}$ be a family of sets.
        The \emph{VC dimension} of $S$ is the size of the largest set $A$ that is shattered by $S$.

    \lemma[Sauer-Shelah]\label{lemma:sauer-shelah}
        Let $S = \parcurly{S_i \mid i \in I}$ be a family of sets.
        If the VC dimension of $S$ is at most $k$, and the union of all sets in $S$ has $n$ elements, then
        $S$ consists of at most $\sum_{i=0}^{k} \binom{n}{i} \leq n^k$ sets.
        \newline

        In order to prove the previous lemma, we first prove a stronger version of this lemma from Pajor.

        \lemma[Sauer-Shelah-Pajor]\label{lemma:pajor}
            Let $S$ be a finite family of sets.
            Then $S$ shatters at least $\parstraight{S}$ sets.
            \begin{proof}
                We will prove this by induction on the cardinality of $S$.
                If $\parstraight{S} = 1$, then $S$ consists of a single set, which only can shutter the empty set.
                If $\parstraight{S} > 1$, we may choose an element $x \in S$ such that some sets of $S$ contain $x$ and some do not.
                Let $S^+ = \parcurly{s \in S \mid x \in S}$ and $S^- = \parcurly{s \in S \mid x \not\in S}$.
                Then $S = S^+ \sqcup S^-$, and both $S^+$ and $S^-$ are non-empty.
                By induction hypothesis, we know that $S^+ \subsetneq S$ shatters at least $\parstraight{S^+}$ sets,
                and $S^- \subsetneq S$ shatters at least $\parstraight{S^-}$ sets.
                Let $T, T^+, T^-$ be the families of sets shattered by $S$, $S^+$ and $S^-$ respectively.
                To conclude the proof, we just need to show that for each element in $T^+$ and $T^-$, there is a corresponding
                one in $T$.
                If a set is shattered by only one of the two families $S^+$ and $S^-$, then it only contributes by one unit
                to $\parstraight{T^+} + \parstraight{T^-}$ and one unit to $\parstraight{T}$.
                Notice that no set shattered by $S^+$ or $S^-$ may contain $x$, otherwise all or none of the intersections
                will contain this element.
                Thus, if a set $s$ is shattered by both $T^+$ and $T^-$, it will contribute by two units to
                $\parstraight{T^+} + \parstraight{T^-}$ and one unit to $\parstraight{T}$.
                But then, for each such set, we can consider $s \cup \parcurly{x}$ which is not in $T^+$ or $T^-$, but it is in $T$.
                This follows the fact that for each subset of $s$, if it does not contain $x$ it is the intersection with some
                set in $S^- \subsetneq S$, and if does contain $x$ it is the intersection with some set in $S^+ \subsetneq S$.
                All in all, we conclude that
                \[
                    \parstraight{T} \geq \parstraight{T^+} + \parstraight{T^-} \geq \parstraight{S^+} + \parstraight{S^-}
                                    \geq \parstraight{S}
                \]
            \end{proof}

        \begin{proof} (of Lemma~\ref{lemma:sauer-shelah})
            Suppose that the union of $S$ has $n$ elements.
            Since there are at most $\sum_{i=0}^k \binom{n}{i}$ subsets of $S$ of size at most $k$, if
            $\parstraight{S} > \sum_{i=0}^k \binom{n}{i}$, by Lemma~\ref{lemma:pajor} $S$ shatters at least
            $\parstraight{S}$ subsets, and thus at least one of them has cardinality larger than $k$.
        \end{proof}

    \lemma\label{lemma:vs_dimension_implies_k_order_property}
        Let $G$ be a graph and $A \subseteq G$.
        Let $S = \parcurly{B_{A,b} \mid b \in G \setminus A}$.
        If $S$ has VC dimension (at least) $k$, then $G$ has the $k$-order property.
        \begin{proof}
            If $S$ has VC dimension $k$, then it shatters a set $A' \subseteq A$ of size $k$.
            Now, choose any order of the vertices of $A' = \partriangle{a_1, \dots, a_k}$.
            Consider now the increasing sequence of subsets $A_1 \subseteq A_2 \subseteq \dots \subseteq A_k = A'$,
            where $A_i = \parcurly{a_j \mid j \in [1,i]}$.
            Since $A'$ is shattered by $S$, for each $i \in [1,k]$ there exists a $b_i \in G$ such that
            $b_i R a$ if and only if $a \in A_i$.
            In particular, the two sequences $\partriangle{a_i \mid i \in [1,k]}$ and $\partriangle{b_i \mid i \in [1,k]}$
            satisfy
            \[
                a_i R b_j \Leftrightarrow i \leq j
            \]
            and thus $G$ has the $k$-order property.
        \end{proof}

    \lemma[Claim 2.6]\label{claim_2.6}
        Let $G$ be a graph with the non-$k$-order property.
        Then, for any finite $A \subseteq G$,
        $$
            \left|\left\{ \left\{ a \in A \mid a R b \right\} \mid b \in G \right\} \right| \leq |A|^k
        $$
        \begin{proof}
            By Lemma~\ref{lemma:vs_dimension_implies_k_order_property}, if $G$ has the non-$k$-order property,
            then the family $\parcurly{B_{A,b} \mid b \in G \setminus A}$ has VC dimension at most $k-1$,
            so by the Sauer-Shelah lemma~\ref{lemma:sauer-shelah} we have
            $\parcurly{B_{A,b} \mid b \in G \setminus A} \leq \sum_{i=0}^{k-1} \binom{|A|}{i}$.
            Since $\parcurly{B_{A,b} \mid b \in A} \leq \parstraight{A}$, we conclude that
            \[
                \parstraight{S} = \parstraight{B_{A,b} \mid b \in G} \leq \sum_{i=0}^{k-1} \binom{|A|}{i} + |A|
            \]
            Finally, when $\parstraight{A} = n,k > 1$: \todo{This conditions should be set at some point of the tfm. Specify that if they are not met, the problem becomes trivial.}
            \begin{itemize}
                \item if $n \leq k$, then $\parstraight{S} \leq 2^n \leq 2^k \leq n^k$.
                \item if $n > k$, then $\parstraight{S} \leq \sum_{i=0}^{k-1} + n \leq n^{k-1} + n \leq 2n^{k-1} \leq n^k$.
            \end{itemize}
            We conclude that $\parstraight{S} \leq n^k$.
        \end{proof}

    We now prove the following equivalent versions of the lemma, which will be useful in the next sections.

    \corollary[Claim 2.6.1]\label{itm:2.6.1}
        Let $G$ be a graph with the non-$k$-order property.
        Then:
        \begin{enumerate}
            \item\label{itm:2.6.1.1} For any finite $A \subseteq G$
                $$
                    \left|\left\{ \left\{ a \in A \mid \neg a R b \right\} \mid b \in G \right\} \right|
                        \leq \sum_{i \leq k} \binom{|A|}{i} \leq |A|^k
                $$
            \item\label{itm:2.6.1.2} For any finite $A \subseteq G$
                $$
                    \left|\left\{ \left\{ a \in A \mid \neg a R b \equiv t(A,b) \right\} \mid b \in G \right\} \right|
                        \leq \sum_{i \leq k} \binom{|A|}{i} \leq |A|^k
                $$
        \end{enumerate}
        \begin{proof}
        \begin{enumerate}
            \item First of all, notice that $B^+_{A,b} = B - B^-{A,b}$, since by definition they are complementary.
                Thus, for any $b, b' \in G$, $B^+_{A,b} = B^+_{A,b'} \Leftrightarrow B^-_{A,b} = B^-_{A,b'}$.
                It follows that
                $$
                    \left| \left\{ B^-_{A,b} \mid b \in G \right\} \right| =
                    \left| \left\{ B^+_{A,b} \mid b \in G \right\} \right| \leq
                    \sum_{i \leq k} \binom{|A|}{i} \leq |A|^k
                $$
                where the last inequalities come from Lemma~\ref{claim_2.6}.
            \item Consider the following map:
                \begin{align*}
                    \pi: \left\{ B_{A,b} \mid b \in G \right\} & \longrightarrow \left\{ B^+_{A,b} \mid b \in G \right\} \\
                                                       B_{A,b} & \longmapsto B^+_{A,b}
                \end{align*}
                We show that the map $\pi$ is injective.
                Let $b,b' \in G$ such that $B_{A,b} = B_{A,b'}$.
                Then, $t(A,b) = t(A,b')$, otherwise (suppose wlog that $t(A,b) = 1$ and $t(A,b') = 0$), we would have
                $$
                    \left| B^-_{A,b'} \right| > \left| B^+_{A,b'} \right| = \left| B^+_{A,b} \right| \geq
                        \left| B^-_{A,b} \right| = \left| B^-_{A,b'} \right|
                $$
                which is a contradiction.
                Then:
                \begin{itemize}
                    \item if $t(A,b) = t(A,b') = 1$, we have that $B_{A,b} = B^+_{A,b} = B^+_{A,b'} = B_{A,b'}$.
                    \item if $t(A,b) = t(A,b') = 0$, we have that
                    $B_{A,b} = B^-_{A,b} = A \setminus B^+_{A,b} = A \setminus B^+_{A,b'} = B^-_{A,b'} = B_{A,b'}$.
                \end{itemize}
                This proves that $\pi$ is injective.
                To conclude,
                $$
                    \left| \left\{ B_{A,b} \mid b \in G \right\} \right| \leq
                    \left| \left\{ B^+_{A,b} \mid b \in G \right\} \right| \leq
                    \sum_{i \leq k} \binom{|A|}{i} \leq |A|^k
                $$
                This concludes the proof.
                Notice that in particular $\pi$ is a bijection, but this is not needed for the proof.
        \end{enumerate}
        \end{proof}

    \todo{Short introduction to the idea of the tree bound and how it is connected with order property.}
    \definition[Definition 2.11]\label{tree_bound}
        Suppose $G$ is a finite graph with the non-$k_*$-property.
        We denote the \emph{tree bound} $k_{**} = k_{**}(G) < \omega$ as the minimal value such that there
        do not exist sequences $\overline{a} = \left< a_\eta \mid \eta \in \left[ 2 \right]^{k_{**}} \right>$ and
        $\overline{b} = \left< b_\rho \mid \rho \in \left[ 2 \right]^{<k_{**}} \right>$ of elements of $G$ satisfying that
        if $\rho \frown \left< \ell \right> \triangleleft \eta$, then $\left( a_\eta R b_\rho \right) \equiv \left( \ell = 1 \right)$.
    \todo{Make a visual example of the tree bound.}
    \todo{Relate tree bound with order property as in Notes for Stable Graphs.}