\section{Conclusion} \label{sec:conclusion}
    In this thesis, we delved into the powerful framework of Szemerédi's Regularity Lemma,
    focusing on a restricted version for the class of stable graphs.
    Our primary contributions have been threefold: we have provided a detailed, self-contained combinatorial proof of the Stable
    Regularity Lemma from~\cite{regularity_lemmas_for_stable_graphs}, clarifying parameters and simplifying arguments;
    we have developed a unified notational framework to bridge concepts from extremal graph theory, stability,
    and property testing; and, most significantly, we have designed an efficient property testing algorithm for
    $H$-freeness specifically for stable graphs.

    Our motivation for this final contribution stemmed from a desire to exploit the most striking feature of the Stable
    Regularity Lemma: the complete absence of irregular pairs in its partitions.
    We hypothesized that this structural guarantee could be a powerful tool in property testing, and we found a
    direct application in testing for forbidden induced subgraphs ($H$-freeness).

    During the design of our tester, we observed that a similar argument could be constructed using the
    (Ultra-) Strong Regularity Lemma for graphs with bounded VC-dimension, as presented
    in~\cite{regularity_partitions_and_the_topology_of_graphons}.
    The stronger regularity conditions of this lemma allows one to effectively \say{avoid} the issue of irregular
    pairs when dealing with induced subgraphs.
    This led to a crucial comparison of the bounds on the number of parts in the partitions.
    The (Ultra-)Strong lemma provides a bound of $(1/\epsilon)^{c \cdot d^2}$, where $d$ is the bound on the VC-dimension,
    while the Stable Regularity Lemma's bound is $(1/\epsilon)^{c \cdot 2^k}$, where $k$ is the stability parameter.
    As established in \Cref{sec:section_3}, stability is a stronger condition than bounded VC-dimension,
    and the parameters $k$ and $d$ are closely related (when comparing the two in this situation $k=d$).
    \todo{Aquest últim parentesis no m'agrada.}

    This comparison suggests that the bound for the Stable Regularity Lemma might not be optimal.
    The fact that a broader class of graphs (bounded VC-dimension) admits a partition with a polynomially better
    exponent raises the possibility that the bound for stable graphs could be improved to something akin to
    $(1/\epsilon)^{c \cdot k^2}$, a question also posed by~\cite{julia_wolf_private_comunication}.
    Despite the potentially suboptimal bound, the Stable Regularity Lemma remains a valuable tool.
    Its guarantee of having no irregular pairs whatsoever allows for a uniquely clean and straightforward
    proof of the testability of $H$-freeness in the context of stable graphs.

    Finally, this work opens several avenues for future research, which could not be included in this thesis due to time
    constraints.
    One promising direction is to leverage the lack of irregular pairs to test for $H_n$-freeness, where the forbidden
    subgraph $H_n$ grows in size with the input graph $G$.
    Another intriguing possibility is to move beyond simple freeness testing towards subgraph counting, using the clean
    structure of the stable regular partition to develop a tester that provides an interval estimate for the number of
    induced copies of a graph $H$.