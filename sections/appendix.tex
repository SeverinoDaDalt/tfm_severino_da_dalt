\appendix

\vfill\newpage \section{Other proofs} \label{sec:appendix_other_proofs}

    For completeness, here we leave secondary proofs we skipped in the thesis.

    \begin{proof}[Proof of \Cref{cor:k_order_propery_bounds_BAbs}]
        \begin{enumerate}
            \item First of all, notice that $B^+_{A,b} = A - B^-_{A,b}$, since by definition they are complementary.
                Thus, for any $b, b' \in G$, $B^+_{A,b} = B^+_{A,b'} \Leftrightarrow B^-_{A,b} = B^-_{A,b'}$.
                It follows that
                \[
                    \parstraight{\parcurly{B^-_{A,b} \mid b \in G}} =
                    \parstraight{\parcurly{B^+_{A,b} \mid b \in G}} \leq |A|^k,
                \]
                where the last inequality follows from \Cref{lem:k_order_property_bounds_BAbs}.
            \item Consider the following map:
                \begin{align*}
                    \pi: \parcurly{B^+_{A,b} \mid b \in G} & \longrightarrow \parcurly{\overline{B}_{A,b} \mid b \in G}. \\
                                                 B^+_{A,b} & \longmapsto \overline{B}_{A,b}
                \end{align*}
                We first prove that the map $\pi$ is well-defined.
                If $B^+_{A,b}$ and $B^+_{A,b'}$ are equal, then they have the same size, and thus the same truth value.
                Then,
                \begin{itemize}
                    \item if $t(A,b) = t(A,b') = 1$, we have that $\overline{B}_{A,b} = B^+_{A,b} = B^+_{A,b'} = \overline{B}_{A,b'}$.
                    \item if $t(A,b) = t(A,b') = 0$, we have that
                    $\overline{B}_{A,b} = B^-_{A,b} = A \setminus B^+_{A,b} = A \setminus B^+_{A,b'} = B^-_{A,b'} = \overline{B}_{A,b'}$.
                \end{itemize}
                which proves that the map is well-defined.
                The map $\pi$ is also surjective, since for each $b \in G$, and thus for each $\overline{B}_{A,b}$,
                the set $B^+_{A,b}$ is mapped to $\overline{B}_{A,b}$ by construction.
                Hence,
                \[
                    \parstraight{\parcurly{\overline{B}_{A,b} \mid b \in G}} \leq
                    \parstraight{\parcurly{B^+_{A,b} \mid b \in G}} \leq
                    \sum_{i \leq k} \binom{|A|}{i} \leq |A|^k.
                \]
                This concludes the proof.
                Notice that, actually, the map $\pi$ is a not necessarily a bijection, since (at most) two $b$'s with
                different truth value with respect to $A$ may induce the same set $\overline{B}_{A,b}$. \qedhere
        \end{enumerate}
    \end{proof}

    \begin{proof}[Proof of \Cref{lem:exceptions_bound_of_f_indivisible_sets}]
        Notice that, by the average condition of the pair $(A,B)$:
        \begin{itemize}
            \item there are at most $f(|A|)$ vertices of $A$ (hence in $A' \subseteq A$), say $S$, which are exceptional
                with respect to $B$, so the number of edges $(a,b) \in S \times B'$ which are exceptional is at most
                $|S| \cdot |B'|$, and
            \item for each $a \in A$ (hence in $A' \subseteq A$) not in $S$, there are at most $g(|B|)$ elements
                $b \in B$ such that $(a,b)$ does not satisfy the truth value $t(A,B)$, i.e. that are exceptional.
                Thus, we have at most $(a,b) \in (A' \setminus S) \times B'$ is at most $(|A'| - |S|) g(|B|)$.
        \end{itemize}
        The overall worse case in this scenario is when $S$ is maximum ($|S| = f(|A|)$), and thus we have at most
        $f(|A|) |B'| + (|A'| - f(|A|)) g(|B|)$ exceptional edges in $A' \times B'$, as $|B'| \geq g(|B|)$.
        Putting it all together:
        \[
            \begin{split}
                \frac{|\parcurly{(a,b) \in (A',B') \mid a R b \equiv \neg t(A,B)}|}{|A' \times B'|}
                    &\leq \frac{f(|A|) |B'| + (|A'| - f(|A|)) g(|B|)}{|A'| |B'|} \\
                    &= \frac{f(|A|)}{|A'|} + \frac{|A'| - f(|A|)}{|A'|} \frac{g(|B|)}{|B'|} \\
                    &\leq \frac{f(|A|)}{|A'|} + \frac{g(|B|)}{|B'|} \\
                    &\leq \frac{f(|A|)}{f(|A|) |A|^{\epsilon_1}} + \frac{g(|B|)}{g(|B|) |B|^{\zeta_1}} \\
                    &= \frac{1}{|A|^{\epsilon_1}} + \frac{1}{|B|^{\zeta_1}}.
            \end{split}
        \]
        This finishes the proof.
    \end{proof}

    \begin{proof}[Proof of \Cref{clm:floor_exponential_composition_bound}]
        We first note that for all natural number $n \geq 1$, and real values $x \geq 1$ and $\epsilon < 1$,
        we have that:
        \begin{equation} \label{eq:exp_bound}
            (x+n)^{\epsilon} \leq x^{\epsilon} + n
        \end{equation}
        and
        \begin{equation} \label{eq:floor_bound}
            \floor{x+n} \leq \floor{x} + n.
        \end{equation}
        We now prove the statement by induction on $k$.
        If $k=2$,
        \[
            f_{\epsilon_1 \epsilon_2} = \floor{x^{\epsilon_1 \epsilon_2}}
                \leq \floor{\floor{x^{\epsilon_1}+1}^{\epsilon_2}}
                \leq \floor{\floor{x^{\epsilon_1}}^{\epsilon_2}} + 1,
        \]
        where the second inequality uses~\eqref{eq:exp_bound} and~\eqref{eq:floor_bound}.
        If $k > 2$,
        \begin{align*}
            f_{\epsilon_1 \epsilon_2 \dots \epsilon_k} (x)
                & \leq f_{\epsilon_1 \epsilon_2} \circ f_{\epsilon_3} \circ \dots \circ f_{\epsilon_{k}}(x) + k - 2 \\
                & = f_{\epsilon_1 \epsilon_2} (f_{\epsilon_3} \circ \dots \circ f_{\epsilon_{k}}(x)) + k - 2 \\
                & \leq f_{\epsilon_1} \circ f_{\epsilon_2} (f_{\epsilon_3} \circ \dots \circ f_{\epsilon_{k}}(x)) + k - 1,
        \end{align*}
        where the first inequality uses I.H. for $k-1$, and the second inequality uses I.H. for $2$.
        This proves the statement.
    \end{proof}

    \begin{proof}[Proof of \Cref{lem:existance_of_excellent_subsets_fixed_size_choices}]
        Suppose the converse.
        We use this fact to build sets $\parcurly{b_\eta \mid \eta \in \parcurly{0,1}^{<k_{**}}}$ and
        $\parcurly{A_\eta \mid \eta \in \parcurly{0,1}^{\leq k_{**}}}$ on induction over $k<k_{**}$, where $k = |\eta|$,
        satisfying:
        \begin{enumerate}
            \item\label{itm:existance_of_excellent_subsets_fixed_size_choices.1} $A_{\Partriangle{\cdot}} \subseteq A$, with $|A|_{\Partriangle{\cdot}} = m_0$.
            \item\label{itm:existance_of_excellent_subsets_fixed_size_choices.2} $B_\eta$ is an $\zeta$-good set witnessing that $A_\eta$ is not
                $\parround{\frac{m_{k+1}}{m_{k}}, \zeta}$-excellent, for all $k < k_{**}$.
            \item\label{itm:existance_of_excellent_subsets_fixed_size_choices.3} $A_{\eta \frown \Partriangle{i}} = \parcurly{a \in A_\eta \mid t(a, B_\eta) \equiv i}$
                for all $i \in \parcurly{0,1}$ and $k < k_{**}$.
            \item\label{itm:existance_of_excellent_subsets_fixed_size_choices.4} $|A_{\eta}| = m_k$, for all $k \leq k_{**}$.
            \item\label{itm:existance_of_excellent_subsets_fixed_size_choices.6} $A_{\eta \frown \Partriangle{0}} \sqcup A_{\eta \frown \Partriangle{1}} \subseteq A_\eta$,
                for all $k < k_{**}$.
            \item\label{itm:existance_of_excellent_subsets_fixed_size_choices.7} $\overline{A_k} = \parcurly{A_\eta \mid \eta \in \parcurly{0,1}^k}$ is a partition of
                a subset of $A$, for all $k \leq k_{**}$.
        \end{enumerate}
        Notice that, by \dref{itm:existance_of_excellent_subsets_fixed_size_choices.1} and
        \dref{itm:existance_of_excellent_subsets_fixed_size_choices.4}, the size of $A_\eta$ is $m_k$,
        so by IH none of the sets $A_\eta$ is $\parround{\frac{m_{k+1}}{m_{k}}, \zeta}$-excellent.
        Then, $B_\eta$ in \dref{itm:existance_of_excellent_subsets_fixed_size_choices.2} is well-defined.
        Also, by $\zeta$-goodness of $B_\eta$, $t(a, B_\eta)$ in \dref{itm:existance_of_excellent_subsets_fixed_size_choices.3} is well-defined.
        Then, since $B_\eta$ is witnessing the non-$\parround{\frac{m_{k+1}}{m_{k}}, \zeta}$-excellence of $A_\eta$,
        we have that $|A_{\eta \frown \Partriangle{i}}| \geq \frac{m_{k+1}}{m_k} m_{k} = m_{k+1}$ for all
        $i \in \parcurly{0,1}$, satisfying \dref{itm:existance_of_excellent_subsets_fixed_size_choices.4}.
        Finally, by definition \dref{itm:existance_of_excellent_subsets_fixed_size_choices.3}, we have the disjoint union
        \dref{itm:existance_of_excellent_subsets_fixed_size_choices.6} which by itself
        ensures \dref{itm:existance_of_excellent_subsets_fixed_size_choices.7}.

        Now, our goal is to build two sequences $\parcurly{b_\eta \mid \eta \in \parcurly{0,1}^{<k_{**}}}$ and
        $\parcurly{a_\eta \mid \eta \in \parcurly{0,1}^{k_{**}}}$ to contradict the tree bound $k_{**}$.
        First of all, notice that, for $\eta \in \parcurly{0,1}^{k_{**}}$
        \[
            |A_\eta| = m_k \geq m_{k_{**}} \geq 1
        \]
        So, for each $\eta \in \parcurly{0,1}^{k_{**}}$, $A_\eta \neq \emptyset$ and we may choose an $a_\eta \in A_\eta$.
        Now, for $\nu \in \parcurly{0,1}^{<k_{**}}$ and $\eta \in \parcurly{0,1}^{k_{**}}$ such that $\nu \triangleleft \eta$, let
        \[
            U_{\nu,\eta} = \parcurly{b \in B_\nu \mid (a_\eta R b) \not\equiv t(a_\eta, B_\nu)}
        \]
        be the subset of elements of $B_\nu$ that do not relate with $a_\eta$ in the expected way.
        By $\zeta$-goodness of $B_\nu$, $|U_{\nu, \eta}| < \zeta |B_\nu|$, and thus for every $\nu \in \parcurly{0,1}^{<k_{**}}$,
        \[
            \parstraight{\bigcup\parcurly{ U_{\nu,\eta} \mid \nu \triangleleft \eta \in \parcurly{0,1}^{k_{**}}}} <
            2^{k_{**}} \zeta |B_\nu| \leq |B_\nu|
        \]
        We may choose $b_\nu \in B_\nu \setminus \bigcup\parcurly{U_{\nu,\eta} \mid \nu \triangleleft \eta \in \parcurly{0,1}^{k_{**}}}$,
        for all $\nu \in \parcurly{0,1}^{<k_{**}}$.
        Finally, the sequences $\Partriangle{a_\eta \mid \eta \in \parcurly{0,1}^{k_{**}}}$ and
        $\Partriangle{b_\nu \mid \nu \in \parcurly{0,1}^{<k_{**}}}$ satisfy that $\forall \eta, \nu$ such that
        $\nu \frown \Partriangle{i} \triangleleft \eta$, $a_\eta R b_\nu \equiv i$, which follows
        \dref{itm:existance_of_excellent_subsets_fixed_size_choices.3}.
        This contradicts \Cref{def:tree_bound} of tree bound $k_{**}$.
    \end{proof}

\vfill\newpage \section{Main changes} \label{sec:main_changes}
    This section of the appendix is dedicated at showing the main changes this thesis applies to the original results
    of~\cite{regularity_lemmas_for_stable_graphs} and~\cite{notes_on_the_stable_regularity_lemma}.

    \begin{itemize}
        % Section 2
        \item Definition 2.3 of the $k$-order property in~\cite{regularity_lemmas_for_stable_graphs} does not
            specify adjacency (or not) of vertices with the same index.
        % Section 4
        \item In order for the arguments of Section 4 to work, most results require that the function
            $f$ (of the $f$-indivisibility) satisfies $x \geq f(x)$, instead of the \emph{non-decreasing} condition
            given in Definition 4.2 in~\cite{regularity_lemmas_for_stable_graphs}, which is redundant.
        \item In order for the average condition to be satisfied, and thus being able to apply Claim 4.8 in the proof
            of Claim 4.10 in~\cite{regularity_lemmas_for_stable_graphs}, something like the extra condition provided
            by \Cref{rmk:sufficient_requirement_for_average_condition} needs to be added to the claim statement.
        \item Second to last inequality in the equation of $P_1$ at page 1569 of~\cite{regularity_lemmas_for_stable_graphs}
            is actually opposite (the $<$ should be a $>$).
            The same occurs, with last inequality of $P_2$ equation at the same page.
            This breaks the proof's argument, requiring extra conditions and some (non-trivial) changes in the argument.
            The most important change in the result is the extra condition $m_0 \geq n^\epsilon$ in
            \Cref{lem:bound_on_the_probability_of_a_subpair_having_no_exceptions}, which strongly reduces the
            interval of possible choices of parts size in the result, and needs to be carried until the end of the subsection.
        \item Condition $m_{**} > k_{**}$, which is persistent in results of Section 4
            in~\cite{regularity_lemmas_for_stable_graphs} can be relaxed into $m_{**} \geq 1$.
        \item Proof of Theorem 4.16 in~\cite{regularity_lemmas_for_stable_graphs} is unclear, even more when previous points
            are noted.
            \Cref{thm:existance_of_equitative_partition_with_perfect_pairs_but_with_bound_exceptional_pairs} provides
            a complete proof of a weaker (but coherent) version of the same result.
        \item Theorem 4.23 proof construction first finds an $\epsilon$-indivisible set, and then applies Claim 4.21
            to find a $c$-indivisible set.
            But Claim 4.21 itself does not require an $\epsilon$-indivisible set as input, as it is constructed in its
            own proof.
            Noticing this allows to fully rewrite the theorem for a stronger (and more interpretable) result
            (\Cref{thm:equitative_partition_high_regularity_parts_grow_with_n}).
        % Section 5
        \item Non-monotonicity\todo{To do...}

    \end{itemize}
    % Section 3
    Also, we note that\dots\todo{Section 3 argument does not work because...}

\vfill\newpage \section{Excellence is not monotonic.} \label{sec:excellence_is_not_monotonic}
    Here give more details on the counterexample to the monotonicity of the excellence property given in
    \Cref{fig:non-monotonic_example}.
    We see that this example is in fact the smallest bipartite graph of a family of counterexamples.
    Each element of the family can be described by the following adjacency matrix, defined by blocks:
    \[
        G_r = \left[
            \begin{array}{c|c}
                0 & H_r \\
                \hline
                H_r^T & 0 \\
            \end{array}
        \right]
        \text{, with }\quad
        H_r = \left[
            \begin{array}{c|c}
                0 & \mathbbm{1}_r - \mathbb{I}_r \\
                \hline
                \mathbb{J}_r & \mathbb{I}_r \\
            \end{array}
        \right]
    \]
    where $\mathds{1}_r$ is the $r \times r$ matrix of all $1$'s,
    $\mathbb{I}_r$ is the $r \times r$ diagonal matrix,
    and $\mathbb{J}_r$ is the $r \times r$ anti-diagonal matrix.
    Also, we use $H^T$ to refer to the transpose of $H$.

    By calling $A$ the set of the first (as indices) $2r$ vertices of $G_r$, and $B$ the last $2r$,
    we have the desired counterexample:
    $A \subseteq G$ is $\frac{1}{2r-1}$-excellent, but $B$ witnesses that $A$ is not $\frac{1}{2r-1}$-excellent.
    The example in \Cref{fig:non-monotonic_example} shows $G_r$ for $r=3$.

    A sufficient proof of this is a simple exhaustive check, and code for this precise purpose is provided with all
    the material of this thesis in a GitHub repository\footnote{See \url{https://github.com/SeverinoDaDalt/tfm_severino_da_dalt/}}.
    There are two main relevant scripts in the repository.
    One allows to check whether $G_r$ for a given value of $r$ is in fact $\frac{1}{2r-1}$-excellent and not
    $\frac{1}{2r-1}$-excellent.
    The other allows for an exhaustive search of \emph{possible} counterexamples under some given parameters, which is how this
    counterexamples were found in the first place.
    Read the documentation for more information on how to run the code.