\section{Introduction} \label{sec:introduction}

    Szemer\'edi's Regularity Lemma~\cite{regular_partitions_of_graphs} is a powerful tool in graph theory,
    stating that any sufficiently large graph can have its vertex set decomposed into an equitable partition such that most,
    but not all, pairs of parts are \emph{regular}.
    A regular pair is one whose edge distribution resembles that of a random bipartite graph, a powerful property with many
    applications in extremal graph theory.
    On top of the presence of a small number of irregular pairs, a major drawback of the lemma is the immense
    bound on the required number of parts, which grows as a tower of exponentials whose height depends on the regularity parameter.

    In the general setting, both limitations have been proven to be unavoidable.
    In~\cite{lower_bounds_of_tower_type_for_szeremedis_uniformity_lemma} the author shows that there exist a family of graphs
    for which the lower bound on the number of parts is still a tower of exponentials\footnote{
    To be more specific, the author shows that the number of parts is lower bounded by an exponenetial tower of $2$'s where
        the height of the tower is at least proportional to $\log\parround{1/\epsilon}$.
        Meanwhile, in the usual proof of the theorem, the upper bound on the height of the tower is proportional to
        $\epsilon^{-5}$.
    }.
    On the other hand, it is folklore knowledge that large-enough half-graphs present irregular pairs in any
    regular partition (\cite{irregular_pairs_in_half_graphs_szemeredi_regularity} gives a written proof of this fact).
    Having seen this unavoidability, it is natural to ask for the underlying reasons of those limitations and which
    additional conditions can be imposed or levied so that the parameters can be improved.

    In this context, one of the first attempts was to make the regularity condition weaker so that the bound on the
    number of parts can be improved: this is now known as
    \emph{the weak regularity lemma}~\cite{quick_approximation_to_matrices_and_applications} and,
    besides its own applications, allows to put the SzRL in the context of a spectra of regularity lemmas, with
    different strengths (the stronger the conclusion the larger the number of
    parts)~\cite{szemeredis_lemma_for_the_analyst, regularity_partitions_and_the_topology_of_graphons};
    a notable example on the other direction (making the regularity lemma stronger) allows its use for Property Testing
    as it is suitable for the study of induced subgraphs~\cite{efficient_testing_of_large_graphs}.

    Another option is to restrict the class of graphs where we want to find a regular partition.
    An example of this effort is the class of graphs with bounded VC-dimension: this concept was introduced
    in~\cite{the_uniform_convergence_of_frequencies_of_the_appearance_of_events_to_their_probabilities}\footnote{
        See~\cite{on_the_uniform_convergence_of_relative_frequencies_of_events_to_their_probabilities}
        for a translated version.}
    and one can view it as a graph with \say{low complexity} (but not necessarily sparse), and the reader can find more
    details in \Cref{sec:section_3}.
    For these graphs the number of parts is highly reduced from a tower type to a polynomial in $1/\epsilon$, whose
    power depends on the bound on the dimension of the graph~\cite{regularity_partitions_and_the_topology_of_graphons,
        erdos_hajnal_conjecture_for_graphs_with_bounded_vc_dimension,
        efficient_testing_of_bipartite_graphs_for_forbidden_induced_subgraphs}.
    Even more, when the graphs have bounded VC-dimension, the density of edge in the regular pairs are either close to $1$,
    or close to $0$.
    However, the issue on the presence of irregular pairs remains, as any half-graph has bounded VC-dimension\footnote{
        Indeed, the fact that the neighbourhoods of the vertices on the same stable set can be ordered by inclusion,
        prevents the VC-dimension to grow beyond $2$.
        Alternatively, in \cite{regularity_partitions_and_the_topology_of_graphons} it is shown that any graph with
        VC-dimension bounded by $k$ can be seen as not containing a bi-induced copy of a bipartite graph where the
        smaller size is $k$; in our case the half-graph has no bi-induced copy of $K_{3,3}$ minus a perfect matching.}.

    In this work we focus our attention on the result by Malliaris and Shelah~\cite{regularity_lemmas_for_stable_graphs,
        notes_on_the_stable_regularity_lemma}
    which states that, if one cannot find a bi-induced copy of a half-graph, then a regular partition can be found, with
    not many pairs ($1/\epsilon$ to the power of an exponential on the size of the half-graph), and where no irregular
    pairs are found.

    The graphs where no large half-graph can be found are called stable graphs, and their study stems from results in
    Model Theory and Logic.
    We shall stress that these stable graphs have, in fact, bounded VC-dimension (since we are forbidding a bi-induced
    copy of a fixed bipartite graph).

    One of the many applications of the regularity lemma for which these bounds on the number of parts become relevant is in
    \emph{property testing}.
    A property testing algorithm for a decision problem $P$ is a randomized algorithm that, by querying only a small portion
    of its input, can distinguish with high probability between objects that satisfy $P$ and those that are \say{far} from
    satisfying it.
    For instance, in~\cite{efficient_testing_of_large_graphs} the authors use Szemer\'edi's Regularity Lemma
    to prove that it is possible to test the property of a graph $G$ being $H$-free (for a fixed graph $H$) using an algorithm
    which query complexity is independent on the size of the input graph $G$.

    The query complexity of such testers, however, is intrinsically linked to the number of parts in the underlying regular
    partition.
    Consequently, the power-tower bounds of the standard regularity lemma lead to prohibitively large, although constant,
    query counts.
    This raises a natural question: can the superior bounds of the stable regularity lemma be exploited to create more
    efficient property testers for graphs in a half-graph-restricted setting?

    In this thesis, we present an algorithm for testing $H$-freeness in stable graphs, thereby providing
    a concrete application that highlights the practical strength and utility of stable regularity partitions.

    \subsection{Main Contributions} \label{subsec:main_contributions}

        The main contributions of this thesis are:
        \begin{itemize}
            \item We place a larger emphasis on the combinatorial part of the result
                in~\cite{regularity_lemmas_for_stable_graphs}, making it self-contained and making some of the argument that
                previously used some Model Theory fully combinatorial.
                Further, we make some of the relations between the parameters explicit while correcting some of the typos
                that inevitably occur.
                In addition, we simplify some of the arguments, while making others more explicit and detailed.
                In particular, we make explicit that the excellence (see \Cref{sec:section_5}) depends on two parameters
                with opposite monotonic properties (see \Cref{def:epsilon_excellent} and \Cref{rmk:excellence_is_not_monotonic}).
            \item The construction of an efficient property testing algorithm for H-freeness tailored to stable graphs.
                The algorithm's analysis leverages the stable regularity lemma to achieve a query complexity with significantly
                improved bounds compared to the general case.
            \item The development of a unified notational framework that cohesively integrates the concepts from
                extremal graph theory, stability, and property testing used throughout the thesis.
        \end{itemize}

    \subsection{Summary} \label{subsec:summary}

        The remainder of this thesis is organized as follows.
        \Cref{sec:section_2} reviews fundamental concepts from graph theory, culminating in a formal statement of Szemer\'edi's
        Regularity Lemma.
        \Cref{sec:section_3} introduces the graph-theoretic notion of stability and proves some basic results in this context.
        \Cref{sec:section_4} presents and analyzes some weaker variants of the stable regularity lemma, and illustrate both its
        strengths and its inherent limitations.
        \Cref{sec:section_5} is dedicated to the proof of the main Stable Regularity Lemma, which forms the technical core of this
        work.
        Finally, \Cref{sec:section_6} applies this previous results to prove our property testing algorithm for
        $H$-freeness in stable graphs works, providing explicit bounds on its query complexity.
