\section{Introduction} \label{sec:introduction}

    \subsection{Szemer\'edi's Regularity Lemma}

    Szemer\'edi's Regularity Lemma~\cite{regular_partitions_of_graphs} is a powerful tool in graph theory,
    stating that every graph can have its vertex set decomposed into an equitable partition such that most,
    but not all, pairs of parts are \emph{regular}.
    A regular pair is one whose edge distribution resembles that of a random bipartite graph, a powerful property with
    many applications in extremal graph theory.

    % \subsection{Applications}
    This theorem has seen applications in many areas of mathematics
    (see an early survey from the mid 1990's~\cite{survey of regularity lemma}), such as extremal graph
    theory (such as~\cite{kuhn ostus, ...}), graph limits, number theory (Szemer\'edi's Theorem), Ramsey Theory,
    in combinatorics, and in areas of computer science, such as property testing.

    % \subsection{Problems with SzRL}
    Since this result is applicable to \emph{any} graph and the number of parts of this partition with good properties is constant,
    it should not be surprising that some limitations arise.
    Firstly, not necessarily all pairs are regular, but most crucially, the required upper bound on the number of parts is
    constant, but very large.
    More specifically, it is a tower of exponentials (it has the form $2^{2^{2^{\text{\reflectbox{$\dots$}}}}}$)
    whose height depends on the regularity parameter.

    In the general setting, both limitations have been proven to be unavoidable.
    In~\cite{lower_bounds_of_tower_type_for_szeremedis_uniformity_lemma}, Gowers shows that there exists a family of graphs
    for which the lower bound on the number of parts is still a tower of exponentials\footnote{
    To be more specific, the author shows that the number of parts is lower bounded by an exponenetial tower of $2$'s where
        the height of the tower is at least proportional to $\log\parround{1/\epsilon}$.
        Meanwhile, in the usual proof of the theorem, the upper bound on the height of the tower is proportional to
        $\epsilon^{-5}$.
    }.
    On the other hand, it is folklore knowledge that large-enough half-graphs present irregular pairs in any
    regular partition (\cite{irregular_pairs_in_half_graphs_szemeredi_regularity} gives a written proof of this fact).
    Having seen this unavoidability, it is natural to ask for the underlying reasons of those limitations and which
    additional conditions can be imposed or levied so that the parameters can be improved.

    \subsection{Versions of SzRL}
    % \subsection{Weak and strong versions}
    In order to reduce the bound on the number of parts, one can relax the property required on the partition.
    One of the first successful implementations of such approach was given
    by~\cite[Theorem 12]{quick_approximation_to_matrices_and_applications} which is now known as
    \emph{the weak regularity lemma}.
    Towards the other direction one can strengthen the property on the
    partition~\cite[Lemma 4.1]{efficient_testing_of_large_graphs}, at the cost of increasing the number of parts.
    The previous results can be thought as three instances of a family of regularity lemmas, with varying
    strength~\cite{regularity_partitions_and_the_topology_of_graphons, szemeredis_lemma_for_the_analyst}.
    This is hinted in~\cite[Lemma 4.1 and its discussion]{szemeredis_lemma_for_the_analyst}, and for example, an even
    stronger instance in this family is given explicitly in~\cite[Section 5.1 - pg. 439]{regularity_partitions_and_the_topology_of_graphons},
    where is referred to as an \emph{ultra-strong} regularity lemma.

    % \subsection{Subclasses versions}
    Now, another way of tackling the limitations of the SzRL is to reduce the scope to an appropriate subclass of graphs.
    A relevant example of this approach is the class of graphs with bounded VC-dimension;
    the notion of VC-dimension was firstly introduced by Vapnik \& Chervonenkis
    in~\cite{the_uniform_convergence_of_frequencies_of_the_appearance_of_events_to_their_probabilities}
    \hspace{-3pt}\footnote{
        See~\cite{on_the_uniform_convergence_of_relative_frequencies_of_events_to_their_probabilities}
        for a translated version.}
    and one can view it as a graph with \say{low complexity} (but not necessarily sparse), the reader can find more
    details in \Cref{sec:section_3}.
    For this class of graphs the bound on the number of parts can be greatly reduced.
    Indeed, if a given graph has VC dimension bounded by $d$, we can obtain a regular partition with only
    $\parround{{1}/{\epsilon}}^{f(d)}$ parts, where $\epsilon$ is the regularity parameter.
    Furthermore, the low complexity of graphs with bounded VC-dimension, translates into the additional property that the
    regular pairs are either almost fully connected or almost empty.

    Another class of graphs that has been considered to alleviate the limitations of SzRL is that of the \emph{stable}
    graphs.
    The concept of stability originates in Model Theory
    (see~\cite{classification_theory_and_the_number_of_non_isomorphic_models}).
    A graph is $k$-stable if it avoids any \emph{bi-induced} (See \Cref{def:bi_induced}) copy of the \emph{half-graph}
    on $2k$ vertices, which is a bipartite graph that behaves in a very \emph{non-quasi-random} way (See~\Cref{fig:half-graph}).
    In fact,~\cite[Theorem 5.19]{regularity_lemmas_for_stable_graphs} shows that restricting to this class of graphs not only achieves
    a partition with a bound on the number of parts which is only exponential on $1/\epsilon$, but also completely avoids
    irregular pairs; the exponent depends on the size of the avoided half-graph.
    Again, (all) pairs are either almost fully connected or almost empty.
    This result is the pivotal point of this work, and \Cref{sec:section_5} is devoted to its proof
    (\Cref{thm:existance_of_regular_partitions}).

    Note that the stable graphs is a subclass of graphs with bounded VC-dimension.
    Indeed, if a graph does not contain a bi-induced copy of a bipartite graph with stable sets of size $\geq k$,
    then it has VC-dimension strictly bounded by $k$~\cite{regularity_partitions_and_the_topology_of_graphons}.
    Hence, $k$-stable graphs have VC-dimension strictly bounded by $k$.
    Additionally, any half-graph has VC-dimension $1$\footnote{
        Indeed, the fact that the neighbourhoods of the vertices on the same stable set of a half-graph can be
        ordered by inclusion, and it is a bipartite graph, results in a VC-dimension of $1$.
        Alternatively, in \cite{regularity_partitions_and_the_topology_of_graphons} it is shown that if a graph does not
        contain a bi-induced copy of a bipartite graph where the smaller size is k, then it has VC-dimension (strictly)
        bounded by k; in our case the half-graph has no bi-induced copy of $K_{3,3}$ minus a perfect matching.},
    so the $k$-stable graphs is a proper subclass of the graphs with VC-dimension strictly bounded by $k$.

    Bipartite bounded VC~\cite{efficient_testing_of_bipartite_graphs_for_forbidden_induced_subgraphs}.
    First general bounded VC~\cite{regularity_partitions_and_the_topology_of_graphons}.
    Best general bounded VC~\cite{erdos_hajnal_conjecture_for_graphs_with_bounded_vc_dimension}.
    Stable~\cite{regularity_lemmas_for_stable_graphs}.

    % up to here it was compiling with no problem.




    \subsection{Property Testing}



    Another option is to restrict the class of graphs where we want to find a regular partition.
    An example of this effort is the class of graphs with bounded VC-dimension: this concept was introduced
    in~\cite{the_uniform_convergence_of_frequencies_of_the_appearance_of_events_to_their_probabilities}\footnote{
        See~\cite{on_the_uniform_convergence_of_relative_frequencies_of_events_to_their_probabilities}
        for a translated version.}
    and one can view it as a graph with \say{low complexity} (but not necessarily sparse).
    The reader can find more details in \Cref{sec:section_3}.
    For these graphs the number of parts is highly reduced from a tower type to a polynomial in $1/\epsilon$, whose
    power depends on the bound on the dimension of the graph~\cite{regularity_partitions_and_the_topology_of_graphons,
        erdos_hajnal_conjecture_for_graphs_with_bounded_vc_dimension,
        efficient_testing_of_bipartite_graphs_for_forbidden_induced_subgraphs}.
    Even more, when the graphs have bounded VC-dimension, the density of edge in the regular pairs are either close to $1$,
    or close to $0$.
    However, the issue on the presence of irregular pairs remains, as any half-graph has bounded VC-dimension\footnote{
        Indeed, the fact that the neighbourhoods of the vertices on the same stable set can be ordered by inclusion,
        prevents the VC-dimension to grow beyond $2$.
        Alternatively, in \cite{regularity_partitions_and_the_topology_of_graphons} it is shown that if a graph does not
        contain a bi-induced copy of a bipartite graph where the smaller size is k, then it has VC-dimension (strictly)
        bounded by k; in our case the half-graph has no bi-induced copy of $K_{3,3}$ minus a perfect matching.}.

    In this work we focus our attention on the result by Malliaris and Shelah~\cite{regularity_lemmas_for_stable_graphs,
        notes_on_the_stable_regularity_lemma}
    which states that, if one cannot find a bi-induced copy of a half-graph, then a regular partition can be found, with
    not many pairs ($1/\epsilon$ to the power of an exponential on the size of the half-graph), and where no irregular
    pairs are found.

    The graphs where no large half-graph can be found are called stable graphs, and their study stems from results in
    Model Theory and Logic.
    We shall stress that these stable graphs have, in fact, bounded VC-dimension (since we are forbidding a bi-induced
    copy of a fixed bipartite graph).

    One of the many applications of the regularity lemma for which these bounds on the number of parts become relevant is in
    \emph{property testing}.
    A property testing algorithm for a decision problem $P$ is a randomized algorithm that, by querying only a small portion
    of its input, can distinguish with high probability between objects that satisfy $P$ and those that are \say{far} from
    satisfying it.
    For instance, in~\cite{efficient_testing_of_large_graphs} the authors use Szemer\'edi's Regularity Lemma
    to prove that it is possible to test the property of a graph $G$ being $H$-free (for a fixed graph $H$) using an algorithm
    which query complexity is independent on the size of the input graph $G$.

    The query complexity of such testers, however, is intrinsically linked to the number of parts in the underlying regular
    partition.
    Consequently, the power-tower bounds of the standard regularity lemma lead to prohibitively large, although constant,
    query counts.
    This raises a natural question: can the superior bounds of the stable regularity lemma be exploited to create more
    efficient property testers for graphs in a half-graph-restricted setting?

    In this thesis, we present an algorithm for testing $H$-freeness in stable graphs, thereby providing
    a concrete application that highlights the practical strength and utility of stable regularity partitions.

    \subsection{Main Contributions} \label{subsec:main_contributions}

        The main contributions of this thesis are:
        \begin{itemize}
            \item We place a larger emphasis on the combinatorial part of the result
                in~\cite{regularity_lemmas_for_stable_graphs}, making it self-contained and making some of the argument that
                previously used some Model Theory fully combinatorial.
                Further, we make some of the relations between the parameters explicit while correcting some of the typos
                that inevitably occur.
                In addition, we simplify some of the arguments, while making others more explicit and detailed.
                In particular, we make explicit that the excellence (see \Cref{sec:section_5}) depends on two parameters
                with opposite monotonic properties (see \Cref{def:epsilon_excellent} and \Cref{rmk:excellence_is_not_monotonic}).
            \item The construction of an efficient property testing algorithm for H-freeness tailored to stable graphs.
                The algorithm's analysis leverages the stable regularity lemma to achieve a query complexity with significantly
                improved bounds compared to the general case.
            \item The development of a unified notational framework that cohesively integrates the concepts from
                extremal graph theory, stability, and property testing used throughout the thesis.
        \end{itemize}

    \subsection{Summary} \label{subsec:summary}

        The remainder of this thesis is organized as follows.
        \Cref{sec:section_2} reviews fundamental concepts from graph theory, culminating in a formal statement of Szemer\'edi's
        Regularity Lemma.
        \Cref{sec:section_3} introduces the graph-theoretic notion of stability and proves some basic results in this context.
        \Cref{sec:section_4} presents and analyzes some weaker variants of the stable regularity lemma, and illustrate both its
        strengths and its inherent limitations.
        \Cref{sec:section_5} is dedicated to the proof of the main Stable Regularity Lemma, which forms the technical core of this
        work.
        Finally, \Cref{sec:section_6} applies this previous results to prove our property testing algorithm for
        $H$-freeness in stable graphs works, providing explicit bounds on its query complexity.
