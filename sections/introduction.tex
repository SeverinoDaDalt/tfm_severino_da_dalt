\section{Introduction} \label{sec:introduction}

\todo{Things to talk about:
- Szemer\'edi's regularity lemma. \\
- Half-graphs and stable regularity lemma. \\
- Property testing. \\
- Stable regularity lemma for testing whether a graph has the property of not containing a fixed graph as a subgraph.
    (Specify this is a $\forall$ P first order property)
}

Szemer\'edi's regularity lemma is a powerful tool in graph theory, stating that any sufficiently large graph can be
decomposed into an equitable partition of its vertices such that most pairs of parts are \emph{regular}.
A regular pair is one whose edge distribution resembles that of a random bipartite graph, a powerful property with many
applications in extremal graph theory.
The primary drawback of the lemma, however, is the immense bound on the required number of parts, which grows as a tower
of exponentials whose height depends on the regularity parameter.

The source of this combinatorial complexity can be traced to the presence of specific induced subgraphs.
As demonstrated by Malliaris and Shelah in their seminal work~\cite{regularity_lemmas_for_stable_graphs}, a key structure
responsible for irregularity is the \emph{half-graph}.\todo{Lluis: is there a better prior cite for this?}
For graphs that exclude large half-graphs, a class known as \emph{stable graphs}, they proved that a much stronger form
of regularity is achievable.
Their \emph{stable regularity lemma} not only yield vastly improved bounds on the partition size but, remarkably,
can guarantee a decomposition entirely free of irregular pairs.

Regularity lemmas are particularly useful in the field of \emph{property testing}.
A property testing algorithm for a decision problem $P$ is a randomized algorithm that, by querying only a small portion
of its input, can distinguish with high probability between objects that satisfy $P$ and those that are \say{far} from
satisfying it.
For instance, in~\cite{efficient_testing_of_large_graphs} the authors use Szemer\'edi's regularity lemma
to prove that it is possible to test the property of a graph $G$ being $H$-free (for a fixed graph $H$) using an algorithm
which query complexity is independent on the size of the input graph $G$.

The query complexity of such testers, however, is intrinsically linked to the number of parts in the underlying regular
partition.
Consequently, the power-tower bounds of the standard regularity lemma lead to prohibitively large, although constant,
query counts.
This raises a natural question: can the superior bounds of the stable regularity lemma be exploited to create more
efficient property testers for graphs in a half-graph-restricted setting?

In this thesis, we present an algorithm for testing $H$-freeness in stable graphs, thereby providing
a concrete application that highlights the practical strength and utility of stable regularity partitions.

The main contributions of this thesis are:
\begin{itemize}
    \item \textbf{A rigorous reformulation and correction of the central proofs}
        in~\cite{regularity_lemmas_for_stable_graphs}.
        Our contribution provides a self-contained, combinatorial framework for these results, systematically
        resolving foundational gaps and inaccuracies in the original arguments to ensure their validity.
        This reworking also makes the associated combinatorial bounds fully explicit for the first time.
    \item \textbf{The construction of an efficient property testing algorithm} for H-freeness tailored to stable graphs.
        The algorithm's analysis leverages the stable regularity lemma to achieve a query complexity with significantly
        improved bounds compared to the general case.
    \item \textbf{The development of a unified notational framework} that cohesively integrates the concepts from
        extremal graph theory, stability, and property testing used throughout the thesis.
\end{itemize}

The remainder of this thesis is organized as follows.
\Cref{sec:section_2} reviews fundamental concepts from graph theory, culminating in a formal statement of Szemerédi's
Regularity Lemma.
\Cref{sec:section_3} introduces the graph-theoretic notion of stability and proves some basic results in this context.
\Cref{sec:section_4} presents and analyzes a weaker variant of the stable regularity lemma, and illustrate both its
strengths and its inherent limitations.
\Cref{sec:section_5} dedicated to the proof of the main Stable Regularity Lemma, which forms the technical core of this
work.
Finally, \Cref{sec:section_6} applies this previous results to prove our property testing algorithm for
$H$-freeness in stable graphs works, providing explicit bounds on its query complexity.
