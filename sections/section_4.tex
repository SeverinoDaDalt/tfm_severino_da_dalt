\section{Section 4} \label{sec:section_4}

    This section works around the concept of \emph{$\epsilon$-indivisible} sets, a strong condition on the \regularity~of a subset
    respect to all the vertices of the graph.
    This condition results in pairs of sufficiently large subsets of vertices satisfying
    the \emph{average condition}, which (asymmetrically) strictly bounds the number of exceptional edges in the pair.
    Using these tools we obtain the first result in \Cref{lem:existance_of_ordered_f_indivisible_partitions_with_exceptions_bound},
    which proves the existence of a partition of highly \regular~pairs with no exceptions, at the cost of a
    non-homogeneous partition.
    Next, we improve the results obtaining an equitable partition in
    \Cref{thm:existance_of_equitative_partition_with_perfect_pairs_but_with_bound_exceptional_pairs}, but this time with a
    small number of exceptional pairs, and a tradeoff between a non-negligible remainder set and even smaller parts.
    The final result, \Cref{thm:equitative_partition_high_regularity_parts_grow_with_n}, achieves removing \irregular~pairs
    and reduce the size of the remainder set.
    All in all, even though the partitions of this section present a very strong form of \regularity, they all share
    the same drawback: a large number of parts that grows with the size of the graph, something that we will be dealing
    with in the next section.

    First step is defining \emph{indivisibility}.
    The general definition is for any function $f$, but for the rest of the section we are mostly interested in the case
    of $f(n) = n^\epsilon$, which we call $\epsilon$-indivisible, and at the end in the constant case $f(n) = c$.

    \begin{definition}[Definition 4.2(b)] \label{def:f_indivisible}
        Let $f: \mathbb{R} \longrightarrow \mathbb{R}$ be a function.
        We say that $A \subseteq G$ is \emph{$f$-indivisible} if for every $b \in G$,
        \[
            \parstraight{\overline{B}_{A,b}} < f(|A|)
        \]
    \end{definition}

    \begin{definition}[Definition 4.2(a)] \label{def:epsilon_indivisible}
        Let $\epsilon \in (0,1)$.
        We say that $A \subseteq G$ is \emph{$\epsilon$-indivisible} if for every $b \in G$,
        \[
            \parstraight{\overline{B}_{A,b}} < |A|^{\epsilon}
        \]
    \end{definition}

    \begin{remark}
        An $\epsilon$-indivisible set is $f$-indivisible for $f(n) = n^\epsilon$.
    \end{remark} \todo{Redundant.}

    A natural follow-up question, is how strongly bounded are exceptions in the context of two indivisible sets.
    The following lemma measures precisely that, although doing so in asymmetrically.

    \begin{lemma}[Claim 4.6)] \label{lem:average_condition_statement}
        Let $G$ be a finite graph.
        Suppose $A, B \subseteq G$ such that $A$ is $f$-indivisible, $B$ is $g$-indivisible, and $f(|A|) g(|B|) < \frac{1}{2} |B|$.
        Then, the truth value $t = t(A,B)$ satisfies that for all but $< f(|A|)$ of the $a \in A$ for all but $< g(|B|)$ of
        the $b \in B$ we have that $a R b \equiv t$.
        \begin{proof}
            Since $B$ is $g$-indivisible, for each $a \in A$ we have that $\parstraight{\overline{B}_{B,a}} < g(|B|)$.
            Let $U_i = \parcurly{a \in A \mid t(a,B) \equiv i}$ for $i \in \parcurly{0,1}$.
            If either $U_i$ satisfies $|U_i| < f(|A|)$ then the statement is true.
            Suppose not.
            Then, there are $W_i \subseteq U_i$ with $|W_i| = f(|A|)$ for $i \in \parcurly{0,1}$.
            Now, let $V = \parcurly{b \in B \mid (\exists a \in W_0 \mid a R b) \vee (\exists a \in W_1 \mid \lnot a R b)}$,
            i.e. the $b$'s which are an exception for some $a \in W_0 \cup W_1$.
            Then, $|V| < |W_0| g(|B|) + |W_1| g(|B|) = 2 f(|A|) g(|B|) < |B|$, where the first inequality follows the
            $g$-indivisibility of $B$.
            Finally, there is a $b_* \in B \setminus V$ such that $\forall a \in W_0$ $\lnot a R b_*$ and
            $\forall a \in W_1$ $a R b_*$ with $|W_0| = |W_1| = f(|A|)$, which contradicts the $f$-indivisibility of $A$.
        \end{proof}
    \end{lemma}

    \begin{definition}
        We say that the pair $(A,B)$ with $A$ $f$-indivisible and $B$ $g$-indivisible satisfies the \emph{average condition} if
        $f(|A|) g(|B|) < \frac{1}{2} |B|$ and thus the statement of \Cref{lem:average_condition_statement} is true for
        the pair $(A,B)$.
    \end{definition}

    \begin{remark}
        The condition $f(|A|) g(|B|) < \frac{1}{2} |B|$ makes ordering of the pair $(A,B)$ matter, that is,
        \[
            (A,B) \text{ has the average condition } \not\Rightarrow (B,A) \text{ has the average condition }
        \]
    \end{remark}

    \begin{remark}[Remark 4.7]
        When $f(n) = n^\epsilon$ and $g(n) = n^\zeta$, the average condition is $|A|^\epsilon |B|^\zeta < \frac{1}{2} |B|$.
    \end{remark}
    \todo{I don't think this is useful in any way.}

    Next, we are interested in studying how the average condition of an indivisible pair controls the homogeneity
    \todo{Define homogeneity.}
    of large enough subpairs, in the sense of bounding exceptional edges.
    We study the $f$ and $\epsilon$ case separately, as the specific case of $\epsilon$ gives a slightly better condition
    on the range of the size of the subpair.

    \begin{lemma}[Claim 4.8] \label{lem:exceptions_bound_of_epsilon_indivisible_sets}
        Let $A$ be $\epsilon$-indivisible, $B$ $\zeta$-indivisible and let the pair $(A,B)$ satisfy the average condition.
        Then, for all $\epsilon_1 \in \parround{0, 1-\epsilon}$, $\zeta_1 \in \parround{0, 1-\zeta}$, $A' \subseteq A$
            and $B' \subseteq B$ such that $|A'| \geq |A|^{\epsilon + \epsilon_1}$ and $|B'| \geq |B|^{\zeta + \zeta_1}$,
            we have that:
        \[
            \frac{|\parcurly{(a,b) \in (A',B') \mid a R b \equiv \neg t(A,B)}|}{|A' \times B'|} \leq
                \frac{1}{|A|^{\epsilon_1}} + \frac{1}{|B|^{\zeta_1}}
        \]
        \begin{proof}
            Notice:
            \begin{itemize}
                \item There are at most $|A|^\epsilon$ vertices of $A$ (hence in $A' \subseteq A$) which are exceptional
                    (in the sense of the average condition).
                \todo{This is not the same use of the word exceptional as defined in \Cref{sec:section_3}.}
                \item For each $a \in A$ (hence in $A' \subseteq A$) not exceptional, there are at most $|B|^\zeta$ elements
                    $b \in B$ such that $(a,b)$ does not satisfy the truth value $t(A,B)$, i.e. that are exceptional.
            \end{itemize}
            Putting it all together:
            \todo{This makes my eyes bleed.}
            \[
                \begin{split}
                    \frac{|\parcurly{(a,b) \in (A',B') \mid a R b \equiv \neg t(A,B)}|}{|A' \times B'|}
                        &\leq \frac{|A|^\epsilon |B'| + (|A'| - |A|^\epsilon) |B|^\zeta}{|A'| |B'|} \\
                        &= \frac{|A|^\epsilon}{|A'|} + \frac{|A'| - |A|^\epsilon}{|A'|} \frac{|B|^\zeta}{|B'|} \\
                        &\leq \frac{|A|^\epsilon}{|A'|} + \frac{|B|^\zeta}{|B'|} \\
                        &\leq \frac{|A|^\epsilon}{|A|^{\epsilon + \epsilon_1}} + \frac{|B|^\zeta}{|B|^{\zeta + \zeta_1}} \\
                        &= \frac{1}{|A|^{\epsilon_1}} + \frac{1}{|B|^{\zeta_1}}
                \end{split}
            \]
        \end{proof}
    \end{lemma}

    \todo{This next generalization may seem to make previous lemma redundant. But they actually prove different results.
        But is it worth to keep both?}

    \begin{lemma}[$f$-indivisible version] \label{lem:exceptions_bound_of_f_indivisible_sets}
        Let $A$ be $f$-indivisible, $B$ $g$-indivisible and let the pair $(A,B)$ satisfy the average condition.
        Then, for all $\epsilon_1 \in \parround{0, 1-\frac{f(|A|)}{|A|}}$, $\zeta_1 \in \parround{0, 1-\frac{g(|B|)}{|B|}}$, $A' \subseteq A$
            and $B' \subseteq B$ such that $|A'| \geq f(|A|) |A|^{\epsilon_1}$ and $|B'| \geq g(|B|) |B|^{\zeta_1}$,
            we have that:
        \[
            \frac{|\parcurly{(a,b) \in (A',B') \mid a R b \equiv \neg t(A,B)}|}{|A' \times B'|} \leq
                \frac{1}{|A|^{\epsilon_1}} + \frac{1}{|B|^{\zeta_1}}
        \]
        \begin{proof}
            Notice:
            \begin{itemize}
                \item There are at most $f(|A|)$ elements of $A$ (hence in $A' \subseteq A$) which are exceptional
                    (in the sense of the average condition).
                \item For each $a \in A$ (hence in $A' \subseteq A$) not exceptional, there are at most $g(|B|)$ elements
                    $b \in B$ such that $(a,b)$ does not satisfy the truth value $t(A,B)$, i.e. that are exceptional.
            \end{itemize}
            Putting it all together:
            \[
                \begin{split}
                    \frac{|\parcurly{(a,b) \in (A',B') \mid a R b \equiv \neg t(A,B)}|}{|A' \times B'|}
                        &\leq \frac{f(|A|) |B'| + (|A'| - f(|A|)) g(|B|)}{|A'| |B'|} \\
                        &= \frac{f(|A|)}{|A'|} + \frac{|A'| - f(|A|)}{|A'|} \frac{g(|B|)}{|B'|} \\
                        &\leq \frac{f(|A|)}{|A'|} + \frac{g(|B|)}{|B'|} \\
                        &\leq \frac{f(|A|)}{f(|A|) |A|^{\epsilon_1}} + \frac{g(|B|)}{g(|B|) |B|^{\zeta_1}} \\
                        &= \frac{1}{|A|^{\epsilon_1}} + \frac{1}{|B|^{\zeta_1}}
                \end{split}
            \]
        \end{proof}
    \end{lemma}

    For later use, we are particularly interested in the case when $f(n) = c$.
    \todo{This may be skipped, and be directly commented in the appropiate remark following the theorem.}

    \begin{corollary}[Corollary 4.9]
        Let $A$ and $B$ be $f$-indivisible with $f(n) = c$ and $(A,B)$ satisfy the average condition.
        Then, for all $\epsilon_1 \in (0, 1 - \frac{c}{|A|})$, $\zeta_1 \in (0, 1 - \frac{c}{|B|})$, $A' \subseteq A$ and
            $B' \subseteq B$ with $|A'| \geq c |A|^{\epsilon_1}$ and $|B'| \geq c |B|^{\zeta_1}$, we have:
        \[
            \frac{|\parcurly{(a,b) \in (A',B') \mid a R b \equiv \neg t(A,B)}|}{|A' \times B'|} \leq
                \frac{1}{|A|^{\epsilon_1}} + \frac{1}{|B|^{\zeta_1}}
        \]
        \begin{proof}
            Use \Cref{lem:exceptions_bound_of_f_indivisible_sets} with $f(n) = c$.
        \end{proof}
    \end{corollary}

    \begin{remark}\label{rmk:sufficient_requirement_for_average_condition}
        Notice that the average condition is easily satisfied if the pair satisfies a condition on the size of its sets.
        If $f(n) = n^\epsilon$, $A$ and $B$ are $f$-indivisible, and $\parstraight{B} \geq \parstraight{A} \geq m$,
        then $m^{1-2\epsilon} > 2$ is sufficient for the average condition to hold for the pair $(A,B)$:
        \[
            \frac{|A|^\epsilon |B|^\epsilon}{|B|}
                \leq \frac{|B|^{2\epsilon}}{|B|}
                = \frac{1}{|B|^{1-2\epsilon}}
                = \frac{1}{m^{1-2\epsilon}}
                < \frac{1}{2}
        \]
        We will be using this fact in the context of a sequence of non-zero natural numbers
        $\parcurly{m_\ell \mid \ell \in \parcurly{0, \dots, k_{**}}}$ where $\floor{m_\ell^\epsilon} = m_{\ell+1}$
        for some $\epsilon \in (0, \frac{1}{2})$ and for all $\ell \in \parcurly{0, \dots, k_{**}-1}$.
        Here, $2 < (m_{k_{**}-1})^{1-2\epsilon}$ is sufficient for any $f$-indivisible $A$ and $B$, with
        $|A|, |B| \in \parcurly{m_0, \dots, m_{k_{**}-1}}$, to satisfy the average condition.
    \end{remark}

    Now that we have proven some properties of indivisible sets, we are actually interested in whether they can be
    found in a graph.
    It turns out that the non-$k$-order property, or more specifically the associated tree bound, is sufficient for
    proving it.
    The proof resumes in assuming that there is no indivisible set to recursively refine a \say{semi-partition}
    \todo{Lluis: hi ha alguna manera de dir una partició que no cubreix tots els vertex amb una paraula?}
    which by construction contains a $k_{**}$-tree.

    \begin{lemma}[Claim 4.3] \label{lem:existance_of_indivisible_sets}
        Let $G$ be a finite graph with the non-$k_*$-property and $f: \mathbb{R} \longrightarrow \mathbb{R}$ a function
        such that $x \geq f(x)$.
        Let $\parcurly{m_\ell \mid \ell \in \parcurly{0, \dots, k_{**}}}$ be a sequence of non-zero natural numbers such that
        for all $\ell \in \parcurly{0, \dots, k_{**}-1}$, $f(m_{\ell}) \geq m_{\ell+1}$.
        If $A \subseteq G$, $|A| \geq m_0$, then for some $\ell \in \parcurly{0, \dots, k_{**}-1}$ there is a subset $B \subseteq A$
        of size $m_\ell$ which is $f$-indivisible.
        % It may seem that m_{k_{**}} has no use, but it actually ensures that the smallest A_\eta is not empty.
        \begin{proof}
            Suppose not.
            Then we can construct the sequences $\Partriangle{b_\eta \mid \eta \in \parcurly{0,1}^{<k}}$ and $\Partriangle{A_\eta \mid \eta \in \parcurly{0,1}^{\leq k}}$
            on induction over $k = |\eta|$, satisfying:
            \begin{enumerate}
                \item\label{itm:existance_of_indivisible_sets.1} $A_{\eta^\frown \Partriangle{i}} \subseteq A_{\eta}$, $\forall i \in \parcurly{0,1}$, $\forall k \in \parcurly{0, \dots, k_{**}-1}$
                \item\label{itm:existance_of_indivisible_sets.2} $A_{\eta^\frown \Partriangle{0}} \cap A_{\eta^\frown \Partriangle{1}} = \emptyset$, $\forall k \in \parcurly{0, \dots, k_{**}-1}$
                \item\label{itm:existance_of_indivisible_sets.3} $|A_\eta| = m_k$, $\forall k \in \parcurly{0, \dots, k_{**}}$
                \item\label{itm:existance_of_indivisible_sets.4} $b_\eta \in G$ witnessing that $A_\eta$ is not $f$-indivisible, $\forall k \in \parcurly{0, \dots, k_{**}-1}$
                \item\label{itm:existance_of_indivisible_sets.5} $A_{\eta^\frown \Partriangle{i}} \subseteq A_\eta^{(i)} = \parcurly{a \in A_\eta \mid a R b_\eta \equiv (i=1)}$,
                    $\forall \in \parcurly{0,1}$, $\forall k \in \parcurly{0, \dots, k_{**}-1}$
            \end{enumerate}
            Let's prove the induction.
            For $k=0$, consider any $A_{\Partriangle{\cdot}} \subseteq A$, satisfying $|A_{\Partriangle{\cdot}}| = m_0$, and
            any $b_{\Partriangle{\cdot}}$ witnessing the non-$f$-indivisibility of $A_{\Partriangle{\cdot}}$.
            For $k > 0$ we can assume by hypothesis that $A_\eta$, with $|A_\eta| = m_{k}$, is not $f$-indivisible.
            Thus, there exists $b_\eta$ such that $A_\eta^{(i)} \geq f(m_{k}) \geq m_{k+1}$ (\dref{itm:existance_of_indivisible_sets.4}), and we can choose
            $A_{\eta^\frown \Partriangle{i}} \subseteq A_\eta^{(i)}$ (\dref{itm:existance_of_indivisible_sets.5}), such that
            $|A_{\eta^\frown \Partriangle{i}}| = m_{k+1}$ $\forall i \in \parcurly{0,1}$ (\dref{itm:existance_of_indivisible_sets.3}).
            \dref{itm:existance_of_indivisible_sets.1} and \dref{itm:existance_of_indivisible_sets.2} are satisfied by the definition of $A_\eta^{(i)}$.
            Now, for all $\eta$ such that $|\eta| = k_{**}$, consider some element $a_\eta \in A_\eta$, which exists since $m_\ell > 0$
            for all $\ell$.
            Then, we have two sequences $\Partriangle{b_\eta \mid \eta \in \parcurly{0,1}^{<k_{**}}}$ and $\Partriangle{a_\eta \mid \eta \in \parcurly{0,1}^{k_{**}}}$
            satisfying the $k_{**}$-tree property: for all $\rho \in \parcurly{0,1}^{<k_{**}}$ and $\eta \in \parcurly{0,1}^{k_{**}}$
            if given $\ell \in \parcurly{0, 1}$ we have $\rho^\frown \Partriangle{\ell} \trianglelefteq \eta$ then
            $(a_\eta R b_\rho) \equiv \parround{\ell = 1}$ since $a_\eta \in A_\eta \subseteq A_{\rho ^\frown \Partriangle{i}}$.
            This contradicts the $k_{**}$ tree bound.
        \end{proof}
    \end{lemma}

    The previous proof can be iteratively used to partition the graph into indivisible parts, with a small reminder.
    As the average condition cares about the ordering, we define the partition as a tuple instead of a family of sets,
    and fix an ascending order on the size of the parts.

    \begin{lemma}[Claim 4.4 + 4.5] \label{lem:existance_of_ordered_f_indivisible_partitions}
        Let $G$ be a finite graph with the non-$k_{*}$-order property and $f: \mathbb{R} \longrightarrow \mathbb{R}$ a function
        such that $x \geq f(x)$.
        Let $\parcurly{m_\ell \mid \ell \in \parcurly{0, \dots, k_{**}}}$ be a sequence of non-zero natural numbers such that
        for all $\ell \in \parcurly{0, \dots, k_{**}-1}$, $f(m_{\ell}) \geq m_{\ell+1}$.
        If $A \subseteq G$ with $|A| = n$, then we can find a sequence $\overline{A} = \Partriangle{A_j \mid j \in \parcurly{1, \dots, j(*)}}$
        and reminder $B = A \setminus \bigcup \overline{A}$ such that:
        \begin{enumerate}
            \item \label{itm:existance_of_ordered_f_indivisible_partitions.1} For each $j \in \parcurly{1, \dots, j(*)}$, $A_j$ is $f$-indivisible.
            \item \label{itm:existance_of_ordered_f_indivisible_partitions.2} For each $j \in \parcurly{1, \dots, j(*)}$, $|A_j| \in \parcurly{m_0, \dots, m_{k_{**}-1}}$.
            \item \label{itm:existance_of_ordered_f_indivisible_partitions.3} $A_j \subseteq A \setminus \bigcup\parcurly{A_i \mid i < j}$, in particular $A_i \cap A_j = \emptyset$ $\forall i \neq j$.
            \item \label{itm:existance_of_ordered_f_indivisible_partitions.4} $|B| < m_0$.
            \item \label{itm:existance_of_ordered_f_indivisible_partitions.5} $\overline{A}$ is $\leq$-increasing.
        \end{enumerate}
        \begin{proof}
            Iteratively, apply \Cref{lem:existance_of_indivisible_sets} to the remainder $A \setminus \bigcup \parcurly{A_i \mid i < j}$
            (\dref{itm:existance_of_ordered_f_indivisible_partitions.3}) to get an $f$-indivisible $A_j$ (\dref{itm:existance_of_ordered_f_indivisible_partitions.1}) of size $m_\ell$, $\ell \in \parcurly{0, \dots, k_{**}-1}$
            (\dref{itm:existance_of_ordered_f_indivisible_partitions.2}) until less then $m_0$ vertices are available (\dref{itm:existance_of_ordered_f_indivisible_partitions.4}).
            To conclude, reorder the indices of the $A_j$'s in ascending size order (\dref{itm:existance_of_ordered_f_indivisible_partitions.5}).
        \end{proof}
    \end{lemma}

    Finally, we ensure the pairs satisfy the average condition by simply requiring a minimal size of the parts,
    a condition that can be easily integrated in the definition of the sequence of integers.

    \begin{lemma}[Claim 4.10] \label{lem:existance_of_ordered_f_indivisible_partitions_with_exceptions_bound}
        Let $G$ be a finite graph with the non-$k_{*}$-order property.
        Let $\parcurly{m_\ell \mid \ell \in \parcurly{0, \dots, k_{**}}}$ be a sequence of non-zero natural numbers such that
        $n \geq m_0$ and for all $\ell \in \parcurly{0, \dots, k_{**}-1}$, $\floor{m_\ell^\epsilon} = m_{\ell+1}$,
        for some $\epsilon \in (0, \frac{1}{2})$ such that $2 < (m_{k_{**}-1})^{1-2\epsilon}$.
        \todo{Last condition can be changed (and probably should) for a condition on $m_{k_{**}}$}
        If $A \subseteq G$ with $|A| = n$, then we can find a sequence $\overline{A} = \Partriangle{A_i \mid i \in \parcurly{1, \dots, i(*)}}$
        and reminder $B = A \setminus \bigcup \overline{A}$ satisfying:
        \begin{enumerate}
            \item \label{itm:existance_of_ordered_f_indivisible_partitions_with_exceptions_bound.1} For each $i \in \parcurly{1, \dots, i(*)}$, $A_i$ is $\epsilon$-indivisible.
            \item \label{itm:existance_of_ordered_f_indivisible_partitions_with_exceptions_bound.2} For each $i \in \parcurly{1, \dots, i(*)}$, $|A_i| \in \parcurly{m_0, \dots, m_{k_{**}-1}}$.
            \item \label{itm:existance_of_ordered_f_indivisible_partitions_with_exceptions_bound.3} $A_i \cap A_j = \emptyset$ for all $i \neq j$.
            \item \label{itm:existance_of_ordered_f_indivisible_partitions_with_exceptions_bound.4} $|B| < m_0$.
            \item \label{itm:existance_of_ordered_f_indivisible_partitions_with_exceptions_bound.5} $\overline{A}$ is $\leq$-increasing.
            \item \label{itm:existance_of_ordered_f_indivisible_partitions_with_exceptions_bound.6} If $\zeta \in \parround{0,\epsilon^{k_{**}}}$ then for every $i,j \in \parcurly{1, \dots, i(*)}$ with $i < j$,
                $A \subseteq A_i$ ad $B \subseteq A_j$ such that $|A| \geq |A_i|^{\epsilon + \zeta}$ and $|B| \geq |A_j|^{\epsilon + \zeta}$
                we have that:
                \[
                    \begin{split}
                        \frac{|\parcurly{(a,b) \in (A,B) \mid a R b \equiv \neg t(A_i,A_j)}|}{|A \times B|}
                            &\leq \frac{1}{|A_i|^\zeta} + \frac{1}{|A_j|^\zeta} \\
                            &\leq \frac{1}{|A|^\zeta} + \frac{1}{|B|^\zeta}
                    \end{split}
                \]
        \end{enumerate}
        \begin{proof}
            The five points are direct consequence of \Cref{lem:existance_of_ordered_f_indivisible_partitions},
            setting $f(x) = x^\epsilon$.
            Now, by \dref{itm:existance_of_ordered_f_indivisible_partitions_with_exceptions_bound.2}, for any $A_i, A_j \in \overline{A}$ with $i < j$
            there is some $\ell \in \parcurly{0, \dots, k_{**}-1}$ such that $|A_i| \leq |A_j| = m_\ell$.
            Also, it follows the condition $2 < (m_{k_{**}-1})^{1-2\epsilon}$ and \Cref{rmk:sufficient_requirement_for_average_condition}
            that the pair $(A_i,A_j)$ satisfies the average condition.
            Finally, notice that $\epsilon^{k_{**}} < \epsilon < 1 - \epsilon$ since $\epsilon \in (0, \frac{1}{2})$,
            so that $\zeta \in (0, \epsilon ^ {k_{**}}) \subseteq (0, 1 - \epsilon)$ and the condition for
            \Cref{lem:exceptions_bound_of_epsilon_indivisible_sets} is satisfied.
            This gives us \dref{itm:existance_of_ordered_f_indivisible_partitions_with_exceptions_bound.6} and concludes the proof of the statement.
        \end{proof}
    \end{lemma}
    \todo{Maybe merge the last two lemmas?}

    \begin{remark}
        For sufficiently small $\epsilon$, the condition $2 < (m_{k_{**}-1})^{1-2\epsilon}$ is almost, trivial.
        For example, if $\epsilon < \frac{1}{4}$, then we are just requiring that $m_{k_{**}-1} \geq 4$.
    \end{remark}

    As stated earlier, the principal drawback of the previous result is that the obtained partition is not equitable.
    To deal with this, we study the event of randomly partitioning a pair of indivisible sets
    \todo{Hehe, "partition indivisible sets", sona contradictiu.}
    into subparts of equal size.
    We prove that the event of a pair of subparts of the refinement being either fully connected or completely empty,
    is satisfied with very high probability.

    \begin{definition}
        Let $A, B$ be $f$-indivisible sets with $f(A) f(B) < \frac{1}{2} |B|$.
        Let $\Partriangle{A_i \mid i \in \parcurly{1, \dots, i_A}}$ be a partition of $A$ with $|A_i| = m$ for all
        $i \in \parcurly{1, \dots, i_A}$ and $\Partriangle{B_i \mid i \in \parcurly{1, \dots, i_B}}$ be a partition of
        $B$ with $|B_i| = m$ for all $i \in \parcurly{1, \dots, i_B}$.
        We define $\varepsilon^+_{A_i,A_j,m}$ as the event:
        \[
            \forall a \in A_i \ \forall b \in B_i, a R b = t(A,B)
        \]
    \end{definition}

    \begin{lemma}[Claim 4.13] \label{lem:bound_on_the_probability_of_a_subpair_having_no_exceptions}
        Let $G$ be a finite graph with the non-$k_{*}$-order property.
        Let $\Partriangle{m_\ell \mid \ell \in \parcurly{0, \dots, k_{**}}}$ be a sequence of non-zero natural numbers such that
        $n \geq m_0 \geq n^\epsilon$ \todo{Should be mentioned that this is a strong limitation which is not mentioned in the original
        paper, but required for the calculations (sin hacer trampa).} and for all $\ell \in \parcurly{0, \dots, k_{**}-1}$, $m_\ell^\epsilon = m_{\ell+1}$,
        \todo{Here we have enforced the equality. Should be commented that this is easily achievable as we do in the next results.}
        for some $\epsilon \in (0, \frac{1}{2})$ such that $2 < (m_{k_{**}-1})^{1-2\epsilon}$.
        Let $A_1, A_2 \subseteq G$ be two $\epsilon$-indivisible subsets such that $|A_1| = m_{\ell_1}$ and $|A_2| = m_{\ell_2}$
        for some $\ell_1, \ell_2 \in \parcurly{0, \dots, k_{**}-1}$ and $|A_1| \leq |A_2|$.
        Let $c \in (0, 1-\epsilon)$ and $\zeta \leq \frac{1 - \epsilon - c}{3}\epsilon^{k_{**}}$ such that
        $m \coloneqq n^\zeta$ divides $|A_1|$ and $|A_2|$.
        Then, let $\Partriangle{A_{1,s} \mid s \in \bparcurly{1, \dots, \frac{|A_1|}{m}}}$ and
        $\Partriangle{A_{2,t} \mid t \in \bparcurly{1, \dots, \frac{|A_2|}{m}}}$ be random partitions of $A_1$ and $A_2$
        respectively, with pieces of size $m$.
        We have that
        \[
            P(\varepsilon^+_{A_{1,s},A_{2,t},m}) \geq 1 - \frac{2}{n^{c\epsilon^{k_{**}}}}
        \]
        \begin{proof}
            Fix $s \in \frac{|A_1|}{m}$, $t \in \frac{|A_2|}{m}$.
            It follows from the condition $2 < (m_{k_{**}-1})^{1-2\epsilon}$ and \Cref{rmk:sufficient_requirement_for_average_condition}
            that the pair $(A_{1}, A_{2})$ satisfies the average condition.
            Let $U_1 = \bparcurly{a \in A_1 \mid \bparstraight{\parcurly{b \in A_2 \mid a R b \equiv \neg t(A_1, A_2)}} \geq |A_2|^\epsilon}$
            and for each $a \in A_1 \setminus U_1$ let $U_{2,a} = \parcurly{b \in A_j \mid a R b \equiv \neg t(A_1, A_2)}$.
            By \Cref{lem:average_condition_statement}, $|U_1| \leq |A_1|^\epsilon$ and $\forall a \in A_1 \setminus U_1$,
            $|U_{2,a}| \leq |A_2|^\epsilon$.
            Now, given $\bparcurly{1, \dots, \frac{|A_1|}{m}}$, we can bound the probability $P_1$ that
            $A_{1,s} \cap U_1 \neq \emptyset$ as follows:
            \[
                \begin{split}
                    P_1
                        & \leq \frac{|U_1|}{|A_1|} + \dots + \frac{|U_1|}{|A_1|-m+1}
                            < \frac{m |U_1|}{|A_1| - m}
                            \leq \frac{m |A_1|^\epsilon}{|A_1| - m} \\
                        & \leq \frac{m^2 |A_1|^\epsilon}{|A_1|}
                            = \frac{m^2}{|A_1|^{1-\epsilon}}
                            = \frac{m^2}{m_0^{\parround{1-\epsilon} \epsilon^{\ell_1}}} % the requirement of equality in the condition of m_l's is required here
                            \leq \frac{n^{2 \zeta}}{n^{(1-\epsilon)\epsilon^{\ell_1 + 1}}} \\
                        & \leq \frac{n^{2\frac{1-\epsilon-c}{3} \epsilon^{k_{**}}}}{n^{(1-\epsilon)\epsilon^{k_{**}}}}
                            \leq \frac{n^{(1-\epsilon-c) \epsilon^{k_{**}}}}{n^{(1-\epsilon)\epsilon^{k_{**}}}}
                            = \frac{1}{n^{c \epsilon^{k_{**}}}}
                \end{split}
            \]
            The forth inequality comes from the fact that $\frac{(|A_i| - m) m}{|A_i|} \geq 1$.
            Then, if $A_{1,s} \cap U_1= \emptyset$, we have that $|\bigcup_{a \in A_{1,s}} U_{2,a}| \leq |A_{1,s}| |A_2|^\epsilon$.
            So, given $\bparcurly{1, \dots, \frac{|A_2|}{m}}$, we can bound $P_2$, the probability that
            $A_{2,t} \cap \bigcup_{a \in A_{1,s}} U_{2,a} \neq \emptyset$, by:
            \[
                \begin{split}
                    P_2
                        & \leq \frac{|\bigcup_{a \in A_{1,s}} U_{2,a}|}{|A_2|} + \dots + \frac{|\bigcup_{a \in A_{1,s}} U_{2,a}|}{|A_2|-m+1}
                            < \frac{m |\bigcup_{a \in A_{1,s}} U_{2,a}|}{|A_2| - m}
                            \leq \frac{m m |A_2|^\epsilon}{|A_2| - m} \\
                        & \leq \frac{m^3 |A_2|^\epsilon}{|A_2|}
                            = \frac{m^3}{|A_2|^{1-\epsilon}}
                            = \frac{m^3}{m_0^{\parround{1-\epsilon} \epsilon^{\ell_2}}} % the requirement of equality in the condition of m_l's is required here
                            \leq \frac{n^{3 \zeta}}{n^{(1-\epsilon)\epsilon^{\ell_2 + 1}}} \\
                        & \leq \frac{n^{3\frac{1-\epsilon-c}{3} \epsilon^{k_{**}}}}{n^{(1-\epsilon)\epsilon^{k_{**}}}}
                            \leq \frac{n^{(1-\epsilon-c) \epsilon^{k_{**}}}}{n^{(1-\epsilon)\epsilon^{k_{**}}}}
                            = \frac{1}{n^{c \epsilon^{k_{**}}}}
                \end{split}
            \]
            Putting it all together:
            \[
                P(\varepsilon^+_{A_{1,s},A_{2,t},m})
                    \geq (1 - P_1) (1 - P_2)
                    \geq \parround{1 - \frac{1}{n^{c \epsilon^{k_{**}}}}}^2
                    \geq 1 - \frac{2}{n^{c\epsilon^{k_{**}}}}
            \]
        \end{proof}
    \end{lemma}

    \begin{remark}
        The condition on the size of $m_0$, which is both an upper and lower bound, is very strong and will be carried over
        up to \Cref{thm:existance_of_equitative_partition_with_perfect_pairs_but_with_bound_exceptional_pairs}.
        The greater limitations of this resides in the fact that the size of the parts of the resulting partition $m_{**}$
        is set by the size of $m_0$, and thus inherits the same limitations.
    \end{remark}

    Now, since the event of a given subpair not satisfying the desired property is very unlikely, it can be easily proven
    that a random refinement of the partition given by \Cref{lem:existance_of_ordered_f_indivisible_partitions} only has
    a small number of exceptional pairs.

    \begin{lemma}[Claim 4.14] \label{lem:existance_of_equitative_partition_with_bound_exceptional_pairs}
        Let $G$ be a finite graph with the non-$k_{*}$-order property.
        Let $\Partriangle{m_\ell \mid \ell \in \parcurly{0, \dots, k_{**}}}$ be a sequence of non-zero natural numbers such that
        for all $\ell \in \parcurly{0, \dots, k_{**}-1}$, $m_\ell^\epsilon = m_{\ell+1}$,
        for some $\epsilon \in (0, \frac{1}{2})$ such that $2 < (m_{k_{**}-1})^{1-2\epsilon}$.
        Also, suppose $m_0$ satisfies $n^\epsilon \leq m_0 < \min \parround{\frac{\sqrt{2}-1}{\sqrt{2}} n, \frac{n}{n^{c \epsilon^{k_{**}}}}}$,
        with $c \in (0, 1-\epsilon)$.
        Finally, let $m_{**}$ be a divisor of $m_\ell$ for all $\ell \in \parcurly{0, \dots, k_{**}-1}$ and
        $m_{**} \leq n^{\frac{1 - \epsilon - c}{3}\epsilon^{k_{**}}}$.
        If $A \subseteq G$ with $|A| = n$, then we can find a partition $\overline{A} = \Partriangle{A_i \mid i \in \parcurly{1, \dots, r}}$
        with reminder $B = A \setminus \bigcup_{i \in \parcurly{1, \dots, r}} A_i$ such that:
        \begin{enumerate}
            \item \label{itm:existance_of_equitative_partition_with_bound_exceptional_pairs.1} $|A_i| = m_{**}$ for all $i \in \parcurly{1, \dots, r}$.
            \item \label{itm:existance_of_equitative_partition_with_bound_exceptional_pairs.2} For all but $\frac{2}{n^{c\epsilon^{k_{**}}}}r^2$ of the pairs
                $(A_i, A_j)$ with $i<j$ there are no exceptional edges, i.e.
                \[
                    \parcurly{(a,b) \in A_i \times A_j \mid a R b \not\equiv t(A_i, A_j)} = \emptyset
                \]
            \item \label{itm:existance_of_equitative_partition_with_bound_exceptional_pairs.3} $|B| < m_0$
        \end{enumerate}
        \begin{proof}
            We can use \Cref{lem:existance_of_ordered_f_indivisible_partitions} to get a partition
            $\overline{A'} = \Partriangle{A'_i \mid i \in \parcurly{1, \dots, i(*)}}$ and remainder $B' = A \setminus \bigcup A'$.
            We can refine the partition by randomly splitting each $A'_i$ into pieces of size $m_{**}$ (\dref{itm:existance_of_equitative_partition_with_bound_exceptional_pairs.1}).
            Consider the resulting partition $\overline{A} = \Partriangle{A_i \mid i \in \parcurly{1, \dots, r}}$ with remainder $B = B'$
            (\dref{itm:existance_of_equitative_partition_with_bound_exceptional_pairs.3}).
            First of all, notice that for each pair $(A_i, A_j)$ such that $A_i \subseteq A'_{i_1}$ and
            $A_j \subseteq A'_{j_1}$ with $i_1 \neq j_1$, the probability of the pair having exceptional edges is
            upper bounded by $\frac{2}{n^{c\epsilon^{k_{**}}}}$.
            This follows \Cref{lem:bound_on_the_probability_of_a_subpair_having_no_exceptions}.
            Thus, given $X$ the random variable counting the number of exceptional pairs of this kind, we have
            \[
                \mathbb{E}(X) = \sum_{\substack{A_i,A_j \text{ s.t.}\\A_i\subseteq A'_{i_1},A_j\subseteq A'_{j_1}\\i_1\neq j_1}} \mathbb{E}(X_{A_i, A_j})
                     = \sum_{\substack{A_i,A_j \text{ s.t.}\\A_i\subseteq A'_{i_1},A_j\subseteq A'_{j_1}\\i_1\neq j_1}} P(\varepsilon_{A_i, A_j,m_{**}})
                     \leq \frac{r^2}{2} \frac{2}{n^{c\epsilon^{k_{**}}}}
            \]
            where $X_{A_i,A_j}$ is the random variable giving $1$ if $(A_i, A_j)$ is exceptional, and $0$ otherwise.
            Since the expectation is an average, for some refinement $\overline{A}$ of $\overline{A'}$ we have that
            the number of exceptional pairs when $i_1 \neq j_1$ is at most $\frac{r^2}{n^{c\epsilon^{k_{**}}}}$.
            Now, we have no control if $i_1 = j_1$, so let's bound how many of these we have:
            \[
                \begin{split}
                    \bparstraight{\bparcurly{\text{Exceptional } (A_i, A_j) \mid A_i, A_j \subseteq A'_{i_1}, i_1 \in \parcurly{1, \dots, i(*)}}}
                        & \leq {\frac{m_0}{m_{**}} \choose 2} \frac{n}{m_0} \\
                        & \leq \frac{\parround{\frac{m_0}{m_{**}}}^2}{2} \frac{n}{m_0}
                            = \frac{m_0 n}{2 m_{**}^2}
                            = \frac{m_0}{n} \bbparround{\frac{n}{\sqrt{2}m_{**}}}^2 \\
                        & \leq \frac{m_0}{n} \bbparround{\frac{n - m_0}{m_{**}}}^2
                            \leq \frac{m_0}{n} r^2
                            < \frac{r^2}{n^{c \epsilon^{k_{**}}}}
                \end{split}
            \]
            Notice that the third inequality comes after the condition $m_0 \leq \frac{\sqrt{2}-1}{\sqrt{2}} n$.
            Putting it all together, we see that the number of exceptional pairs is upper bounded by
                $\frac{2r^2}{n^{c\epsilon^{k_{**}}}}$ satisfying \dref{itm:existance_of_equitative_partition_with_bound_exceptional_pairs.2}.
        \end{proof}
    \end{lemma}

    \begin{remark}[Remark 4.15]
        In the previous proof, the condition $m_0 < \frac{n}{n^{c\epsilon^{k_{**}}}}$ can be
        weakened at the cost of increasing the number of exceptional pairs.
        More specifically, since this condition is only used to bound the exceptional sub-pairs in the same pair
        (the second part of the proof), the number of exceptional pairs can be generally bounded by
        \[
            |\parcurly{\text{Exceptional pairs}}|
                \leq \bbparround{\frac{m_0}{n} + \frac{2}{n^{c\epsilon^{k_{**}}}}} r^2
        \]
    \end{remark}

    We now resume the previous results in a theorem with minimal conditions.

    \todo{Notation here is confusing. $r$ is another thing, and $m$ becomes the number of parts.}

    \begin{theorem}[Theorem 4.16] \label{thm:existance_of_equitative_partition_with_perfect_pairs_but_with_bound_exceptional_pairs}
        Let $\epsilon = \frac{1}{r} \in \parround{0, \frac{1}{2}}$ with $r \in \mathbb{N}$ (this avoids rounding errors),
        $c\in \parround{0,1-\epsilon}$ and $k_*$ be given.
        Let $G$ be a finite graph with the non-$k_*$-order property.
        Let $A \subseteq G$ with $|A| = n$, and $n > 2^{\frac{r^{k_{**}}}{1-2\epsilon}}$.
        Then, for any $m_{**} \in \bparsquared{n^{\frac{(1-\epsilon-c)}{3}\epsilon^{k_{**}+1}},
        \parround{\frac{\sqrt{2}-1}{\sqrt{2}}}^{\frac{1-\epsilon-c}{3}\epsilon^{k_{**}}} n^{\frac{(1-\epsilon-c)}{3}\epsilon^{k_{**}} -
        \frac{(1-\epsilon-c)c}{3}\epsilon^{2k_{**}}}}$, there is a partition
        $\overline{A} = \Partriangle{A_i \mid i \in \parcurly{1, \dots, m}}$ of $A$ with remainder
        $B = A \setminus \bigcup \overline{A}$ such that:
        \begin{enumerate}
            \item\label{itm:existance_of_equitative_partition_with_perfect_pairs_but_with_bound_exceptional_pairs.1}
                $|A_i| = m_{**}$ for all $i \in \parcurly{1, \dots, m}$.
            \item\label{itm:existance_of_equitative_partition_with_perfect_pairs_but_with_bound_exceptional_pairs.2}
                $|B| < m_{**}^{\frac{3}{(1-\epsilon-c)}r^{k_{**}}}$.
            \item\label{itm:existance_of_equitative_partition_with_perfect_pairs_but_with_bound_exceptional_pairs.3}
                $\bparstraight{\bparcurly{ (i,j) \mid i,j \in \parcurly{1, \dots, m}, i < j \text{ and }
                \parcurly{(a,b) \in A_i \times A_j \mid a R b} \notin
                \parcurly{A_i \times A_j, \emptyset}}}
                \leq \frac{2}{n^{c\epsilon^{k_{**}}}} m^2$
        \end{enumerate}
        \begin{proof}
            Let $m_{k_{**}} = m_{**}^{\frac{3}{1-\epsilon-c}}$, and consider the sequence
            \[
                m_{**} \leq m_{k_{**}} < \dots < m_0
            \]
            such that for all $\ell \in \parcurly{1, \dots, k_{**}}$ we have that $m_{\ell-1} = m_\ell^r$.
            Notice that:
            \begin{enumerate}
                \item $m_{**}$ divides $m_\ell$ for all $\ell \in \parcurly{0, \dots, k_{**}}$ since the $m_\ell$'s are powers of $m_{k_{**}}$
                    and $m_{**}$ divides $m_{k_{**}}$ by construction.
                \todo{Probably it is not needed that $m_{**}$ divides $m_{k_{**}}$, with $m_{k_{**}-1}$ is enough, but it comes for free.}
                \item $(m_{\ell-1})^\epsilon = m_\ell$ for all $\ell \in \parcurly{1, \dots, k_{**}}$, by construction.
                \item $m_{**} \leq n^{\frac{1-\epsilon-c}{3}\epsilon^{k_{**}}}$, by choice of $m_{**}$.
                \item $m_0 = m_{**}^{\frac{3}{1-\epsilon-c}r^{k_{**}}}$, so on one hand
                    \[
                        m_0 = m_{**}^{\frac{3}{1-\epsilon-c}r^{k_{**}}} \geq n^{\frac{1-\epsilon-c}{3}\epsilon^{k_{**}+1} \frac{3}{1-\epsilon-c}r^{k_{**}}}
                            \geq n^{\epsilon}
                    \]
                    and on the other hand,
                    \[
                        m_0 = m_{**}^{\frac{3}{1-\epsilon-c}r^{k_{**}}} \leq \parround{\frac{\sqrt{2}-1}{\sqrt{2}}} n^{1 - c \epsilon^{k_{**}}}
                    \]
                    and thus $n$ is both smaller than $\parround{\frac{\sqrt{2}-1}{\sqrt{2}}} n$ and
                    smaller than $n^{1 - c \epsilon^{k_{**}}}$.
                \item $m_{k_{**}-1} = m_{**}^{\frac{3}{1-\epsilon-c}r} \geq n^{\epsilon^{k_{**}}} > 2^{\frac{1}{1-2\epsilon}}$.
            \end{enumerate}
            So, all the conditions of \Cref{lem:existance_of_equitative_partition_with_bound_exceptional_pairs} are satisfied,
            and we can use it to get a partition $\overline{A}$ with remainder $B$ satisfying the statement.
            Notice that \dref{itm:existance_of_equitative_partition_with_perfect_pairs_but_with_bound_exceptional_pairs.2}
            is satisfied by the fact that $|B| < m_0 \leq m_{**}^{\frac{3}{(1-\epsilon-c)}r^{k_{**}}}$.
        \end{proof}
    \end{theorem}

    \begin{remark}
        Some notes on the partition obtained in the previous theorem:
        \begin{itemize}
            \item With any choice of $c$ and $m_{**}$, the fraction of exceptional pairs is asymptotically small,
                but we obtain very small parts, that is, $m_{**} \thickapprox n^{\epsilon^{k_{**}}}$.
            \item A smaller value of $c$ results in larger parts and smaller reminder, at the cost of a larger fraction
                of exceptional pairs.
            \item The window of choice of $m_{**}$ is very small, and taking a larger value
                (in the given interval), results in a strongly larger reminder.
                The edge case of choosing $m_{**}$ as the larger value, results in the bound on the size of the
                reminder becoming $\parstraight{B} < \frac{\sqrt{2}-1}{\sqrt{2}} n^{1-\epsilon^{k_{**}}}$.
        \end{itemize}
    \end{remark}

    Next, we will follow another approach to obtain an equitable partition.
    That is, if a given \regular~property is transitive to subsets, then creating such a partition is not a hard
    problem, as we can simply choose a subset of the desired size from each \regular~part.
    A very simple such property is $f_c$-indivisibility, where $f_c$ is the constant function $f_c(x) = c$.
    With this property, the number of exceptional edges is at most $c$ for each vertex of the graph,
    and taking a subset can only reduce the absolute number of exceptional edges.
    This property is much stronger then $\epsilon$-indivisibility, but in the next few lemmas we will show that actually,
    in each $\epsilon$-indivisible set, there is a subset of vertices which is $f_c$-indivisible
    (\Cref{lem:many_values_to_equitative_partition_with_bound_exceptional_pairs}).

    To prove so, we use a probabilistic argument, and show that the event of there existing a subset which has
    intersection smaller than $c$ with every $\overline{B}_{A,b}$ (\Cref{def:n_large_enough_property}) is highly
    probable under some very specific conditions (\Cref{lem:n_large_enough_valid_values}).

    \todo{In what follows, $c$ should be another letter, it collides with previous definition. Also, what about renaming
        $c$-indivisible to $f_c$-indivisible or something like that?}

    \begin{definition}[Definition 4.18] \label{def:n_large_enough_property}
        For $n, c \in \mathbb{N}$ and $\epsilon, \zeta, \xi \in \mathbb{R}$, let $\oplus[n, \epsilon, \zeta, \xi, c]$ be
        the statement:
        For any set $A$ and family of subsets $P \subseteq \mathcal{P}(A)$ such that $|A| = n$ and $|P| \leq n^{\frac{1}{\zeta}}$,
        and for all $B \in P$ with $|B| \leq n^\epsilon$, there exists $U \subseteq A$ with $|U| = \lfloor n^\xi \rfloor$ such that
        for all $B \in P$, $|U \cap B| \leq c$.
    \end{definition}

    \begin{lemma}[Lemma 4.19] \label{lem:n_large_enough_valid_values}
        If the reals $\epsilon, \zeta, \xi$ satisfy $\epsilon \in (0,1)$, $\zeta > 0$ and $0 < \xi < \frac{1}{2}$,
        the natural number $n$ is sufficiently large ($n > N(\epsilon, \zeta, \xi, c)$) to satisfy the equation
            \begin{equation} \label{eq:n_large_enough_valid_values.1}
                \frac{1}{2n^{1-2\xi}} + \frac{1}{n^{(1 - \xi - \epsilon)c - \frac{1}{\zeta}}} < 1
            \end{equation}
        and $c > \frac{1}{\zeta (1 - \xi - \epsilon)}$,
        then $\oplus[n, \epsilon, \zeta, \xi, c]$ holds.
        \begin{proof}
            First of all, notice that the condition on $c$ implies that $(1 - \xi - \epsilon) > 0$, and thus $\xi < 1 -\epsilon$.
            Let $m = \lfloor n^\xi \rfloor$ be the size of the set $U$ we want to build, and let $\mathcal{F}_* = [A]^m$
            the set of sequences of elements of $A$ with length $m$.
            Let $\mu$ be a probability distribution on $\mathcal{F}_*$ such that for all $F \in \mathcal{F}_*$,
            $\mu(F) = \frac{|F|}{|\mathcal{F}_*|}$.
            We want to prove that the probability that a random $U$ satisfies:
            \begin{enumerate}
                \item\label{itm:n_large_enough_valid_values.1} All elements of $U$ are distinct.
                \item\label{itm:n_large_enough_valid_values.2} For all $B \in P$, $|U \cap B| < c$.
            \end{enumerate}
            is non-zero.
            First of all let's bound the converse of \dref{itm:n_large_enough_valid_values.1}, i.e. the probability that there are two equal elements
            in $U$:
            \[
                P_1 = P(\exists s < t \in [m] \mid U_s = U_t)
                    \leq {m \choose 2} \frac{n}{n^2}
                    \leq \frac{m^2}{2n}
                    \leq \frac{n^{2\xi}}{2n}
                    < \frac{1}{2n^{1-2\xi}}
            \]
            Now, in order to bound \dref{itm:n_large_enough_valid_values.2}, let's first bound the probability that at least $c$ elements of
            $U$ are in a given $B \in P$:
            \[
                P_B = P(\exists^{\geq c} t\in [m] \mid U_t \in B)
                    \leq {m \choose c} \bbparround{\frac{|B|}{n}}^c
                    \leq \frac{m^c |B|^c}{n^c}
                    \leq \frac{n^{\xi c} n^{\epsilon c}}{n^c}
                    = \frac{1}{n^{c (1 - \xi - \epsilon)}}
            \]
            Then, we can bound the converse of \dref{itm:n_large_enough_valid_values.2}, i.e. the probability that this happens for some $B \in P$,
            by:
            \[
                P_2 = P(\exists B \in P \mid \exists^{\geq c} t\in [m], U_t \in B)
                    \leq \sum_{B \in P} P_B
                    = \frac{|P|}{n^{c (1 - \xi - \epsilon)}}
                    \leq \frac{1}{n^{c (1 - \xi - \epsilon) - \frac{1}{\zeta}}}
            \]
            Putting it all together, we have that
            \[
                P((\dref{itm:n_large_enough_valid_values.1}) \cup (\dref{itm:n_large_enough_valid_values.2}))
                    \leq P_1 + P_2
                    < \frac{1}{2n^{1-2\xi}} + \frac{1}{n^{c (1 - \xi - \epsilon) - \frac{1}{\zeta}}}
            \]
            Notice that
            \begin{itemize}
                \item Since $\xi < \frac{1}{2}$ we have that $1 - 2\xi > 0$.
                \item $c (1 - \xi - \epsilon) - \frac{1}{\zeta}> 0$.
            \end{itemize}
            so, the $n$-large enough condition \eqref{eq:n_large_enough_valid_values.1} is well defined and
            \[
                P((\dref{itm:n_large_enough_valid_values.1}) \cup (\dref{itm:n_large_enough_valid_values.2}))
                    < \frac{1}{2n^{1-2\xi}} + \frac{1}{n^{c (1 - \xi - \epsilon) - \frac{1}{\xi}}}
                    < 1
            \]
            holds.
            We conclude that the probability that there exists a $U \subseteq A$ satisfying the condition is non-trivial,
            and $\oplus[n, \epsilon, \zeta, \xi, c]$ holds.
        \end{proof}
    \end{lemma}

    \begin{lemma}[Claim 4.21] \label{lem:many_values_to_equitative_partition_with_bound_exceptional_pairs}
        Let $k_*, k, c \in \mathbb{N}$ and $\epsilon, \xi \in \mathbb{R}$ such that:
        \begin{enumerate}
            \item\label{itm:many_values_to_equitative_partition_with_bound_exceptional_pairs.1} $G$ is a graph with the non-$k_*$-order property.
            \item\label{itm:many_values_to_equitative_partition_with_bound_exceptional_pairs.2} $\epsilon \in \parround{0, \frac{1}{2}}$.
            \item\label{itm:many_values_to_equitative_partition_with_bound_exceptional_pairs.3} $\xi \in \parround{0, \frac{\epsilon^{k_{**}}}{2}}$.
            \item\label{itm:many_values_to_equitative_partition_with_bound_exceptional_pairs.4} $c$ satisfies \[
                c > \frac{1}{\frac{1}{k} (1 - \frac{\xi}{\epsilon^{k_{**}}} - \epsilon)}
            \]
        \end{enumerate}
        Then, for every sufficiently large $n \in \mathbb{N}$ (it suffices that $n >
        g_\epsilon^{k_{**}}\parround{N_{\ref{lem:n_large_enough_valid_values}}\parround{\epsilon, \frac{1}{k}, \frac{\xi}{\epsilon^{k_{**}}}, c}}$,
        where $g_\epsilon(x) = \parround{x+1}^{\frac{1}{\epsilon}}$),
        if $A \subseteq G$ with $|A| = n$, there is $Z \subseteq A$ such that
        \begin{enumerate}[label=(\alph*), ref=\alph*]
            \item\label{itm:many_values_to_equitative_partition_with_bound_exceptional_pairs.a} $|Z| = \lfloor n^\xi \rfloor$.
            \item\label{itm:many_values_to_equitative_partition_with_bound_exceptional_pairs.b} $Z$ is $c$-indivisible in $G$.
        \end{enumerate}
        \begin{proof}
            Let $n = m_0 > m_1 > \dots > m_{k_{**}}$ with $m_\ell = \floor{m_{\ell-1}^\epsilon} \geq g_\epsilon^{-1}(m_{\ell-1}) \geq g_\epsilon^{-\ell}(n)$.
            Then, $m_\ell \geq m_{\ell+1}$ and we can use \Cref{lem:existance_of_indivisible_sets}
            to getan $\epsilon$-indivisible subset $A_1 \subseteq A$, with $|A_1| = m_\ell$ for some $\ell \in \parcurly{0, \dots, k_{**}-1}$.
            Notice that:
            \begin{itemize}
                \item $\epsilon \in (0,1)$ by \dref{itm:many_values_to_equitative_partition_with_bound_exceptional_pairs.2}.
                \item We can set $\zeta \coloneqq \frac{1}{k} > 0$.
                \item By \dref{itm:many_values_to_equitative_partition_with_bound_exceptional_pairs.3},
                    $\frac{\xi}{\epsilon^\ell} \leq \frac{\xi}{\epsilon^{k_{**}}} < \frac{1}{2}$, so $0 < \frac{\xi}{\epsilon^\ell} < \frac{1}{2}$.
                \item For all $\ell \in \parcurly{0,\dots,k_{**}}$, $m_\ell$ is sufficiently large:
                    \[
                        m_\ell \geq g_\epsilon^{-\ell}(n) \geq g_\epsilon^{-k_{**}}(n) > n\parround{\epsilon, \frac{1}{k},
                        \frac{\xi}{\epsilon^{k_{**}}}, c} > n\parround{\epsilon, \zeta, \frac{\xi}{\epsilon^{\ell}}, c}
                    \]
                \item $c > \frac{1}{\frac{1}{k} (1 - \frac{\xi}{\epsilon^{k_{**}}} - \epsilon)}
                    = \frac{1}{\zeta (1 - \frac{\xi}{\epsilon^{k_{**}}} - \epsilon)}$, by \dref{itm:many_values_to_equitative_partition_with_bound_exceptional_pairs.4}.
            \end{itemize}
            By \Cref{lem:n_large_enough_valid_values} then, $\oplus\parsquared{m_\ell, \epsilon, \zeta, \frac{\xi}{\epsilon^\ell}}$
            (as in \Cref{def:n_large_enough_property}) holds, and we can take
            $A_{(\ref{def:n_large_enough_property})}$ and $P_{(\ref{def:n_large_enough_property})}$ as $A_1$ and
            $P \coloneqq \bparcurly{\overline{B}_{A_1,b} \mid b \in G}$ respectively, which then satisfy the conditions:
            \begin{itemize}
                \item $|A_1| = m_\ell$.
                \item $|P| \leq m_\ell^k = m_\ell^{\frac{1}{\zeta}}$, where first inequality follows \dref{itm:k_order_propery_bounds_BAbs.2}
                    of \Cref{cor:k_order_propery_bounds_BAbs}.
                \item $\forall B \in P$, $|B| \leq |A_1|^\epsilon$ by $\epsilon$-indivisibility of $A_1$.
            \end{itemize}
            Thus, by \Cref{def:n_large_enough_property} we have that there exists $Z \subseteq A_1$ such that:
            \begin{itemize}
                \item $|Z| = \lfloor m_\ell^{\frac{\xi}{\epsilon^\ell}} \rfloor = \lfloor n^{\epsilon^\ell \frac{\xi}{\epsilon^\ell}} \rfloor
                    = \lfloor n^\xi \rfloor$ satisfying \dref{itm:many_values_to_equitative_partition_with_bound_exceptional_pairs.a}.
                \item $Z$ is $c$-indivisible since $|B \cap Z| \leq c$ for all $B \in P$,
                    satisfying \dref{itm:many_values_to_equitative_partition_with_bound_exceptional_pairs.b}.
            \end{itemize}
            This proves the statement.
        \end{proof}
    \end{lemma}

    We know use the previous results to build an equitable partition.
    The strategy is to use \Cref{lem:existance_of_indivisible_sets} to find an $\epsilon$-indivisible set of vertices in
    the graph, to which \Cref{lem:many_values_to_equitative_partition_with_bound_exceptional_pairs} is used to obtain
    a subset which is $f_c$-indivisible.
    Since $f_c$-indivisibility is transitive to taking subsets, we can take a smaller subset with the same property.
    We move aside this as the first part of our partition, and repeat the strategy on the reminder.
    This process finishes when the reminder is too small to apply \Cref{lem:existance_of_indivisible_sets}.

    \begin{theorem}[Theorem 4.23] \label{thm:equitative_partition_high_regularity_parts_grow_with_n}
        Let $G$ be a graph with the non-$k_*$-property.
        For any $c \in \mathbb{N}$, $\epsilon, \xi \in \mathbb{R}$ satisfying the hypothesis of \Cref{lem:many_values_to_equitative_partition_with_bound_exceptional_pairs}
        (with $k = k_*$ and $\zeta = \frac{1}{k_*}$), any $\theta \in (0,1)$ and $A \subseteq G$ large enough
        $\parstraight{A} = n > N\parround{c, \epsilon, \zeta, \xi, \theta}$,
        there is a partition $\overline{A} = \Partriangle{A_i \mid i \in \parcurly{1, \dots, i(*)}}$ of $A$ with remainder
        $B = A \setminus \bigcup_{i \in \parcurly{1, \dots, i(*)}} A_i$ satisfying:
        \begin{itemize}
            \item $|A_i| = \lfloor \lfloor n^\theta \rfloor ^\zeta \rfloor$ for all $i \in \parcurly{1, \dots, i(*)}$.
            \item $A_i$ is $c$-indivisible for all $i \in \parcurly{1, \dots, i(*)}$ where $c$ is the constant function $f(x) = c$.
            \item $|B| < m_0 \approx n^{\frac{\theta}{\epsilon^{k_{**}}}}$. \todo{$m_0$ cannot be the bound as it is not defined in the statement. There is no clear
                to write the real bound in a clear way without error or absurdly worse bounds. I think the better
                solution is to force $\epsilon$ (and possibly other parameters) to be a fraction $\frac{1}{r}$.
                If so, make a remark saying that this condition is not necessary but makes bounds cleaner.}
        \end{itemize}
        \begin{proof}
            Let $n > N\parround{c, \epsilon, \zeta, \xi, \theta} \coloneqq \bparround{g_\epsilon^{k_{**}}\bparround{N_{\ref{lem:n_large_enough_valid_values}}
                \parround{ \epsilon, \frac{1}{k_*}, \frac{\xi}{\epsilon^{k_{**}}}, c}} + 1 }^{\frac{1}{\theta}}$,
            so that $\lfloor n^\theta \rfloor$ satisfies the large enough condition of
            \Cref{lem:many_values_to_equitative_partition_with_bound_exceptional_pairs}:
            \[
                \lfloor n^\theta \rfloor
                    > g_\epsilon^{k_{**}}\bparround{N_{\ref{lem:n_large_enough_valid_values}}
                        \parround{\epsilon, \frac{1}{k_*}, \frac{\xi}{\epsilon^{k_{**}}}, c}}
            \]
            Now, we define a decreasing sequence $m_0 > m_1 > \dots > m_{k_{**}}$ with $m_{k_{**}} = \lfloor n^\theta \rfloor$
            and $m_{\ell} = \lceil \parround{m_{\ell+1}}^{\frac{1}{\epsilon}} \rceil$ for all $\ell \in \parcurly{0, \dots, k_{**}-1}$.
            This sequence satisfies the condition of \Cref{lem:existance_of_indivisible_sets} for $f(n) = n^\epsilon$.
            We will build a sequence of disjoint $c$-indivisible subsets $A_i$ by induction on $i$ as follows.
            Let $R_i = A \setminus \bigcup_{j<i} A_j$ (so $R_1 = A$).
            If $R_i < m_0$, then
            $\overline{A} = \Partriangle{A_j \mid j < i = i(*)}$ and $B = R_i$, and we are done.
            Otherwise, we can apply \Cref{lem:existance_of_indivisible_sets} to $R_i$ with the sequence
            $\Partriangle{m_\ell}_{\ell \leq k_{**}}$, to obtain an $\epsilon$-indivisible subset $B_i \subseteq R_i$ of
            size $m_{k_{**}-\ell}$. \todo{WHY IN THE HELL DO WE NEED TO HAVE EPSILON_INDIVISIBLE SETS??}
            Then, since $|B_i| = m_{k_{**}-\ell} \geq m_{k_{**}} = \lfloor n^\theta \rfloor$ by the $n$-large-enough assumption,
            we can apply \Cref{lem:many_values_to_equitative_partition_with_bound_exceptional_pairs} and get a
            $c$-indivisible subset $Z_i$ of size $|Z_i| = \lfloor m_{k_{**}-\ell}^\zeta \rfloor
            \geq \lfloor \lfloor n^{\frac{\theta}{\epsilon^\ell}} \rfloor ^\zeta \rfloor
            \geq \lfloor \lfloor n^{\theta} \rfloor ^\zeta \rfloor$.
            Since $c$-indivisibility is preserved when taking subsets,
            we can choose $A_i \subseteq Z_i$ to be a $c$-indivisible subset of size $\lfloor \lfloor n^{\theta} \rfloor ^\zeta \rfloor$.
        \end{proof}
    \end{theorem}

    \todo{Make some remark on the fact that $\theta$ needs to be smaller then $\epsilon^{k_{**}}$ for this to make sense.}

    \begin{remark}
        Some notes on the partition obtained in the previous theorem:
        \begin{itemize}
            \item For the theorem to have any value $\theta$ needs to be smaller then $\epsilon^{k_{**}}$, otherwise the
                reminder may be as large as the whole graph.
            \item \jojo
        \end{itemize}
    \end{remark}

    % up to here it was compiling with no problem.
