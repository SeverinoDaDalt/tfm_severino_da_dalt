\section{Unbounded stable regularity lemmas} \label{sec:section_4}

    This section is centered around the concept of \emph{$\epsilon$-indivisible} sets, a strong condition on bounding the
    non-homogeneous behaviour of a subset respect to all the vertices of the graph.
    This condition results in pairs of sufficiently large subsets of vertices satisfying
    the \emph{average condition}, which (asymmetrically) strictly bounds the number of exceptional edges in the pair.
    Using these tools we obtain the first result in \Cref{lem:existance_of_ordered_f_indivisible_partitions_with_exceptions_bound},
    which proves the existence of a partition where \emph{all} pairs are highly close to full or empty, at the cost of a
    uneven partition.
    Next, we improve the results obtaining an even partition in
    \Cref{thm:existance_of_equitative_partition_with_perfect_pairs_but_with_bound_exceptional_pairs}, but this time with a
    small number of exceptional pairs, and a tradeoff between a non-negligible remainder set and parts of even smaller size.
    The final result, \Cref{thm:equitative_partition_high_regularity_parts_grow_with_n}, achieves removing exceptional pairs
    and reduces the size of the remainder set.
    All in all, even though the partitions of this section present a very strong form of \regularity, they all share
    the same drawback: a large number of parts that grows with the size of the graph, something that we will be dealing
    with in \Cref{sec:section_5}.

    \subsection{Indivisibility and the average condition} \label{subsec:subsection_4.1}

        Let's first define \emph{indivisibility}.
        The general definition is dependent on a certain function $f$, but we shall be mostly
        interested in the case of $f(n) = n^\epsilon$, which we call $\epsilon$-indivisible, and in the constant case $f(n) = c$.

        \begin{definition}[Definition 4.2 (2) in~\cite{regularity_lemmas_for_stable_graphs}] \label{def:f_indivisible}
            Let $f: \mathbb{R} \longrightarrow \mathbb{R}$ be a function.
            We say that $A \subseteq G$ is \emph{$f$-indivisible} if for every $b \in G$,
            \[
                \parstraight{\overline{B}_{A,b}} < f(|A|).
            \]
        \end{definition}

        \begin{definition}[Definition 4.2 (1) in~\cite{regularity_lemmas_for_stable_graphs}] \label{def:epsilon_indivisible}
            Let $\epsilon \in (0,1)$.
            We say that $A \subseteq G$ is \emph{$\epsilon$-indivisible} if for every $b \in G$,
            \[
                \parstraight{\overline{B}_{A,b}} < |A|^{\epsilon}.
            \]
        \end{definition}

        As mentioned before, an $\epsilon$-indivisible set is $f$-indivisible for $f(n) = n^\epsilon$.

        Thus, indivisibility upper bounds the number of exceptional edges with respect to each vertex.
        A natural follow-up question is how two indivisible sets interact between each other.
        The following lemma tackles precisely that, although doing so asymmetrically.

        \begin{lemma}[Claim 4.6)] \label{lem:average_condition_statement}
            Let $G$ be a finite graph.
            Suppose $A, B \subseteq G$ such that $A$ is $f$-indivisible, $B$ is $g$-indivisible, and $\ceil{f(|A|)} g(|B|) < \frac{1}{2} |B|$.
            Then, the truth value $t = t(A,B)$ satisfies the following.
            For all but $< f(|A|)$ of the $a \in A$ we have that for all but $< g(|B|)$ of the $b \in B$ satisfy $a R b \equiv t$.
            \begin{proof}
                Since $B$ is $g$-indivisible, for each $a \in A$ we have that $\parstraight{\overline{B}_{B,a}} < g(|B|)$.
                Let $U_i = \parcurly{a \in A \mid t(a,B) \equiv i}$ for $i \in \parcurly{0,1}$.
                If either $U_i$ satisfies $|U_i| < f(|A|)$ then the statement is true.
                Suppose not.
                Then, $|U_i| \geq f(|A|)$ and we can take $W_i \subseteq U_i$ with $|W_i| = \ceil{f(|A|)}$ for $i \in \parcurly{0,1}$.
                Now, let $V = \parcurly{b \in B \mid (\exists a \in W_0 \mid a R b) \vee (\exists a \in W_1 \mid \lnot a R b)}$,
                i.e. the $b$'s which are an exception for some $a \in W_0 \cup W_1$.
                Then, $|V| < |W_0| g(|B|) + |W_1| g(|B|) = 2 \ceil{f(|A|)} g(|B|) < |B|$, where the first inequality follows the
                $g$-indivisibility of $B$.
                Since $|V| < |B|$, there is a $b_* \in B \setminus V$ such that $\forall a \in W_0$ $\lnot a R b_*$ and
                $\forall a \in W_1$ $a R b_*$ with $|W_0| = |W_1| = \ceil{f(|A|)}$, which contradicts the $f$-indivisibility of $A$.
            \end{proof}
        \end{lemma}

        \begin{definition}
            We say that the pair $(A,B)$ with $A$ $f$-indivisible and $B$ $g$-indivisible satisfies the \emph{average condition} if
            $\ceil{f(|A|)} g(|B|) < \frac{1}{2} |B|$ and thus the statement of \Cref{lem:average_condition_statement} is true for
            the pair $(A,B)$.
        \end{definition}

        \begin{remark}
            The condition $\ceil{f(|A|)} g(|B|) < \frac{1}{2} |B|$ makes ordering of the pair $(A,B)$ matter, that is,
            \[
                (A,B) \text{ has the average condition } \not\Rightarrow (B,A) \text{ has the average condition. }
            \]
        \end{remark}

        Next, we are interested in studying how the average condition of an indivisible pair bounds the number of
        exceptional edges of large subpairs.
        We study the $f$-indivisible and $\epsilon$-indivisible case separately, as the specific case of $\epsilon$-indivisibility gives
        a slightly better condition on the range of the size of the subpair.
        In any case, the two proofs are very similar, so we prove the $\epsilon$-indivisible case here and the
        $f$-indivisible case in \Cref{sec:appendix_other_proofs}.

        \begin{lemma}[Claim 4.8 in~\cite{regularity_lemmas_for_stable_graphs}] \label{lem:exceptions_bound_of_epsilon_indivisible_sets}
            Let $A$ be $\epsilon$-indivisible, $B$ $\zeta$-indivisible and let the pair $(A,B)$ satisfy the average condition.
            Then, for all $\epsilon_1 \in \parround{0, 1-\epsilon}$, $\zeta_1 \in \parround{0, 1-\zeta}$, $A' \subseteq A$
                and $B' \subseteq B$ such that $|A'| \geq |A|^{\epsilon + \epsilon_1}$ and $|B'| \geq |B|^{\zeta + \zeta_1}$,
                we have that:
            \[
                \frac{|\parcurly{(a,b) \in (A',B') \mid a R b \equiv \neg t(A,B)}|}{|A' \times B'|} \leq
                    \frac{1}{|A|^{\epsilon_1}} + \frac{1}{|B|^{\zeta_1}}.
            \]
            \begin{proof}
                Notice that, by the average condition of the pair $(A,B)$:
                \begin{itemize}
                    \item there are at most $|A|^\epsilon$ vertices of $A$ (hence in $A' \subseteq A$), say $S$, which are exceptional
                        with respect to $B$, so the number of edges $(a,b) \in S \times B'$ which are exceptional is at most
                        $|S| \cdot |B'|$, and
                    \item for each $a \in A$ (hence in $A' \subseteq A$) not in $S$, there are at most $|B|^\zeta$ elements
                        $b \in B$ such that $(a,b)$ does not satisfy the truth value $t(A,B)$, i.e. that are exceptional.
                        Thus, we have at most $(a,b) \in (A' \setminus S) \times B'$ is at most $(|A'| - |S|) |B|^\zeta$.
                \end{itemize}
                The overall worse case in this scenario is when $S$ is maximum ($|S| = |A|^\epsilon$), and thus we have at most
                $|A|^\epsilon |B'| + (|A'| - |A|^\epsilon) |B|^\zeta$ exceptional edges in $A' \times B'$, as $|B'| \geq |B|^\zeta$.
                Putting it all together:
                \[
                    \begin{split}
                        \frac{|\parcurly{(a,b) \in (A',B') \mid a R b \equiv \neg t(A,B)}|}{|A' \times B'|}
                            &\leq \frac{|A|^\epsilon |B'| + (|A'| - |A|^\epsilon) |B|^\zeta}{|A'| |B'|} \\
                            &= \frac{|A|^\epsilon}{|A'|} + \frac{|A'| - |A|^\epsilon}{|A'|} \frac{|B|^\zeta}{|B'|} \\
                            &\leq \frac{|A|^\epsilon}{|A'|} + \frac{|B|^\zeta}{|B'|} \\
                            &\leq \frac{|A|^\epsilon}{|A|^{\epsilon + \epsilon_1}} + \frac{|B|^\zeta}{|B|^{\zeta + \zeta_1}} \\
                            &= \frac{1}{|A|^{\epsilon_1}} + \frac{1}{|B|^{\zeta_1}}.
                    \end{split}
                \]
                This finishes the proof.
            \end{proof}
        \end{lemma}

        \begin{lemma}[$f$-indivisible version] \label{lem:exceptions_bound_of_f_indivisible_sets}
            Let $A$ be $f$-indivisible, $B$ $g$-indivisible and let the pair $(A,B)$ satisfy the average condition.
            Then, for all $\epsilon_1 \in \parround{0, 1-\frac{f(|A|)}{|A|}}$, $\zeta_1 \in \parround{0, 1-\frac{g(|B|)}{|B|}}$, $A' \subseteq A$
                and $B' \subseteq B$ such that $|A'| \geq f(|A|) |A|^{\epsilon_1}$ and $|B'| \geq g(|B|) |B|^{\zeta_1}$,
                we have that:
            \[
                \frac{|\parcurly{(a,b) \in (A',B') \mid a R b \equiv \neg t(A,B)}|}{|A' \times B'|} \leq
                    \frac{1}{|A|^{\epsilon_1}} + \frac{1}{|B|^{\zeta_1}}.
            \]
            \begin{proof}[Proof in \Cref{sec:appendix_other_proofs}]
            \end{proof}
        \end{lemma}

        For later use, we are particularly interested in the case when $f(n) = c$.

        \begin{corollary}[Corollary 4.9 in~\cite{regularity_lemmas_for_stable_graphs}] \label{cor:exceptions_bound_of_c_indivisible_sets}
            Let $A$ and $B$ be $f$-indivisible with $f(n) = c$ and $(A,B)$ satisfy the average condition.
            Then, for all $\epsilon_1 \in (0, 1 - \frac{c}{|A|})$, $\zeta_1 \in (0, 1 - \frac{c}{|B|})$, $A' \subseteq A$ and
            $B' \subseteq B$ with $|A'| \geq c |A|^{\epsilon_1}$ and $|B'| \geq c |B|^{\zeta_1}$, we have:
            \[
                \frac{|\parcurly{(a,b) \in (A',B') \mid a R b \equiv \neg t(A,B)}|}{|A' \times B'|} \leq
                    \frac{1}{|A|^{\epsilon_1}} + \frac{1}{|B|^{\zeta_1}}.
            \]
            \begin{proof}
                Use \Cref{lem:exceptions_bound_of_f_indivisible_sets} with $f(n) = c$.
            \end{proof}
        \end{corollary}

        \begin{remark} \label{rmk:sufficient_requirement_for_average_condition}
            Notice that the average condition is easily satisfied if the pair satisfies a condition on the size of its sets.
            Namely, if $f(n) = n^\epsilon$, $A$ and $B$ are $f$-indivisible, and $\parstraight{B} \geq \parstraight{A} \geq m$,
            then $m^{1-2\epsilon} > 4$ is sufficient for the average condition to hold for the pair $(A,B)$:
            \[
                \frac{\ceil{|A|^\epsilon} |B|^\epsilon}{|B|}
                    \leq \frac{2|B|^{2\epsilon}}{|B|}
                    = \frac{2}{|B|^{1-2\epsilon}}
                    = \frac{2}{m^{1-2\epsilon}}
                    < \frac{1}{2}.
            \]
            We will be using this fact in the context of a sequence of non-zero natural numbers
            $\parcurly{m_\ell \mid \ell \in \parcurly{0, \dots, k_{**}}}$ where $\floor{m_\ell^\epsilon} = m_{\ell+1}$
            for some $\epsilon \in (0, \frac{1}{2})$ and for all $\ell \in \parcurly{0, \dots, k_{**}-1}$.
            Here, $4 < (m_{k_{**}-1})^{1-2\epsilon}$ is sufficient for any $f$-indivisible $A$ and $B$, with
            $|A|, |B| \in \parcurly{m_0, \dots, m_{k_{**}-1}}$, to satisfy the average condition.
        \end{remark}

        Now that we have proven some properties of indivisible sets, we are actually interested in whether they can be
        found in a graph.
        It turns out that the non-$k$-order property, or more specifically the associated tree bound, is sufficient for
        proving it.
        The proof resumes in assuming that there is no indivisible set to recursively refine a partition of a subset,
        which by construction must contain a $k_{**}$-tree, and gives a contradiction with the non-$k_*$-property.

        \begin{lemma}[Claim 4.3 in~\cite{regularity_lemmas_for_stable_graphs}] \label{lem:existance_of_indivisible_sets}
            Let $G$ be a finite graph with the non-$k_*$-property and $f: \mathbb{R} \longrightarrow \mathbb{R}$ a function
            such that $x \geq f(x)$.
            Let $\parcurly{m_\ell \mid \ell \in \parcurly{0, \dots, k_{**}}}$ be a sequence of non-zero natural numbers such that
            for all $\ell \in \parcurly{0, \dots, k_{**}-1}$, $f(m_{\ell}) \geq m_{\ell+1}$.
            If $A \subseteq G$, $|A| \geq m_0$, then for some $\ell \in \parcurly{0, \dots, k_{**}-1}$ there is a subset $B \subseteq A$
            of size $m_\ell$ which is $f$-indivisible.
            \begin{proof}
                Suppose that such subset $B$ does not exist.
                Then we shall construct by induction on $k = |\eta|$ the sequences $\Partriangle{b_\eta \mid \eta \in \parcurly{0,1}^{<k}}$
                and $\Partriangle{A_\eta \mid \eta \in \parcurly{0,1}^{\leq k}}$, satisfying the following properties:
                \begin{enumerate}[label={\Roman*}., ref={\Roman*}, font=\rmfamily]
                    % \item\label{itm:existance_of_indivisible_sets.1} $A_{\eta^\frown \Partriangle{i}} \subseteq A_{\eta}$, $\forall i \in \parcurly{0,1}$, $\forall k \in \parcurly{0, \dots, k_{**}-1}$.
                    % \item\label{itm:existance_of_indivisible_sets.2} $A_{\eta^\frown \Partriangle{0}} \cap A_{\eta^\frown \Partriangle{1}} = \emptyset$, $\forall k \in \parcurly{0, \dots, k_{**}-1}$.
                    \item\label{itm:existance_of_indivisible_sets.3} $|A_\eta| = m_k$, $\forall k \in \parcurly{0, \dots, k_{**}}$.
                    \item\label{itm:existance_of_indivisible_sets.4} $b_\eta \in G$ witnessing that $A_\eta$ is not $f$-indivisible, $\forall k \in \parcurly{0, \dots, k_{**}-1}$.
                    \item\label{itm:existance_of_indivisible_sets.5} $A_{\eta^\frown \Partriangle{i}} \subseteq A_\eta^{(i)} = \parcurly{a \in A_\eta \mid a R b_\eta \equiv i}$,
                        $\forall \in \parcurly{0,1}$, $\forall k \in \parcurly{0, \dots, k_{**}-1}$.
                \end{enumerate}
                Let's prove the induction.
                For $k=0$, consider any $A_{\Partriangle{\cdot}} \subseteq A$, satisfying $|A_{\Partriangle{\cdot}}| = m_0$, and
                any $b_{\Partriangle{\cdot}}$ witnessing the non-$f$-indivisibility of $A_{\Partriangle{\cdot}}$.
                For $k > 0$ we can assume by hypothesis that $A_\eta$, with $|A_\eta| = m_{k}$, is not $f$-indivisible.
                Thus, there exists a $b_\eta$ such that $A_\eta^{(i)} \geq f(m_{k}) \geq m_{k+1}$ (\dref{itm:existance_of_indivisible_sets.4}), and we can choose
                $A_{\eta^\frown \Partriangle{i}} \subseteq A_\eta^{(i)}$ (\dref{itm:existance_of_indivisible_sets.5}), such that
                $|A_{\eta^\frown \Partriangle{i}}| = m_{k+1}$ $\forall i \in \parcurly{0,1}$ (\dref{itm:existance_of_indivisible_sets.3}).
                % \dref{itm:existance_of_indivisible_sets.1} and \dref{itm:existance_of_indivisible_sets.2} are satisfied \dref{itm:existance_of_indivisible_sets.5}.
                Now, for all $\eta$ such that $|\eta| = k_{**}$, consider some element $a_\eta \in A_\eta$, which exists since $m_\ell > 0$
                for all $\ell$.
                Then, we have two sequences $\Partriangle{b_\eta \mid \eta \in \parcurly{0,1}^{<k_{**}}}$ and $\Partriangle{a_\eta \mid \eta \in \parcurly{0,1}^{k_{**}}}$
                are a $k_{**}$-tree in $G$: for all $\rho \in \parcurly{0,1}^{<k_{**}}$ and $\eta \in \parcurly{0,1}^{k_{**}}$
                if given $\ell \in \parcurly{0, 1}$ we have $\rho^\frown \Partriangle{\ell} \trianglelefteq \eta$ then
                $a_\eta R b_\rho \equiv \ell$ since $a_\eta \in A_\eta \subseteq A_{\rho ^\frown \Partriangle{\ell}}$ by \dref{itm:existance_of_indivisible_sets.5}.
                This contradicts the tree bound $k_{**}$ (see \Cref{thm:tree_implies_order}).
            \end{proof}
        \end{lemma}

        The previous proof can be iteratively used to partition the graph into indivisible parts, with a small reminder.
        As the average condition cares about the ordering of the elements of the pair, we define the partition as a tuple
        instead of a family of sets, and fix an ascending order on the size of the parts.

        \begin{lemma}[Claim 4.4 and 4.5 in~\cite{regularity_lemmas_for_stable_graphs}] \label{lem:existance_of_ordered_f_indivisible_partitions}
            Let $G$ be a finite graph with the non-$k_{*}$-order property and $f: \mathbb{R} \longrightarrow \mathbb{R}$ a function
            such that $x \geq f(x)$.
            Let $\parcurly{m_\ell \mid \ell \in \parcurly{0, \dots, k_{**}}}$ be a sequence of non-zero natural numbers such that
            for all $\ell \in \parcurly{0, \dots, k_{**}-1}$, $f(m_{\ell}) \geq m_{\ell+1}$.
            If $A \subseteq G$ with $|A| = n$, then we can find a sequence $\overline{A} = \Partriangle{A_j \mid j \in \parcurly{1, \dots, j(*)}}$
            and reminder $B = A \setminus \bigcup_{A_i \in \overline{A}} A_i$ such that:
            \begin{enumerate}[label={\Roman*}., ref={\Roman*}, font=\rmfamily]
                \item \label{itm:existance_of_ordered_f_indivisible_partitions.1} For each $j \in \parcurly{1, \dots, j(*)}$, $A_j$ is $f$-indivisible.
                \item \label{itm:existance_of_ordered_f_indivisible_partitions.2} For each $j \in \parcurly{1, \dots, j(*)}$, $|A_j| \in \parcurly{m_0, \dots, m_{k_{**}-1}}$.
                \item \label{itm:existance_of_ordered_f_indivisible_partitions.3} $A_j \subseteq A \setminus \bigcup\parcurly{A_i \mid i < j}$, in particular $A_i \cap A_j = \emptyset$ $\forall i \neq j$.
                \item \label{itm:existance_of_ordered_f_indivisible_partitions.4} $|B| < m_0$.
                \item \label{itm:existance_of_ordered_f_indivisible_partitions.5} $|A_i| \leq |A_j| \Leftrightarrow i \leq j$.
            \end{enumerate}
            \begin{proof}
                Iteratively, apply \Cref{lem:existance_of_indivisible_sets} to the remainder $A \setminus \bigcup \parcurly{A_i \mid i < j}$
                (\dref{itm:existance_of_ordered_f_indivisible_partitions.3}) to get an $f$-indivisible $A_j$ (\dref{itm:existance_of_ordered_f_indivisible_partitions.1}) of size $m_\ell$, $\ell \in \parcurly{0, \dots, k_{**}-1}$
                (\dref{itm:existance_of_ordered_f_indivisible_partitions.2}) until less than $m_0$ vertices are available (\dref{itm:existance_of_ordered_f_indivisible_partitions.4}).
                To conclude, reorder the indices of the $A_j$'s in ascending size order (\dref{itm:existance_of_ordered_f_indivisible_partitions.5}).
            \end{proof}
        \end{lemma}

        Finally, we ensure the pairs satisfy the average condition by simply requiring a minimal size of the parts
        using \Cref{rmk:sufficient_requirement_for_average_condition},
        which can be easily integrated in the definition of the sequence of integers.

        \begin{lemma}[Claim 4.10 in~\cite{regularity_lemmas_for_stable_graphs}] \label{lem:existance_of_ordered_f_indivisible_partitions_with_exceptions_bound}
            Let $G$ be a finite graph with the non-$k_{*}$-order property.
            Let $\parcurly{m_\ell \mid \ell \in \parcurly{0, \dots, k_{**}}}$ be a sequence of non-zero natural numbers such that
            $n \geq m_0$ and for all $\ell \in \parcurly{0, \dots, k_{**}-1}$, $\floor{m_\ell^\epsilon} = m_{\ell+1}$,
            for some $\epsilon \in (0, \frac{1}{2})$ such that $4 < (m_{k_{**}-1})^{1-2\epsilon}$.
            If $A \subseteq G$ with $|A| = n$, then we can find a sequence $\overline{A} = \Partriangle{A_i \mid i \in \parcurly{1, \dots, i(*)}}$
            and reminder $B = A \setminus \bigcup_{A_i \in \overline{A}} A_i$ satisfying:
            \begin{enumerate}[label={\Roman*}., ref={\Roman*}, font=\rmfamily]
                \item \label{itm:existance_of_ordered_f_indivisible_partitions_with_exceptions_bound.1} For each $i \in \parcurly{1, \dots, i(*)}$, $A_i$ is $\epsilon$-indivisible.
                \item \label{itm:existance_of_ordered_f_indivisible_partitions_with_exceptions_bound.2} For each $i \in \parcurly{1, \dots, i(*)}$, $|A_i| \in \parcurly{m_0, \dots, m_{k_{**}-1}}$.
                \item \label{itm:existance_of_ordered_f_indivisible_partitions_with_exceptions_bound.3} $A_i \cap A_j = \emptyset$ for all $i \neq j$.
                \item \label{itm:existance_of_ordered_f_indivisible_partitions_with_exceptions_bound.4} $|B| < m_0$.
                \item \label{itm:existance_of_ordered_f_indivisible_partitions_with_exceptions_bound.5} $\overline{A}$ is $\leq$-increasing.
                \item \label{itm:existance_of_ordered_f_indivisible_partitions_with_exceptions_bound.6} If $\zeta \in \parround{0,\epsilon^{k_{**}}}$ then for every $i,j \in \parcurly{1, \dots, i(*)}$ with $i < j$,
                    $A \subseteq A_i$ ad $B \subseteq A_j$ such that $|A| \geq |A_i|^{\epsilon + \zeta}$ and $|B| \geq |A_j|^{\epsilon + \zeta}$
                    we have that:
                    \[
                        \begin{split}
                            \frac{|\parcurly{(a,b) \in (A,B) \mid a R b \equiv \neg t(A_i,A_j)}|}{|A \times B|}
                                &\leq \frac{1}{|A_i|^\zeta} + \frac{1}{|A_j|^\zeta} \\
                                &\leq \frac{1}{|A|^\zeta} + \frac{1}{|B|^\zeta}.
                        \end{split}
                    \]
            \end{enumerate}
            \begin{proof}
                The five points are direct consequence of \Cref{lem:existance_of_ordered_f_indivisible_partitions},
                setting $f(x) = x^\epsilon$.
                Now, by \dref{itm:existance_of_ordered_f_indivisible_partitions_with_exceptions_bound.2}, for any $A_i, A_j \in \overline{A}$ with $i < j$
                there is some $\ell \in \parcurly{0, \dots, k_{**}-1}$ such that $|A_i| \leq |A_j| = m_\ell$.
                Also, it follows the condition $4 < (m_{k_{**}-1})^{1-2\epsilon}$ and \Cref{rmk:sufficient_requirement_for_average_condition}
                that the pair $(A_i,A_j)$ satisfies the average condition.
                Finally, notice that $\epsilon^{k_{**}} < \epsilon < 1 - \epsilon$ since $\epsilon \in (0, \frac{1}{2})$,
                so that $\zeta \in (0, \epsilon ^ {k_{**}}) \subseteq (0, 1 - \epsilon)$ and the condition for
                \Cref{lem:exceptions_bound_of_epsilon_indivisible_sets} is satisfied.
                This gives us \dref{itm:existance_of_ordered_f_indivisible_partitions_with_exceptions_bound.6},
                where in last inequality we used that $A_i\subseteq A, A_j\subseteq B$.
            \end{proof}
        \end{lemma}

        \begin{remark}
            For sufficiently small $\epsilon$, the condition $4 < (m_{k_{**}-1})^{1-2\epsilon}$ is rather mild.
            For example, if $\epsilon < \frac{1}{4}$, then we are just requiring that $m_{k_{**}-1} \geq 16$.
        \end{remark}

    \subsection{$\epsilon$-indivisible even partition} \label{subsec:subsection_4.2}

        As stated earlier, the principal drawback of the previous result is that the obtained partition is not even.
        To deal with this, we study the event of randomly partitioning a pair of indivisible sets
        into subparts of equal size.
        In particular, we prove that the event of a pair of subparts of the refinement being either \emph{completely} connected or
        \emph{completely} empty, is satisfied with very high probability.

        \begin{definition}
            Let $A, B$ be $f$-indivisible sets with $\ceil{f(A)} f(B) < \frac{1}{2} |B|$.
            Let $\Partriangle{A_i \mid i \in \parcurly{1, \dots, i_A}}$ be a partition of $A$ with $|A_i| = m$ for all
            $i \in \parcurly{1, \dots, i_A}$ and $\Partriangle{B_j \mid j \in \parcurly{1, \dots, j_B}}$ be a partition of
            $B$ with $|B_j| = m$ for all $j \in \parcurly{1, \dots, j_B}$.
            We define $\varepsilon^+_{A_i,B_j,m}$ as the event:
            \[
                \forall a \in A_i \ \forall b \in B_j, a R b = t(A,B).
            \]
        \end{definition}

        Notice that, in the previous definition, we are implicitly requiring that $|A| = i_A \cdot m$ and $|B| = j_B \cdot m$.

        \begin{lemma}[Claim 4.13 in~\cite{regularity_lemmas_for_stable_graphs}] \label{lem:bound_on_the_probability_of_a_subpair_having_no_exceptions}
            Let $G$ be a finite graph with the non-$k_{*}$-order property.
            \sloppy Let $\Partriangle{m_\ell \mid \ell \in \parcurly{0, \dots, k_{**}}}$ be a sequence of non-zero natural numbers such that
            $n \geq m_0 \geq n^\epsilon$ and for all $\ell \in \parcurly{0, \dots, k_{**}-1}$,
            $m_\ell^\epsilon = m_{\ell+1}$,\footnote{
                Implicitly, we are requiring each $m_\ell^\epsilon$ to be a natural number.
                This can be easily achieved by fixing $\epsilon$ to be a fraction $\frac{1}{r}$ for some natural number
                $r$, and constructing the sequence starting from $m_{k_{**}}$ as $m_\ell = m_{\ell+1}^r$ for all
                $\ell \in \parcurly{0, \dots, k_{**}-1}$.
                This is precisely the strategy we will use in following results.}
            for some $\epsilon \in (0, \frac{1}{2})$ such that $4 < (m_{k_{**}-1})^{1-2\epsilon}$.
            Let $A_1, A_2 \subseteq G$ be two $\epsilon$-indivisible subsets such that $|A_1| = m_{\ell_1}$ and $|A_2| = m_{\ell_2}$
            for some $\ell_1, \ell_2 \in \parcurly{0, \dots, k_{**}-1}$ and $|A_1| \leq |A_2|$.
            Let $c \in (0, 1-\epsilon)$ and $m \leq n^{\frac{1 - \epsilon - c}{3}\epsilon^{k_{**}}}$ such that $m$ divides $|A_1|$ and $|A_2|$.
            Then, let $\Partriangle{A_{1,s} \mid s \in \bparcurly{1, \dots, \frac{|A_1|}{m}}}$ and
            $\Partriangle{A_{2,t} \mid t \in \bparcurly{1, \dots, \frac{|A_2|}{m}}}$ be random partitions of $A_1$ and $A_2$
            respectively, with pieces of size $m$.
            We have that
            \[
                \mathbb{P}(\varepsilon^+_{A_{1,s},A_{2,t},m}) \geq 1 - \frac{2}{n^{c\epsilon^{k_{**}}}}.
            \]
            \begin{proof}
                Fix $s \in \frac{|A_1|}{m}$, $t \in \frac{|A_2|}{m}$.
                It follows from the condition $4 < (m_{k_{**}-1})^{1-2\epsilon}$ and \Cref{rmk:sufficient_requirement_for_average_condition}
                that the pair $(A_{1}, A_{2})$ satisfies the average condition.
                Let $U_1 = \bparcurly{a \in A_1 \mid \bparstraight{\parcurly{b \in A_2 \mid a R b \equiv \neg t(A_1, A_2)}} \geq |A_2|^\epsilon}$
                and for each $a \in A_1 \setminus U_1$ let $U_{2,a} = \parcurly{b \in A_j \mid a R b \equiv \neg t(A_1, A_2)}$.
                By \Cref{lem:average_condition_statement}, $|U_1| \leq |A_1|^\epsilon$ and $\forall a \in A_1 \setminus U_1$,
                $|U_{2,a}| \leq |A_2|^\epsilon$.
                Now, given $s \in \bparcurly{1, \dots, \frac{|A_1|}{m}}$, we can bound the probability $P_1$ that
                $A_{1,s} \cap U_1 \neq \emptyset$ as follows:
                \[
                    \begin{split}
                        P_1
                            & \leq \frac{|U_1|}{|A_1|} + \dots + \frac{|U_1|}{|A_1|-m+1}
                                < \frac{m |U_1|}{|A_1| - m}
                                \leq \frac{m |A_1|^\epsilon}{|A_1| - m} \\
                            & \leq \frac{m^2 |A_1|^\epsilon}{|A_1|}
                                = \frac{m^2}{|A_1|^{1-\epsilon}}
                                = \frac{m^2}{m_0^{\parround{1-\epsilon} \epsilon^{\ell_1}}} \\ % the requirement of equality in the condition of m_l's is required here
                            & \leq \frac{n^{2\frac{1-\epsilon-c}{3} \epsilon^{k_{**}}}}{n^{(1-\epsilon)\epsilon^{k_{**}}}}
                                \leq \frac{n^{(1-\epsilon-c) \epsilon^{k_{**}}}}{n^{(1-\epsilon)\epsilon^{k_{**}}}}
                                = \frac{1}{n^{c \epsilon^{k_{**}}}}.
                    \end{split}
                \]
                The forth inequality comes from the fact that $\frac{(|A_i| - m) m}{|A_i|} \geq 1$.
                Then, if $A_{1,s} \cap U_1= \emptyset$, we have that $|\bigcup_{a \in A_{1,s}} U_{2,a}| \leq |A_{1,s}| |A_2|^\epsilon$.
                So, given $t \in \bparcurly{1, \dots, \frac{|A_2|}{m}}$, we can similarly bound $P_2$, the probability that
                $A_{2,t} \cap \bigcup_{a \in A_{1,s}} U_{2,a} \neq \emptyset$, by:
                \[
                    \begin{split}
                        P_2
                            & \leq \frac{|\bigcup_{a \in A_{1,s}} U_{2,a}|}{|A_2|} + \dots + \frac{|\bigcup_{a \in A_{1,s}} U_{2,a}|}{|A_2|-m+1}
                                < \frac{m |\bigcup_{a \in A_{1,s}} U_{2,a}|}{|A_2| - m}
                                \leq \frac{m m |A_2|^\epsilon}{|A_2| - m} \\
                            & \leq \frac{m^3 |A_2|^\epsilon}{|A_2|}
                                = \frac{m^3}{|A_2|^{1-\epsilon}}
                                = \frac{m^3}{m_0^{\parround{1-\epsilon} \epsilon^{\ell_2}}} \\ % the requirement of equality in the condition of m_l's is required here
                            & \leq \frac{n^{3\frac{1-\epsilon-c}{3} \epsilon^{k_{**}}}}{n^{(1-\epsilon)\epsilon^{k_{**}}}}
                                \leq \frac{n^{(1-\epsilon-c) \epsilon^{k_{**}}}}{n^{(1-\epsilon)\epsilon^{k_{**}}}}
                                = \frac{1}{n^{c \epsilon^{k_{**}}}}.
                    \end{split}
                \]
                Putting it all together:
                \[
                    \mathbb{P}(\varepsilon^+_{A_{1,s},A_{2,t},m})
                        \geq (1 - P_1) (1 - P_2)
                        \geq \parround{1 - \frac{1}{n^{c \epsilon^{k_{**}}}}}^2
                        \geq 1 - \frac{2}{n^{c\epsilon^{k_{**}}}}.
                \]
            \end{proof}
        \end{lemma}

        \begin{remark}
            The condition $n \geq m_0 \geq n^\epsilon$, which is both an upper and lower bound of $m_0$, is very strong and will be carried over
            up to \Cref{thm:existance_of_equitative_partition_with_perfect_pairs_but_with_bound_exceptional_pairs}.
            The greater limitations of this resides in the fact that the size of the parts of the resulting partition $m_{**}$
            is set by the size of $m_0$, and thus inherits the same limitations.
        \end{remark}

        Now, since the event of a given subpair not satisfying the desired property is very unlikely, it can be easily proven
        that a random refinement of the partition given by \Cref{lem:existance_of_ordered_f_indivisible_partitions} only has
        a small number of exceptional pairs.

        \begin{lemma}[Claim 4.14 in~\cite{regularity_lemmas_for_stable_graphs}] \label{lem:existance_of_equitative_partition_with_bound_exceptional_pairs}
            Let $G$ be a finite graph with the non-$k_{*}$-order property.
            \sloppy Let $\Partriangle{m_\ell \mid \ell \in \parcurly{0, \dots, k_{**}}}$ be a sequence of non-zero natural numbers such that
            for all $\ell \in \parcurly{0, \dots, k_{**}-1}$, $m_\ell^\epsilon = m_{\ell+1}$,
            for some $\epsilon \in (0, \frac{1}{2})$ such that $4 < (m_{k_{**}-1})^{1-2\epsilon}$.
            Also, suppose $m_0$ satisfies $n^\epsilon \leq m_0 < \min \parround{\frac{\sqrt{2}-1}{\sqrt{2}} n, \frac{n}{n^{c \epsilon^{k_{**}}}}}$,
            with $c \in (0, 1-\epsilon)$.
            Finally, let $m_{**}$ be a divisor of $m_\ell$ for all $\ell \in \parcurly{0, \dots, k_{**}-1}$ and
            $m_{**} \leq n^{\frac{1 - \epsilon - c}{3}\epsilon^{k_{**}}}$.
            If $A \subseteq G$ with $|A| = n$, then we can find a partition $\overline{A} = \Partriangle{A_i \mid i \in \parcurly{1, \dots, r}}$
            with reminder $B = A \setminus \bigcup_{i \in \parcurly{1, \dots, r}} A_i$ such that:
            \begin{enumerate}[label={\Roman*}., ref={\Roman*}, font=\rmfamily]
                \item \label{itm:existance_of_equitative_partition_with_bound_exceptional_pairs.1} $|A_i| = m_{**}$ for all $i \in \parcurly{1, \dots, r}$.
                \item \label{itm:existance_of_equitative_partition_with_bound_exceptional_pairs.2} For all but $\frac{2}{n^{c\epsilon^{k_{**}}}}r^2$ of the pairs
                    $(A_i, A_j)$ with $i<j$ there are no exceptional edges, i.e.
                    \[
                        \parcurly{(a,b) \in A_i \times A_j \mid a R b \not\equiv t(A_i, A_j)} = \emptyset.
                    \]
                \item \label{itm:existance_of_equitative_partition_with_bound_exceptional_pairs.3} $|B| < m_0$.
            \end{enumerate}
            \begin{proof}
                We can use \Cref{lem:existance_of_ordered_f_indivisible_partitions} to get a partition
                $\overline{A'} = \Partriangle{A'_i \mid i \in \parcurly{1, \dots, i(*)}}$ and remainder $B' = A \setminus \bigcup A'$.
                We can refine the partition by randomly splitting each $A'_i$ into pieces of size $m_{**}$ (\dref{itm:existance_of_equitative_partition_with_bound_exceptional_pairs.1}).
                Consider the resulting partition $\overline{A} = \Partriangle{A_i \mid i \in \parcurly{1, \dots, r}}$ with remainder $B = B'$
                (\dref{itm:existance_of_equitative_partition_with_bound_exceptional_pairs.3}).
                First of all, notice that for each pair $(A_i, A_j)$ such that $A_i \subseteq A'_{i_1}$ and
                $A_j \subseteq A'_{j_1}$ with $i_1 \neq j_1$, the probability of the pair having exceptional edges is
                upper bounded by $\frac{2}{n^{c\epsilon^{k_{**}}}}$.
                This follows \Cref{lem:bound_on_the_probability_of_a_subpair_having_no_exceptions}.
                Thus, given $X$ the random variable counting the number of exceptional pairs of this kind, we have
                \[
                    \mathbb{E}(X) = \hspace{-1.6em} \sum_{\substack{A_i,A_j \text{ s.t.}\\A_i\subseteq A'_{i_1},A_j\subseteq A'_{j_1}\\i_1\neq j_1}} \hspace{-1.6em} \mathbb{E}(X_{A_i, A_j})
                         = \hspace{-1.6em} \sum_{\substack{A_i,A_j \text{ s.t.}\\A_i\subseteq A'_{i_1},A_j\subseteq A'_{j_1}\\i_1\neq j_1}} \hspace{-1.6em} 1 - \mathbb{P}(\varepsilon^+_{A_i, A_j,m_{**}})
                         \leq \frac{r^2}{2} \frac{2}{n^{c\epsilon^{k_{**}}}},
                \]
                where $X_{A_i,A_j}$ is the random variable giving $1$ if $(A_i, A_j)$ is exceptional, and $0$ otherwise.
                Since the expectation is an average, for some refinement $\overline{A}$ of $\overline{A'}$ we have that
                the number of exceptional pairs when $i_1 \neq j_1$ is at most $\frac{r^2}{n^{c\epsilon^{k_{**}}}}$.
                Now, we have no control if $i_1 = j_1$, so let's bound how many of these we have:
                \[
                    \begin{split}
                        \bparstraight{\bparcurly{\text{Exceptional } (A_i, A_j) \mid A_i, A_j \subseteq A'_{i_1}, i_1 \in \parcurly{1, \dots, i(*)}}}
                            & \leq {\frac{m_0}{m_{**}} \choose 2} \frac{n}{m_0} \\
                            & \leq \frac{\parround{\frac{m_0}{m_{**}}}^2}{2} \frac{n}{m_0}
                                = \frac{m_0 n}{2 m_{**}^2}
                                = \frac{m_0}{n} \bbparround{\frac{n}{\sqrt{2}m_{**}}}^2 \\
                            & \leq \frac{m_0}{n} \bbparround{\frac{n - m_0}{m_{**}}}^2
                                \leq \frac{m_0}{n} r^2
                                < \frac{r^2}{n^{c \epsilon^{k_{**}}}}.
                    \end{split}
                \]
                Notice that the third inequality comes after the condition $m_0 \leq \frac{\sqrt{2}-1}{\sqrt{2}} n$.
                Putting it all together, we see that the number of exceptional pairs is upper bounded by
                    $\frac{2r^2}{n^{c\epsilon^{k_{**}}}}$ satisfying \dref{itm:existance_of_equitative_partition_with_bound_exceptional_pairs.2}.
            \end{proof}
        \end{lemma}

        \begin{remark}[Remark 4.15 in~\cite{regularity_lemmas_for_stable_graphs}]
            In the previous proof, the condition $m_0 < \frac{n}{n^{c\epsilon^{k_{**}}}}$ can be
            weakened at the cost of increasing the number of exceptional pairs.
            More specifically, since this condition is only used to bound the exceptional sub-pairs in the same pair
            (the second part of the proof), the number of exceptional pairs can be generally bounded by
            \[
                |\parcurly{\text{Exceptional pairs}}|
                    \leq \bbparround{\frac{m_0}{n} + \frac{2}{n^{c\epsilon^{k_{**}}}}} r^2.
            \]
        \end{remark}

        We now resume the previous results in a theorem with minimal conditions.

        \begin{theorem}[Theorem 4.16 in~\cite{regularity_lemmas_for_stable_graphs}] \label{thm:existance_of_equitative_partition_with_perfect_pairs_but_with_bound_exceptional_pairs}
            Let $r, r' > 1$ be two natural numbers,
            and let $\epsilon = \frac{1}{r}$ and \mbox{$c = \frac{r' - 1}{r'} - \epsilon$}.\footnote{Choosing $\epsilon$ and
                $c$ this way ensures avoiding aproximation errors, and makes the proof more readable}
            Let $G$ be a finite graph with the non-$k_*$-order property.
            Let $A \subseteq G$ with $|A| = n$, and $n > 4^{\frac{r^{k_{**}}}{1-2\epsilon}}$.
            Then, for any $m_{**} \in \bparsquared{n^{\frac{(1-\epsilon-c)}{3}\epsilon^{k_{**}+1}},
            \parround{\frac{\sqrt{2}-1}{\sqrt{2}}}^{\frac{1-\epsilon-c}{3}\epsilon^{k_{**}}} n^{\frac{(1-\epsilon-c)}{3}\epsilon^{k_{**}} -
            \frac{(1-\epsilon-c)c}{3}\epsilon^{2k_{**}}}}$, there is a partition
            $\overline{A} = \Partriangle{A_i \mid i \in \parcurly{1, \dots, m}}$ of $A$ with remainder
            $B = A \setminus \bigcup \overline{A}$ such that:
            \begin{enumerate}[label={\Roman*}., ref={\Roman*}, font=\rmfamily]
                \item\label{itm:existance_of_equitative_partition_with_perfect_pairs_but_with_bound_exceptional_pairs.1}
                    $|A_i| = m_{**}$ for all $i \in \parcurly{1, \dots, m}$.
                \item\label{itm:existance_of_equitative_partition_with_perfect_pairs_but_with_bound_exceptional_pairs.2}
                    $|B| < m_{**}^{\frac{3}{(1-\epsilon-c)}r^{k_{**}}}$.
                \item\label{itm:existance_of_equitative_partition_with_perfect_pairs_but_with_bound_exceptional_pairs.3}
                    $\bparstraight{\bparcurly{ (i,j) \mid i,j \in \parcurly{1, \dots, m}, i < j \text{ and }
                    \parcurly{(a,b) \in A_i \times A_j \mid a R b} \notin
                    \parcurly{A_i \times A_j, \emptyset}}}
                    \leq \frac{2}{n^{c\epsilon^{k_{**}}}} m^2$.
            \end{enumerate}
            \begin{proof}
                Let $m_{k_{**}} = m_{**}^{\frac{3}{1-\epsilon-c}}$, and consider the sequence
                \[
                    m_{**} \leq m_{k_{**}} < \dots < m_0
                \]
                such that for all $\ell \in \parcurly{1, \dots, k_{**}}$ we have that $m_{\ell-1} = m_\ell^r$.
                Notice that:
                \begin{enumerate}[label={\roman*}., ref={\roman*}, font=\rmfamily]
                    \item $c \in (0, 1 - \epsilon)$, since $0 < \frac{1}{2} - \epsilon \leq \frac{r'-1}{r'} - \epsilon < 1 - \epsilon$.
                    \item All $m_\ell$'s are powers of $m_{k_{**}}$, and so they are divisible by it.
                        Also, since $\frac{3}{1-\epsilon-c} = 3r'$ is an integer
                        by choice of $c$, $m_{**}$ divides $m_{k_{**}}$.
                        Thus, $m_{**}$ divides $m_\ell$ for all $\ell \in \parcurly{0, \dots, k_{**}}$.
                    \item $(m_{\ell-1})^\epsilon = m_\ell$ for all $\ell \in \parcurly{1, \dots, k_{**}}$, by construction.
                    \item $m_{**} \leq n^{\frac{1-\epsilon-c}{3}\epsilon^{k_{**}}}$, by choice of $m_{**}$.
                    \item $m_0 = m_{**}^{\frac{3}{1-\epsilon-c}r^{k_{**}}}$, so on one hand
                        \[
                            m_0 = m_{**}^{\frac{3}{1-\epsilon-c}r^{k_{**}}} \geq n^{\frac{1-\epsilon-c}{3}\epsilon^{k_{**}+1} \frac{3}{1-\epsilon-c}r^{k_{**}}}
                                \geq n^{\epsilon},
                        \]
                        and on the other hand,
                        \[
                            m_0 = m_{**}^{\frac{3}{1-\epsilon-c}r^{k_{**}}} \leq \parround{\frac{\sqrt{2}-1}{\sqrt{2}}} n^{1 - c \epsilon^{k_{**}}},
                        \]
                        and thus $n$ is both smaller than $\parround{\frac{\sqrt{2}-1}{\sqrt{2}}} n$ and
                        smaller than $n^{1 - c \epsilon^{k_{**}}}$.
                    \item $m_{k_{**}-1} = m_{**}^{\frac{3}{1-\epsilon-c}r} \geq n^{\epsilon^{k_{**}}} > 4^{\frac{1}{1-2\epsilon}}$.
                \end{enumerate}
                So, all the conditions of \Cref{lem:existance_of_equitative_partition_with_bound_exceptional_pairs} are satisfied,
                and we can use it to get a partition $\overline{A}$ with remainder $B$ satisfying the statement.
                Notice that \dref{itm:existance_of_equitative_partition_with_perfect_pairs_but_with_bound_exceptional_pairs.2}
                is satisfied by the fact that $|B| < m_0 \leq m_{**}^{\frac{3}{(1-\epsilon-c)}r^{k_{**}}}$.
            \end{proof}
        \end{theorem}

        \begin{remark}
            Some notes on the partition obtained in the previous theorem:
            \begin{itemize}
                \item With any choice of $c$ and $m_{**}$, the fraction of exceptional pairs is asymptotically small,
                    but we obtain very small parts, that is, $m_{**} \thickapprox n^{\epsilon^{k_{**}}}$.
                \item A smaller value of $c$ results in larger parts and smaller reminder, at the cost of a larger fraction
                    of exceptional pairs.
                \item The window of choice of $m_{**}$ is very small, and taking a larger value
                    (in the given interval), results in a strongly larger reminder.
                    The edge case of choosing $m_{**}$ as the larger value, results in the bound on the size of the
                    reminder becoming $\parstraight{B} < \frac{\sqrt{2}-1}{\sqrt{2}} n^{1-\epsilon^{k_{**}}}$.
            \end{itemize}
        \end{remark}

    \subsection{$f_c$-indivisible even partition} \label{subsec:subsection_4.3}

        Next, we will follow another approach to obtain an even partition.
        That is, we prove a result similar to that of \Cref{lem:existance_of_indivisible_sets}, but this time the size of the resulting
        \regular~set can be chosen in advance.
        The resulting \Cref{lem:many_values_to_equitative_partition_with_bound_exceptional_pairs} has also the advantage that
        the associated \regular~property is $f_c$-indivisibility, where $f_c$ is the constant function $f_c(x) = c$, which
        is much stronger then $\epsilon$-indivisibility as the bound on the number of exceptions is constant.

        To prove this result, we use a probabilistic argument, and show that the event of there existing a subset which has
        intersection smaller than $c$ with every $\overline{B}_{A,b}$ (\Cref{def:n_large_enough_property}) is highly
        probable under some very specific conditions (\Cref{lem:n_large_enough_valid_values}).

        \begin{definition}[Definition 4.18 in~\cite{regularity_lemmas_for_stable_graphs}] \label{def:n_large_enough_property}
            For $n, c \in \mathbb{N}$ and $\epsilon, \zeta, \xi \in \mathbb{R}$, let $\oplus[n, \epsilon, \zeta, \xi, c]$ be
            the statement:
            For any set $A$ such that $|A| = n$, and for any family of subsets $P \subseteq \mathcal{P}(A)$ such that
            $|P| \leq n^{\frac{1}{\zeta}}$ and each $B \in P$ satisfies $|B| \leq n^\epsilon$, then there exists $U \subseteq A$
            with $|U| = \lfloor n^\xi \rfloor$ such that for all $B \in P$, $|U \cap B| \leq c$.
        \end{definition}

        \begin{lemma}[Lemma 4.19 in~\cite{regularity_lemmas_for_stable_graphs}] \label{lem:n_large_enough_valid_values}
            If the reals $\epsilon, \zeta, \xi$ satisfy $\epsilon \in (0,1)$, $\zeta > 0$ and $0 < \xi < \min \parround{\frac{1}{2}, 1 - \epsilon}$,
            the natural number $n$ is sufficiently large ($n > N(\epsilon, \zeta, \xi, c)$) to satisfy the equation
                \begin{equation} \label{eq:n_large_enough_valid_values.1}
                    \frac{1}{2n^{1-2\xi}} + \frac{1}{n^{(1 - \xi - \epsilon)c - \frac{1}{\zeta}}} < 1,
                \end{equation}
            and $c > \frac{1}{\zeta (1 - \xi - \epsilon)}$,
            then $\oplus[n, \epsilon, \zeta, \xi, c]$ holds.
            \begin{proof}
                Let $m = \lfloor n^\xi \rfloor$ be the size of the set $U$ we want to build, and let $\mathcal{F}_* = [A]^m$ be
                the set of sequences of length $m$ of (not necessarily distinct) elements of $A$.
                Let $\mu$ be the uniform distribution on $\mathcal{F}_*$, i.e. for all $F \subseteq \mathcal{F}_*$,
                $\mu(F) = \frac{|F|}{|\mathcal{F}_*|}$.
                Let $P \subseteq \mathcal{F}_*$ be a family of sequences where the elements of the sequences are pairwise distinct.
                We want to prove that the probability that a random $U \in \mathcal{F}_*$ satisfies:
                \begin{enumerate}[label={\Roman*}., ref={\Roman*}, font=\rmfamily]
                    \item\label{itm:n_large_enough_valid_values.1} All elements of $U$ are distinct.
                    \item\label{itm:n_large_enough_valid_values.2} For all $B \in P$, $|U \cap B| < c$
                        (where the intersection counts repetitions).
                \end{enumerate}
                is non-zero.
                First of all let's bound the converse of \dref{itm:n_large_enough_valid_values.1}, i.e. the probability that there are two equal elements
                in $U$:
                \[
                    P_1 = \mathbb{P}(\exists s < t \in [m] \mid U_s = U_t)
                        \leq {m \choose 2} \frac{n}{n^2}
                        \leq \frac{m^2}{2n}
                        \leq \frac{n^{2\xi}}{2n}
                        < \frac{1}{2n^{1-2\xi}}.
                \]
                Now, in order to bound \dref{itm:n_large_enough_valid_values.2}, let's first bound $P_B$: the probability that at least $c$ elements of
                $U$ are in a given $B \in P$.
                \[
                    P_B = \mathbb{P}(\exists^{\geq c}\; t\in [m] \mid U_t \in B)
                        \leq {m \choose c} \bbparround{\frac{|B|}{n}}^c
                        \leq \frac{m^c |B|^c}{n^c}
                        \leq \frac{n^{\xi c} n^{\epsilon c}}{n^c}
                        = \frac{1}{n^{c (1 - \xi - \epsilon)}}.
                \]
                Then, we can bound the converse of \dref{itm:n_large_enough_valid_values.2}, i.e. the probability that this happens for some $B \in P$,
                by:
                \[
                    P_2 = \mathbb{P}(\exists B \in P \mid \exists^{\geq c}\; t\in [m], U_t \in B)
                        \leq \sum_{B \in P} P_B
                        = \frac{|P|}{n^{c (1 - \xi - \epsilon)}}
                        \leq \frac{1}{n^{c (1 - \xi - \epsilon) - \frac{1}{\zeta}}}.
                \]
                Putting it all together, we have that
                \[
                    \mathbb{P}(\text{U does not satisfy } \dref{itm:n_large_enough_valid_values.1} \text{ or does not satisfy } \dref{itm:n_large_enough_valid_values.2})
                        \leq P_1 + P_2
                        < \frac{1}{2n^{1-2\xi}} + \frac{1}{n^{c (1 - \xi - \epsilon) - \frac{1}{\zeta}}}.
                \]
                Notice that
                \begin{itemize}
                    \item Since $\xi < \frac{1}{2}$ we have that $1 - 2\xi > 0$.
                    \item $c (1 - \xi - \epsilon) - \frac{1}{\zeta}> 0$, since $\xi < 1 - \epsilon$ and $c > \frac{1}{\zeta (1 - \xi - \epsilon)}$.
                \end{itemize}
                so, the $n$-large enough condition \eqref{eq:n_large_enough_valid_values.1} is well defined and
                \[
                    \mathbb{P}(\text{U does not satisfy } \dref{itm:n_large_enough_valid_values.1} \text{ or does not satisfy } \dref{itm:n_large_enough_valid_values.2})
                        < \frac{1}{2n^{1-2\xi}} + \frac{1}{n^{c (1 - \xi - \epsilon) - \frac{1}{\zeta}}}
                        < 1
                \]
                holds.
                We conclude that the probability that there exists a $U \subseteq A$ satisfying the condition is non-trivial,
                and $\oplus[n, \epsilon, \zeta, \xi, c]$ holds.
            \end{proof}
        \end{lemma}

        \begin{remark} \label{rmk:n_grows}
            In the context of the condition $c > \frac{1}{\zeta (1 - \xi - \epsilon)}$ from the previous lemma, we note that
            the lower bound on $c$ increases as $\xi$ and $\epsilon$ grow, and as $\zeta$ decreases.
            A similar pattern is also followed by the large enough condition of $n$ given
            by~\eqref{eq:n_large_enough_valid_values.1}.
            For the condition to be met, $n$ needs to grow as the exponents $1-2\xi$ and
            $(1 - \xi - \epsilon)c - \frac{1}{\zeta}$ becomes smaller.
            That is, the lower bound $N_{\ref{lem:n_large_enough_valid_values}}(\epsilon, \zeta, \xi, c)$ on $n$ becomes larger as $\xi$ and $\epsilon$ grow, and as $\zeta$ and $c$ decrease.
        \end{remark}

        The following claim will be useful in the proof of \Cref{lem:many_values_to_equitative_partition_with_bound_exceptional_pairs}.
        \begin{claim} \label{clm:floor_exponential_composition_bound}
            Let $f_\epsilon(x)$ be the map such that $f_\epsilon(x) = \floor{x^\epsilon}$.
            Then, for any $x \geq 1$ and $\epsilon_1, \dots, \epsilon_k \in (0,1)$, the following
            inequality is satisfied:
            \[
                f_{\epsilon_1 \epsilon_2 \dots \epsilon_k}(x)
                    \leq f_{\epsilon_1} \circ f_{\epsilon_2} \circ \dots \circ f_{\epsilon_k}(x) + k - 1
            \]
            \begin{proof}[Proof in \Cref{sec:appendix_other_proofs}]
            \end{proof}
        \end{claim}

        \begin{lemma}[Claim 4.21 in~\cite{regularity_lemmas_for_stable_graphs}] \label{lem:many_values_to_equitative_partition_with_bound_exceptional_pairs}
            Let $k_*, c \in \mathbb{N}$ and $\epsilon, \xi \in \mathbb{R}$ such that:
            \begin{enumerate}[label={\Roman*}., ref={\Roman*}, font=\rmfamily]
                \item\label{itm:many_values_to_equitative_partition_with_bound_exceptional_pairs.1} $G$ is a graph with the non-$k_*$-order property.
                \item\label{itm:many_values_to_equitative_partition_with_bound_exceptional_pairs.2} $\epsilon \in (0, \frac{1}{2}]$.
                \item\label{itm:many_values_to_equitative_partition_with_bound_exceptional_pairs.3} $\xi \in \parround{0, \frac{\epsilon^{k_{**}}}{2}}$.
                \item\label{itm:many_values_to_equitative_partition_with_bound_exceptional_pairs.4} $c$ satisfies \[
                    c > \frac{1}{\frac{1}{k_*} (1 - \frac{\xi}{\epsilon^{k_{**}}} - \epsilon)}.
                \]
            \end{enumerate}
            Then, for every sufficiently large $n \in \mathbb{N}$ (it suffices that $n >
            g_\epsilon^{k_{**}}\parround{N_{\ref{lem:n_large_enough_valid_values}}\parround{\epsilon, \frac{1}{k_*}, \frac{\xi}{\epsilon^{k_{**}}}, c}}$,
            where $g_\epsilon(x) = \parround{x+1}^{\frac{1}{\epsilon}}$),
            if $A \subseteq G$ with $|A| = n$, there is $Z \subseteq A$ such that
            \begin{enumerate}[label=\roman*., ref=\roman*, font=\rmfamily]
                \item\label{itm:many_values_to_equitative_partition_with_bound_exceptional_pairs.a} $|Z| \geq \lfloor n^\xi \rfloor - k_{**}$.
                \item\label{itm:many_values_to_equitative_partition_with_bound_exceptional_pairs.b} $Z$ is $f_c$-indivisible in $G$.
            \end{enumerate}
            \begin{proof}
                Let $n = m_0 > m_1 > \dots > m_{k_{**}}$ with $m_\ell = \floor{m_{\ell-1}^\epsilon} \geq g_\epsilon^{-1}(m_{\ell-1}) \geq g_\epsilon^{-\ell}(n)$.
                Then, $m_\ell \geq m_{\ell+1}$ and we can use \Cref{lem:existance_of_indivisible_sets}
                to get an $\epsilon$-indivisible subset $A_1 \subseteq A$, with $|A_1| = m_\ell$ for some $\ell \in \parcurly{0, \dots, k_{**}-1}$.
                Notice that:
                \begin{itemize}
                    \item $\epsilon \in (0,1)$ by \dref{itm:many_values_to_equitative_partition_with_bound_exceptional_pairs.2}.
                    \item We can set $\zeta \coloneqq \frac{1}{k_*} > 0$.
                    \item By \dref{itm:many_values_to_equitative_partition_with_bound_exceptional_pairs.3},
                        $0 < \frac{\xi}{\epsilon^\ell} \leq \frac{\xi}{\epsilon^{k_{**}}} < \frac{1}{2}$.
                    \item For all $\ell \in \parcurly{0,\dots,k_{**}}$, $m_\ell$ is sufficiently large:
                        \[
                            m_\ell \geq g_\epsilon^{-\ell}(n) \geq g_\epsilon^{-k_{**}}(n)
                                > N_{\ref{lem:n_large_enough_valid_values}}\parround{\epsilon, \frac{1}{k_*}, \frac{\xi}{\epsilon^{k_{**}}}, c}
                                > N_{\ref{lem:n_large_enough_valid_values}}\parround{\epsilon, \zeta, \frac{\xi}{\epsilon^{\ell}}, c}.
                        \]
                        Second inequality follows from the fact that function $g_\epsilon^{-1}$ satisfies that
                        $g_\epsilon^{-1}(x) \leq x$ for $x \geq 1$.
                        For third inequality we use that $g_\epsilon^{-1}$ is decreasing.
                        Forth inequality uses \Cref{rmk:n_grows} and by definition $\zeta = \frac{1}{k_{**}}$.
                    \item $c > \frac{1}{\frac{1}{k_*} (1 - \frac{\xi}{\epsilon^{k_{**}}} - \epsilon)}
                        = \frac{1}{\zeta (1 - \frac{\xi}{\epsilon^{k_{**}}} - \epsilon)}$, by \dref{itm:many_values_to_equitative_partition_with_bound_exceptional_pairs.4}.
                \end{itemize}
                Conditions of \Cref{lem:n_large_enough_valid_values} are met, so
                $\oplus\parsquared{m_\ell, \epsilon, \zeta, \frac{\xi}{\epsilon^\ell}, c}$
                (as defined in \Cref{def:n_large_enough_property}) holds.
                We can take $A_{(\ref{def:n_large_enough_property})}$ and $P_{(\ref{def:n_large_enough_property})}$ to be
                $A_1$ and $P \coloneqq \bparcurly{\overline{B}_{A_1,b} \mid b \in G}$ respectively, which satisfy:
                \begin{itemize}
                    \item $|A_1| = m_\ell$.
                    \item $|P| \leq m_\ell^{k_*} = m_\ell^{\frac{1}{\zeta}}$, where first inequality follows \dref{itm:k_order_propery_bounds_BAbs.2}
                        of \Cref{cor:k_order_propery_bounds_BAbs}.
                    \item $\forall B \in P$, $|B| \leq |A_1|^\epsilon$ by $\epsilon$-indivisibility of $A_1$.
                \end{itemize}
                Thus, by $\oplus\parsquared{m_\ell, \epsilon, \zeta, \frac{\xi}{\epsilon^\ell}, c}$ we have that there exists $Z \subseteq A_1$ such that:
                \begin{itemize}
                    \item Condition \dref{itm:many_values_to_equitative_partition_with_bound_exceptional_pairs.a} is satisfied:
                        \[
                            |Z| = \floor{m_\ell^{\xi\epsilon^{-\ell}}}
                            = f_{\xi\epsilon^{-\ell}} \circ f_{\epsilon}^{\ell} (n)
                            \stackrel{\ref{clm:floor_exponential_composition_bound}}{\geq} f_{\xi\epsilon^{-\ell} \epsilon^{\ell}} (n) - \ell
                            \geq f_\xi (n) - k_{**}
                            = \lfloor n^\xi \rfloor - k_{**},
                        \]
                        where $f_\epsilon(x)$ is defined in \Cref{clm:floor_exponential_composition_bound}.
                    \item $Z$ is $f_c$-indivisible since $|B \cap Z| \leq c$ for all $B \in P$,
                        satisfying \dref{itm:many_values_to_equitative_partition_with_bound_exceptional_pairs.b}.
                \end{itemize}
                This proves the statement.
            \end{proof}
        \end{lemma}

        We now use the previous result to build an even partition.
        Similarly to \Cref{lem:existance_of_ordered_f_indivisible_partitions}, we will iteratively extract an $f_c$-indivisible
        set from the reminder using \Cref{lem:many_values_to_equitative_partition_with_bound_exceptional_pairs}, while
        the sufficiently large condition holds.

        \begin{theorem}[Theorem 4.23 in~\cite{regularity_lemmas_for_stable_graphs}] \label{thm:equitative_partition_high_regularity_parts_grow_with_n}
            Let $G$ be a graph with the non-$k_*$-property.
            For any $\epsilon \in (0, \frac{1}{2}]$, $\xi \in \parround{0, \frac{\epsilon^{k_{**}}}{2}}$ and
            $c > \frac{k_*}{1-\frac{\xi}{\epsilon^{k_{**}}} - \epsilon}$, any $A \subseteq G$ with $\parstraight{A} = n$
            has a partition $\overline{A} = \Partriangle{A_i \mid i \in \parcurly{1, \dots, i(*)}}$ of $A$ with remainder
            $B = A \setminus \bigcup_{i \in \parcurly{1, \dots, i(*)}} A_i$ satisfying:
            \begin{itemize}
                \item $|A_i| = \floor{n^\xi} - k_{**}$ for all $i \in \parcurly{1, \dots, i(*)}$.
                \item $A_i$ is $f_c$-indivisible for all $i \in \parcurly{1, \dots, i(*)}$, where $f_c(x) = c$ is a constant function.
                \item $|B| \leq N \coloneqq g_\epsilon^{k_{**}}\parround{N_{\ref{lem:n_large_enough_valid_values}}\parround{\epsilon, \frac{1}{k_*}, \frac{\xi}{\epsilon^{k_{**}}}, c}}$
                    where $g_\epsilon(x) = \parround{x+1}^{\frac{1}{\epsilon}}$.
            \end{itemize}
            \begin{proof}
                We build a sequence of disjoint $f_c$-indivisible subsets $A_i$ by induction on $i$ as follows.
                Let $R_i = A \setminus \bigcup_{j<i} A_j$ (so $R_1 = A$).
                At each step, if $\parstraight{R_i} > N$, we can apply \Cref{lem:many_values_to_equitative_partition_with_bound_exceptional_pairs} to $R_i$
                with the values $f_c$, $\epsilon$ and $\xi$ of the statement of this theorem,
                to obtain a $f_c$-indivisible subset $A_i$ of $R_i$ of size (at least\footnote{
                    Since $f_c$-indivisibility is preserved under taking subsets, we can take $A_i$ to have exactly size $\floor{n^\xi} - k_{**}$.
                }) $\floor{n^\xi} - k_{**}$, which will be disjoint with all $A_j$ with $j < i$.
                Otherwise, we stop and let $\overline{A} = \Partriangle{A_j \mid j < i = i(*)}$ and $B = R_i$.
                By the case hypothesis, $\parstraight{B} = \parstraight{R_i} \leq N$, and we are done.
            \end{proof}
        \end{theorem}

        \begin{remark}
            Some notes on the partition obtained in the previous theorem:
            \begin{itemize}
                \item The partition is exceptionally \regular, and the number of exceptional edges in each pair of parts
                    and subparts is strongly bounded as shown by \Cref{cor:exceptions_bound_of_c_indivisible_sets}.
                \item As the upper bound on the size of the remainder is constant with respect to the size of the graph $n$,
                    the remainder as a fraction of the total graph can be made as small as desired (but not completely avoided).
                    If we want the remainder to be at most $\frac{1}{x}$ of the total graph, we can simply impose
                    $n \geq x \cdot N$, and we are done.
                \item The $- k_{**}$ factor in the size of the parts is just a by-product of using \Cref{clm:floor_exponential_composition_bound}
                    to deal with floor functions, and not a crucial dependence of the size of the parts
                    on this parameter.
                    Careful computations may reduce this factor to a constant independent of any parameter.
                \item The parts are exponentially smaller than the size of the graph.
                    Hence, the number of parts grows with the size of the graph, which is actually the principal drawback of
                    this theorem.
                    This will be solved in the partition studied in \Cref{sec:section_5}.
            \end{itemize}
        \end{remark}

        % up to here it was compiling with no problem.
