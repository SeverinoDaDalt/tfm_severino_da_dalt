\documentclass[12,twoside]{utils/mammeTFM}
\usepackage{amsthm,amsmath,amssymb,amsfonts,amscd}
\usepackage{graphicx}
\usepackage{enumerate}
\usepackage[all]{xy}
\usepackage{booktabs}
\usepackage{mathtools}
\usepackage{enumitem}
\usepackage{hyperref}
\usepackage{dirtytalk}
\usepackage{bbm}
\usepackage{pdfpages}


% SKIP INDENTATION
\setlength{\parindent}{0pt}


% ALGORITHM
\usepackage{algorithm}
\usepackage{algorithmicx}
\usepackage{algpseudocode}
\makeatletter
\AddToHook{env/algorithmic/before}{\def\@currentcounter{ALG@line}}
\makeatother
\usepackage[nameinlink]{cleveref}  % this is for a lot of things, not only algorithms
\crefalias{ALG@line}{line}


% PICTURES
\usepackage{calc}
\usepackage{tikz}
\usepackage{caption}
\usetikzlibrary{fit, positioning}


% BIBLIOGRAPHY
\usepackage[
    backend=biber,
    style=trad-abbrv,
    sortcites
]{biblatex}
\addbibresource{sections/bibliography.bib}


% INDEX
\usepackage{tocbibind}


% Temporal
\usepackage[textsize=tiny]{todonotes}
\setlength{\marginparwidth}{15mm}
\usepackage{dsfont}


% Theorem Environments: add extra ones at the end if you need it.

\newtheorem*{theoremA}{Theorem A}
\newtheorem*{theoremB}{Theorem B}
\newtheorem*{theoremC}{Theorem C}
\newtheorem{namedthm}{Theorem}
\newenvironment{thm*}[1]{%
  \renewcommand\thenamedthm{#1}%
  \namedthm}{\endnamedthm}
\newtheorem{theorem}{Theorem}[section]

\newtheorem{proposition}[theorem]{Proposition}
\newtheorem{lemma}[theorem]{Lemma}
\newtheorem{claim}[theorem]{Claim}  % Seve
\newtheorem{corollary}[theorem]{Corollary}
\newtheorem{conjecture}[theorem]{Conjecture}

\theoremstyle{definition}
\newtheorem{definition}[theorem]{Definition}
\newtheorem{example}[theorem]{Example}

\theoremstyle{remark}
\newtheorem{remark}[theorem]{Remark}
\newtheorem*{remarknonumber}{Remark}
\newtheorem{observation}[theorem]{Observation}



%%%%%%%%%%%%%%%%%%
% macros/abbreviations: Include here your own.
%%%%%%%%%%%%%%%%%%

\newcommand{\N}{\ensuremath{\mathbb{N}}}

% Parenthesis
\newcommand{\Parcurly}[1]{\left\{ #1 \right\}}
\newcommand{\parcurly}[1]{\{ #1 \}}
\newcommand{\bparcurly}[1]{\bigl\{ #1 \bigr\}}

\newcommand{\Parround}[1]{\left( #1 \right)}
\newcommand{\parround}[1]{( #1 )}
\newcommand{\bparround}[1]{\bigl( #1 \bigr)}
\newcommand{\bbparround}[1]{\Bigl( #1 \Bigr)}

\newcommand{\Parsquared}[1]{\left[ #1 \right]}
\newcommand{\parsquared}[1]{[ #1 ]}
\newcommand{\bparsquared}[1]{\bigl[ #1 \bigr]}

\newcommand{\Parstraight}[1]{\left| #1 \right|}
\newcommand{\parstraight}[1]{| #1 |}
\newcommand{\bparstraight}[1]{\bigl| #1 \bigr|}

\newcommand{\Partriangle}[1]{\left< #1 \right>}
\newcommand{\partriangle}[1]{< #1 >}


\newcommand{\ceil}[1]{\left\lceil #1 \right\rceil}
\newcommand{\floor}[1]{\left\lfloor #1 \right\rfloor}

\newcommand{\naturals}{\mathbb{N}}
\newcommand{\dref}[1]{\text{\fontfamily{rm}\rmfamily \ref{#1}}}
\newcommand{\jojo}{TO BE CONTINUED... \emph{(JoJo's Bizarre Adventure theme plays in the background)}}
\newcommand{\regular}{quasi-random}
\newcommand{\regularity}{quasi-randomness}
\newcommand{\irregular}{non-\regular}
%\newcommand{\NAME}{COMMAND}

% Body of document

\titol{The Regularity Lemma for Stable Graphs and its applications in Property Testing}
\titolcurt{Regularity Lemmas for Stable Graphs}
\authorStudent{Severino Da Dalt}
\supervisors{(Llu\'is Vena Cros)}
\monthYear{\today}

\paraulesclau{
    Graph Theory,
    Stable Graphs,
    Stability,
    VC-dimension,
    Szemer\'edi Regularity Lemma,
    Property Testing
}
\agraiments{
Thanks to...}


\abstracteng{
    Szemerédi's Regularity Lemma is a cornerstone of modern graph theory, asserting that any graph can be partitioned
    into a bounded number of vertex sets, where the connections between most pairs of sets behave quasi-randomly.
    Despite its wide-ranging applications in areas like number theory, combinatorics and computer science, the lemma suffers from
    two major limitations:
    a partition size bounded by a tower of exponentials,
    and the presence of irregular pairs, both unavoidable in the general case.

    This work focuses on a specific subclass of graphs, the \emph{stable graphs}, where these limitations can be overcome.
    By avoiding a bipartite substructure known as the half-graph, stable graphs admit a much stronger regularity lemma.
    This specialized lemma, originally developed by Malliaris and Shelah, guarantees a partition where all pairs are
    regular and the number of parts is bounded by a single exponential, a significant improvement over the general
    tower-type bound.

    This thesis first presents a self-contained, combinatorial, and complete presentation of the proof of the stable
    regularity lemma, developing a unified notational framework to bridge concepts from extremal graph theory, stability,
    and property testing.
    Building on this theoretical foundation, we then construct an efficient algorithm for testing \emph{$H$-freeness}
    (the property of not containing an induced copy of a fixed graph $H$) for stable graphs.
    This application leverages the lemma's superior properties to achieve a query complexity with significantly
    improved bounds compared to testers for general graphs.}

%%%%%%%%%
\begin{document}

    \includepdf[pages={1}]{utils/front_page_tfm.pdf}

    \maketitle

    \addtocontents{toc}{\protect\enlargethispage{2\baselineskip}}  % This allows the table of contents to fit in one page.
    \tableofcontents

    % sections
    \section{Introduction} \label{sec:introduction}

    Szemer\'edi's Regularity Lemma~\cite{regular_partitions_of_graphs} is a powerful tool in graph theory,
    stating that any sufficiently large graph can have its vertex set decomposed into an equitable partition such that most,
    but not all, pairs of parts are \emph{regular}.
    A regular pair is one whose edge distribution resembles that of a random bipartite graph, a powerful property with many
    applications in extremal graph theory.
    On top of the presence of a small number of irregular pairs, a major drawback of the lemma is the immense
    bound on the required number of parts, which grows as a tower of exponentials whose height depends on the regularity parameter.

    In the general setting, both limitations have been proven to be unavoidable.
    In~\cite{lower_bounds_of_tower_type_for_szeremedis_uniformity_lemma} the author shows that there exist a family of graphs
    for which the lower bound on the number of parts is still a tower of exponentials\footnote{
    To be more specific, the author shows that the number of parts is lower bounded by an exponenetial tower of $2$'s where
        the height of the tower is at least proportional to $\log\parround{1/\epsilon}$.
        Meanwhile, in the usual proof of the theorem, the upper bound on the height of the tower is proportional to
        $\epsilon^{-5}$.
    }.
    On the other hand, it is folklore knowledge that large-enough half-graphs present irregular pairs in any
    regular partition (\cite{irregular_pairs_in_half_graphs_szemeredi_regularity} gives a written proof of this fact).
    Having seen this unavoidability, it is natural to ask for the underlying reasons of those limitations and which
    additional conditions can be imposed or levied so that the parameters can be improved.

    In this context, one of the first attempts was to make the regularity condition weaker so that the bound on the
    number of parts can be improved: this is now known as
    \emph{the weak regularity lemma}~\cite{quick_approximation_to_matrices_and_applications} and,
    besides its own applications, allows to put the SzRL in the context of a spectra of regularity lemmas, with
    different strengths (the stronger the conclusion the larger the number of
    parts)~\cite{szemeredis_lemma_for_the_analyst, regularity_partitions_and_the_topology_of_graphons};
    a notable example on the other direction (making the regularity lemma stronger) allows its use for Property Testing
    as it is suitable for the study of induced subgraphs~\cite{efficient_testing_of_large_graphs}.

    Another option is to restrict the class of graphs where we want to find a regular partition.
    An example of this effort is the class of graphs with bounded VC-dimension: this concept was introduced
    in~\cite{the_uniform_convergence_of_frequencies_of_the_appearance_of_events_to_their_probabilities}\footnote{
        See~\cite{on_the_uniform_convergence_of_relative_frequencies_of_events_to_their_probabilities}
        for a translated version.}
    and one can view it as a graph with \say{low complexity} (but not necessarily sparse), and the reader can find more
    details in \Cref{sec:section_3}.
    For these graphs the number of parts is highly reduced from a tower type to a polynomial in $1/\epsilon$, whose
    power depends on the bound on the dimension of the graph~\cite{regularity_partitions_and_the_topology_of_graphons,
        erdos_hajnal_conjecture_for_graphs_with_bounded_vc_dimension,
        efficient_testing_of_bipartite_graphs_for_forbidden_induced_subgraphs}.
    Even more, when the graphs have bounded VC-dimension, the density of edge in the regular pairs are either close to $1$,
    or close to $0$.
    However, the issue on the presence of irregular pairs remains, as any half-graph has bounded VC-dimension\footnote{
        Indeed, the fact that the neighbourhoods of the vertices on the same stable set can be ordered by inclusion,
        prevents the VC-dimension to grow beyond $2$.
        Alternatively, in \cite{regularity_partitions_and_the_topology_of_graphons} it is shown that if a graph does not
        contain a bi-induced copy of a bipartite graph where the smaller size is k, then it has VC-dimension (strictly)
        bounded by k; in our case the half-graph has no bi-induced copy of $K_{3,3}$ minus a perfect matching.}.

    In this work we focus our attention on the result by Malliaris and Shelah~\cite{regularity_lemmas_for_stable_graphs,
        notes_on_the_stable_regularity_lemma}
    which states that, if one cannot find a bi-induced copy of a half-graph, then a regular partition can be found, with
    not many pairs ($1/\epsilon$ to the power of an exponential on the size of the half-graph), and where no irregular
    pairs are found.

    The graphs where no large half-graph can be found are called stable graphs, and their study stems from results in
    Model Theory and Logic.
    We shall stress that these stable graphs have, in fact, bounded VC-dimension (since we are forbidding a bi-induced
    copy of a fixed bipartite graph).

    One of the many applications of the regularity lemma for which these bounds on the number of parts become relevant is in
    \emph{property testing}.
    A property testing algorithm for a decision problem $P$ is a randomized algorithm that, by querying only a small portion
    of its input, can distinguish with high probability between objects that satisfy $P$ and those that are \say{far} from
    satisfying it.
    For instance, in~\cite{efficient_testing_of_large_graphs} the authors use Szemer\'edi's Regularity Lemma
    to prove that it is possible to test the property of a graph $G$ being $H$-free (for a fixed graph $H$) using an algorithm
    which query complexity is independent on the size of the input graph $G$.

    The query complexity of such testers, however, is intrinsically linked to the number of parts in the underlying regular
    partition.
    Consequently, the power-tower bounds of the standard regularity lemma lead to prohibitively large, although constant,
    query counts.
    This raises a natural question: can the superior bounds of the stable regularity lemma be exploited to create more
    efficient property testers for graphs in a half-graph-restricted setting?

    In this thesis, we present an algorithm for testing $H$-freeness in stable graphs, thereby providing
    a concrete application that highlights the practical strength and utility of stable regularity partitions.

    \subsection{Main Contributions} \label{subsec:main_contributions}

        The main contributions of this thesis are:
        \begin{itemize}
            \item We place a larger emphasis on the combinatorial part of the result
                in~\cite{regularity_lemmas_for_stable_graphs}, making it self-contained and making some of the argument that
                previously used some Model Theory fully combinatorial.
                Further, we make some of the relations between the parameters explicit while correcting some of the typos
                that inevitably occur.
                In addition, we simplify some of the arguments, while making others more explicit and detailed.
                In particular, we make explicit that the excellence (see \Cref{sec:section_5}) depends on two parameters
                with opposite monotonic properties (see \Cref{def:epsilon_excellent} and \Cref{rmk:excellence_is_not_monotonic}).
            \item The construction of an efficient property testing algorithm for H-freeness tailored to stable graphs.
                The algorithm's analysis leverages the stable regularity lemma to achieve a query complexity with significantly
                improved bounds compared to the general case.
            \item The development of a unified notational framework that cohesively integrates the concepts from
                extremal graph theory, stability, and property testing used throughout the thesis.
        \end{itemize}

    \subsection{Summary} \label{subsec:summary}

        The remainder of this thesis is organized as follows.
        \Cref{sec:section_2} reviews fundamental concepts from graph theory, culminating in a formal statement of Szemer\'edi's
        Regularity Lemma.
        \Cref{sec:section_3} introduces the graph-theoretic notion of stability and proves some basic results in this context.
        \Cref{sec:section_4} presents and analyzes some weaker variants of the stable regularity lemma, and illustrate both its
        strengths and its inherent limitations.
        \Cref{sec:section_5} is dedicated to the proof of the main Stable Regularity Lemma, which forms the technical core of this
        work.
        Finally, \Cref{sec:section_6} applies this previous results to prove our property testing algorithm for
        $H$-freeness in stable graphs works, providing explicit bounds on its query complexity.

        \newpage

    \section{Section 2} \label{sec:section_2}
        \newpage

    \section{Section 3} \label{sec:section_3}

    In this section we introduce the class of \emph{stable} graphs.
    A graph is considered stable, if it does not contain bi-induced (as defined in~\cite{induced_subgraph_density_vi_bounded_vc_dimension})
    \todo{Maybe move cite to section 2, and mention there other references such as "induced sub-bigraph"}
    large half-graphs, a particularly \irregular\todo{Explain somewhere what this means.} structure in graphs.
    See \Cref{pic:half_graph} for an example of such a graph. \todo{Add visual example of a half-graph}

    First, stability implies a bounded \emph{Vapnik-Chervonenkis (VC) dimension}, which limits the variety of
    neighborhoods of vertices within the graph.
    While stability implies a bounded VC-dimension for the entire graph
    (See~\cite{regularity_partitions_and_the_topology_of_graphons}), our work primarily focuses on bounding
    the VC-dimension restricted to a subset of vertices.
    This is formalized in \Cref{lem:k_order_property_bounds_BAbs}.

    Second, stability implies a finite \emph{tree bound}.
    This property is the foundational tool we use to prove the existence of parts that are \regular
    with respect to the rest of the graph.
    We use this to establish the existence of indivisible parts in \Cref{sec:section_4}
    (\Cref{lem:existance_of_indivisible_sets}) and
    excellent parts in \Cref{sec:section_5} (\Cref{lem:existance_of_excellent_subsets}).

    First, we formally define stability as the non-$k$-order property, where $k$ determines the size of the
    excluded half-graphs.

    \begin{definition} \label{def:k_order_property}
        Let $G$ be a graph.
        We say that $G$ has the \emph{$k$-order property} if there exist two sequences of vertices
        $\Partriangle{a_i \mid i \in \parcurly{1, \dots, k}}$ and $\Partriangle{b_i \mid i \in \parcurly{1, \dots, k}}$ such that
        for all $i,j \leq k$, $a_i R b_j$ if and only if $i \geq j$.
        Otherwise, we say that $G$ has the \emph{non-$k$-order property} or that $G$ is \emph{$k$-stable}.
    \end{definition}

    \begin{remark}
        It is important to note what is left unspecified in \Cref{def:k_order_property}.
        First, the vertices within each sequence must be distinct, as their neighborhoods within the other sequence
        differ.
        However, the sequences themselves need not be disjoint.
        One may have $a_i=b_j$, provided $i < j$ (so that $\neg(a_i R b_j)$).
        Furthermore, the definition does not specify the presence or absence of edges within the same sequence.
        Consequently, the non-$k$-order property requires the containment of a subgraph from a broad class of structures,
        not merely a k-half-graph.
        \todo{Possibly add visual example of this too.}
    \end{remark}

    \begin{remark}
        $G$ having the $k$-order property implies that $G$ has the $k'$-order property for all $k' \leq k$.
        Conversely, $G$ having the non-$k$-order property implies that $G$ has the non-$k'$-order property for all $k' \geq k$.
    \end{remark}

    An important concept used all over the thesis is that of \emph{exceptional edges} and \emph{exceptional vertices}.
    That is, edges and vertices that, in the context of a pair of sets of vertices, do not \say{behave} as the rest.
    \todo{Echarle un ojo.}
    In order to classify what is the expected behaviour in a graph, or more specifically, in a pair of sets of vertices,
    we define the \emph{truth value}.

    \begin{definition}[Truth value] \label{def:truth_value}
        Let $G$ be a graph.
        For any (not necessarily disjoint) $A, B \subseteq G$, we say that
        \[
            t(A,B) =
            \begin{cases}
                0 & \text{if } \parstraight{\parcurly{\parround{a, b} \in A \times B \mid a R b, a \neq b}} <
                    \parstraight{\parcurly{\parround{a, b} \in A \times B \mid \neg a R b, a \neq b}} \\
                1 & \text{otherwise}
            \end{cases}
        \]
        is the \emph{truth value} of the pair $\parround{A,B}$.
        That is, $t(A,B) = 0$ if $A$ and $B$ are mostly disconnected, and $t(A,B) = 1$ if they are mostly connected.
        When $B = \parcurly{b}$, we write $t(A,b)$ instead of $t(A,\parcurly{b})$, and we say that it is the truth value of $A$
        with respect to $b$.
    \end{definition}

    In this context, we say that a vertex $a \in A$ is \emph{exceptional} with respect to $B \subseteq G$ if $t(a,B) \not\equiv t(A,B)$,
    or that it is exceptional with respect to $b \in G$ if $a R b \not\equiv t(A,b)$.
    On the other hand, we say that an edge $ab$ with $a \in A$ and $b \in B$ is exceptional in $(A,B)$ if $a R b \not\equiv t(A,B)$.
    Also, it is useful to define the following set of vertices.
    \begin{itemize}
        \item $B_{A,b} = \parcurly{a \in A \mid a R b \equiv t(A,b)}$, i.e. the set of non-exceptional vertices of $A$
            with respect to $B$.
        \item $\overline{B}_{A,b} = \parcurly{a \in A \mid a R b \not\equiv t(A,b)}$, the set of exceptional vertices of $A$ with
            respect to $B$.
        \item $B^+_{A,b} = \parcurly{a \in A \mid a R b}$, the vertices of $A$ connected to $b$.
        \item $B^-_{A,b} = \parcurly{a \in A \mid \neg a R b}$, the vertices of $A$ that are not connected to $b$.
    \end{itemize}
    With this notation, notice that either $t(A,b) = 1$ and thus $B_{A,b} = B^+_{A,b}$, or $t(A,b) = 0$ and $B_{A,b} = B^-_{A,b}$.

    Sets of vertices $A$ with a large number of large $\overline{B}_{A,b}$ are a great obstacle towards creating a \regular,
    and more specifically homogeneous \todo{Define homogeneous} partition, as the number of exceptional edges with respect
    to the entire graph is large and concentrated.
    A useful tool to deal with them is \Cref{lem:k_order_property_bounds_BAbs}, which gives a bound on the number of such sets under the
    non-$k$-order property.
    In order to prove it, we first need to introduce the \emph{VC dimension} of a family of sets, and relate it to the
    $k$-order property.
    This, together with \Cref{lem:sauer-shelah}, will give us the desired result.

    \begin{definition} \label{def:shattered}
        Let $G$ be a set and $S = \parcurly{S_i \subseteq G \mid i \in I}$ be a family of sets.
        A set $A \subseteq G$ is said to be \emph{shattered} by $S$ (and $S$ is said to \emph{shatter} $A$) if
        for every $B \subseteq A$, there exists $S_i \in S$ such that $S_i \cap A = B$.
    \end{definition}

    \begin{definition} \label{def:VC_dimension}
        Let $G$ be a set and $S = \parcurly{S_i \subseteq G \mid i \in I}$ be a family of sets.
        The \emph{VC dimension} of $S$ is the size of the largest set $A \subseteq G$ that is shattered by $S$.
    \end{definition}

    \begin{lemma}[Sauer-Shelah] \label{lem:sauer-shelah}
        Let $G$ be a set and $S = \parcurly{S_i \subseteq G \mid i \in I}$ be a family of sets.
        If the VC dimension of $S$ is at most $k$, and the union of all the sets in $S$ has $n$ elements, then
        $S$ consists of at most $\sum_{i=0}^{k} \binom{n}{i} \leq n^k$ sets.
    \end{lemma}

    We'll begin by proving a stronger version of this lemma from Pajor, for which Sauer-Shelah will be a straightforward
    consequence.

    \begin{lemma}[Sauer-Shelah-Pajor] \label{lem:pajor}
        Let $G$ be a set and $S$ be a finite family of sets in $G$.
        Then $S$ shatters at least $\parstraight{S}$ sets.
        \begin{proof}
            We will prove this by induction on the cardinality of $S$.
            If $\parstraight{S} = 1$, then $S$ consists of a single set, which only shatters the empty set.
            If $\parstraight{S} > 1$, we may choose an element $x \in S$ such that some sets of $S$ contain $x$ and some do not.
            Let $S^+ = \parcurly{s \in S \mid x \in S}$ and $S^- = \parcurly{s \in S \mid x \not\in S}$.
            Then $S = S^+ \sqcup S^-$, and both $S^+$ and $S^-$ are non-empty.
            By induction hypothesis, we know that $S^+ \subsetneq S$ shatters at least $\parstraight{S^+}$ sets,
            and $S^- \subsetneq S$ shatters at least $\parstraight{S^-}$ sets.
            Let $T, T^+, T^-$ be the families of sets shattered by $S$, $S^+$ and $S^-$ respectively.
            To conclude the proof, we just need to show that for each element in $T^+$ and $T^-$, there is a corresponding
            one in $T$.
            If a set is shattered by only one of the two families $S^+$ and $S^-$, then it only contributes by one unit
            to $\parstraight{T^+} + \parstraight{T^-}$ and one unit to $\parstraight{T}$.
            Notice that no set shattered by $S^+$ or $S^-$ may contain $x$, otherwise all or none of the intersections
            will contain this element.
            Thus, if a set $s$ is shattered by both $S^+$ and $S^-$, it will contribute by two units to
            $\parstraight{T^+} + \parstraight{T^-}$ and one unit to $\parstraight{T}$.
            But then, for each such set, we can consider $s \cup \parcurly{x}$ which is not in $T^+$ or $T^-$, but it is in $T$.
            Indeed, for each subset of $s$, if it does not contain $x$ it is the intersection with some
            set in $S^- \subsetneq S$, and if it does contain $x$ it is the intersection with some set in $S^+ \subsetneq S$.
            All in all, we conclude that
            \[
                \parstraight{T} \geq \parstraight{T^+} + \parstraight{T^-} \geq \parstraight{S^+} + \parstraight{S^-}
                                \geq \parstraight{S}
            \]
        \end{proof}
    \end{lemma}

    \begin{proof} (of \Cref{lem:sauer-shelah})
        Suppose that $\bigcup S$ has $n$ elements.
        By \Cref{lem:pajor}, $S$ shatters at least $\parstraight{S}$ subsets, and since there are at most
        $\sum_{i=0}^k \binom{n}{i}$ subsets of $S$ of size at most $k$, if
        $\parstraight{S} > \sum_{i=0}^k \binom{n}{i}$, at least one of the shattered sets has cardinality larger than $k$,
        and hence the VC dimension of $S$ is larger than $k$.
    \end{proof}

    Next, we want to prove that if $G$ has the non-$k$-order property, then the size of the family of exceptional
    sets of $A$, relative to each vertex $b \in G$, is bounded by $|A|^k$.
    Instead, we prove a stronger result, that is we prove this same bound with only the condition that $G$
    has the \say{disjoint} non-$k$-order property, in which the two sequences of vertices in the \Cref{def:k_order_property}
    are in fact disjoint.
    This stronger version is neither more useful nor easier to prove, but remarks that the non-disjointness of the sequences,
    and thus the broadening of the excluded structures, is not needed to obtain the bound, but later on.
    \todo{Lluis: pot semblar raro aixó, pero crec que aixi deixa clar que és més endavant on de veritat necessitem que
    puguin ser non disjoint.}

    \begin{lemma} \label{lem:vc_dimension_implies_k_order_property}
        Let $G$ be a graph and $A \subseteq G$.
        Let $S = \parcurly{B^+_{A,b} \mid b \in G \setminus A}$.
        If $S$ has VC dimension (at least) $k$, then $G$ has the $k$-order property.
        \begin{proof}
            If $S$ has VC dimension $k$, then it shatters a set $A' \subseteq A$ of size $k$.
            Now, choose any order of the vertices of $A' = \Partriangle{a_1, \dots, a_k}$.
            Then, consider the increasing sequence of subsets $A_1 \subseteq A_2 \subseteq \dots \subseteq A_k = A'$,
            where $A_i = \parcurly{a_j \mid j \in \parcurly{1, \dots, i}}$.
            Since $A'$ is shattered by $S$, for each $i \in \parcurly{1, \dots, k}$ there exists a $b_i \in G$ such that
            $b_i R a$ if and only if $a \in A_i$.
            In particular, the two sequences $\Partriangle{a_i \mid i \in \parcurly{1, \dots, k}}$ and
            $\Partriangle{b_i \mid i \in \parcurly{1, \dots, k}}$ satisfy
            \[
                a_i R b_j \Leftrightarrow i \leq j
            \]
            and thus $G$ has the $k$-order property.
        \end{proof}
    \end{lemma}

    \begin{lemma}[Claim 2.6] \label{lem:k_order_property_bounds_BAbs}
        Let $G$ be a graph with the non-$k$-order property.
        Then, for any finite non-trivial $A \subseteq G$,
        \[
            \parstraight{\parcurly{B^+_{A,b} \mid b \in G}} \leq |A|^k
        \]
        \begin{proof}
            By \Cref{lem:vc_dimension_implies_k_order_property}, if $G$ has the non-$k$-order property,
            then the family $\parcurly{B^+_{A,b} \mid b \in G \setminus A}$ has VC dimension at most $k-1$,
            so by the Sauer-Shelah \Cref{lem:sauer-shelah} we have
            $\parcurly{B^+_{A,b} \mid b \in G \setminus A} \leq \sum_{i=0}^{k-1} \binom{|A|}{i}$.
            Since $\parcurly{B^+_{A,b} \mid b \in A} \leq \parstraight{A}$, we conclude that
            \[
                \parstraight{S} = \parstraight{\parcurly{B^+_{A,b} \mid b \in G}} \leq \sum_{i=0}^{k-1} \binom{|A|}{i} + |A|
            \]
            Finally, when $\parstraight{A} = n,k > 1$: \todo{This conditions should be set at some point of the tfm. Specify that if they are not met, the problem becomes trivial.}
            \begin{itemize}
                \item if $n \leq k$, then $\parstraight{S} \leq 2^n \leq 2^k \leq n^k$.
                \item if $n > k$, then $\parstraight{S} \leq \sum_{i=0}^{k-1} {n \choose i} + n \leq n^{k-1} + n \leq 2n^{k-1} \leq n^k$.
            \end{itemize}
            We conclude that $\parstraight{S} \leq n^k$.
        \end{proof}
    \end{lemma}

    \begin{remark}
        The condition $n,k > 1$ is trivial.
        If $n=1$ then $A$ is the trivial graph with a single vertex.
        If $k=1$ we are not allowing even a single edge, so $G$ is the empty graph.
    \end{remark}

    We now prove the following equivalent versions of the lemma, which will be useful in the different sections of the
    thesis.
    The idea is that any choice of either the exceptional or the non-exceptional vertices set of $A$ with respect to
    each vertex $b \in G$, have the same bound.

    \begin{corollary}[Claim 2.6.1] \label{cor:k_order_propery_bounds_BAbs}
        Let $G$ be a graph with the non-$k$-order property.
        Then:
        \begin{enumerate}
            \item\label{itm:k_order_propery_bounds_BAbs.1} For any finite $A \subseteq G$
                \[
                    \parstraight{\parcurly{B^-_{A,b} \mid b \in G}}
                        \leq |A|^k
                \]
            \item\label{itm:k_order_propery_bounds_BAbs.2} For any finite $A \subseteq G$
                \[
                    \parstraight{\parcurly{\overline{B}_{A,b} \mid b \in G}}
                        \leq |A|^k
                \]
        \end{enumerate}
        \begin{proof}
        \begin{enumerate}
            \item First of all, notice that $B^+_{A,b} = A - B^-_{A,b}$, since by definition they are complementary.
                Thus, for any $b, b' \in G$, $B^+_{A,b} = B^+_{A,b'} \Leftrightarrow B^-_{A,b} = B^-_{A,b'}$.
                It follows that
                \[
                    \parstraight{\parcurly{B^-_{A,b} \mid b \in G}} =
                    \parstraight{\parcurly{B^+_{A,b} \mid b \in G}} \leq |A|^k
                \]
                where the last inequality follows from \Cref{lem:k_order_property_bounds_BAbs}.
            \item Consider the following map:
                \begin{align*}
                    \pi: \parcurly{\overline{B}_{A,b} \mid b \in G} & \longrightarrow \parcurly{B^+_{A,b} \mid b \in G} \\
                                                       \overline{B}_{A,b} & \longmapsto B^+_{A,b}
                \end{align*}
                We show that the map $\pi$ is injective.
                \todo{Should I prove it is well defined?}
                Let $b,b' \in G$ such that $\overline{B}_{A,b} = \overline{B}_{A,b'}$.
                Then, $t(A,b) = t(A,b')$, otherwise (w.l.o.g., suppose that $t(A,b) = 1$ and $t(A,b') = 0$), we would have
                \[
                    \parstraight{B^-_{A,b'}} > \parstraight{B^+_{A,b'}} = \parstraight{B^+_{A,b}} \geq
                        \parstraight{B^-_{A,b}} = \parstraight{B^-_{A,b'}}
                \]
                which is a contradiction.
                Then:
                \begin{itemize}
                    \item if $t(A,b) = t(A,b') = 1$, we have that $B_{A,b} = B^+_{A,b} = B^+_{A,b'} = B_{A,b'}$.
                    \item if $t(A,b) = t(A,b') = 0$, we have that
                    $B_{A,b} = B^-_{A,b} = A \setminus B^+_{A,b} = A \setminus B^+_{A,b'} = B^-_{A,b'} = B_{A,b'}$.
                \end{itemize}
                This proves that $\pi$ is injective.
                Hence, we have that
                \[
                    \parstraight{\parcurly{\overline{B}_{A,b} \mid b \in G}} \leq
                    \parstraight{\parcurly{B^+_{A,b} \mid b \in G}} \leq
                    \sum_{i \leq k} \binom{|A|}{i} \leq |A|^k
                \]
                This concludes the proof.
                Notice that actually $\pi$ is a bijection, but this is not needed for the proof.
        \end{enumerate}
        \end{proof}
    \end{corollary}

    During the next sections, it will be a key point proving that some sort of \say{regular} subgraphs
    (\emph{independent} in \Cref{sec:section_4} and \emph{excellent} in \Cref{sec:section_5}) exist in a given
    stable graph.
    In order to do so, a useful structure strongly related to the $k$-order property is the \emph{$k$-tree}.

    \begin{definition} \label{def:k-tree}
        A \emph{$k$-tree} is an ordered pair $H = (\overline{c},\overline{b})$ comprising:
        \begin{itemize}
            \item $\overline{c} = \parcurly{ c_\eta \mid \eta \in \parcurly{0,1}^{<k_{**}} }$, the set of \emph{nodes}.
            \item $\overline{b} = \parcurly{ b_\rho \mid \rho \in \parcurly{0,1}^{k_{**}} }$, the set of \emph{branches}.
        \end{itemize}
        satisfying that, for all $\eta \in \parcurly{0,1}^{<k_{**}}$ and $\rho \in \parcurly{0,1}^{k_{**}}$,
        if given $\ell \in \parcurly{0, 1}$ we have $\eta \frown \Partriangle{\ell} \triangleleft \rho$, then
        $\parround{b_\rho R c_\eta} \equiv \parround{\ell = 1}$.
    \end{definition}

    \todo{Make a visual example of the k-tree.}

    Similarly to stability, we can define the \emph{tree bound} of a graph to measure the level of freeness from $k$-trees
    of graph.

    \begin{definition}[Definition 2.11] \label{def:tree_bound}
        Suppose $G$ is a finite graph.
        We denote the \emph{tree bound} $k_{**} = k_{**}(G)$ as the minimal positive integer such that there is no $k_{**}$-tree
        $H = (\overline{c},\overline{b})$, where $\overline{b}$ and $\overline{c}$ are two sets of vertices of $G$.
    \end{definition}

    As mentioned earlier, the tree bound is closely related to the $k$-order property.
    The following theorem states that if a graph has a sufficiently large tree bound, then it has the $k$-order property
    and vice versa.

    \begin{theorem} \label{thm:tree_implies_order}
        If a graph $G$ has the $2^{k_{**}}$-order property, then the tree bound of $G$ is at least $k_{**} + 1$.
        On the other hand, if a graph $G$ has tree bound at least $k_{**} = 2^{k_*+1}-3$, then it has the $k_*$-order
        property.
        \begin{proof}
            For the first implication, just consider $\Partriangle{a_i \mid i \in \parcurly{1, \dots, 2^{k_{**}}-1}}$ and
            $\Partriangle{b_i \mid i \in \parcurly{0, \dots, 2^{k_{**}}-1}}$ to be the two sequences of vertices witnessing the
            $2^{k_{**}}$-order property in $G$, and thus for all $i,j \leq k$, $a_i R b_j$ if and only if $i \geq j$.
            It is straightforward to build a $k_{**}$-tree using these vertices.
            Take $\Partriangle{b_i \mid i \in \parcurly{0, \dots, 2^{k_{**}}-1}}$ to be the branches of the tree, indexing them by
            the binary decomposition of their index, and run the following construction for the nodes:
            \begin{itemize}
                \item Initiate $C_\empty = \Partriangle{a_i \mid i \in \parcurly{0, \dots, 2^{k_{**}}-2}}$.
                \item At each step $k \in \parcurly{0, k_{**}-1}$, for each $\eta \in \parcurly{0,1}^k$, take the middle
                    element of the sequence $C_\eta$ and set it to be the node $c_\eta$.
                    Then, the remaining first half of $C_\eta$ becomes the sequence $C_{\eta \frown \Partriangle{0}}$
                    and the second half is $C_{\eta \frown \Partriangle{1}}$.
            \end{itemize}
            Notice that at each step, the sequence $C_\eta$ has an odd number of elements.
            The resulting two sequences of nodes and branches form a $k_{**}$-tree.
            See \Cref{pic:tree_implies_order} for a visual example of this construction.
            \todo{Add visual example of order implies tree.}

            During the proof of the second implication, we will say that a set of nodes $N$ of a $k$-tree
            $H = (\overline{c},\overline{b})$ \emph{contains} a $k'$-tree, if there exists a map
            $f \colon \parcurly{0,1}^{<k'} \longrightarrow \parcurly{0,1}^{<k}$ such that for all $\eta, \eta' \in \parcurly{0,1}^{<k'}$,
            $c_{f(\eta)}$ and $c_{f(\eta')}$ are in $N$, and if $\eta \frown \Partriangle{i} = \eta'$ then
            $f(\eta) \frown \Partriangle{i} \triangleleft f(\eta')$, for all $i \in \parcurly{0, 1}$.

            This clearly implies that there is a $k'$-tree $H'$ with nodes in $N$ and branches in $\overline{b}$.
            Simply, for each $\eta \in \parcurly{0,1}^{k'-1}$, pick exactly two branches $b_{\rho_0}$ and $b_{\rho_1}$ such that
            $f(\eta)\frown\Partriangle{i} \triangleleft \rho_i$ for $i \in \parcurly{0,1}$.

            Also, we will use $H_i$ to denote the subtree of $H$ consisting of the nodes $c_{f(\eta)}$ and branches
            $b_{f(\rho)}$ such that $\Partriangle{i} \triangleleft \eta$ and $\Partriangle{i} \triangleleft \rho$.
            Notice that, if $H$ is an $h$-tree, $H_0$ and $H_1$ are $(h-1)$-trees, and together with the root node
            $c_{f(\emptyset)}$, they partition $H$.

            Next, we prove the following claim, which shows that we can always find a tree
            in one of the parts of a bipartition of the nodes of a larger tree.

            \begin{claim}
                For all $n, k \geq 0$, if $H$ is a $(n + k)$-tree and the nodes of $H$ are partitioned into two sets $N$ and $P$,
                then either $N$ contains an $n$-tree or $P$ contains a $k$-tree.
                \begin{proof} \it{(of claim)}
                    We prove this by induction on $n + k$.
                    Clearly, the statement is true for the trivial case $n = k = 0$.
                    Suppose $n + k > 0$.
                    Without loss of generality, we may assume that the root node $c_\emptyset$ is in $N$.
                    Let $Z_i$ be the set of nodes of $H_i$.
                    By H.I., for each $i \in \parcurly{0,1}$, either $N \cap Z_i$ contains an $(n-1)$-tree or
                    $P \cap Z_i$ contains a $k$-tree.
                    If either $P \cap Z_0$ or $P \cap Z_1$ contains a $k$-tree, then $P$ contains a $k$-tree, and we are done.
                    Otherwise, both $N \cap Z_0$ and $N \cap Z_1$ contain an $(n-1)$-tree.
                    Since $c_\emptyset$ is in $N$, the root with the two $(k-1)$-tree are in $N$ and make an $n$-tree.
                    Thus, $N$ contains an $n$-tree.
                \end{proof}
            \end{claim}

            Suppose that $G$ has tree bound at least $2^{k_*+1}-3$, and thus contains a $(2^{k_*+1}-2)$-tree.
            We show by induction on $k_*-r$, with $1 \leq r \leq k_*$, that the following scenario $S_r$ holds:
            \begin{enumerate}
                \item\label{itm:tree_implies_order.1} There are
                    \[
                        b_0, c_0, \dots, b_{q-1}, c_{q-1}, H, b_q, c_q, \dots, b_{k_*-r-1}, c_{k_*-r-1}
                    \]
                    such that:
                \item\label{itm:tree_implies_order.2} for all $i \in \parcurly{0, \dots, k_*-r-1}$, $b_i$ and $c_i$ are vertices in $G$,
                    and $H$ is a $(2^{r+1}-2)$-tree in $G$.
                \item\label{itm:tree_implies_order.3} for all $i,j \in \parcurly{0, \dots, k_*-r-1}$, $b_i R c_j \Leftrightarrow i \geq j$.
                \item\label{itm:tree_implies_order.4} if $c$ is a node of $H$, $b_i R c \Leftrightarrow i \geq q$.
                \item\label{itm:tree_implies_order.5} if $b$ is a branch of $H$, $b R c_i \Leftrightarrow i < q$.
            \end{enumerate}

            The initial case $S_{k_*}$ only requires the existence of a $(2^{k_*+1}-2)$-tree in $G$, which is the premise.
            If the final case $S_1$ is true, then we are done:
            this case assumes that $H$ is a $2$-tree, in which case there is a node $c_*$ and branch $b_*$ in $H$ which
            are connected.
            These vertices satisfy conditions \dref{itm:tree_implies_order.4} and \dref{itm:tree_implies_order.5}, so
            the sequence resulting by replacing $H$ in \dref{itm:tree_implies_order.1} by $b_*$, $c_*$ implies that $G$
            has the $k_*$-order property.

            To conclude the proof it remains to prove that if $S_r$ holds, then so does $S_{r-1}$ for $r>1$.
            Assume $S_r$.
            Fixing $h = 2^r - 2$, by \dref{itm:tree_implies_order.2} we have that $H$ is a $(2h +2)$-tree.
            For each branch $b$ of $H$ we denote $Z(b)$ the set of nodes $c$ of $H$ such that $b R c$.

            We have two cases:
            \begin{itemize}
                \item \emph{Case 1.} There is a branch $b_*$ such that $Z(b_*)$ contains an $(h+1)$-tree $H'$.
                    In that case, we can take $c_*$ to be the top node of the $(h+1)$-tree, and $H_*$ to be the
                    $h$-subtree $H'_0$.
                    Replacing $H$ in \dref{itm:tree_implies_order.1} with $H_*$, $b_*$, $c_*$ in this order, the
                    conditions for $S_{r-1}$ are satisfied.
                \item \emph{Case 2.} There is no branch $b$ such that $Z(b)$ contains an $(h+1)$-tree.
                    Now, let $c_*$ be the top node of $H$, $Z_1$ the set of nodes of $H_1$, and
                    $b_*$ any branch of $H_1$.
                    By the case assumption, $Z(b) \cap Z_1$ contains no $(h+1)$-tree, so by the claim,
                    $Z_1 \setminus Z(b)$ contains an $h$-tree $H_*$.
                    Finally, replacing $H$ in \dref{itm:tree_implies_order.1} by $b_*$, $c_*$, $H_*$ in this order, the
                    conditions for $S_{r-1}$ are satisfied.
            \end{itemize}
            In any case, $S_{r-1}$ is satisfied, and the proof is complete.
        \end{proof}
    \end{theorem}

    \begin{remark}
        The key point of the proof of the second implication of \Cref{thm:tree_implies_order} is that the found $k$-order
        does not only utilize edges and non-edges of the $k$-tree structure itself.
        Instead, it relies on the fact that, for a tall enough tree, a $k$-order must appear in some way, leveraging
        some \say{unknown} edges, independently on the choice of those.
    \end{remark}

    Given that stability of the studied graph is fixed for all proofs in the next sections, from now on we will use
    $k_*$ as the value of the non-$k$-property of the studied graphs, and $k_{**}$ for the associated tree bound.
        \newpage

    \section{Section 4} \label{sec:section_4}

    \todo{Make an intrduction explaining what is the goal of this section, and how we will reach it. Explain what is the purpose of each Claim.}

    \definition[Definition 4.2(a)]\label{def:epsilon_indivisible}
    Let $\epsilon \in (0,1)$.
    We say that $A \subseteq G$ is \emph{$\epsilon$-indivisible} if for every $B \in G$, the truth value $t = t(A,b)$ satisfies
    \[
        |\left\{ a\in A \mid a R b \not\equiv t \right\}| < |A|^{\epsilon}
    \]

    \definition[Definition 4.2(b)]\label{def:f_indivisible}
    Let $f: \mathbb{R} \longrightarrow \mathbb{R}$ be a non-decreasing function.
    We say that $A \subseteq G$ is \emph{$f$-indivisible} if for every $B \in G$, the truth value $t = t(A,b)$ satisfies
    \[
        |\left\{ a\in A \mid a R b \not\equiv t \right\}| < f(|A|)
    \]

%    \remark
%        Notice that the previous two definitions become meaningful only if $|A|^{\epsilon} \leq \frac{\parstraight{A}}{2}$ and
%        $f(|A|) \leq \frac{\parstraight{A}}{2}$ respectively.
%        In this context, the definition of the truth value $t$ is compatible with that of Definition~\ref{def:truth_value}.
%        In particular, Definition~\ref{def:epsilon_indivisible} would require that $\parstraight{A} \geq \parround{2}^{\frac{1}{1-\epsilon}}$.

    \remark
        If $f(n) = \epsilon n$, then $f$-indivisible $\equiv$ $\epsilon$-good.

    \remark
    $\epsilon$-indivisible is a much stronger condition then $\epsilon$-good.

    \todo{From here on, change all $l$'s with $\ell$}
    \todo{Change all sequences of $m$'s as in section 5}

    \lemma[Claim 4.3]\label{existance_of_indivisible_sets}
        Let $G$ be a finite graph with the non-$k_*$-property.
        Let $\left< m\ell \mid \ell \in [0, k_{**}] \right>$ be a sequence of non-zero natural numbers such that
        for all $\ell \in [k_{**}]$, $f(m_{\ell-1}) \geq m_\ell$.
        If $A \subseteq G$, $|A| = m_0$, then for some $\ell < k_{**}$ there is a subset $B \subseteq A$ of size $m_\ell$ which is
        $f$-indivisible.
        \begin{proof}
            Suppose not.
            Then we can construct the sequences $\left< b_\eta \mid \eta \in [2]^{<k} \right>$ and $\left< A_\eta \mid \eta \in [2]^{\leq k} \right>$
            on induction over $k < k_{**}$, where $k = |\eta|$, satisfying:
        \begin{enumerate}
            \item\label{itm:4.3.1} $A_{\eta^\frown \left< i \right>} \subseteq A_{\eta}$, $\forall i \in \left\{0,1\right\}$, $\forall k < k_{**}$
            \item\label{itm:4.3.2} $A_{\eta^\frown \left< 0 \right>} \cap A_{\eta^\frown \left< 1 \right>} = \emptyset$, $\forall k < k_{**}$
            \item\label{itm:4.3.3} $|A_\eta| = m_k$, $\forall k \leq k_{**}$
            \item\label{itm:4.3.4} $b_\eta \in G$ witnessing that $A_\eta$ is not $f$-indivisible, $\forall k < k_{**}$
            \item\label{itm:4.3.5} $A_{\eta^\frown \left< i \right>} \subseteq A_\eta^{(i)} = \left\{ a \in A_\eta \mid a R b_\eta \equiv (i=1) \right\}$,
                $\forall \in \left\{ 0,1 \right\}$, $\forall k < k_{**}$
        \end{enumerate}
        Let's prove the induction:
        \begin{itemize}
            \item \underline{$k=0$}.
                Consider $A_{\left< \cdot \right>} = A$, which satisfies $|A_{\left< \cdot \right>}| = m_0$ and
                $|b_{\left< \cdot \right>}|$ witnessing the non-$f$-indivisibility of $A_{\left< \cdot \right>}$.
            \item \underline{$k \Rightarrow k+1$}.
                We can assume $|A_\eta| = m_k$ and by hypothesis $A_\eta$ is not $f$-indivisible.
                So, there exists $b_\eta$ such that $A_\eta^{(i)} \geq f(m_k) \geq m_{k+1}$ (\ref{itm:4.3.4}), and we can choose
                $A_{\eta^\frown \left< i \right>} \subseteq A_\eta^{(i)}$ (\ref{itm:4.3.5}), such that
                $|A_{\eta^\frown \left< i \right>}| = m_{k+1} \forall i \in \left\{ 0,1 \right\}$ (\ref{itm:4.3.3}).
                (\ref{itm:4.3.1}) and (\ref{itm:4.3.2}) are satisfied by the definition of $A_\eta^{(i)}$.
        \end{itemize}
        Now, for all $\eta$ such that $|\eta| = k_{**}$, consider some element $a_\eta \in A_\eta$.
        Then, we have two sequences $\left< b_\eta \mid \eta \in [2]^{<k_{**}} \right>$ and $\left< A_\eta \mid \eta \in [2]^{k_{**}} \right>$
        with the property:
        \[
            \forall \rho \in [2]^{<k_{**}} \forall \eta \in [2]^{k_{**}} \text{ such that } \rho^\frown \left< i \right> \trianglelefteq
                \eta \text{, } (a_\eta R b_\rho)
        \]
        since $a_\eta \in A_\eta \subseteq A_{\rho ^\frown \left< i \right>}$.
        This contradicts the $k_{**}$ tree bound.
        \end{proof}

    \todo{since we changed $l \in [k_**]$ to $l < k_**$, you should adapt consequently all that follows this point...}
    \lemma[Claim 4.4]\label{existance_of_f_indivisible_partitions}
    Let $G$ be a finite graph with the non-$k_{*}$-order property.
    Assume $m_0 > \dots > m_{k_{**}}$ is a sequence of non-zero natural numbers and for all $l \in [k_{**}]$, $f(m_{l-1}) \geq m_l$.
    If $A \subseteq G$ with $|A| = n$, then we can find a sequence $\overline{A} = \left< A_j \mid j \in [j(*)] \right>$
        and reminder $B = A \setminus \bigcup \overline{A}$ such that:
    \begin{enumerate}
        \item \label{itm:4.4.1} For each $j \in [j(*)]$, $A_j$ is $f$-indivisible
        \item \label{itm:4.4.2} For each $j \in [j(*)]$, $|A_j| \in \left\{ m_0, \dots, m_{k_{**}-1} \right\}$
        \item \label{itm:4.4.3} $A_j \subseteq A \setminus \bigcup\left\{ A_i \mid i < j \right\}$, in particular $A_i \cap A_j = \emptyset$ $\forall i \neq j$
        \item \label{itm:4.4.4} $|B| < m_0$
    \end{enumerate}
        \begin{proof}
        Iteratively, apply Claim ~\ref{existance_of_indivisible_sets} to the remainder $A \setminus \bigcup \left\{ A_i \mid i < j \right\}$
            (\ref{itm:4.4.3}) to get an $f$-indivisible $A_j$ (\ref{itm:4.4.1}) of size $m_l$, $l \in \left\{ 0, \dots, k_{**}-1 \right\}$
            (\ref{itm:4.4.2}) until less then $m_0$ vertices are available (\ref{itm:4.4.4}).
        \end{proof}

    \lemma[Claim 4.5]\label{existance_of_ordered_epsilon_indivisible_partitions}
    Let $G$ be a graph with the non-$k_{*}$-order property.
    Assume $n \geq m_0 > \dots > m_{k_{**}}$ is a sequence of non-zero natural numbers satisfying that for all $l \in [k_{**}]$
        $\lfloor (m_{l-1})^\epsilon \rfloor = m_l$, for $\epsilon \in (0, \frac{1}{2})$.
    If $A \subseteq G$, $|A| = n$, then we can find $\overline{A} = \left< A_i \mid i \in [i(*)] \right>$ with remainder
        $B = A \setminus \bigcup \overline{A}$ such that:
    \begin{enumerate}  % If you do any changes here, update 4.10 too...
        \item \label{itm:4.5.1} For each $j \in [j(*)]$, $A_j$ is $\epsilon$-indivisible
        \item \label{itm:4.5.2} For each $j \in [j(*)]$, $|A_j| \in \left\{ m_0, \dots, m_{k_{**}-1} \right\}$
        \item \label{itm:4.5.3} $A_i \cap A_j = \emptyset$ $\forall i \neq j$
        \item \label{itm:4.5.4} $|B| < m_0$
        \item \label{itm:4.5.5} $\overline{A}$ is $\leq$-increasing
    \end{enumerate}
        \begin{proof}
            The first four clauses are direct consequence of applying Claim~\ref{existance_of_f_indivisible_partitions}
                with $f(n) = n^\epsilon$.
            By renaming the $A_i$'s in ascending-size order, we get (\ref{itm:4.5.5}).
        \end{proof}

    \remark
    In this context, if $m_{k_{**}} > k_{**}$
    \[
        n^{\epsilon^{k_{**}}} \geq m_0^{\epsilon^{k_{**}}} \geq m_1^{\epsilon^{k_{**}-1}} \geq \dots \geq m_{k_{**}} > k_{**}
    \]
    So, $n^{\epsilon^{k_{**}}} > k_{**}$.

    \lemma[Claim 4.6)]\label{average_condition_statement}
    Let $G$ be a finite graph.
    Suppose $A, B \subseteq G$ such that $A$ is $f$-indivisible, $B$ is $g$-indivisible, and $f(|A|) g(|B|) < \frac{1}{2} |B|$.
    Then, for some truth value $t = t(A,B)$ for all but $< f(|A|)$ of the $a \in A$ for all but $< g(|B|)$ of the $b \in B$
        we have that $a R b \equiv t$.
        \begin{proof}
            Since $B$ is $g$-indivisible, for each $a \in A$ there is a truth value $t_a = t(a,B)$ such that
                $\left\{ b \in B \mid a R b \not\equiv t_a \right\} < g(|B|)$.
            Let $U_i = \left\{ a \in A \mid t_a = i \right\}$ for $i \in \left\{ 0,1 \right\}$.
            If either $U_i$ satisfies $|U_i| < f(|A|)$ then the statement is true.
            Suppose not.
            Then, there are $W_i \subseteq U_i$ with $|W_i| = f(|A|)$ for $i \in \left\{ 0,1 \right\}$.
            Now, let $V = \left\{ b \in B \mid (\exists a \in W_0 \mid a R b) \vee (\exists a \in W_1 \mid \lnot a R b) \right\}$,
                i.e. the $b$'s which are an exception for some $a \in W_0 \cup W_1$.
            Then, $|V| < |W_0| g(|B|) + |W_1| g(|B|) = 2 f(|A|) g(|B|) < |B|$, where the first inequality follows the
                $g$-indivisibility of $B$.
            Finally, there is a $b_* \in B \setminus V$ such that $\forall a \in W_0$ $\lnot a R b_*$ and
                $\forall a \in W_1$ $a R b_*$ with $|W_0| = |W_1| = f(|A|)$, which contradicts the $f$-indivisibility of $A$.
        \end{proof}

    \definition
    We say that the pair $(A,B)$ with $A$ $f$-indivisible and $B$ $g$-indivisible satisfies the \emph{average condition} if
        $f(|A|) g(|B|) < \frac{1}{2} |B|$ and thus the statement of Claim~\ref{average_condition_statement} is true for
        the pair $(A,B)$.

    \remark
    The condition $f(|A|) g(|B|) < \frac{1}{2} |B|$ makes ordering of the pair $(A,B)$ matter.
    Thus,
    \[
        (A,B) \text{ has the average condition } \Rightarrow (B,A) \text{ has the average condition }
    \]

    \remark[Remark 4.7]\label{sufficient_requirement_for_average_condition}
        When $f(n) = n^\epsilon$ and $g(n) = n^\zeta$, the average condition is $|A|^\epsilon |V|^\zeta < \frac{1}{2} |B|$.

    \lemma[Claim 4.8]\label{exceptions_bound_of_epsilon_indivisible_sets}
        Let $A$ be $\epsilon$-indivisible, $B$ $\zeta$-indivisible and let the pair $(A,B)$ satisfy the average condition.
        Then, for all $\epsilon_1 \in \left( 0, 1-\epsilon \right)$, $\zeta_1 \in \left( 0, 1-\zeta \right)$, $A' \subseteq A$
            and $B' \subseteq B$ such that $|A'| \geq |A|^{\epsilon + \epsilon_1}$ and $|B'| \geq |B|^{\zeta + \zeta_1}$,
            we have that:
        \[
            \frac{|\left\{ (a,b) \in (A',B') \mid a R b \equiv \neg t(A,B) \right\}|}{|A' \times B'|} \leq
                \frac{1}{|A|}\epsilon_1 + \frac{1}{|B|}\zeta_1
        \]
        \begin{proof}
            Notice:
            \begin{itemize}
                \item There are at most $|A|^\epsilon$ elements of $A$ (hence in $A' \subseteq A$) which are exceptional
                    (in the sense of the average condition).
                \item For each $a \in A$ (hence in $A' \subseteq A$) not exceptional, there are at most $|B|^\zeta$ elements
                    $b \in B$ such that $(a,b)$ does not satisfy the truth value $t(A,B)$, i.e. that are exceptional.
            \end{itemize}
            Putting it all together:
            \[
                \begin{split}
                    \frac{|\left\{ (a,b) \in (A',B') \mid a R b \equiv \neg t(A,B) \right\}|}{|A' \times B'|}
                        &\leq \frac{|A|^\epsilon |B'| + (|A'| - |A|^\epsilon) |B|^\zeta}{|A'| |B'|} \\
                        &= \frac{|A|^\epsilon}{|A'|} + \frac{|A'| - |A|^\epsilon}{|A'|} \frac{|B|^\zeta}{|B'|} \\
                        &\leq \frac{|A|^\epsilon}{|A'|} + \frac{|B|^\zeta}{|B'|} \\
                        &\leq \frac{|A|^\epsilon}{|A|^{\epsilon + \epsilon_1}} + \frac{|B|^\zeta}{|B|^{\zeta + \zeta_1}} \\
                        &= \frac{1}{|A|^\epsilon_1} + \frac{1}{|B|^\zeta_1}
                \end{split}
            \]
        \end{proof}

    \lemma[$f$-indivisible version]\label{exceptions_bound_of_f_indivisible_sets}
        Let $A$ be $f$-indivisible, $B$ $g$-indivisible and let the pair $(A,B)$ satisfy the average condition.
        Then, for all $\epsilon_1 \in \left( 0, 1-\frac{f(|A|)}{|A|} \right)$, $\zeta_1 \in \left( 0, 1-\frac{g(|B|)}{|B|} \right)$, $A' \subseteq A$
            and $B' \subseteq B$ such that $|A'| \geq f(|A|) |A|^{\epsilon_1}$ and $|B'| \geq g(|B|) |B|^{\zeta_1}$,
            we have that:
        \[
            \frac{|\left\{ (a,b) \in (A',B') \mid a R b \equiv \neg t(A,B) \right\}|}{|A' \times B'|} \leq
                \frac{1}{|A|}\epsilon_1 + \frac{1}{|B|}\zeta_1
        \]
        \begin{proof}
            Notice:
            \begin{itemize}
                \item There are at most $f(|A|)$ elements of $A$ (hence in $A' \subseteq A$) which are exceptional
                    (in the sense of the average condition).
                \item For each $a \in A$ (hence in $A' \subseteq A$) not exceptional, there are at most $g(|B|)$ elements
                    $b \in B$ such that $(a,b)$ does not satisfy the truth value $t(A,B)$, i.e. that are exceptional.
            \end{itemize}
            Putting it all together:
            \[
                \begin{split}
                    \frac{|\left\{ (a,b) \in (A',B') \mid a R b \equiv \neg t(A,B) \right\}|}{|A' \times B'|}
                        &\leq \frac{f(|A|) |B'| + (|A'| - f(|A|)) g(|B|)}{|A'| |B'|} \\
                        &= \frac{f(|A|)}{|A'|} + \frac{|A'| - f(|A|)}{|A'|} \frac{g(|B|)}{|B'|} \\
                        &\leq \frac{f(|A|)}{|A'|} + \frac{g(|B|)}{|B'|} \\
                        &\leq \frac{f(|A|)}{f(|A|) |A|^{\epsilon_1}} + \frac{g(|B|)}{g(|B|) |B|^{\zeta_1}} \\
                        &= \frac{1}{|A|^\epsilon_1} + \frac{1}{|B|^\zeta_1}
                \end{split}
            \]
        \end{proof}

    \corollary[Corollary 4.9]
        Let $A$ and $B$ be $f$-indivisible with $f(n) = c$ and $(A,B)$ satisfy the average condition.
        Then, for all $\epsilon_1 \in (0, 1 - \frac{c}{|A|})$, $\zeta_1 \in (0, 1 - \frac{c}{|B|})$, $A' \subseteq A$ and
            $B' \subseteq B$ with $|A'| \geq c |A|^{\epsilon_1}$ and $|B'| \geq c |B|^{\zeta_1}$, we have:
        \[
            \frac{|\left\{ (a,b) \in (A',B') \mid a R b \equiv \neg t(A,B) \right\}|}{|A' \times B'|} \leq
                \frac{1}{|A|}\epsilon_1 + \frac{1}{|B|}\zeta_1
        \]
        \begin{proof}
            Use Claim~\ref{exceptions_bound_of_f_indivisible_sets} with $f(n) = c$.
        \end{proof}

    \lemma[Claim 4.10]
        Let $G$ be a finite graph wit the non-$k_{*}$-order property.
        Assume $n \geq m_0 > \dots > m_{k_{**}}$ is a sequence of non-zero natural numbers and for all $l \in [k_{**}]$,
            $\lfloor (m_{l-1})^\epsilon \rfloor = m_l$, for some $\epsilon \in (0, \frac{1}{2})$ such that $2 < (m_{k_{**}})^{1-2\epsilon}$.
        If $A \subseteq G$ with $|A| = n$, then we can find a sequence $\overline{A} = \left< A_i \mid i \in [i(*)] \right>$
            and reminder $B = A \setminus \bigcup \overline{A}$ satisfying:
        \begin{enumerate}
            \item \label{itm:4.10.1} For each $i \in [i*)]$, $A_i$ is $\epsilon$-indivisible
            \item \label{itm:4.10.2} For each $i \in [i(*)]$, $|A_i| \in \left\{ m_0, \dots, m_{k_{**}-1} \right\}$
            \item \label{itm:4.10.3} $A_i \cap A_j = \emptyset$ $\forall i \neq j$
            \item \label{itm:4.10.4} $|B| < m_0$
            \item \label{itm:4.10.5} $\overline{A}$ is $\leq$-increasing
            \item \label{itm:4.10.6} If $\zeta \in \left(0,\epsilon^{k_{**}}\right)$ then for every $i,j \in [i(*)]$ with $i < j$,
                $A \subseteq A_i$ ad $B \subseteq A_j$ such that $|A| \geq |A_i|^{\epsilon + \zeta}$ and $|B| \geq |A_j|^{\epsilon + \zeta}$
                we have that:
                \[
                    \begin{split}
                        \frac{|\left\{ (a,b) \in (A,B) \mid a R b \equiv \neg t(A_i,A_j) \right\}|}{|A \times B|}
                            &\leq \frac{1}{|A_i|}\zeta + \frac{1}{|A_j|}\zeta \\
                            &\leq \frac{1}{|A|}\zeta + \frac{1}{|B|}\zeta
                    \end{split}
                \]
        \end{enumerate}
        \begin{proof}
            The five points are direct consequence of Claim~\ref{existance_of_ordered_epsilon_indivisible_partitions}.
            Now, for any $A_i, A_j \in \overline{A}$ with $i < j$.
            By (\ref{itm:4.10.2}), there is some $l < k_{**}$ such that $|A_i| \leq |A_j| = m_l$ for some $l < k_{**}$.
            Then, it follows the condition $2 < (m_{k_{**}})^{1-2\epsilon}$ that:
            \[
                \frac{|A|^\epsilon |B|^\epsilon}{|B|}
                    \leq \frac{|B|^{2\epsilon}}{|B|}
                    = \frac{1}{|B|^{1-2\epsilon}}
                    = \frac{1}{m_l^{1-2\epsilon}}
                    \leq \frac{1}{m_{k_{**}}}
                    < \frac{1}{2}
            \]
            i.e. $|A|^\epsilon |B|^\epsilon < \frac{1}{2} |B|$ and by Claim~\ref{sufficient_requirement_for_average_condition}
                the average condition is satisfied.
            Finally, notice that $\epsilon^{k_{**}} < \epsilon < 1 - \epsilon$ since $\epsilon \in (0, \frac{1}{2})$,
                so that $\zeta \in (0, \epsilon ^ {k_{**}}) \subseteq (0, 1 - \epsilon)$ and the condition for
                Claim~\ref{exceptions_bound_of_epsilon_indivisible_sets} is satisfied.
            This gives us (\ref{itm:4.10.6}) and concludes the proof of the statement.
        \end{proof}

    \definition
    Let $A, B$ be $f$-indivisible sets with $f(A) \times f(B) < \frac{1}{2} |B|$.
    Let $\left< A_i \mid i < i_A \right>$ be a partition of $A$ with $|A_i| = m \forall i<i_A$ and
        $\left< B_i \mid i < i_B \right>$ be a partition of $B$ with $|B_i| = m \forall i<i_B$.
    $\varepsilon^+_{A_i,A_j,m}$ is the event:
    \[
        \forall a \in A_i \forall b \in B_i, a R b = t(A,B)
    \]

    \lemma[Claim 4.13]\label{bound_on_the_probability_of_a_subpair_having_no_exceptions}
        Let $G$ be a finite graph wit the non-$k_{*}$-order property.
        Assume $n \geq m_0 > \dots > m_{k_{**}}$ is a sequence of non-zero natural numbers and for all $l \in [k_{**}]$,
            $\lfloor (m_{l-1})^\epsilon \rfloor = m_l$, for some $\epsilon \in (0, \frac{1}{2})$ such that $2 < (m_{k_{**}})^{1-2\epsilon}$.
        Let $A_1, A_2 \subseteq G$ two $\epsilon$-indivisible subsets such that $|A_1| = m_{l_1}$ and $|A_2| = m_{l_2}$
            for some $l_1, l_2 < k_{**}$ and $|A_1| \leq |A_2|$.
        We will assume some approximation error by supposing $m_l = (m_{l-1})^\epsilon$.
        Suppose that, for some $c \in (0, 1-\epsilon)$ and $\zeta \leq \frac{1 - \epsilon - c}{3}\epsilon^{k_{**}}$,
            $m = n^\zeta$ divides $|A_1|$ and $|A_2|$.
        Then, let $\left< A_{1,s} \mid s \in \left[ \frac{|A_1|}{m} \right] \right>$ and
            $\left< A_{2,t} \mid t \in \left[ \frac{|A_2|}{m} \right] \right>$ be random partitions of $A_1$ and $A_2$
            respectively, with pieces of size $m$.
        We have that
        \[
            P(\varepsilon^+_{A_{1,s},A_{2,t},m}) \geq 1 - \frac{2}{n^{c\epsilon^{k_{**}}}}
        \]
        \begin{proof}
            Fix $s \in \frac{|A_1|}{m}$, $t \in \frac{|A_2|}{m}$.
            % TODO: copy the necessary section in Claim 4.10 or make it a separate claim/remark and reference it here.
            % (to prove average condition is satisfied)
            \[
                \text{UPS, something is missing here}
            \]
            \dots and thus the average condition is satisfied.
            Let $U_1 = \left\{ a \in A_1 \mid |\left\{ b \in A_2 \mid a R b \equiv \neg t(A_1, A_2) \right\}| \geq |A_2|^\epsilon \right\}$
                and for each $a \in A_1 \setminus U_1$ let $U_{2,a} = \left\{ b \in A_j \mid a R b \equiv \neg t(A_1, A_2) \right\}$
            By Claim~\ref{average_condition_statement}, $|U_1| \leq |A_1|^\epsilon$ and $\forall a \in A_1 \setminus U_1$,
                $|U_2| \leq |A_2|^\epsilon$.
            Now, we can bound the probability $P_1$ that $A_{1,s} \cap U_1 \neq \emptyset$ as follows:
            \[
                \begin{split}
                    P_1
                        & \leq \frac{|U_1|}{|A_1|} + \dots + \frac{|U_1|}{|A_1|-m+1}
                            < \frac{m |U_1|}{|A_1| - m}
                            \leq \frac{m |A_1|^\epsilon}{|A_1| - m} \\
                        & \leq \frac{m^2 |A_1|^\epsilon}{|A_1|}
                            = \frac{m^2}{|A_1|^{1-\epsilon}}
                            = \frac{n^{2 \zeta}}{n^{(1-\epsilon)\epsilon^{l_1}}} \\ % the approximation error is here
                        & \leq \frac{n^{2\frac{1-\epsilon-c}{3} \epsilon^{k_{**}}}}{n^{(1-\epsilon)\epsilon^{k_{**}}}}
                            \leq \frac{n^{(1-\epsilon-c) \epsilon^{k_{**}}}}{n^{(1-\epsilon)\epsilon^{k_{**}}}}
                            = \frac{1}{n^{c \epsilon^{k_{**}}}}
                \end{split}
            \]
            The forth inequality comes from the fact that $\frac{(|A_i| - m) m}{|A_i|} \geq 1$.
            Then, if $A_{1,s} \cap U_1= \emptyset$, we have that $|\bigcup_{a \in A_{1,s}} U_{2,a}| \leq |A_{1,s}| |A_2|^\epsilon$.
            So we can bound $P_2$, the probability that $A_{2,t} \cap \bigcup_{a \in A_{1,s}} U_{2,a} = \emptyset$, by:
            \[
                \begin{split}
                    P_2
                        & \leq \frac{|\bigcup_{a \in A_{1,s}} U_{2,a}|}{|A_2|} + \dots + \frac{|\bigcup_{a \in A_{1,s}} U_{2,a}|}{|A_2|-m+1}
                            < \frac{m |\bigcup_{a \in A_{1,s}} U_{2,a}|}{|A_2| - m}
                            \leq \frac{m m |A_2|^\epsilon}{|A_2| - m} \\
                        & \leq \frac{m^3 |A_2|^\epsilon}{|A_2|}
                            = \frac{m^3}{|A_2|^{1-\epsilon}}
                            = \frac{n^{3 \zeta}}{n^{(1-\epsilon)\epsilon^{l_2}}} \\ % the approximation error is here
                        & \leq \frac{n^{3\frac{1-\epsilon-c}{3} \epsilon^{k_{**}}}}{n^{(1-\epsilon)\epsilon^{k_{**}}}}
                            \leq \frac{n^{(1-\epsilon-c) \epsilon^{k_{**}}}}{n^{(1-\epsilon)\epsilon^{k_{**}}}}
                            = \frac{1}{n^{c \epsilon^{k_{**}}}}
                \end{split}
            \]
            Putting it all together:
            \[
                P(\varepsilon^+_{A_{1,s},A_{2,t},m})
                    \geq (1 - P_1) (1 - P_2)
                    \geq \left(1 - \frac{1}{n^{c \epsilon^{k_{**}}}}\right)^2
                    \geq 1 - \frac{2}{n^{c\epsilon^{k_{**}}}}
            \]
        \end{proof}

    \remark\label{subpair_bound_specification}
    Since $\epsilon < \frac{1}{2}$, we can take $c = 1 - 2\epsilon$.
    In this context, $\zeta \leq \frac{\epsilon^{k_{**}+1}}{3}$.

    \lemma[Claim 4.14]\label{existance_of_equitative_partition_with_bound_exceptional_pairs}
        Let $G$ be a finite graph wit the non-$k_{*}$-order property.
        Assume $n \geq m_0 > \dots > m_{k_{**}}$ is a sequence of non-zero natural numbers and for all $l \in [k_{**}]$,
            $\lfloor (m_{l-1})^\epsilon \rfloor = m_l$, for some $\epsilon \in (0, \frac{1}{2})$ such that $2 < (m_{k_{**}})^{1-2\epsilon}$.
        Also, let $m_0$ be small enough to satisfy $m_0 < \frac{n}{n^{(1 - 2\epsilon)\epsilon^{k_{**}}}}$ and
            $m_0 \leq \frac{\sqrt{2}-1}{\sqrt{2}} n$.
        Finally, let $m_{**}$ be a divisor of $m_l$ for all $l < k_{**}$ and $m_{**} \leq n^{\frac{\epsilon^{k_{**}+1}}{3}}$.
        % TODO: probably, this last condition can be removed.
        If $A \subseteq G$ with $|A| = n$, then we can find a partition $\overline{A} = \left< A_i \mid i \in [r] \right>$
            with reminder $B = A \setminus \bigcup \overline{A}$ such that:
        \begin{enumerate}
            \item\label{itm:4.14.1} $|A_i| = m_{**} \forall i \in [r]$
            \item\label{itm:4.14.2} For all but $\frac{2r^2}{n^{(1-2\epsilon)\epsilon^{k_{**}}}}$ of the pairs
                $(A_i, A_j)$ with $i<j$ there are no exceptional edges, i.e.
                \[
                    \left\{ (a,b) \in A_i \times A_j \mid a R b \not\equiv t(A_i, A_j) \right\} = \emptyset
                \]
            \item\label{itm:4.14.3} $|B| < m_0$
        \end{enumerate}
        \begin{proof}
            We can use Claim~\ref{existance_of_ordered_epsilon_indivisible_partitions} to get a partition
                $\overline{A'} = \left< A'_i \mid i \in [i(*)] \right>$ and remainder $B' = A \setminus \bigcup A'$.
            We can refine the partition by randomly splitting each $A'_i$ into pieces of size $m_{**}$ (\ref{itm:4.14.1}).
            Consider the resulting partition $\overline{A} = \left< A_i \mid i \in [r] \right>$ with remainder $B = B'$
                (\ref{itm:4.14.3}).
            First of all, notice that for each pair $(A_i, A_j)$ such that $A_i \subseteq A'_{i_1}$ and
                $A_j \subseteq A'_{j_1}$ with $i_1 \neq j_1$, the probability of the pair having exceptional edges is
                upper bounded by $\frac{2}{n^{(1-2\epsilon)\epsilon^{k_{**}}}}$.
            This follows Claim~\ref{bound_on_the_probability_of_a_subpair_having_no_exceptions} in the context of
                Remark~\ref{subpair_bound_specification}.
            Thus, given $X$ the random variable counting the number of exceptional pairs of this kind, we have
            \[
                E(X) = \sum_{\substack{A_i,A_j \text{ s.t.}\\A_i\subseteq A'_{i_1},A_j\subseteq A'_{j_1}\\i_1\neq j_1}} E(X_{A_i, A_j})
                    = \sum_{\substack{A_i,A_j \text{ s.t.}\\A_i\subseteq A'_{i_1},A_j\subseteq A'_{j_1}\\i_1\neq j_1}} P(\varepsilon_{A_i, A_j,m_{**}})
                    \leq \frac{r^2}{2} \frac{2}{n^{(1-2\epsilon)\epsilon^{k_{**}}}}
            \]
                where $X_{A_i,A_j}$ is the random variable giving $1$ if $(A_i, A_j)$ is exceptional, and $0$ otherwise.
            Now, we have no control if $i_1 = j_1$, so let's bound how many of these we have:
            \[
                \begin{split}
                    |\left\{ \text{Esceptional } (A_i, A_j) \mid A_i, A_j \subseteq A'_{i_1}, i_1 \in [i(*)] \right\}|
                        & \leq {\frac{m_0}{m_{**}} \choose 2} \frac{n}{m_0} \\
                        & \leq \frac{{\frac{m_0}{m_{**}} \choose 2}^2}{2} \frac{n}{m_0}
                            = \frac{m_0 n}{2 m_{**}^2}
                            = \frac{m_0}{n} \left( \frac{n}{\sqrt{2}m_{**}} \right)^2 \\
                        & \leq \frac{m_0}{n} \left( \frac{n - m_0}{m_{**}} \right)^2
                            \leq \frac{m_0}{n} r^2
                            < \frac{r^2}{n^{(1-2\epsilon) \epsilon^{k_{**}}}}
                \end{split}
            \]
            Putting it all together, we see that the number of exceptional pairs is upper bounded by
                $\frac{2r^2}{n^{(1-2\epsilon)\epsilon^{k_{**}}}}$ satisfying (\ref{itm:4.14.2}).
        \end{proof}

    \remark[Remark 4.15]
        Notice that, in the previous proof, the condition $m_0 < \frac{n}{n^{(1-2\epsilon)\epsilon^{k_{**}}}}$ can be
            weakened at the cost of increasing the number of exceptional pairs.
        More specifically, since this condition is only used to bound the exceptional sub-pairs in the same pair
            (the second part of the proof), the number of exceptional pairs can be generally bounded by
            \[
                |\left\{ \text{Exceptional pairs} \right\}|
                    \leq \left( \frac{m_0}{n} + \frac{2}{n^{(1-2\epsilon)\epsilon^{k_{**}}}} \right) r^2
            \]

    \theorem[Theorem 4.16]
        Let $\epsilon = \frac{1}{r} \in \left( 0, \frac{1}{2} \right)$ with $r \in \mathbb{N}$ (this avoids rounding error)
            and $k_*$ be given.
        Let $G$ be a finite graph with the non-$k_*$-order property.
        Let $A \subseteq G$ with $|A| = n$.
        Then, for any $m_{**} \leq n^{\frac{\epsilon^{k_{**}+1}}{3}}$, there is a partition
            $\overline{A} = \left< A_i \mid i \in [m] \right>$ of $A$ with remainder $B = A \setminus \bigcup \overline{A}$
            such that:
        \begin{enumerate}
            \item\label{itm:4.16.1} $|A_i| = m_{**} \forall i \in [m]$
            \item\label{itm:4.16.2} $|B| < n^{\frac{\epsilon}{3}}$
            \item\label{itm:4.16.3} $|\left\{ (i,j) \mid i,j \in [m], i < j \text{ and }
                \left\{ (a,b) \in A_i \times A_j \mid a R b \right\} \notin
                \left\{ A_i \times A_j, \emptyset \right\} \right\}|
                \leq \frac{2}{n^{(1-2\epsilon)\epsilon^{k_{**}}}} m^2$
        \end{enumerate}
        \begin{proof}
            Let $m_{k_{**}}$ be the smaller multiple of $m_{**}$ such that $2 < (m_{k_{**}})^{1-2\epsilon}$.
            Then, consider the sequence
                \[
                    m_{**} \leq m_{k_{**}} < \dots < m_0
                \]
                such that for all $l \in [k_{**}]$ we have that $m_{l-1} = m_l^r$.
            Notice that:
            \begin{enumerate}
                \item $m_{**}$ divides $m_l$ for all $l \in [0, k_{**}]$ since the $m_l$'s are powers of $m_{k_{**}}$
                    and $m_{**}$ divides $m_{k_{**}}$ by construction.
                \item $(m_{l-1})^\epsilon = m_l \forall l \in [k_{**}]$
                \item \[
                    \begin{split}
                        \underline{m_0}
                            & = m_{k_{**}}^{r^{k_{**}}}
                                \leq m_{**}^{r^{k_{**}}}
                                \leq n^{\frac{\epsilon}{3}\epsilon^{k_{**}}r^{k_{**}}}
                                = \underline{n^{\frac{\epsilon}{3}}} \\
                            & < n^{\frac{1}{6}}
                                < n^{1-\frac{1}{2}\epsilon^{k_{**}}}
                                = \underline{\frac{n}{n^{\frac{1}{2}\epsilon^{k_{**}}}}}
                                < \underline{n}
                    \end{split}
                \]
            \end{enumerate}
            So, all the conditions are satisfied to apply Claim~\ref{existance_of_equitative_partition_with_bound_exceptional_pairs},
                which gives us the partition $\overline{A}$ with remainder $B$ satisfying the statement.
            Notice that (\ref{itm:4.16.2}) is satisfied by the fact that $|B| < m_0 \leq n^{\left(\frac{1}{6} - \frac{\epsilon}{3}\right)}$.
        \end{proof}
    % TODO: make a complementary of this theorem with another value of c, which removes the need of (1-2\epsilon) in (3).

    \remark
    Let $n^{\frac{\epsilon^{k_{**}+1}}{3}}$ be an integer and let $m_{**}$ take this value.
    Then, the number of pieces of the partition is at most $n^c$ with $c = 1 - \frac{\epsilon^{k_{**}+1}}{3}$.

    \definition[Definition 4.18]\label{n_large_enough_property}
    For $n, c \in \mathbb{N}$ and $\epsilon, \zeta, \xi \in \mathbb{R}$, let $\oplus[n, \epsilon, \zeta, \xi, c]$ be
        the statement:
    For any set $A$ and $P \subseteq \mathcal{P}(A)$ such that $|A| = n$, $|P| \leq n^{\frac{1}{\zeta}}$ and for all
        $B \in P$ $|B| \leq n^\epsilon$, there exists $U \subseteq A$ with $|U| = \lfloor n^\xi \rfloor$ such that
        for all $B \in P$ $|U \cap B| \leq c$.

    \lemma[Lemma 4.19]\label{n_large_enough_valid_values}
        If the reals $\epsilon, \zeta, \xi$ and the natural numbers $n, c$ satisfy:
        \begin{itemize}
            \item $\epsilon \in (0,1)$
            \item $\zeta > 0$
            \item $0 < \xi < \min(1-\epsilon, \frac{1}{2})$
            \item $n$ sufficiently large ($n > n(\epsilon, \zeta, \xi, c)$) to satisfy the equation:\[
                \frac{1}{2n^{1-2\xi}} + \frac{1}{n^{(1 - \xi - \epsilon)c - \frac{1}{\zeta}}} < 1
            \]
            \item $c > \frac{1}{\zeta (1 - \xi - \epsilon)}$
        \end{itemize}
            then $\oplus[n, \epsilon, \zeta, \xi, c]$ holds.
        \begin{proof}
            Let $m = \lfloor n^\xi \rfloor$ the size of the set $U$ we want to build, and let $\mathcal{F}_* = [A]^m$
                the set of sequences of elements of $A$ with length $m$.
            Let $\mu$ be a probability distribution on $\mathcal{F}_*$ such that for all $F \in \mathcal{F}_*$
                $\mu(F) = \frac{|F|}{|\mathcal{F}_*|}$.
            We want to prove that the probability that a random $U$ satisfies:
            \begin{enumerate}
                \item\label{itm:4.19.1} All elements of $U$ are distinct
                \item\label{itm:4.19.2} For all $B \in P$ $|U \cap B| < K$
            \end{enumerate}
                is not trivial.
            First of all let's bound the converse (\ref{itm:4.19.1}) i.e. the probability that there are two equal elements
                in $U$:
            \[
                P_1 = P(\exists s < t \in [m] \mid U_s = U_t)
                    \leq {m \choose 2} \frac{n}{n^2}
                    \leq \frac{m^2}{2n}
                    \leq \frac{n^{2\xi}}{2n}
                    < \frac{1}{2n^{1-2\xi}}
            \]
            Now, in order to bound (\ref{itm:4.19.2}), let's first bound the probability that at least $c$ elements of
                $U$ are in a given $B \in P$:
            \[
                P_B = P(\exists^{\geq c} t\in [m] \mid U_t \in B)
                    \leq {m \choose c} \left( \frac{|B|}{n} \right)^c
                    \leq \frac{m^c |B|^c}{n^c}
                    \leq \frac{n^{\xi c} n^{\epsilon c}}{n^c}
                    = \frac{1}{n^{c (1 - \xi - \epsilon)}}
            \]
            Then, we can bound the converse of (\ref{itm:4.19.2}), i.e. the probability that this happens for some $B \in P$,
                by:
            \[
                P_2 = P(\exists B \in P \mid \exists^{\geq c} t\in [m], U_t \in B)
                    \leq \sum_{B \in P} P_B
                    = \frac{|P|}{n^{c (1 - \xi - \epsilon)}}
                    \leq \frac{1}{n^{c (1 - \xi - \epsilon) - \frac{1}{\xi}}}
            \]
            Putting it all together, we have that
            \[
                P((\ref{itm:4.19.1}) \cup (\ref{itm:4.19.2}))
                    \leq P_1 + P_2
                    < \frac{1}{2n^{1-2\xi}} + \frac{1}{n^{c (1 - \xi - \epsilon) - \frac{1}{\xi}}}
            \]
            Notice that
            \begin{itemize}
                \item Since $\xi < \frac{1}{2}$ we have that $1 - 2\xi > 0$
                \item Since $\xi < 1 - \epsilon$, we have that $1 - \epsilon - \xi > 0$ and given that $c$ is natural
                    $c (1 - \xi - \epsilon) > 0$
            \end{itemize}
                so, the $n$-large enough condition of the forth point of the statement is well defined and
            \[
                P((\ref{itm:4.19.1}) \cup (\ref{itm:4.19.2}))
                    < \frac{1}{2n^{1-2\xi}} + \frac{1}{n^{c (1 - \xi - \epsilon) - \frac{1}{\xi}}}
                    < 1
            \]
            Thus, the probability that there exists a $U \subseteq A$ satisfying the condition is non-trivial,
                and $\oplus[n, \epsilon, \zeta, \xi, c]$ holds.
        \end{proof}

    \lemma[Claim 4.21]\label{many_values_to_equitative_partition_with_bound_exceptional_pairs}
        Let $k_*, k, c \in \mathbb{N}$ and $\epsilon, \xi \in \mathbb{R}$ such that:
        \begin{enumerate}
            \item\label{itm:4.21.1} $G$ is a graph with the non-$k_*$-order property
            \item\label{itm:4.21.2} $A \subseteq G$ implies $|\left\{ \left\{ a \in A \mid a R b \equiv t(a,b) \right\} \mid b \in G \right\}|
                \leq |A|^k$
            \item\label{itm:4.21.3} $\epsilon \in \left(0, \frac{1}{2}\right)$
            \item\label{itm:4.21.4} $\xi \in \left(0, \frac{\epsilon^{k_{**}}}{2} \right)$
            \item\label{itm:4.21.5} $c$ satisfies \[
                c > \frac{1}{\frac{1}{k} (1 - \frac{\xi}{\epsilon^{k_{**}}} - \epsilon)}
            \]
        \end{enumerate}
        Then, for every sufficiently large $n \in \mathbb{N}$ ($n^{\epsilon^{k_{**}}} > n\left( \epsilon, \frac{1}{k},
            \frac{\xi}{\epsilon^{k_{**}}}, c \right)$ in the sense of Lemma~\ref{n_large_enough_valid_values} (d)), if
            $A \subseteq G$ with $|A| = n$, then there is $Z \subseteq A$ such that
        \begin{enumerate}[label=(\alph*), ref=\alph*]
            \item\label{itm:4.21.a} $|Z| = \lfloor n^\xi \rfloor$
            \item\label{itm:4.21.b} $Z$ is $\epsilon$-indivisible in $G$
        \end{enumerate}
        \begin{proof}
            In order to simplify the calculation, we will assume that $n^{\epsilon^l} \in \mathbb{N} \forall l \leq k_{**}$.
            Notice that can be easily achieved by setting $\epsilon$ as $\epsilon = \frac{1}{r}$ with $r \in \mathbb{N}$.
            Let $n = m_0 > m_1 > \dots > m_{k_{**}}$ with $m_l = n^{\epsilon^{l}}$.
            So $m_{l+1} = m_l^\epsilon = \lfloor (m_l)^\epsilon \rfloor$ and we can use Claim~\ref{existance_of_indivisible_sets}
                to get $A_1 \subseteq A$ with $|A_1| = m_l$ for some $l \leq k_{**}$ and $A_1$ $\epsilon$-indivisible.
            By (\ref{itm:4.21.2}) we have that $|P_1| \leq |A_1|^k = m_l^k$.
            Notice that:
            \begin{itemize}
                \item $\epsilon \in (0,1)$ by (\ref{itm:4.21.3})
                \item $\zeta \coloneqq \frac{1}{k} > 0$
                \item since $\epsilon \in \left( 0, \frac{1}{2} \right)$ by (\ref{itm:4.21.3}), then by (\ref{itm:4.21.4})
                    $\frac{\xi}{\epsilon^l} \leq \frac{\xi}{\epsilon^{k_{**}}} < \frac{1}{2} < 1 - \epsilon$ and
                    thus $0 < \xi < \min(1-\epsilon, \frac{1}{2})$
                \item $m_l$ sufficiently large: $m_l = n^{\epsilon^l} \geq n^{\epsilon^{k_**}} > n\left( \epsilon, \frac{1}{k},
                    \frac{\xi}{\epsilon^{k_{**}}}, c \right) > n\left( \epsilon, \zeta, \frac{\xi}{\epsilon^{l}}, c \right)$
                \item $c > \frac{1}{\frac{1}{k} (1 - \frac{\xi}{\epsilon^{k_{**}}} - \epsilon)}
                    \geq \frac{1}{\zeta (1 - \frac{\xi}{\epsilon^{k_{**}}} - \epsilon)}$
            \end{itemize}
            By Lemma~\ref{n_large_enough_valid_values} then, $\oplus\left[ m_l, \epsilon, \zeta, \frac{\xi}{\epsilon^l} \right]$
                holds, and by taking $A_{(\ref{n_large_enough_property})} \coloneqq A_1$ and
                $P_{(\ref{n_large_enough_property})} \coloneqq P_1$ we have that:
            \begin{itemize}
                \item $|A_1| = m_l$
                \item $|P_1| \leq m_l^k = m_l^{\frac{1}{\zeta}}$
                \item $\forall B \in P_1$, $|B| \leq |A_1|^\epsilon$ by $\epsilon$-indivisibility of $A_1$
            \end{itemize}
            Thus, by Definition~\ref{n_large_enough_property} we have that there exists $Z \subseteq A_1$ such that:
            \begin{itemize}
                \item $|U| = \lfloor m_l^{\frac{\xi}{\epsilon^l}} \rfloor = \lfloor n^{\epsilon^l \frac{\xi}{\epsilon^l}} \rfloor
                    \lfloor n^\xi \rfloor$ satisfying (\ref{itm:4.21.a})
                \item $Z$ is $c$-indivisible since $|B \cap Z| \leq c \forall B \in P_1$, satisfying (\ref{itm:4.21.b})
            \end{itemize}
            This proves the statement.
        \end{proof}

    \lemma[Remark 4.22]\label{k_asterisk_enough_for_k}
    Notice that if $k = k_*$, the condition (\ref{itm:4.21.2}) will be satisfied by Claim ??? % TODO: add the claim
        and the non-$k_*$-order of $G$.

    \theorem[Theorem 4.23]
        Let $G$ be a graph with the non-$k_*$-property.
        For any $c \in \mathbb{N}$, $\epsilon, \xi \in \mathbb{R}$ satisfying the hypothesis of Claim~\ref{many_values_to_equitative_partition_with_bound_exceptional_pairs}
            (with $k = k_*$ and $\zeta = \frac{1}{k_*}$), any $\theta \in (0,1)$ and $A \subseteq G$ with
            $A = n > n\left( c, \epsilon, \zeta, \xi, \theta \right)$ (i.e. $n$ large enough in the sense of Claim~\ref{n_large_enough_valid_values}),
            there is a partition $\overline{A} = \left< A_i \mid i \in [i(*)] \right>$ of $A$ with remainder $B = A \setminus \bigcup \overline{A}$
            satisfying:
        \begin{itemize}
            \item $|A_i| = \lfloor \lfloor n^\theta \rfloor ^\zeta \rfloor \forall i \in [i(*)]$
            \item $A_i$ is $c$-indivisible $\forall i \in [i(*)]$ where $c$ is the constant function $f(x) = c$
            \item $|B| < \lfloor n^{\frac{\theta}{\epsilon^{k_{**}}}} \rfloor$
        \end{itemize}
        \begin{proof}
            Let $n > \left( n\left( \epsilon, \frac{1}{k_*}, \frac{\xi}{\epsilon^{k_{**}}}, c \right)^{\frac{1}{\epsilon^{k_{**}}} + 1} \right)^{\frac{1}{\theta}}$
                in the sense of Lemma~\ref{n_large_enough_valid_values}, so that $\lfloor n^\theta \rfloor$ satisfies the
                large enough condition of Claim~\ref{many_values_to_equitative_partition_with_bound_exceptional_pairs}:
                \[
                    \left( \lfloor n^\theta \rfloor \right)^{\epsilon^{k_{**}}}
                        > n\left( \epsilon, \frac{1}{k_*}, \frac{\xi}{\epsilon^{k_{**}}}, c \right)
                \]
            Notice that condition (\ref{itm:4.21.2}) in Claim~\ref{many_values_to_equitative_partition_with_bound_exceptional_pairs}
                is satisfied by Remark~\ref{k_asterisk_enough_for_k}.
            Now, we define a decreasing sequence $m_0 > m_1 > \dots > m_{k_{**}}$ with $m_{k_{**}} = \lfloor n^\theta \rfloor$
                and $m_{k_{**}-j} = \lceil \left( m_{k_{**}-j+1} \right)^{\frac{1}{\epsilon}} \rceil \forall j \in [1, k_{**}]$.
            This sequence satisfies the condition of Claim~\ref{existance_of_indivisible_sets} for $f(n) = n^\epsilon$.
            We will build a sequence of disjoint $c$-indivisible subsets $A_i$ by induction on $i$ as follows.
            Let $R_i = A \setminus \bigcup_{j<i} A_j$ (so $R_1 = A$).
            If $R_i < \lfloor n^{\frac{\theta}{\epsilon^{k_{**}}}} \rfloor$, then
                $\overline{A} = \left< A_j \mid j < i = i(*) \right>$ and $B = R_i$, and we are done.
            Otherwise, we can apply Claim~\ref{existance_of_indivisible_sets} to $R_i$ with the sequence
                $\left< m_l \right>_{l \leq k_{**}}$, to obtain an $\epsilon$-indivisible subset $B_i \subseteq R_i$ of
                size $m_{k_{**}-l}$.
            Then, since $|B_i| = m_{k_{**}-l} \geq m_{k_{**}} = \lfloor n^\theta \rfloor$ by the $n$-large-enough assumption,
                we can apply Claim~\ref{many_values_to_equitative_partition_with_bound_exceptional_pairs} and get a
                $c$-indivisible subset $Z_i$ of size $|Z_i| = \lfloor m_{k_{**}-l}^\zeta \rfloor
                \geq \lfloor \lfloor n^{\frac{\theta}{\epsilon^l}} \rfloor ^\zeta \rfloor
                \geq \lfloor \lfloor n^{\theta} \rfloor ^\zeta \rfloor$.
            Since $c$-indivisible is preserved when taking subsets, % TODO: make it a remark
                we can choose $A_i \subseteq Z_i$ $c$-indivisible of size $\lfloor \lfloor n^{\theta} \rfloor ^\zeta \rfloor$.
        \end{proof}

    % up to here it was compiling with no problem.

        \newpage

    \section{Section 5} \label{sec:section_5}

    \definition[Definition 5.2(a)]
        Let $G$ be a finite graph with the non-$k_*$-property.
        We say that $A \subseteq G$ is $\epsilon$-\emph{good} when for every $b \in G$ the truth value
        $t = t(b, A) \in \left\{ 0, 1 \right\}$ satisfies $|\left\{ a\in A \mid aRb \not\equiv t \right\}| < \epsilon |A|$.

    \definition[Definition 5.2(b)]
        Let $G$ be a finite graph with the non-$k_*$-property.
        We say that $A \subseteq G$ is $(\epsilon, \zeta)$-\emph{excellent} when $A$ is $\epsilon$-good and, if $B$ is
        $\zeta$-good, then the truth value $t = t(B,A)$ satisfies $|\left\{ a \in A \mid t(a,B) \neq t(A,B) \right\}| < \epsilon |A|$.

        In particular, we say $A$ is $\epsilon$-\emph{excellent} if $A$ is $(\epsilon, \epsilon)$-excellent.

    \begin{remark}\label{remark_excellence_imply_little_exceptions}
        Notice that, if $A, B \subseteq G$ are two (not necessarily disjoint) subsets of vertices
        with $A$ $(\epsilon, \epsilon')$-excellent and $B$ $\epsilon'$-good set, then the number of exceptional edges between $A$ and $B$,
        i.e. these vertex pairs that do not follow $t(A,B)$, is relatively small:
        $$
            \parstraight{\parcurly{\text{Exceptional edges between } A \text{ and } B}} <
                \epsilon |A| |B| + (1- \epsilon) |A| \epsilon' |B| = \parround{\epsilon + \parround{1-\epsilon} \epsilon'} |A| |B|
        $$
        A relevant example is that of two disjoint $\epsilon$-excellent sets, in which case we have that the fraction
        of exceptional edges between them is less than $2\epsilon$.
        If they are not disjoint, we can still use the same reasoning to conclude that the fraction of exceptional edges
        is less than $2 \epsilon \frac{\parstraight{A} \parstraight{B}}{e(A,B)} < 8 \epsilon$, since
        $e(A,B) > \frac{\parstraight{A} \parstraight{B}}{4}$.
        \todo{Discuss with Luis, this may be reduced but I am not sure.}
    \end{remark}

    \lemma[Claim 5.4]\label{existance_of_excellent_subsets}
        Let $G$ be a finite graph with the non-$k_{*}$-order property.
        Let $\zeta \leq \frac{1}{2^{k_{**}}}$, $\epsilon \in \parround{0, \frac{1}{2}}$.
        Then, for every $A \subseteq G$ with $|A| \geq \frac{1}{\epsilon^{k_{**}}}$ there exists an $(\epsilon, \zeta)$-excellent
        subset $A' \subseteq A$ such that $|A'| \geq \epsilon^{k_{**}-1} |A|$.
        \begin{proof}
            Suppose the converse.
            We use this fact to build sets $\left\{ b_\eta \mid \eta \in [2]^{<k_{**}} \right\}$ and
            $\left\{ A_\eta \mid \eta \in [2]^{\leq k_{**}} \right\}$ on induction over $k<k_{**}$, where $k = |\eta|$,
            satisfying:
            \begin{enumerate}
                \item\label{itm:5.4.1} $A_{\left< \cdot \right>} = A$.
                \item\label{itm:5.4.2} $B_\eta$ is a $\zeta$-good set witnessing that $A_\eta$ is not
                    $(\epsilon, \zeta)$-excellent, for $k < k_{**}$.
                \item\label{itm:5.4.3} $A_{\eta \frown \left< i \right>} = \left\{ a \in A_\eta \mid t(a, B_\eta) \equiv i \right\}$
                    for all $i \in \left\{ 0,1 \right\}$ and $k < k_{**}$.
                \item\label{itm:5.4.4} $|A_{\eta \frown \left< i \right>}| \geq \epsilon |A_\eta|$
                    for all $i \in \left\{ 0,1 \right\}$ and $k < k_{**}$.
                \item\label{itm:5.4.5} $|A_\eta| \geq \epsilon^k |A|$, for $k \leq k_{**}$.
                \item\label{itm:5.4.6} $A_\eta = A_{\eta \frown \left< 0 \right>} \sqcup A_{\eta \frown \left< 1 \right>}$,
                    for $k < k_{**}$.
                \item\label{itm:5.4.7} $\overline{A_k} = \left\{ A_\eta \mid \eta \in [2]^k \right\}$ is a partition of $A$,
                    for $k \leq k_{**}$.
            \end{enumerate}
            First of all, notice that at each step, the non-$(\epsilon, \zeta)$-excellence of $A_\eta$ comes by IH
            from (\ref{itm:5.4.1}) or (\ref{itm:5.4.5}), and thus allows the existence of $B_\eta$ in (\ref{itm:5.4.2}).
            (\ref{itm:5.4.4}) follows the definition of $A_{\eta \frown \left< i \right>}$ in (\ref{itm:5.4.3}) and
            the fact $B_\eta$ is witnessing that $A_\eta$ is not $(\epsilon, \zeta)$-excellent.
            Applying recursively this last point we obtain (\ref{itm:5.4.5}).
            Finally, by definition (\ref{itm:5.4.3}), we have the disjoint union (\ref{itm:5.4.6}) which ensures
            the partition (\ref{itm:5.4.7}).

            Now, our goal is to build two sequences $\left\{ b_\eta \mid \eta \in [2]^{<k_{**}} \right\}$ and
            $\left\{ a_\eta \mid \eta \in [2]^{k_{**}} \right\}$ to contradict the tree bound $k_{**}$.
            First of all, notice that, for $\eta \in [2]^{k_{**}}$
            \[
                |A_\eta| \geq \epsilon^{k_{**}} |A| \geq
                \epsilon^{k_{**}} \frac{1}{\epsilon^{k_{**}}} = 1
            \]
            So, for each $\eta \in [2]^{k_{**}}$, $A_\eta \neq \emptyset$ and we may choose an $a_\eta \in A_\eta$.
            Now, for $\nu \in [2]^{<k_{**}}$ and $\eta \in [2]^{k_{**}}$ such that $\nu \triangleleft \eta$, let
            \[
                U_{\nu,\eta} = \left\{ b \in B_\nu \mid a_\eta R b \not\equiv t(a_\eta, B_\nu) \right\}
            \]
            be the subset of elements of $B_\nu$ that do not relate with $a_\eta$ in the expected way.
            By $\zeta$-goodness of $B_\nu$, $|U_{\nu, \eta}| < \zeta |B_\nu|$, and thus for every $\nu \in [2]^{<k_{**}}$,
            \[
                \left|\bigcup\left\{ U_{\nu,\eta} \mid \nu \triangleleft \eta \in [2]^{k_{**}} \right\}\right| <
                2^{k_{**}} \zeta |B_\nu| \leq |B_\nu|
            \]
            We may choose $b_\nu \in B_\nu \setminus \bigcup\left\{ U_{\nu,\eta} \mid \nu \triangleleft \eta \in [2]^{k_{**}} \right\}$,
            for all $\nu \in [2]^{<k_{**}}$.
            Finally, the sequences $\left< a_\eta \mid \eta \in [2]^{k_{**}} \right>$ and $\left< b_\nu \mid \nu \in [2]^{<k_{**}} \right>$
            satisfy that $\forall \eta, \nu$ such that $\nu \frown \left< i \right> \triangleleft \eta$, $\left( a_\eta R b_\nu \right)^i$
            by (\ref{itm:5.4.3}) and (\ref{itm:5.4.6}).
            This contradicts the definition of tree bound $k_{**}$ (\ref{tree_bound}).
        \end{proof}

    \lemma[Claim 5.4.1]\label{existance_of_excellent_subsets_fixed_size_choices}
        Let $G$ be a finite graph with the non-$k_{*}$-order property.
        Let $\zeta < \frac{1}{2^{k_{**}}}$, $\epsilon \in \parround{0, \frac{1}{2}}$.
        Let $\left< m\ell \mid \ell \in \parcurly{0, \dots, k_{**}} \right>$ be a decreasing sequence of natural numbers such that
        $\epsilon m_{\ell} \geq m_{\ell+1}$ for all $\ell \in \parcurly{0, \dots, k_{**}-1}$ and $m_{k_{**}} \geq 1$.
        % Notice that we needed to add m_{k_{**}} to the sequence, since it is needed in (c)' in the paper
        % (m_{l+1} may take that value!!).
        Then, for every $A \subseteq G$ with $|A| \geq m_0$ there exists
        $\left(\frac{m_{\ell+1}}{m_{\ell}}, \zeta\right)$-excellent subset $A' \subseteq A$ such that $|A'| = m_\ell$ for
        some $\ell \in \parcurly{0, \dots, k_{**}-1}$.
        \begin{proof}
            Suppose the converse.
            We use this fact to build sets $\left\{ b_\eta \mid \eta \in [2]^{<k_{**}} \right\}$ and
            $\left\{ A_\eta \mid \eta \in [2]^{\leq k_{**}} \right\}$ on induction over $k<k_{**}$, where $k = |\eta|$,
            satisfying:
            \begin{enumerate}
                \item\label{itm:5.4.1.1} $A_{\left< \cdot \right>} \subseteq A$, with $|A|_{\left< \cdot \right>} = m_0$.
                \item\label{itm:5.4.1.2} $B_\eta$ is an $\zeta$-good set witnessing that $A_\eta$ is not
                    $\left(\frac{m_{k+1}}{m_{k}}, \zeta\right)$-excellent, for all $k < k_{**}$.
                \item\label{itm:5.4.1.3} $A_{\eta \frown \left< i \right>} = \left\{ a \in A_\eta \mid t(a, B_\eta) \equiv i \right\}$
                    for all $i \in \left\{ 0,1 \right\}$ and $k < k_{**}$.
                \item\label{itm:5.4.1.4} $|A_{\eta}| = m_k$, for all $k \leq k_{**}$.
                \item\label{itm:5.4.1.6} $A_{\eta \frown \left< 0 \right>} \sqcup A_{\eta \frown \left< 1 \right>} \subseteq A_\eta$,
                    for all $k < k_{**}$.
                \item\label{itm:5.4.1.7} $\overline{A_k} = \left\{ A_\eta \mid \eta \in [2]^k \right\}$ is a partition of
                    a subset of $A$, for all $k \leq k_{**}$.
            \end{enumerate}
            Notice that, by (\ref{itm:5.4.1.1}) and (\ref{itm:5.4.1.4}), the size of $A_\eta$ is $m_k$,
            so by IH none of the sets $A_\eta$ is $\left(\frac{m_{k+1}}{m_{k}}, \zeta\right)$-excellent.
            Then, $B_\eta$ in (\ref{itm:5.4.1.2}) is well-defined.
            Also, by $\zeta$-goodness of $B_\eta$, $t(a, B_\eta)$ in (\ref{itm:5.4.1.3}) is well-defined.
            Then, since $B_\eta$ is witnessing the non-$\left(\frac{m_{k+1}}{m_{k}}, \zeta\right)$-excellence of $A_\eta$,
            we have that $|A_{\eta \frown \left< i \right>}| \geq \frac{m_{k+1}}{m_k} m_{k} = m_{k+1}$ for all
            $i \in \left\{ 0,1 \right\}$, satisfying (\ref{itm:5.4.1.4}).
            Finally, by definition (\ref{itm:5.4.1.3}), we have the disjoint union (\ref{itm:5.4.1.6}) which by itself
            ensures (\ref{itm:5.4.1.7}).

            Now, our goal is to build two sequences $\left\{ b_\eta \mid \eta \in [2]^{<k_{**}} \right\}$ and
            $\left\{ a_\eta \mid \eta \in [2]^{k_{**}} \right\}$ to contradict the tree bound $k_{**}$.
            First of all, notice that, for $\eta \in [2]^{k_{**}}$
            $$
                |A_\eta| = m_k \geq m_{k_{**}} \geq 1
            $$
            So, for each $\eta \in [2]^{k_{**}}$, $A_\eta \neq \emptyset$ and we may choose an $a_\eta \in A_\eta$.
            Now, for $\nu \in [2]^{<k_{**}}$ and $\eta \in [2]^{k_{**}}$ such that $\nu \triangleleft \eta$, let
            $$
                U_{\nu,\eta} = \left\{ b \in B_\nu \mid (a_\eta R b) \not\equiv t(a_\eta, B_\nu) \right\}
            $$
            be the subset of elements of $B_\nu$ that do not relate with $a_\eta$ in the expected way.
            By $\zeta$-goodness of $B_\nu$, $|U_{\nu, \eta}| < \zeta |B_\nu|$, and thus for every $\nu \in [2]^{<k_{**}}$,
            \[
                \left|\bigcup\left\{ U_{\nu,\eta} \mid \nu \triangleleft \eta \in [2]^{k_{**}} \right\}\right| <
                2^{k_{**}} \zeta |B_\nu| \leq |B_\nu|
            \]
            We may choose $b_\nu \in B_\nu \setminus \bigcup\left\{ U_{\nu,\eta} \mid \nu \triangleleft \eta \in [2]^{k_{**}} \right\}$,
            for all $\nu \in [2]^{<k_{**}}$.
            Finally, the sequences $\left< a_\eta \mid \eta \in [2]^{k_{**}} \right>$ and
            $\left< b_\nu \mid \nu \in [2]^{<k_{**}} \right>$ satisfy that $\forall \eta, \nu$ such that
            $\nu \frown \left< i \right> \triangleleft \eta$, $\left( a_\eta R b_\nu \right)^i$, which follows
            (\ref{itm:5.4.1.3}).
            This contradicts the definition of tree bound $k_{**}$ (\ref{tree_bound}).
        \end{proof}

%    \lemma[Fact 5.9]\label{fact_5.9}
%        Let $p,q \in \left( 0,1 \right)$.
%        Let $A$ be a set of size $n$, $B \subseteq A$ a subset of size $p|A|$, and $A' \subseteq A$ a random subset
%        of size $\geq q|A|$. % TODO: check conditions
%        Then, for $\zeta > 0$,
%        $$
%            P\left( \frac{\left| A' \cap B \right|}{\left| A' \right|} \in
%                 \left( \frac{\left| B \right|}{\left| A \right|} -
%                 \zeta, \frac{\left| B \right|}{\left| A \right|} + \zeta \right) \right)
%        $$
%        can be modeled by a random variable which is asymptotically normally distributed when $n \to +\infty$.
%
%    \lemma[Fact 5.10]\label{fact_5.10}
%        Let $A$ be a set of events measured with a probability $P_A$.
%        Let $S$ a family of subsets of $A$, which are measurable with $P_A$.
%        Let $A_r = \left\{ a_1, \dots, a_r \right\} \subseteq A$ be a random sample of size $r$.
%        For each $B \in S$, we may define $v_B^{A_r}$ the relative frequency of events of $B$ in $A_r$, i.e.,
%        $$
%            v_B^{A_r} = \frac{P_A(A_r \cap B)}{P_A(A_r)}
%        $$
%        Let
%        $$
%            \pi ^{A_r} = \sup_{B \in S} \left| v_B^{A_r} - P_A(B) \right|
%        $$
%        i.e. the upperbound of error of $v_B^{A_r}$ as an approximation of $P_A(B)$.
%        Also let $\Delta^s(A_r)$ be the number of subsets of $A_r$ induced by sets of $S$ ($B\in S$ induces
%        $B \cap A_r \subseteq A_r$), i.e.
%        $$
%            \Delta^s(A_r) = \left| \left\{ B \cap A_r \mid B \in S \right\} \right|
%        $$
%        Finally, let
%        $$
%            m^S(r) = \max_{C \in {A \choose r}} \Delta^s(C)
%        $$
%        Then, if there exists a finite $k > 0$ such that $m^s(r) \leq r^k +1$ for all $r > 0$, we have that,
%        for all $\epsilon > 0$,
%        $$
%            \lim_{r \to +\infty} P_A\left( \pi^{A_r} > \epsilon \right) = 0
%        $$
%
%    \begin{remark}[Fact 5.12]\label{fact_5.12}
%        If there exists $k > 0$ such that $m^s(r) \leq r^k + 1$ and $r$ satisfies:
%        $$
%            r \geq \frac{16}{\zeta^2} \left( k \log \frac{16 k}{\zeta^2} - \log \frac{\eta}{4} \right)^{k+1}
%        $$
%        for some $\eta > 0$, then
%        $$
%            P\left( \pi^{A_r} < \zeta \right) \geq 1 - \eta
%        $$
%        In particular, if we suppose that all events in $A$ are equiprobable and sampled without replacement, then
%        $$
%            P\left( \forall B \in S, \; \frac{\left| A_r \cap B \right|}{\left| A_r \right|} \in
%                 \left( \frac{\left| B \right|}{\left| A \right|} -
%                 \zeta, \frac{\left| B \right|}{\left| A \right|} + \zeta \right) \right) \geq 1 - \eta
%        $$
%        or in other words, for all but a fraction $\eta$ of all possible choices of $A_r$, we have that
%        $$
%            \forall B \in S, \; \frac{\left| A_r \cap B \right|}{\left| A_r \right|} \in
%                 \left( \frac{\left| B \right|}{\left| A \right|} -
%                 \zeta, \frac{\left| B \right|}{\left| A \right|} + \zeta \right)
%        $$
%    \end{remark}

    \lemma\label{ineq_5.13}
        For $k > 1$, $\zeta, \eta \in \parround{0,1}$ the function $f(m) = m^k \cdot e^{-2 \zeta^2 m}$ satisfies
        $f(m) \leq \eta$ for all $m \geq \frac{1}{\zeta^2}\parround{k \log{\frac{1}{\zeta^2} k} - \log{\eta}}$.
        \begin{proof}
            First of all, notice that for $m = \frac{1}{\zeta^2}\parround{k \log{\frac{1}{\zeta^2} k} - \log{\eta}}$,
            \[
                f(m) = \frac{m^k}{e^{2 \zeta^2 m}}
                     = \frac{\parround{\frac{1}{\zeta^2}\parround{k \log{\frac{1}{\zeta^2} k} - \log{\eta}}}^k}{
                        \parround{\frac{k}{\zeta^2}}^{2k} \eta^{-2}}
                     \leq \frac{k^k \parround{\log{\frac{k}{\zeta^2}\parround{\frac{1}{\eta}}^{\frac{1}{k}}}}^k}
                        {k^k \parround{\frac{k}{\zeta^2} \parround{\frac{1}{\eta}}^{\frac{1}{k}}}^k} \eta
                     < \eta
            \]
            To conclude, we prove that $f$ is decreasing for larger values of $m$:
            \[
                f'(m)
                    = \frac{k m^{k-1} e^{2 \zeta^2 m} - 2 \zeta^2 m^k e^{2 \zeta^2 m}}{\parround{e^{2 \zeta^2 m}}^2}
                    = \parround{k - 2m\zeta^2} \frac{m^{k-1}}{e^{2 \zeta^2 m}}
            \]
            The second factor is always positive, and $m > \frac{k}{\zeta^2} > \frac{k}{2\zeta^2}$, proving that $f'(m) < 0$
            and thus $f$ is decreasing.
        \end{proof}

    \lemma[Claim 5.13]\label{claim_5.13}
        Let $G$ be a finite graph with the non-$k_{*}$-order property.
        Then:
        \begin{enumerate}[label=(\alph*), ref=\alph*]
            \item \label{itm:5.13.1} For every $\epsilon \in \left(0, \frac{1}{2}\right)$,
                $\zeta \in \left(0, \frac{1}{2} - \epsilon \right)$, $\xi \in \left(0, 1 \right)$ and
                $m \geq \frac{1}{\zeta^2}\parround{k_* \log{\frac{1}{\zeta^2}k_*} - \log{\xi}}$,
                if $A \subseteq G$ is an $\epsilon$-good subset of size $n \geq m$,
                then a random subset $A' \subseteq A$ of size $m$ is $(\epsilon + \zeta)$-good with probability $1-\xi$.
            \item \label{itm:5.13.1*} Moreover, such $A'$ satisfies $t(b, A') = t(b, A)$ for all $b \in G$.
            \item \label{itm:5.13.2} For every $\zeta \in \left(0, \frac{1}{2}\right)$ and $\zeta' < \zeta$, there is
                $\epsilon_1 = \epsilon_1(\zeta, \zeta')$ such that for every $\epsilon < \epsilon' \leq \epsilon_1$, if
                \begin{itemize}
                    \item $A \subseteq G$ is $\left( \epsilon, \epsilon' \right)$-excellent.
                    \item $A' \subseteq A$ is $\left( \epsilon + \zeta' \right)$-good.
                \end{itemize}
                then, $A'$ is $\left( \epsilon + \zeta, \epsilon' \right)$-excellent.
            \item \label{itm:5.13.3} For all $\zeta \in \left(0, \frac{1}{2}\right)$, $\zeta' < \zeta$, $r \geq 1$ and for all
                $\epsilon < \epsilon'$ small enough (in the sense of the previous point) there exists
                $N = N\left(k_{*}, \zeta', r \right)$ such that, if $|A| = n > N$, $r$ divides $n$ and $A$ is
                $\left( \epsilon, \epsilon' \right)$-excellent, there exists a partition into $r$ disjoint pieces of equal
                size, each of which is $\left( \epsilon + \zeta, \epsilon' \right)$-excellent.
        \end{enumerate}
        \begin{proof}
        \begin{enumerate}[label=(\alph*), ref=\alph*]
%            % TODO: add some intuition about the proof.
%            \item For each $b \in G$ we say that $\overline{B}_{A,b}$ is \emph{exceptional} if
%                $\left| \overline{B}_{A,b} \right| \geq \epsilon \left| A' \right|$.
%                Notice that, if we prove that, with probability $1-\xi$, $A'$ satisfies that for all exceptional $\overline{B}_{A,b}$:
%                $$
%                    \frac{\left| A' \cap \overline{B}_{A,b} \right|}{\left| A' \right|} \in
%                         \left( \frac{\left| \overline{B}_{A,b} \right|}{\left| A \right|} -
%                         \zeta, \frac{\left| \overline{B}_{A,b} \right|}{\left| A \right|} + \zeta \right)
%                $$
%                then, with the same probability:
%                \begin{equation}\label{eq:equation1}
%                    \left| A' \cap \overline{B}_{A,b} \right| < \left( \frac{\left| \overline{B}_{A,b} \right|}{|A|} + \zeta \right) |A'|
%                        < \left( \epsilon + \zeta \right) |A'|
%                \end{equation}
%                and we are done.
%
%                By Lemma~\ref{fact_5.9}, for $n = |A|$ large enough, we can approximate sampling a set of size $m$ from $A$,
%                with $m$ i.i.d. random variables $x_1, \dots, x_m$, where each $x_i$ picks a vertex uniformly at random from $A$.
%                % TODO: here we ignored the fact that we need to fix p and q in Fact 5.9.
%                Let $S \coloneq \left\{ \text{Exceptional } \overline{B}_{A,b} \right\}$.
%                Since $G$ has the non-$k_{*}$-order property, we can apply Lemma~\ref{itm:2.6.1} to $G_{\ref{itm:2.6.1}} = A$
%                and $A_{\ref{itm:2.6.1}} = A'$, which gets us that:
%                $$
%                    |S| \leq \left|\left\{ \left\{ a \in A' \mid a R b \not\equiv t(A', b) \right\} \mid b \in G \right\} \right|
%                    \leq |A'|^{k_*}
%                $$
%                Then,
%                $$
%                    m^s(\ell) \leq \left| S \right| \leq |A'|^{k_*} \leq \ell^{k_*} \leq \ell^{k_{*}} + 1 \quad \forall \ell \geq |A'|
%                $$
%                Notice that this is enough to satisfy the conditions of Lemma~\ref{fact_5.10}:
%                For each $\ell < |A'|$, let $k_l$ be the smallest integer such that $m^s(\ell) \leq \ell^{k_l} + 1$.
%                Since there are finitely many of them, we can take the maximum
%                $k_{\max} = \max \left\{ k_1, \dots, k_{|A'|-1}, k_* \right\}$, which satisfies
%                $$
%                    m^s(\ell) \leq \ell^{k_{\max}} + 1 \quad \forall \ell
%                $$
%                % TODO: all of this may not be necessary, since I think that if \ell' < \ell, then k_l' < k_l
%                % TODO: in \ref{fact_5.12}, the bound on m grows with k_max, and here k_max grows with m. Need to solve this
%                So we conclude equation (\ref{eq:equation1}), which by itself is sufficient to prove $A'$ is
%                $(\epsilon + \zeta)$-good.
            \item For each $b \in G$, we say that $B_{A,b}$ is \emph{bad} if $\parstraight{B_{A,b}} \geq \epsilon \parstraight{A'}$.
                For each bad $B_{A,b}$, let $X_{A,b}$ be the event that
                $\parstraight{B_{A,b}} \geq \parround{\epsilon + \zeta} \parstraight{A'}$ for a random subset
                $A' \subseteq A$ of size $m$.
                Notice that $X_{A,b}$ is modelled by a hypergeometric distribution, and so the probability of
                upperly deviating from the mean by $\zeta$, can be modeled by
                \[
                    P\parround{X_{A,b} = 1} \leq e^{-2\zeta^2 m}
                \]
                Now we want to study the random variable $X$ counting the number of events $X_{A,b}$ that are satisfied.
                That is, $X = \sum_{\text{bad } B_{A,b}} X_{A,b}$.
                We compute the expectation
                \[
                    \mathbb{E}[X] = \sum_{\text{bad } B_{A,b}} \mathbb{E}[X_{A,b}]
                        = \sum_{\text{bad } B_{A,b}} P\parround{X_{A,b} = 1}
                        \leq \sum_{\text{bad } B_{A,b}} e^{-2\zeta^2 m}
                \]
                Following (\ref{itm:2.6.1.2}), the number of intersections of bad $B_{A,b}$'s with $A'$, can be bounded
                by $m^{k_*}$.
                Thus, using the First Moment Method, we have that:
                \[
                    P\parround{X \geq 1} \leq \mathbb{E}[X] \leq m^{k_*} \cdot e^{-2\zeta^2 m} \leq \xi
                \]
                Last inequality follows Lemma~\ref{ineq_5.13} using the lower bound on $m$.
                Thus, with probability at least $1 - \xi$, we have that $A'$ is $(\epsilon + \zeta)$-good.
            \item Suppose that $A'$ is the subset described in (\ref{itm:5.13.1}).
                We proved that, such set satisfies
                \[
                    \parstraight{A' \cap B_{A,b}} < \parround{\epsilon + \zeta} \parstraight{A'}
                \]
                for all $b \in G$ such that $\parstraight{B_{A,b}} \geq \epsilon m$.
                Thus, we have that:
                \begin{itemize}
                    \item If $\parstraight{B_{A,b}} < \epsilon m$, then
                        $\parstraight{\parcurly{a \in A' \mid a R b \not \equiv t(b,A)}} \leq \parstraight{B_{A,b}}
                        < \epsilon m < \parround{\epsilon + \zeta} m$.
                    \item If $\parstraight{B_{A,b}} \geq \epsilon m$, then
                        $\parstraight{\parcurly{a \in A' \mid a R b \not \equiv t(b,A)}} = \parstraight{A' \cap B_{A,b}}
                        < \parround{\epsilon + \zeta} m$.
                \end{itemize}
            We conclude that $t(b,A) = t(b,A')$ for all $b \in G$.
            \item Let $B \subseteq G$ be an $\epsilon'$-good set.
                We first upperbound the number of exceptional vertices of $B$ with respect to $A'$:
                \begin{align*}
                    \parstraight{\parcurly{b \in B \mid t(b, A') \not\equiv t(B,A)}}
                        & = \parstraight{\parcurly{b \in B \mid t(b, A) \not\equiv t(B,A)}} \\
                        & \leq \frac{\parround{\epsilon + \parround{1 - \epsilon} \epsilon'}\parstraight{A}\parstraight{B}}
                            {\parround{1 - \epsilon}\parstraight{A}} \\
                        & = \parround{\epsilon' + \frac{\epsilon}{1 - \epsilon}}\parstraight{B}
                \end{align*}
                The first equality follows (\ref{itm:5.13.1*}), and the first inequality follows from remark
                (\ref{remark_excellence_imply_little_exceptions}) for the numerator, and taking the worst case of only
                $(1 - \epsilon) \parstraight{A}$ exceptional edges per exceptional $b \in B$
                (considering that $A$ is $\epsilon$-good).
                
                Now, let $Q$ be the set of exceptional vertices of $A'$ with respect to $B$, i.e.:
                $$
                    Q = \parcurly{a \in A' \mid t(a, B) \not\equiv t(A, B)}
                $$
                We want to double-count the number of exceptional edges between $Q$ and $B$.
                On one hand, we have that:
                $$
                    \parstraight{\parcurly{(a,b) \in Q \times B \mid t(a, b) \not\equiv t(A, B)}} < 
                    \parround{\epsilon' + \frac{\epsilon}{1 - \epsilon}} \parstraight{B} \parstraight{Q} + 
                    \parround{1 - \epsilon' - \frac{\epsilon}{1 - \epsilon}} \parstraight{B} \parround{\epsilon + \zeta'} \parstraight{A'}
                $$
                The first term is the maximum number of exceptional edges associated to exceptional $b \in B$ 
                (considering all edges exceptional), while the second term bounds the number of exceptional edges of 
                non-exceptional $b \in B$, using the fact that $A'$ is $(\epsilon + \zeta')$-good.
            
                On the other hand, we have that:
                $$
                    \parstraight{\parcurly{(a,b) \in Q \times B \mid t(a, b) \not\equiv t(A, B)}} \geq
                    \parstraight{Q} \parround{1 - \epsilon'} \parstraight{B}
                $$
                which follows $B$ being $\epsilon'$-good.

                Putting it all together:
                $$
                    \parround{1 - \epsilon' - \epsilon' - \frac{\epsilon}{1 - \epsilon}} \parstraight{B} \parstraight{Q} <
                    \parround{1 - \epsilon' + \frac{\epsilon}{1 - \epsilon}} \parround{\epsilon + \zeta'} \parstraight{B} \parstraight{A'}
                $$
                So, we have that:
                \begin{align*}
                    \parstraight{Q} & < \frac{\parround{1 - \epsilon' - \frac{\epsilon}{1 - \epsilon}}}
                                        {\parround{1 - \epsilon' - \frac{\epsilon}{1 - \epsilon}} - \epsilon'}
                                        \parround{\epsilon + \zeta'} \parstraight{A'} \\
                                    & = \parround{1 + \frac{\epsilon'}{1 - 2\epsilon' - \frac{\epsilon}{1 - \epsilon}}}
                                        (\epsilon + \zeta') |A'|
                \end{align*}
                Notice that $f(\epsilon, \epsilon') \coloneq \frac{\epsilon'}{1 - 2\epsilon' - \frac{\epsilon}{1 - \epsilon}}$
                decreases with $\epsilon$ and $\epsilon'$.
                In particular,
                $$
                    f(\epsilon, \epsilon') \overset{\epsilon' \to 0}{\longrightarrow} 0
                $$
                and $\epsilon' > \epsilon$.
                Then,
                $$
                    \parstraight{Q} < \parround{\epsilon + \parround{\underbrace{\epsilon f(\epsilon, \epsilon')}_{\to 0} +
                    \underbrace{\parround{1 + f(\epsilon, \epsilon')}}_{\to 1}} \zeta'} |A'|
                    \overset{\epsilon' \to 0}{\longrightarrow} \parround{\epsilon + \zeta'} |A'|
                $$
                So, there exists an $\epsilon_1 = \epsilon_1(\zeta, \zeta')$ small enough such that for all
                $(\epsilon <)$ $\epsilon' \leq \epsilon_1$, we have that $\parstraight{Q} < \parround{\epsilon + \zeta} |A'|$,
                and since $A'$ is $(\epsilon + \zeta')$-good, and thus $(\epsilon + \zeta)$-good, we conclude that
                $A'$ is $(\epsilon + \zeta, \epsilon')$-excellent. \todo{Mention that in the next claim we show valid values for this.}
            \item Let $\zeta, \zeta', \epsilon, \epsilon'$ and $r$ be given satisfying the conditions of the statement.
                Set $\xi = \frac{1}{r + 1}$.
                We will see that the condition
                $n > N = N\left(k_{*}, \zeta', r \right) \coloneq r \frac{1}{\zeta'^2}\parround{k_* \log{\frac{1}{\zeta'^2}k_*} - \log{\frac{1}{r + 1}}}$
                is sufficient.
                First of all, randomly choose a function $h: A \longrightarrow \left\{ 1, \dots, r-1 \right\}$ such that
                for all $s < n$ we have that $\parstraight{\parcurly{a \in A \mid h(a) = s}} = \frac{n}{r}$.
                Since $h$ is random, each $A' \in [A]^\frac{n}{r}$ has the same probability of being part of the partition
                induced by $h$, i.e. to satisfy $A' = h^{-1}(s)$ for some $s \in \left\{ 1, \dots, r-1 \right\}$.
                Since each element of the partition $A'$ has size
                $\frac{n}{r} > \frac{N}{r} = \frac{1}{\zeta'^2}\parround{k_* \log{\frac{1}{\zeta'^2}k_*} - \log{\xi}}$,
                we can apply (\ref{itm:5.13.1}) to get that
                $$
                    P\parround{A' \text{ is not } \parround{\epsilon + \zeta'}\text{-good}} < \xi
                $$
                In particular, since $A$ is $(\epsilon, \epsilon')$-excellent, it follows (\ref{itm:5.13.2}) that if $A'$ is
                $\parround{\epsilon + \zeta'}$-good then it is also $\parround{\epsilon + \zeta, \epsilon'}$-excellent, so:
                $$
                    P\parround{A' \text{ is not } \parround{\epsilon + \zeta, \epsilon'}\text{-excellent}} < \xi
                $$
                To conclude, by the union bound, we have that:
                \begin{align*}
                    P\parround{\bigcup_{s < r} h^{-1}(s) \text{ is not } \parround{\epsilon + \zeta, \epsilon'}\text{-excellent}}
                        & \leq \sum_{s < r} P\parround{h^{-1}(s) \text{ is not } \parround{\epsilon + \zeta, \epsilon'}\text{-excellent}} \\
                        & < r \xi = \frac{r}{r+1} < 1
                \end{align*}
                All in all, there is a non-zero chance that the partition satisfies the statement, i.e. there exists at least one.
        \end{enumerate}
        \end{proof}

    \begin{remark}[Remark 5.13.1]\label{remark_5.13.1}
        For following applications, we would like to use Lemma~\ref{claim_5.13} (\ref{itm:5.13.3}) with
        $\epsilon' > k \parround{\epsilon + \zeta}$, for an arbitrarily large $k \in \naturals$.
        Notice that if $\epsilon, \zeta' \leq \frac{1}{t}, \epsilon' \leq \frac{1}{t'}$ and $t > t' \geq 5$, then:
        \begin{enumerate}[label=(\alph*), ref=\alph*]
            \item $\frac{\epsilon}{1-\epsilon} \leq \frac{\frac{1}{t}}{1-\frac{1}{t}} = \frac{\frac{1}{t}}{\frac{t-1}{t}}
                = \frac{1}{t-1}$
            \item $1 - 2 \epsilon' - \frac{\epsilon}{1-\epsilon} \geq 1 - \frac{2}{t'} - \frac{1}{t-1} > 1 - \frac{3}{t'-1}
                = \frac{t'-4}{t'-1}$
            \item\label{itm:5.13.1.c} $\parround{1 + \frac{\epsilon'}{1 - 2 \epsilon' - \frac{\epsilon}{1-\epsilon}}}
                < 1 + \frac{\epsilon'}{1 - \frac{3}{t'-1}}
                = \parround{1 + \frac{t'-1}{t'-4} \epsilon'} \parround{\epsilon + \zeta'}$
        \end{enumerate}
        Then, by requiring $\frac{1}{t} \leq \frac{1}{4k}\epsilon'$ we have that
        \begin{align*}
            \epsilon + \zeta'
                & \leq \frac{2}{t} \leq 2 \parround{\frac{1}{4k}\epsilon'} = \frac{1}{2} \parround{\frac{1}{k}\epsilon'} \\
                & < \frac{t'-4}{t'- 3} \frac{1}{k} \epsilon' = \frac{1}{k} \frac{\epsilon'}{1 + \frac{1}{t'-4}} \\
                & < \frac{1}{k} \frac{\epsilon'}{1 + \frac{t'-1}{t'}\frac{1}{t'-4}} = \frac{1}{k} \frac{\epsilon'}{1 + \frac{t'-1}{t'-4}\frac{1}{t'}} \\
                & \leq \frac{1}{k} \frac{\epsilon'}{1 + \frac{t'-1}{t'-4}\epsilon'}
        \end{align*}
        i.e., we have:
        $$
            \parround{1 + \frac{t'-1}{t'-4} \epsilon'} \parround{\epsilon + \zeta'} < \frac{1}{k} \epsilon'
        $$
        which by (\ref{itm:5.13.1.c}) gives us:
        $$
            \parround{1 + \frac{\epsilon'}{1 - 2 \epsilon' - \frac{\epsilon}{1-\epsilon}}} < \frac{1}{k} \epsilon'
        $$
        All in all, a sufficient condition, for the lemma to hold under the constraint $\epsilon' \geq k \parround{\epsilon + \zeta}$, is:
        $$
            \epsilon, \zeta' \leq \frac{1}{4k} \epsilon' \quad \text{ and } \quad \epsilon' \leq \frac{1}{5}
        $$
    \end{remark}

    We use this fact to reformulate point (\ref{itm:5.13.3}) of Lemma~\ref{claim_5.13} as:

    \lemma[Claim 5.13.2(3)]\label{existance_of_equitable_excellent_subpartition}
        Let $G$ be a finite graph with the non-$k_*$-property.
        For all $k, r \geq 1$, $\epsilon' \leq \frac{1}{5}$ and $\epsilon \leq \frac{1}{4k} \epsilon'$, there exists
        $N = N\parround{k, k_*, \epsilon', r}$ large enough such that, for all $n > N$ and $r$ dividing $n$,
        if $A \subseteq G$ is $\left( \epsilon, \epsilon' \right)$-excellent, with $|A| = n$, then there exists a
        partition into $r$ disjoint pieces of equal size, each of which is $\parround{\frac{\epsilon'}{k}, \epsilon'}$-excellent.
        \begin{proof}
            Choose any $\zeta' \leq \frac{1}{4k} \epsilon'$ and set $N \coloneq N_{\ref{claim_5.13}}\parround{k_*, \zeta', r}$.
            Remark~\ref{remark_5.13.1} sufficiency condition is satisfied, Claim~\ref{claim_5.13} (\ref{itm:5.13.3}) holds
            and we are done.
        \end{proof}

    \begin{remark}
        A sufficient condition for $N_{\ref{existance_of_equitable_excellent_subpartition}}$ to be large enough is
        to choose $\zeta' = \frac{1}{4k} \epsilon'$ in which case
        $N_{\ref{existance_of_equitable_excellent_subpartition}}\parround{k, k_*, \epsilon', r} \coloneq
        N_{\ref{claim_5.13}}\parround{k_*, \frac{1}{4k} \epsilon', r}$
    \end{remark}

    \lemma[Claim 5.14.1]\label{existance_of_excellent_partition}
        Let $G$ be a finite graph with the non-$k_{*}$-order property.
        Let $\epsilon \in \parround{0, \frac{1}{2}}$ and $\epsilon' \leq \frac{1}{2^{k_{**}}}$.
        Let $A \subseteq G$ such that $|A| = n$.
        Let $\partriangle{m\ell \mid \ell \in \parcurly{0, \dots, k_{**}}}$ be a decreasing sequence of natural numbers such that
        $\epsilon m_{\ell} \geq m_{\ell+1}$ for all $\ell \in \parcurly{0, \dots, k_{**}-1}$ and $m_{k_{**}} \geq 1$.
        Denote $m_* \coloneq m_0$ and $m_{**} \coloneq m_{k_{**}}$.
        Then, there is a partition $\overline{A} = \partriangle{A_j \mid j \in \parcurly{1, \dots, j(*)}}$ with remainder
        $B = A \setminus \bigcup_{j < j(*)} A_j$ such that:
        \begin{enumerate}[label=(\alph*), ref=\alph*]
            \item \label{itm:5.14.1.a} For all $j \in \parcurly{1, \dots, j(*)}$, $|A_j| \in \partriangle{m\ell \mid \ell \in \parcurly{0, \dots, k_{**}-1}}$.
            \item \label{itm:5.14.1.b} For all $i \neq j \in \parcurly{1, \dots, j(*)}$, $A_i \cap A_j = \emptyset$.
            \item \label{itm:5.14.1.c} For all $j \in \parcurly{1, \dots, j(*)}$, $A_j$ is $\parround{\epsilon, \epsilon'}$-excellent.
            \item \label{itm:5.14.1.d} $|B| < m_*$.
        \end{enumerate}
        \begin{proof}
            Apply Lemma~\ref{existance_of_excellent_subsets_fixed_size_choices} recursively to the remainder
            $A \setminus \bigcup_{i < j} A_i$, to obtain $A_j$ at each step.
            The process stops at $j(*)$ when the remainder is smaller than $m_0$, and thus the lemma cannot be applied.
            Notice that, since $\frac{m_\ell}{m_{\ell-1}} \leq \epsilon$, $\parround{\frac{m_\ell}{m_{\ell-1}}, \epsilon'}$-excellence
            implies $\parround{\epsilon, \epsilon'}$-excellence.
        \end{proof}

    % TODO: say that if A is smaller than m_0, then the partition is empty and B = A.

    \lemma[Claim 5.14.1a]\label{existance_of_excellent_partition_with_equal_size}
        Let $G$ be a finite graph with the non-$k_{*}$-order property.
        Let $\epsilon' \leq \min \parround{\frac{1}{5}, \frac{1}{2^{k_{**}}}}$ and $\epsilon \leq \frac{1}{4k} \epsilon'$ for some $k > 1$.
        Let $A \subseteq G$ such that $|A| = n$.
        Let $\partriangle{m\ell \mid \ell \in \parcurly{0, \dots, k_{**}}}$ be a decreasing sequence of natural numbers such that
        $\epsilon m_{\ell} \geq m_{\ell+1}$ for all $\ell \in \parcurly{0, \dots, k_{**}-1}$, $m_{k_{**}} \geq 1$,
        $m_{**} \coloneq m_{k_{**}} \mid m_\ell$ for all $\ell \in \parcurly{0, \dots, k_{**}}$,
        $m_{k_{**}-1} > N\parround{k, k_*, \epsilon', \frac{m_*}{m_{**}}}$
        (in the sense of Claim~\ref{existance_of_equitable_excellent_subpartition}), and $n \geq m_0$.
        Let $m_* \coloneq m_0$.
        Then, for some $i(*) \leq \frac{n}{m_{**}}$, there is a partition $\overline{A} = \partriangle{A_i \mid i \in \parcurly{1, \dots, i(*)}}$
        with remainder $B = A \setminus \bigcup \overline{A}$ such that:
        \begin{enumerate}[label=(\alph*), ref=\alph*]
            \item \label{itm:5.14.1a.a} For all $i \in \parcurly{1, \dots, i(*)}$, $|A_i| = m_{**}$.
%            \item \label{itm:5.14.1a.b} For all $i \neq j \in \parcurly{1, \dots, i(*)}$, $A_i \cap A_j = \emptyset$.
            \item \label{itm:5.14.1a.c} For all $i \in \parcurly{1, \dots, i(*)}$, $A_i$ is $\parround{\frac{\epsilon'}{k}, \epsilon'}$-excellent.
            \item \label{itm:5.14.1a.d} $|B| < m_*$.
        \end{enumerate}
        \begin{proof}
            Use Claim~\ref{existance_of_excellent_partition} to obtain a partition
            $\overline{A}' = \partriangle{A_j' \mid j \in \parcurly{1, \dots, j(*)}}$ and remainder $B$ with $\parstraight{B} < m_*$.
            Then, we can apply Claim~\ref{existance_of_equitable_excellent_subpartition} with $r = \frac{m_*}{m_{**}}$ to each of
            the parts $A_j'$.
            Putting together all the new subparts, we obtain a new partition $\overline{A} = \partriangle{A_i \mid i \in \parcurly{1, \dots, i(*)}}$
            with remainder $B$, satisfying all the conditions of the statement.
        \end{proof}

    \lemma[Claim 5.14.2]\label{existance_of_excellent_partition_with_equal_size_and_no_remainder}
        Under the same condition of Lemma~\ref{existance_of_excellent_partition_with_equal_size}, we can get a
        partition $\overline{A} = \partriangle{A_i \mid i \in \parcurly{1, \dots, i(*)}}$ with no remainder, such that:
        \begin{enumerate}[label=(\alph*), ref=\alph*]
            \item \label{itm:5.14.2.a} For all $i, j \in \parcurly{1, \dots, i(*)}$, $\parstraight{\parstraight{A_i}- \parstraight{A_j}} \leq 1$.
            \item \label{itm:5.14.2.b} For all $i, j \in \parcurly{1, \dots, i(*)}$, $A_i \cap A_j = \emptyset$.
            \item \label{itm:5.14.2.c} For all $i \in \parcurly{1, \dots, i(*)}$, $A_i$ is $\parround{\epsilon'', \epsilon'}$-excellent,
                where
                $$
                    \epsilon'' \leq \frac{\frac{\epsilon'}{k} m_{**} + \ceil{\frac{m_*}{i(*)}}}{m_{**} + \ceil{\frac{m_*}{i(*)}}}
                $$
            \item \label{itm:5.14.2.d} $A = \bigcup \overline{A}$.
        \end{enumerate}
        \begin{proof}
            Let $\overline{A}' = \partriangle{A_i' \mid i \in \parcurly{1, \dots, i(*)}}$ and $B$ from
            Claim~\ref{existance_of_excellent_partition_with_equal_size}.
            We can partition $B$ into $\overline{B} = \partriangle{B_i \mid i \in \parcurly{1, \dots, i(*)}}$ in such a way that
            for all $i \in \parcurly{1, \dots, i(*)}$,
            $$
                |B_i| \in \parcurly{\floor{\frac{\parstraight{B}}{i(*)}}, \ceil{\frac{\parstraight{B}}{i(*)}}}
            $$
            Notice that we are allowing $B_i = \emptyset$.
            Then, the new partition $\overline{A} = \partriangle{A_i' \cup B_i \mid i \in \parcurly{1, \dots, i(*)}}$ satisfies
            (\ref{itm:5.14.2.a}), (\ref{itm:5.14.2.b}) and (\ref{itm:5.14.2.d}) by construction.
            To conclude, notice that for each $\epsilon'$-good set $B$, the number of exceptions is bounded by
            \begin{align*}
                \parstraight{\parcurly{a \in A_i \mid t(a, B) \not\equiv t(A_i, B)}}
                    & \leq \frac{\epsilon'}{k} \parstraight{A_i'} + \parstraight{B_i} \\
                    & = \frac{\frac{\epsilon'}{k} \parstraight{A_i'} + \parstraight{B_i}}{\parstraight{A_i'} + \parstraight{B_i}}
                        \parround{\parstraight{A_i'} + \parstraight{B_i}} \\
                    & \leq \frac{\frac{\epsilon'}{k} m_{**} + \ceil{\frac{m_*}{i(*)}}}{m_{**} + \ceil{\frac{m_*}{i(*)}}}
                        \parstraight{A_i}
            \end{align*}
            which proves that (\ref{itm:5.14.2.c}) can be satisfied.
        \end{proof}

    \begin{remark}[Remark 5.14.3]\label{epsilons_proportion_can_be_k}
        In the context of Lemma~\ref{existance_of_excellent_partition_with_equal_size_and_no_remainder}, if:
        \begin{enumerate}[label=(\alph*), ref=\alph*]
            \item \label{itm:5.14.3.a} $m_{**} \geq \frac{1}{\frac{\epsilon'}{k}}$
            \item \label{itm:5.14.3.b} $m_* \leq \frac{\frac{\epsilon'}{k} n + 1}{\frac{\epsilon'}{k} + 1}$
        \end{enumerate}
        then $\epsilon'' \leq \frac{3 \epsilon'}{k}$.
        \begin{proof}
            Notice that, if $\parstraight{B_i} \leq 2 \frac{\epsilon'}{k} \parstraight{A_i}$ for all $i \in \parcurly{1, \dots, i(*)}$,
            then $\epsilon''$ can be bounded by:
            $$
                \epsilon'' \leq \frac{\frac{\epsilon'}{k} \parstraight{A_i} + \parstraight{B_i}}{\parstraight{A_i} + \parstraight{B_i}}
                \leq \frac{\frac{\epsilon'}{k} \parstraight{A_i} + 2 \frac{\epsilon'}{k} \parstraight{A_i}}{\parstraight{A_i}}
                = \frac{3 \epsilon'}{k}
            $$
            Let's now prove that $\parstraight{B_i} \leq \frac{2 \epsilon'}{k} \parstraight{A_i}$ is satisfied.
            Notice that, by construction:
            $$
                \parstraight{B_i} \leq \ceil{\frac{\parstraight{B}}{i(*)}} \leq \ceil{\frac{m_* - 1}{i(*)}} \leq
                \frac{m_* - 1}{i(*)} + 1
            $$
            Also we can bound $i(*)$ by:
            $$
                \frac{n}{m_{**}} \geq i(*) \geq \frac{n - \parstraight{B}}{m_{**}} \geq \frac{n - m_* + 1}{m_{**}} >
                \frac{n - m_*}{m_{**}}
            $$
            % TODO: lower bound needed?
            Thus, $\parstraight{B_i} - 1 \leq \frac{m_* - 1}{i(*)} \leq \frac{\parround{m_* - 1} m_{**}}{n - m_*}$,
            then $\frac{\parstraight{B_i} - 1}{m_{**}} \leq \frac{m_* - 1}{n - m_*}$, and since $\parstraight{A_i} = m_{**}$
            we get:
            $$
                \frac{\parstraight{B_i}}{\parstraight{A_i}} \leq \frac{m_* - 1}{n - m_*} + \frac{1}{m_{**}}
            $$
            Finally, notice that condition (\ref{itm:5.14.3.a}) implies:
            $$
                \frac{\epsilon'}{k} \geq \frac{1}{m_{**}}
            $$
            and condition (\ref{itm:5.14.3.b}) implies:
            $$
                \frac{\epsilon'}{k} \geq \frac{m_* - 1}{n - m_*}
            $$
            We conclude:
            $$
                \frac{\parstraight{B_i}}{\parstraight{A_i}} \leq \frac{m_* - 1}{n - m_*} + \frac{1}{m_{**}} \leq 2 \frac{\epsilon'}{k}
            $$
            completing the proof.
        \end{proof}
    \end{remark}

    \lemma[Corollary 5.15]\label{resume_of_all_conditions_for_excellent_partitions}
        Let $G$ be a graph with the non-$k_{*}$-order property.
        Suppose that we are given:
        \begin{enumerate}
            \item $\epsilon \leq \min \parround{\frac{1}{5}, \frac{1}{2^{k_{**}}}}$.
            \item A sequence of positive integers $\partriangle{m\ell \mid \ell \in \parcurly{0, \dots, k_{**}}}$, and values $m_*$
                and $m_{**}$, such that:
                \begin{enumerate}[label=(\alph*), ref=2.\alph*]
                    \item \label{itm:5.15.a} $\frac{\epsilon}{12} m_\ell \geq m_{\ell + 1}$.
                    \item \label{itm:5.15.b} $m_{**} \coloneq m_{k_{**}} > \frac{3}{\epsilon}$.
                    \item \label{itm:5.15.c} $m_{**} \mid m_\ell$ for all $\ell \in \parcurly{0, \dots, k_{**}}$.
                    \item \label{itm:5.15.d} $m_{k_{**}-1} > N\parround{3, k_*, \epsilon, \frac{m_*}{m_{**}}}$ (in the sense
                        of Claim~\ref{existance_of_equitable_excellent_subpartition}).
            \end{enumerate}
            \item $A \subseteq G$ such that $|A| = n$, where $n$ is large enough to satisfy:
            \begin{enumerate}[label=(\alph*'), ref=3.\alph*]
                \item \label{itm:5.15.a'} $n \geq m_0$. \todo{This is implied by next condition.}
                \item \label{itm:5.15.b'} $m_* \leq \frac{1 + \frac{\epsilon}{3}n}{1 + \frac{\epsilon}{3}}$.
            \end{enumerate}
        \end{enumerate}
        Then, there exists $i(*) \leq \frac{n}{m_{**}}$ and a partition of $A$ into disjoint pieces
        $\overline{A} = \partriangle{A_i \mid i \in \parcurly{1, \dots, i(*)}}$ such that:
        \begin{enumerate}[label=(\roman*), ref=\roman*]
            \item \label{itm:5.15.i} For all $i, j \in \parcurly{1, \dots, i(*)}$, $\parstraight{\parstraight{A_i}- \parstraight{A_j}} \leq 1$.
            \item \label{itm:5.15.ii} For all $i \in \parcurly{1, \dots, i(*)}$, $A_i$ is $\epsilon$-excellent,
            \item \label{itm:5.15.iii} For all $i, j \in \parcurly{1, \dots, i(*)}$, $\parround{A_i, A_j}$ is $\epsilon$-uniform.
        \end{enumerate}
        \begin{proof}
            Simply apply Lemma~\ref{existance_of_excellent_partition_with_equal_size_and_no_remainder} in the context of
            Remark~\ref{epsilons_proportion_can_be_k} with $k = 3$, ${\epsilon'_{\ref{existance_of_excellent_partition_with_equal_size_and_no_remainder}}} = \epsilon$
            and ${\epsilon_{\ref{existance_of_excellent_partition_with_equal_size_and_no_remainder}}} \leq \frac{1}{12} \epsilon$.
            This results in a partition of $A$ into disjoint pieces that satisfy (\ref{itm:5.15.i}) and that are
            $\parround{{\epsilon''_{\ref{existance_of_excellent_partition_with_equal_size_and_no_remainder}}},
                \epsilon'_{\ref{existance_of_excellent_partition_with_equal_size_and_no_remainder}}}$-excellent,
            with ${\epsilon''_{\ref{existance_of_excellent_partition_with_equal_size_and_no_remainder}}} \leq
                \frac{3 \epsilon'_{\ref{existance_of_excellent_partition_with_equal_size_and_no_remainder}}}{k}$.
            But since $k \geq 3$, ${\epsilon''_{\ref{existance_of_excellent_partition_with_equal_size_and_no_remainder}}} \leq
                \epsilon'_{\ref{existance_of_excellent_partition_with_equal_size_and_no_remainder}}$, they are also
            $\epsilon'_{\ref{existance_of_excellent_partition_with_equal_size_and_no_remainder}}$-excellent, satisfying
            (\ref{itm:5.15.ii}) and (\ref{itm:5.15.iii}).
        \end{proof}

    \theorem[Theorem 5.18]\label{minimal_conditions_for_excellent_partitions}
        Let $k_*$ and therefore $k_{**}$ be given.
        Then, for all $\epsilon \leq \min \parround{\frac{1}{5}, \frac{1}{2^{k_{**}}}}$ and $m > 1$, there is $M = M\parround{\epsilon, m, k_*}$
        and $N = N\parround{\epsilon, m, k_*}$ such that, for every finite graph $G$ with the non-$k_{*}$-order property, and
        every $A \subseteq G$ with $|A| \geq N$, there exists a partition $\overline{A} = \partriangle{A_i \mid i \in \parcurly{1, \dots, i(*)}}$
        of $A$, such that:
        \begin{enumerate}
            \item \label{itm:5.18.0} The number of parts is bounded by
                $m \leq i(*) \leq M \coloneqq \max \parround{\ceil{\frac{12}{\epsilon}}^{k_{**}+1}, 4m}$. \todo{Move the bound on $M$ to another point?}
            \item \label{itm:5.18.1} For all $i, j \in \parcurly{1, \dots, i(*)}$, $\parstraight{\parstraight{A_i}- \parstraight{A_j}} \leq 1$.
            \item \label{itm:5.18.2} For all $i \in \parcurly{1, \dots, i(*)}$, $A_i$ is $\epsilon$-excellent.
            \item \label{itm:5.18.3} For all $i, j \in \parcurly{1, \dots, i(*)}$, $\parround{A_i, A_j}$ is $\epsilon$-uniform.  \todo{Redundant?}
        \end{enumerate}
        \begin{proof}
            Our goal is to apply Lemma~\ref{resume_of_all_conditions_for_excellent_partitions}.
            Let $q = \ceil{\frac{12}{\epsilon}}$.
            For $N\parround{\epsilon, m, k_*}$, and thus $n$, large enough, we can then choose the smallest $m_{**}$ satisfying:
            \begin{enumerate}[label=(\alph*), ref=\alph*]
                \item \label{itm:5.18.a} $m_{**} \in \parsquared{\delta n -1, \delta n}$, where
                    $\delta = \min \parround{\frac{\epsilon}{\parround{3 + \epsilon} q^{k_{**}}}, \frac{1}{m + q^{k_{**}}}}$
                \item \label{itm:5.18.b} $m_{**} > \frac{3}{\epsilon}$.
                \item \label{itm:5.18.c} $m_{**} > \frac{N_{\ref{existance_of_equitable_excellent_subpartition}}
                    \parround{3, k_*, \epsilon, q^{k_{**}}}}{q}$.
            \end{enumerate}
            We set $m_{k_{**}} = m_{**}$ and we build recursively a sequence of integers
            $\partriangle{m_\ell \mid \ell \in \parcurly{0, \dots, k_{**}}}$ such that $m_\ell = q m_{\ell + 1}$ for all
            $\ell \in \parcurly{0, \dots, k_{**}-1}$.
            Also, let $m_* \coloneqq m_0 = q^{k_{**}} m_{**}$.
            By (\ref{itm:5.18.a}) we have that $m_* \leq \frac{\epsilon n}{3 + \epsilon}$.
            This sequence satisfies all the conditions of Lemma~\ref{resume_of_all_conditions_for_excellent_partitions}:
            \begin{itemize}[label={}]
                \item (\ref{itm:5.15.a}) $m_{\ell+1} = \frac{1}{q} m_\ell \leq \frac{\epsilon}{12} m_\ell$.
                \item (\ref{itm:5.15.b}) $m_{**} \geq \frac{3}{\epsilon}$.
                \item (\ref{itm:5.15.c}) $m_{**} \mid m_\ell$ for all $\ell \in \parcurly{0, \dots, k_{**}}$, since $q$ is an integer.
                \item (\ref{itm:5.15.d}) $m_{k_{**}-1} = q m_{**} >
                    q \frac{N_{\ref{existance_of_equitable_excellent_subpartition}}\parround{3, k_*, \epsilon, q^{k_{**}}}}{q} =
                    N_{\ref{existance_of_equitable_excellent_subpartition}}\parround{3, k_*, \epsilon, \frac{m_*}{m_{**}}}$.
                \item (\ref{itm:5.15.b'}) $m_* < \frac{\epsilon n}{3 + \epsilon} < \frac{1 + \frac{\epsilon}{3}n}{1 + \frac{\epsilon}{3}}$.
                \item (\ref{itm:5.15.a'}) $m_0 = m* < \frac{\epsilon n}{3 + \epsilon} < n$
            \end{itemize}
            We can apply Lemma~\ref{resume_of_all_conditions_for_excellent_partitions} to obtain a partition
            satisfying (\ref{itm:5.18.1}), (\ref{itm:5.18.2}) and (\ref{itm:5.18.3}).

            We proceed to bound the number of parst $i(*)$.
            First, the upper bound follows from the fact that
            $m_{**} \geq \frac{1}{2} \min \parround{\frac{\epsilon} {3 + \epsilon}, \frac{1}{m + q^{k_{**}}}} n$:
            \[
                i(*) \leq \frac{n}{m_{**}} \leq \frac{2 \max \parround{\frac{3 + \epsilon}{\epsilon} q^{k_{**}}, m + q^{k_{**}}}n}{n}
                     < 2 \max \parround{\frac{3 + \epsilon}{\epsilon} q^{k_{**}}, 2m}
                     \leq \max \parround{\ceil{\frac{12}{\epsilon}}^{k_{**}+1}, 4m}
            \]
            In the last inequality, we used that if $m < q^{k_{**}}$, then $m + q^{k_{**}} \leq 2q^{k_{**}} < \frac{3 + \epsilon}{\epsilon} q^{k_{**}}$,
            which is dealt in the first argument of the maximum, so we may assume that $m \geq q^{k_{**}}$.
            We also show that the lower bound is satisfied:
            \[
                i(*) \geq \frac{n - m_*}{m_{**}}
                     \geq \frac{n - m_{**}q^{k_{**}}}{m_{**}}
                     = \frac{n}{m_{**}} - q^{k_{**}}
                     \geq \frac{m + q^{k_{**}}}{n} n - q^{k_{**}}
                     = m
            \]
        \end{proof}

    \begin{remark}
        We now see how large $N$, and thus $n$, actually needs to be.
        First of all, we see that:
        \begin{align*}
            \frac{1}{q} N_{\ref{existance_of_equitable_excellent_subpartition}}\parround{4, k_*, \epsilon, q^{k_{**}}}
                & = \frac{1}{q} N_{\ref{claim_5.13}}\parround{k_*, \frac{1}{4 \cdot 3} \epsilon, q^{k_{**}}} \\
                & = \frac{1}{q} q^{k_{**}} \parround{\frac{12}{\epsilon}}^2
                    \parround{k_* \log{\parround{\frac{12}{\epsilon}}^2 k_*} - \log{\frac{1}{q^{k_{**}}+1}}} \\
                & < k_*^2 q^{2 k_{**} + 3}
        \end{align*}
        Also, $\frac{3}{\epsilon}$ is clearly smaller than this value.
        Then, since $m_{**}$ is the smallest integer larger than both values, we conclude:
        \begin{align*}
            \frac{m_{**}}{\delta}
                & \leq \frac{k_*^2 q^{2 k_{**} + 3}}
                    {\min \parround{\frac{\epsilon}{\parround{3 + \epsilon} q^{k_{**}}}, \frac{1}{m + q^{k_{**}}}}} \\
                & = k_*^2 q^{2 k_{**} + 3} \max \parround{\frac{3 + \epsilon}{\epsilon} q^{k_{**}}, m + q^{k_{**}}} \\
                & \leq \max \parround{q^{k_{**} + 1}, 4m} k_*^2 q^{2 k_{**} + 3}
        \end{align*}
    \end{remark}

    % TODO: define uniformity.
    \lemma[Lemma 5.17]\label{excellence_implies_regularity}
        Suppose that $\epsilon_1, \epsilon_2, \epsilon_3 \in \parround{0, {1 \over 2}}$ with
        ${{\epsilon_1 + \epsilon_2} \over \epsilon_3} < {1 \over 2}$ and the pair $\parround{A,B}$ is
        $\parround{\epsilon_1, \epsilon_2}$-uniform.
        Let $A' \subseteq A$ with $\parstraight{A'} \geq \epsilon_3 \parstraight{A}$,
        $B' \subseteq B$ with $\parstraight{B'} \geq \epsilon_3 \parstraight{B}$ and
        denote $Z = \parcurly{\parround{a,b} \in \parround{A \times B} \mid a R b \not\equiv t\parround{A,B}}$ and
        $Z' = \parcurly{\parround{a,b} \in \parround{A' \times B'} \mid a R b \not\equiv t\parround{A,B}}$.
        Then, we have:
        \begin{enumerate}
            \item \label{itm:5.17.1} ${\parstraight{Z} \over {\parstraight{A} \parstraight{B}}} < \epsilon_1 + \epsilon_2$.
            \item \label{itm:5.17.2} ${\parstraight{Z'} \over {\parstraight{A} \parstraight{B}}} <
                {{\epsilon_1 + \epsilon_2} \over \epsilon_3}$.
        \end{enumerate}
        In particular, if for some $\epsilon_0, \epsilon \in \parround{0, {1 \over 2}}$, the pair
        $\parround{A,B}$ is $\epsilon_0$-uniform, for $\epsilon_0 \leq {{\epsilon^2} \over 2}$, then:
        \begin{enumerate}[label=\alph*., ref=\alph*]
            \item \label{itm:5.17.a} $\parround{A,B}$ is $\epsilon$-regular.
            \item \label{itm:5.17.b} If $A' \in \parsquared{A}^{\geq \epsilon \parstraight{A}} and $
                $B' \in \parsquared{B}^{\geq \epsilon \parstraight{B}}$, then $d \parround{A',B'} < \epsilon$ or
                $d \parround{A',B'} \geq 1 - \epsilon$.
        \end{enumerate}
        \begin{proof}
            Let $U = \parcurly{a \in A \mid \parstraight{\overline{B}_{B,a}} > \epsilon_1 \parstraight{A}}$, i.e. the set
            of exceptional vertices $a \in A$.
            Then,
            \[
                Z \subseteq U \times B \cup \bigcup_{a \in A \setminus U} \parcurly{a} \times \overline{B}_{B,a}
            \]
            and
            \[
                Z' \subseteq U \times B' \cup \bigcup_{a \in A' \setminus U} \parcurly{a} \times \overline{B}_{B,a}
            \]
            Notice that, if $a \in A \setminus U$, then $\parstraight{\overline{B}_{B,a}} < \epsilon_2 \parstraight{B}$, so
            \[
                \parstraight{Z} < \epsilon_1 \parstraight{A} \parstraight{B} + \parstraight{A} \epsilon_2 \parstraight{B}
            \]
            which can be written as
            \[
                    \frac{\parstraight{Z}}{\parstraight{A} \parstraight{B}} < \epsilon_1 + \epsilon_2
            \]
            which proves (\ref{itm:5.17.1}).
            Similarly,
            \begin{align*}
                \parstraight{Z'} & \leq \parstraight{U} \parstraight{B'} + \parstraight{A'} \max \parcurly{\parstraight{\overline{B}_{B,a}} \mid a \notin U} \\
                                 & < \epsilon_1 \parstraight{A} \parstraight{B'} + \parstraight{A'} \epsilon_2 \parstraight{B}
            \end{align*}
            By dividing both sides by $\parstraight{A'} \parstraight{B'}$ we conclude
            \[
                {\parstraight{Z'} \over \parstraight{A'} \parstraight{B'}} < \epsilon_1 \frac{\parstraight{A}}{\parstraight{A'}} + \epsilon_2 \frac{\parstraight{B}}{\parstraight{B'}}
                \leq {{\epsilon_1 \parstraight{A}} \over {\epsilon_3 \parstraight{A}}} + {{\epsilon_2 \parstraight{B}} \over {\epsilon_3 \parstraight{B}}}
                = {{\epsilon_1 + \epsilon_2} \over \epsilon_3}
            \]
            proving (\ref{itm:5.17.2}).
            Let's now prove (\ref{itm:5.17.a}) and (\ref{itm:5.17.b}).
            First of all, notice that:
            \todo{This only works so nice when $A$ and $B$ are disjoint. Check what happens when they are not.
                Something more on the line of $d\parround{A,B} > 1 - 4 \parround{\epsilon_1 + \epsilon_2}$}
            \begin{itemize}
                \item \underline{if $t\parround{A,B} = 1$}, then $d\parround{A,B} > 1 - \parround{\epsilon_1 + \epsilon_2}$
                    and $d \parround{A',B'} > 1 - {{\epsilon_1 + \epsilon_2} \over \epsilon_3}$, which follows (\ref{itm:5.17.1}) and
                    (\ref{itm:5.17.2}) respectively.
                    Thus,
                    \begin{align*}
                        \parstraight{d \parround{A,B} - d \parround{A',B'}}
                            & \leq \max \parcurly{d \parround{A,B} - d \parround{A',B'}, d \parround{A',B'} - d \parround{A,B}} \\
                            & < \max \parcurly{ 1 - \parround{1 - {{\epsilon_1 + \epsilon_2} \over \epsilon_3}},
                                1 - \parround{1 - \epsilon_1 - \epsilon_2}} \\
                            & = {{\epsilon_1 + \epsilon_2} \over \epsilon_3}
                    \end{align*}
                \item \underline{if $t\parround{A,B} = 0$}, similarly $d\parround{A,B} < \parround{\epsilon_1 + \epsilon_2}$
                    and $d \parround{A',B'} < {{\epsilon_1 + \epsilon_2} \over \epsilon_3}$.
                    Thus,
                    \begin{align*}
                        \parstraight{d \parround{A,B} - d \parround{A',B'}}
                            & \leq \max \parcurly{d \parround{A,B} - d \parround{A',B'}, d \parround{A',B} - d \parround{A,B}} \\
                            & < \max \parcurly{\parround{\epsilon_1 + \epsilon_2}, {{\epsilon_1 + \epsilon_2} \over \epsilon_3}} \\
                            & = {{\epsilon_1 + \epsilon_2} \over \epsilon_3}
                    \end{align*}
            \end{itemize}
            In both cases, we have that $\parstraight{d \parround{A,B} - d \parround{A',B'}}$ is bounded by
            ${{\epsilon_1 + \epsilon_2} \over \epsilon_3} < {1 \over 2}$.
            Also, $d \parround{A', B'}$ may only differ by ${{\epsilon_1 + \epsilon_2} \over \epsilon_3}$ with either
            $0$ or $1$.
            In particular, we may choose $\epsilon_3 = \epsilon$ and $\epsilon_1 = \epsilon_2 = \epsilon_0 \leq {{\epsilon^2} \over 2}$.
            This way, the condition ${{\epsilon_1 + \epsilon_2} \over \epsilon_3} \leq \epsilon < {1 \over 2}$ is satisfied.
            We conclude that $\parround{A,B}$ is $\epsilon$-regular (\ref{itm:5.17.a}) and that $d \parround{A',B'}$ is either
            $< \epsilon$ or $\geq 1 - \epsilon$ (\ref{itm:5.17.b}).
        \end{proof}

    \theorem[Theorem 5.19]\label{existance_of_regular_partitions}
        For every $k_* \in \naturals$ and $\epsilon \in \parround{0, {1 \over 2}}$ and $m > 1$, there exist $N = N\parround{\epsilon, m, k_*}$
        and $M = M\parround{\epsilon, m, k_*}$ such that, for every finite graph $G$ with the non-$k_{*}$-order property,
        and every $A \subseteq G$ with $|A| \geq N$, there is $m < \ell < M$ and a partition
        $\overline{A} = \partriangle{A_i \mid i \in \parcurly{1, \dots, \ell}}$ of $A$ such that each $A_i$ is
        ${{\epsilon^2} \over 2}$-excellent, and for every $i, j \in \parcurly{1, \dots, \ell}$,
        \begin{enumerate}
            \item \label{itm:5.19.1} $\parstraight{\parstraight{A_i}- \parstraight{A_j}} \leq 1$.
            \item \label{itm:5.19.2} $\parround{A_i, A_j}$ is $\epsilon$-regular, and moreover if
                $B_i \in \parsquared{A_i}^{\geq \epsilon \parstraight{A_i}}$ and $B_j \in \parsquared{A_j}^{\geq \epsilon \parstraight{A_j}}$,
                then either $d\parround{B_i, B_j} < \epsilon$ or $d\parround{B_i, B_j} \geq 1 - \epsilon$.
            \item \label{itm:5.19.3} If $\epsilon \leq \min \parround{\frac{1}{5}, \frac{1}{2^{k_{**}}}}$, then
                $M \leq \max \parround{\ceil{\frac{12}{\epsilon}}^{k_{**}+1}, 4m}$.
        \end{enumerate}
        \begin{proof}
            If $\epsilon \leq \min \parround{\frac{1}{5}, \frac{1}{2^{k_{**}}}}$, then we can apply Theorem~\ref{minimal_conditions_for_excellent_partitions}
            to $A$ with ${{\epsilon^2} \over 2}$, and then use Lemma~\ref{excellence_implies_regularity} to replace the
            ${{\epsilon^2} \over 2}$-uniformity of pairs by $\epsilon$-regularity.
            Otherwise, to get (\ref{itm:5.19.1}) and (\ref{itm:5.19.2}), just do the same process for some
            $\epsilon' = \min \parround{\frac{1}{5}, \frac{1}{2^{k_{**}}}} \leq \epsilon$.
            Then, since regularity is monotone, we get the wanted $\epsilon$-regularity from the resulting $\epsilon'$-regularity.
            In this last case, the bound on $M$ is $M \leq \max \parround{\ceil{\frac{12}{\epsilon'}}^{k_{**}+1}, 4m}$.
        \end{proof}

    \begin{remark}
        By Theorem~\ref{tree_implies_order}, we have that $k_{**} \leq 2^{k_* + 1}-2$ in the context of the non-$k_*$-order
        property.
        Thus, the bound on the number of parts $M$ can clearly be reformulated as a function of only $k_*$, $\epsilon$ and $m$:
        \[
            M \leq \max \parround{\ceil{\frac{12}{\epsilon}}^{2^{k_* + 1}-1}, 4m}
        \]
    \end{remark}

    % up to here it was compiling with no problem.




        \newpage

    \section{Property Testing} \label{sec:section_6}

    Property testing is a field of theoretical computer science, concerned about finding low-complexity algorithms
    for testing (approximate) properties in large objects, such as graphs.
    These algorithms need to be successful with high probability, and are only required to distinguish between objects
    that do not satisfy the property, and those which are \say{far} from satisfying it.
    For the purposes of this thesis, it is useful to formalize these concepts in the context of graphs.

    \begin{definition}
        We say that a graph $G$ is \emph{$\epsilon$-far} from satisfying a graph property $\mathcal{P}$ if no adding or
        removing of up to $\epsilon {|G| \choose 2}$ edges in $G$ results in the graph satisfying the property.
    \end{definition}

    \begin{definition} \label{def:epsilon_test}
        An \emph{$\epsilon$-test} $\mathcal{A}$ deciding a graphs property $\mathcal{P}$ with query complexity
        $q(n)$ is a randomized algorithm that, on input graph $G$ of size $n$,
        satisfies:
        \begin{enumerate}
            \item If $G \in \mathcal{P}$, then $P\parround{\mathcal{A} \text{ accepts } G} \geq \frac{2}{3}$.
            \item If $G$ is $\epsilon$-far from satisfying $\mathcal{P}$,
                then $P\parround{\mathcal{A} \text{ rejects } G} \geq \frac{2}{3}$.
        \end{enumerate}
        The query complexity $q(n)$ is the maximum number of queries the algorithm can make, and (in our case)
        a query discerns whether a desired pair of vertices in the input graph $G$ are adjacent or not.
    \end{definition}

    Of course, the most desirable testers are those with lower query complexity.
    A class of particular interest is that of testers whose complexity does not grow with the size of the
    graph $G$.

    \begin{definition}
        We say that a property $\mathcal{P}$ is \emph{testable} if there exists an $\epsilon$-test deciding $\mathcal{P}$
        with a constant query-complexity with respect to the size of the input graph, that is, it only depends on the
        parameter $\epsilon$.
    \end{definition}

    In~\cite{a_characterization_of_the_natural_graph_properties_testable_with_one_sided_error}, Alon and Shapira
    showed that a large class of properties, a subclass of which will be the center of our attention, are testable.

    \begin{theorem}[Alon \& Shapira Theorem in~\cite{a_characterization_of_the_natural_graph_properties_testable_with_one_sided_error}]
        \label{thm:alon_and_shapira_theorem}
        Every hereditary graph property is testable (with one-sided error).
    \end{theorem}

    A property is said to be \emph{hereditary} if it is preserved under taking induced subgraphs.
    A property is testable \emph{with one-sided error} if the first condition in \Cref{def:epsilon_test}
    is strengthened to $P\parround{\mathcal{A} \text{ accepts } G} = 1$, and thus the associated algorithm does
    not give false negatives.

    Towards Alon \& Shapira Theorem,~\cite{efficient_testing_of_large_graphs} shows that \emph{$H$-freeness} is testable.
    A graph $G$ is said to be $H$-free, where $H$ is another graph, if no copy of $H$ appears as an induced subgraph in $G$.
    Clearly, $H$-freeness is an hereditary property, and it will be the focus of our attention later in the section.

    Both~\cite{} and~\cite{} made use of what it has become known as the ‘Strong Regularity lemma’.
    The Strong Regularity Lemma is a stronger version of Szemer\'edi's Regularity Lemma.
    It features a partition and one refinement of it, both regular, with the second having a stronger regularity then
    the first one.
    The partition's regularity can be any given value, and the refinement's regularity is a function
    on the number of pairs of the first partition.
    Again, such function can be any \say{reasonable} one (See~\cite{efficient_testing_of_large_graphs}[Lemma 4.1]).
    The goal of this strenghthened version of the regularity lemma is to be able to deal with the counting of the
    induced copies of graphs within G. Indeed, in the usual regularity lemma, theorem ~\ref{}, there are irregular
    pairs of clusters.
    Recall that these irregular pairs are unavoidable (see the comment in the introduction, due to the possible presence
    of a large half-graph).
    Since there is no control on the edges of the irregular pairs (the only information is that they are not regular,
    which is not very restrictive), it is hard to know how the induced copies behave when some of the edges or non-edges
    should belong to these irregular pairs.
    See … (put the example of the irregular pairs from the book of Zhao, or put the example that we developed regarding
    the induced path with 4 vertices), where each modification of the edges withing the irregular pair will shift
    induced copies of $P_4$, and thus, it is not completely clear how to eliminate those copies.

    In order to solve these technical challanges, Alon …. and Szegedy \cite[Theorem~]{} applied the regularity lemma
    twice, refining an already regular partition, in order to gain control on most of the parts of the refined
    (aquí pots explicar amb una mica de bla bla bla l statement).
    However, this implies that the number of parts in the final refinement increases.
    \footnote{As the reader might have guess, there is a \emph{weak} version of the regularity lemma, in which the
        number of parts is much smaller, but less information can be gathered from the partition. This result can be found
        in \cite{} and, together with the regular and the strong version of the regularity lemma, can be seen as one of a
        sequence of many possible regularity lemmas with varying strength \cite{regulary lemma for the analyst, lov szegedy}.}

    %%%%
    This requires the use of the so-called Strong Regularity Lemma\footnote{It is a
        stronger version of Szemer\'edi's Regularity Lemma, and it is shown in Lemma 4.1 of~\cite{efficient_testing_of_large_graphs}.
        It features a partition and one refinement of it, both regular,
        with the second having a stronger regularity then the first one.
        The partition's regularity can be any given value, and the refinement's regularity is a function
        on the number of pairs of the first partition.
        Again, such function can be any \say{reasonable} one.},
    which aims at dealing with the irregular
    pairs from the original version, of which edges we know nothing and thus hinder property testing.
    It uses a double partitioning of the vertex set in order to obtain a family of disjoint sets
    with no irregular pairs.
    This double partition increases the number of parts, and thus the query complexity of the associated tester, which
    is already extremely large, although constant, due to the tower of powers function bound which unavoidable in the
    general setting~\cite{lower_bounds_of_tower_type_for_szeremedis_uniformity_lemma}.
    Then, \Cref{thm:alon_and_shapira_theorem} is a generalization of the results for $H$-freeness to all
    hereditary properties.
    %%%%

    A natural solution in order to avoid such immense bound from the Regularity Lemma is to restrict the problem
    to a favorable class of graphs, for example, restricting to the class of graphs with bounded VC-dimension
    (See \Cref{def:VC_dimension}).
    %%START
    In the context of bipartite graphs with bounded VC-dimension,~\cite{efficient_testing_of_bipartite_graphs_for_forbidden_induced_subgraphs}
    shows a family of regularity lemmas (from weaker to stronger than the Szemer\'edi's) which bound on the number of parts
    is only polynomial in $\epsilon^{-1}$.
    %%END

    A first result in that direction, with the focuss on bipartite graphs, appeared in \cite{??}, where the bound on the
    number of parts of the regular partition is greatily improved, and also the information that each regular pair
    is either almost or almost empty of edges is given.
    The version where no restriction was placed on the graph G was given in \cite{??},
    and the one with the best bounds in this context is given by
    the~\cite[theorem ..]{fox suk…},
    which we quote below;

    \begin{theorem}
    \end{theorem}

    i hi poses les cotes.

    The author uses the stronger version to prove testability of $H$-freeness, with a much lower query complexity.
    \todo{Lluis, imagino que no ho diu aixó ultim al paper, pero está clar que baixa, no?}
    Then,~\cite{??} shows a similar result but generalizing to all graphs with bounded VC-dimension,
    and~\cite{erdos_hajnal_conjecture_for_graphs_with_bounded_vc_dimension} improves this bound further.
    These last two results use the so-called Ultra Strong Regularity Lemma, in which (most of) the resulting partition
    pairs have density either at most $\epsilon$ or at least $1 -\epsilon$, which they call \say{$\epsilon$-homogeneous}.

    %%START
    Still, the class of graphs with bounded VC-dimension does not avoid irregular pairs, and thus neither avoids
    the associated increase in parts when further refining the partition when proving the testability of $H$-freeness
    (which is a standard procedure to deal with such pairs~\cite{efficient_testing_of_large_graphs}).
    %%END

    Note that, even with the problem of getting better bounds in the regularity lemma addressed, the regular partitions
    of graphs with bounded VC-dimension may contain irregular pairs (Indeed, the half-graph itself has bounded
    VC-dimension).
    The lack of irregular pairs in the class of stable graphs, allows us to use Theorem~\ref{??}
    (regularity lemma for stable graphs) to give an straightforward argument that proves that  being (induced)
    H-free is testable, with one-sided error, with some explicit (and more reasonable) bounds than the ones given by
    Theorem~\ref{??} (alon shapira).
    The algorithm is given as Algorithm 1, the proof of the correctness and its explanation is given just before.
    Let us remark that using l’esquema de l’Alon-…szegedy usant el resultat per la bounded VC.-dimension, one could be
    able to obtain bounds on the … of the type … which are better than … For further discussion the reader is referred
    to the end of the section (posarem alguna cosa per dir que el del irregular pairs, de fet, permet dir més).

    Now, by moving to a subclass of graphs with bounded VC-dimension, the class of stable graphs,
    all these difficulties are easily avoided by using the Stable Regularity Lemma instead.
    The partition size is only a double exponential of the error parameter $\epsilon$, irregular pairs are
    completely avoided, and all pairs are also $\epsilon$-homogeneous.
    It needs to be noted that in the Stable Regularity Lemma the exponent of $\epsilon^{-1}$ in the bound of the number
    of parts is an exponential on the size of the avoided half-graph, against the lower polynomial exponent of the
    Ultra Strong Regularity Lemma.
    It is an open problem whether the bound from the Stable Regularity Lemma can be improved~\cite{julia_cositas}.

    The remaining of this section will be dedicated to the construction of an $\epsilon$-test for $H$-freeness in stable graphs.
    %A graph $G$ is said to be $H$-free, where $H$ is another graph, if no copy of $H$ appears as an induced subgraph in $G$.
    Such an $\epsilon$-test needs to be able to distinguish between graphs that are $H$-free and graphs that
    are $\epsilon$-far from it, with some error.
    In fact, our $\epsilon$-test will only have one-sided error, as if the input graph is $H$-free the tester will
    report so with probability $1$.

    The first step towards constructing such tester is proving \Cref{thm:property_testing_with_stable_partitions}.
    This theorem uses the Stable Regularity Lemma to prove that a graph being $\epsilon$-far
    from being $H$-free implies it containing many (as a fixed fraction of all induced subgraphs of size $|H|$)
    induced copies of $H$.
    This point is central for the construction, and once proved we can simply let the tester ask for all the edges
    in a sample of vertices of fixed size.
    The algorithm then simply checks whether a copy of $H$ can be found in the subgraph induced by the sample, and report
    accordingly.

    \subsection{Unavoidable is Abundant} \label{subsec:subsection_6.1}

        We now briefly formalize the concepts of being far from $H$-freeness, and containing many copies of $H$ using the
        notation from~\cite{efficient_testing_of_large_graphs}.

        \begin{definition} \label{def:unavoidable}
            A graph $H$ is \emph{$\gamma$-unavoidable} in a graph $G$ if no adding or removing of up to $\gamma {|G| \choose 2}$
            edges in $G$ results in $H$ not appearing as an induced subgraph of $G$.
        \end{definition}

        \begin{definition} \label{def:abundant}
            A graph $H$ is \emph{$\eta$-abundant} in a graph $G$ if $G$ contains at least $\eta |G|^{|H|}$
            induced copies of $H$.
        \end{definition}

        An important property of regularity, which is needed for the proof of the theorem, is that the regularity is
        partially maintained when moving to subsets.
        Not only that, but it also ensures that the density of the pair does not change too much.
        \todo{Possibly move this to section 2.}

        \begin{lemma}[Lemma 3.1 in~\cite{efficient_testing_of_large_graphs}] \label{lem:regularity_is_transitive}
            Let $\epsilon \leq \epsilon' < \frac{1}{2}$ and $\delta \in \parround{0, 1}$.
            If $\parround{A,B}$ is an (not necessarily disjoint) $\epsilon$-regular pair with density $\delta$, $A' \subseteq A$ with
            $A \geq \epsilon' |A|$, and $B' \subseteq B$ with $|B'| \geq \epsilon' |B|$, then $\parround{A', B'}$ is an
            $\parround{\frac{\epsilon}{\epsilon'}}$-regular pair with density at least $\delta - \epsilon$ and at most $\delta + \epsilon$.
            \begin{proof}
                Let $A'' \subseteq A' \subseteq A$, $B'' \subseteq B' \subseteq B$ be such that
                \begin{align*}
                    |A''| & \geq \frac{\epsilon}{\epsilon'} |A'| \geq \frac{\epsilon}{\epsilon'} \epsilon' |A| = \epsilon |A|
                    \text{ and } \\
                    |B''| & \geq \frac{\epsilon}{\epsilon'} |B'| \geq \frac{\epsilon}{\epsilon'} \epsilon' |B| = \epsilon |B|
                \end{align*}
                By $\epsilon$-regularity of $\parround{A,B}$, $\parstraight{d(A,B) - d(A'',B'')} < \epsilon$.
                Thus,
                \begin{align*}
                    \parstraight{d(A',B') - d(A'',B'')}
                        & = \parstraight{d(A',B') - d(A,B) + d(A,B) - d(A'',B'')} \\
                        & \leq \parstraight{d(A',B') - d(A,B)} + \parstraight{d(A,B) - d(A'',B'')} \\
                        & < 2 \epsilon \leq \frac{\epsilon}{\epsilon'}
                \end{align*}
                This proves the $\parround{\frac{\epsilon}{\epsilon'}}$-regularity of $\parround{A',B'}$.

                Also, since $\parround{A,B}$ is $\epsilon$-regular, $\parstraight{d(A,B) - d(A',B')} < \epsilon$,
                and thus,
                \[
                    \delta - \epsilon < d(A',B') < \delta + \epsilon
                \]
            \end{proof}
        \end{lemma}

        The pivotal point in the proof of \Cref{thm:property_testing_with_stable_partitions} is the fact that, if the
        reduced graph \todo{Define reduced subgraph as a remark of the stable regularity lemma.} from a regular partition contains an induced structure resembling $H$, i.e. where pairs of parts are
        mostly connected if the corresponding vertices in $H$ are connected, and mostly not connected otherwise,
        then the original graph contains many induced copies of $H$ (this is a version of the so called \emph{Counting Lemma}
        from~\cite{the_regulariy_lemma_and_its_applications_in_graph_theory}).
        The following lemma formalizes this idea.

        \begin{lemma}[Lemma 3.2 in~\cite{efficient_testing_of_large_graphs}] \label{lem:H_like_partition_implies_H_abundance}
            For every $\delta \in \parround{0, 1}$ and $\ell > 0$ there exist $\epsilon = \epsilon \parround{\delta, \ell}$ and
            $\eta = \eta \parround{\delta, \ell}$ satisfying the following property:

            Let $H$ be a graph with vertices $v_1, \dots,v_\ell$ and let $V_1, \dots, V_\ell$ be an $\ell$-tuple of (not necessarily disjoint)
            sets of vertices of a graph $G$ such that for every $1 \leq i < i' \leq \ell$, the pair $\parround{V_i, V_{i'}}$
            is $\epsilon$-regular, with density at least $\delta$ if $v_i v_{i'}$ is an edge of $H$, and at most $1 - \delta$
            if $v_i v_{i'}$ is not an edge of $H$.
            Then, at least $\eta \prod_{i=1}^\ell \parstraight{V_i}$ of $\ell$-tuples $w_1 \in V_1, \dots, w_\ell \in V_\ell$
            span induced copies of $H$ where $w_i$ plays the role of $v_i$.
            \begin{proof}
                Without loss of generality, we assume that $H$ is the complete graph, since we can simply replace each non-edge
                $v_i v_{i'}$ of $H$ with an edge by exchanging all edges and non-edges between $V_i$ and $V_{i'}$.

                We prove the lemma by induction on $\ell$.
                The case $\ell=1$ is trivial, and the number of induced copies of $H$ is $|V_1|$, so $\eta(\delta, 1) = 1$ and
                $\epsilon(\delta, 1) = 1$ (No regularity needed if no pairs).
                The I.H. is that the values $\eta(\delta, \ell-1)$ and $\epsilon(\delta, \ell-1)$ exist and are known for all
                $\ell$.
                We proceed to prove that the following values $\eta$ and $\epsilon$ hold:
                \begin{align*}
                    \epsilon &= \epsilon(\delta, \ell)
                        = \min \bbparround{\frac{1}{2\ell - 2}, \frac{1}{2} \delta \epsilon(\frac{1}{2}\delta, \ell-1)} \\
                    \eta &= \eta(\delta, \ell)
                        = \frac{1}{2} (\delta - \epsilon)^{\ell-1} \eta(\frac{1}{2}\delta, \ell-1)
                \end{align*}
                For each $1 < i \leq \ell$, the number of vertices of $V_1$ which have less then
                $(\delta - \epsilon) \parstraight{V_i}$ neighbors in $V_i$ is less then $\epsilon \parstraight{V_i}$.
                Otherwise, the set of such vertices, say $U \in [V_1]^{\geq \epsilon \parstraight{V_1}}$ together with $V_i$
                would form a subpair $\parround{U, V_i}$ with density $< \delta - \epsilon$ which, by
                \Cref{lem:regularity_is_transitive} contradicts the $\epsilon$-regularity of the pair $\parround{V_1, V_i}$.

                Therefore, at least $(1 - (\ell -1) \epsilon) \parstraight{V_1}$ of the vertices of $V_1$ have at least
                $(\delta - \epsilon) \parstraight{V_i}$ neighbors in $V_i$ for all $1 < i \leq \ell$.
                In particular, since $\epsilon \leq \frac{1}{2\ell - 2}$ we have that $(\ell - 1) \epsilon \leq \frac{1}{2}$
                and then $1 - (\ell - 1) \epsilon \geq \frac{1}{2}$, so at least half of the vertices of $V_1$ satisfy the
                above condition.

                For each such vertex $w_1 \in V_1$, let $V_i'$ denote the subset of vertices of $V_i$ which are neighbors
                of $w_1$.
                Since $\epsilon \leq \frac{1}{2}\delta$, \Cref{lem:regularity_is_transitive} implies that for all
                $1 < i < i' \leq \ell$, the pair $\parround{V_i', V_{i'}'}$ is $\parround{\frac{\epsilon}{\delta-\epsilon}}$-regular,
                and given that $\parround{\frac{\epsilon}{\delta-\epsilon}} \leq \parround{\frac{2\epsilon}{\delta}} \leq \epsilon\parround{\frac{1}{2}\delta, \ell-1}$,
                it is $\epsilon\parround{\frac{1}{2}\delta, \ell-1}$-regular.
                Also, it has density at least $\delta - \epsilon \geq \frac{1}{2} \delta$.
                By the induction hypothesis, we have at least
                \[
                    \eta \parround{\frac{1}{2}\delta, \ell-1} \prod_{i=2}^\ell \parstraight{V_i'}
                        \geq \eta \parround{\frac{1}{2}\delta, \ell-1} \prod_{i=2}^\ell \parround{\delta - \epsilon} \parstraight{V_i}
                \]
                possible choices of $w_2 \in V_2, \dots, w_\ell \in V_\ell$ such that the induced subgraph spanned by
                $w_1, \dots, w_\ell$ is complete.
                Since there are at least $\frac{1}{2} \parstraight{V_1}$ vertices $w_1$ which satisfy the above condition,
                the chosen values of $\eta$ satisfies the lemma, and we are done.
            \end{proof}
        \end{lemma}

        \begin{remark}
            The non-recursive form of $\epsilon$ and $\eta$ for $\ell > 1$ is:
            \begin{align*}
                \epsilon(\delta, \ell) & = \bbparround{\frac{1}{2}}^{\frac{\ell(\ell-1)}{2}} \cdot \delta^{\ell-1} \\
                \eta(\delta, \ell) & \geq \bbparround{\frac{1}{2}}^{\frac{\ell^3+5\ell-6}{6}} \cdot \delta^{\frac{\ell(\ell-1)}{2}}
            \end{align*}
        \end{remark}

        We are now ready to prove the main theorem of this section.
        The proof is similar to that of~\cite[Theorem~5.1]{efficient_testing_of_large_graphs},
        but with some major simplification and optimization allowed by using the Stable Regularity Lemma.
        The main difference is the fact that we do not need to refine the partition to get rid of irregular pairs.
        To resume, we first apply \Cref{thm:existance_of_regular_partitions} to get a regular partition,
        then, we create a copy of the graph where pairs become either complete or empty, by adding or subtracting,
        overall, less than $\gamma {|G| \choose 2}$ edges.
        By the $\gamma$-unavoidability of $H$, this new graph still contains a copy of $H$.
        This fact ensures the existence of an induced structure in the partition of the original graph which allows
        us to apply \Cref{lem:H_like_partition_implies_H_abundance} and conclude that $H$ is abundant in $G$.
        Such conclusion is formalized in the following theorem.

        \begin{theorem} \label{thm:property_testing_with_stable_partitions}
            For every $k_*, \gamma, \ell$ there is a $\eta(k_*, \gamma, \ell)$ such that if $H$ is a graph with $\ell$
            vertices, $G$ has the non-$k_*$-order property and $H$ is $\gamma$-unavoidable in $G$, then $H$ is
            $\eta$-abundant in $G$.
            \begin{proof}
                Apply \Cref{thm:existance_of_regular_partitions} to $G$ with $\epsilon = \min \bparround{\sqrt{\gamma},
                    \epsilon_{\ref{lem:H_like_partition_implies_H_abundance}} \parround{1 - \sqrt{\gamma}, \ell}}$,
                $k_*$ and $m=0$.
                We have a partition $\overline{A} = \bparcurly{A_i \mid i \in \parcurly{1, \dots, m_*}}$ into $m_* \leq M$
                disjoint parts with,
                \[
                    M \leq \ceil{12 \max\Parround{\frac{1}{\sqrt{\gamma}}, \frac{1}{\epsilon_{\ref{lem:H_like_partition_implies_H_abundance}}
                        \parround{1 - \sqrt{\gamma}, \ell}}}}^{2^{k_*+1}-1}
                \]
                such that all pairs of parts are $\epsilon$-regular.
                Also, by \Cref{rmk:excellence_imply_little_exceptions} and $\frac{\epsilon^2}{2}$-excellence of the parts,
                pairs have density at most $\epsilon^2$ or at least $1 - \epsilon^2$.

                Next, we modify the graph $G$ into $G'$ by only adding and removing no more than $\gamma {|G| \choose 2}$
                edges:
                \begin{itemize}
                    \item For each pair of parts $\parround{A_{i_1}, A_{i_2}}$ with $i_1 \neq i_2$, if the pair's density is
                        at most $\epsilon^2$, we remove all edges between $A_{i_1}$ and $A_{i_2}$.
                        Otherwise, the pair's density is at least $1 - \epsilon^2$, and we add all remaining edges.
                        This changes at most a fraction $\epsilon^2$ of the edges between (disjoint) parts.
                    \item For each self-pair $\parround{A_i, A_i}$, if the pair's density is at most $\epsilon^2$, again we
                        remove all edges in $A_i$.
                        Otherwise, the pair's density is at least $1 - \epsilon^2$, and we add all remaining edges.
                        Notice that, in self-pairs, the density ($1$ minus the density respectively) is at most
                        the fraction of possible edges in the pair that actually are edges (non-edges), as noted in
                        \Cref{rmk:density_vs_real_density}.
                        Thus, the fraction of changed edges in all self-pairs is at most $\epsilon^2$.
                \end{itemize}
                The resulting graph $G'$ differs from $G$ in at most $\epsilon^2 {|G| \choose 2} \leq \gamma {|G| \choose 2}$
                edges, and satisfies that each pair of parts (disjoint or not) is either complete or empty.
                Then, the $\gamma$-unavoidability of $H$ in $G$ ensures that there is still a copy of $H$ in $G'$.
                Denote its vertices $v_{i_1}, \dots, v_{i_\ell}$, choosing $i_1, \dots, i_\ell$ such that
                $v_{i_1} \in A_{i_1}, \dots, v_{i_\ell} \in A_{i_\ell}$.

                Notice that $A_{i_1}, \dots, A_{i_\ell}$ satisfy the conditions of \Cref{lem:H_like_partition_implies_H_abundance}
                with $\delta_{\ref{lem:H_like_partition_implies_H_abundance}} = 1 - \sqrt{\gamma}$:
                each pair $\parround{A_{i_j}, A_{i_{j'}}}$ with $j \neq j'$ is $\epsilon$-regular,
                and since $\epsilon \leq \epsilon_{\ref{lem:H_like_partition_implies_H_abundance}} \parround{1 - \sqrt{\gamma}, \ell}$,
                in particular is $\epsilon_{\ref{lem:H_like_partition_implies_H_abundance}} \parround{1 - \sqrt{\gamma}, \ell}$-regular.
                Hence, the lemma guarantees that there are at least $\eta_{\ref{lem:H_like_partition_implies_H_abundance}}
                    \parround{1 - \sqrt{\gamma}, \ell} \prod_{j=1}^\ell \parcurly{A_{i_j},j}$
                copies of $H$ in $G$.

                The fraction of induced copies of $H$ in $G$ is at least
                \[
                    \frac{\eta_{\ref{lem:H_like_partition_implies_H_abundance}} \parround{1 - \sqrt{\gamma}, \ell}
                        \prod_{j=1}^\ell \parcurly{A_{i_j}}}{n^\ell}
                        \geq \eta_{\ref{lem:H_like_partition_implies_H_abundance}} \parround{1 - \sqrt{\gamma}, \ell}
                            \bbparround{\frac{\frac{n}{M}}{n}}^{\ell}
                        = \eta_{\ref{lem:H_like_partition_implies_H_abundance}} \parround{1 - \sqrt{\gamma}, \ell}
                            \parround{M}^{-\ell}
                        \eqqcolon \eta
                \]
                and $H$ is at least $\eta$-abundant in $G$.
            \end{proof}
        \end{theorem}

        Notice that this same result can be proved in the general context instead of only for stable graphs
        as the original Theorem 5.1 from~\cite{efficient_testing_of_large_graphs} proves.
        The difference is that the resulting $\eta$ is much larger (although not given explicitly).
        The main reasons of such an improvement, as mentioned earlier, is that neither a double partition is required to
        elude irregular pairs, neither the bound on the number of parts is a tower of powers.
        This allows the resulting set of parts $A_{i_1}, \dots, A_{i_\ell}$, the ones used as a restricted copy of $H$ in $G$
        in the previous theorem, to be a larger proportion of the whole graph, and thus inducing more copies of $H$.

        \begin{remark}
            We now provide a more explicit lower bound for $\eta$ only depending on $\gamma$, $k_*$ and $\ell$ is:
            \[
                \eta \geq \Parround{\frac{1}{2}}^{\frac{\ell^3 + 5\ell -6}{6}}
                    \cdot \delta^{\frac{\ell(\ell-1)}{2}}
                    \cdot \Parround{\frac{1}{24} \min\Parcurly{\sqrt{\gamma},
                        \Parround{\frac{1}{2}}^{\frac{\ell(\ell-1)}{2}} \delta^{\ell-1}}}^{\ell \Parround{2^{k_*+1}-1}}
            \]
        \end{remark}

    \subsection{The Algorithm} \label{subsec:subsection_6.2}

        Now we have all the tools needed to build an $\epsilon$-test $\mathcal{A}$, which decides
        $H$-freeness for a given graph $H$ of size $\ell$, in the context of graphs with the non-$k_*$-order property.

        $\mathcal{A} = \mathcal{A}(H,\epsilon,k_*)$ works as follows.
        Given a graph $H$, with all edges known, a natural number $k_*$, a real number $\epsilon$, and a graph $G$ with
        the non-$k_*$-order property, whose edges are unknown, the algorithm computes $\ell = |H|$, and the value
        $t = \frac{\ell \log\parround{\frac{2}{3}}}
            {\log\parround{1 - \eta_{\ref{thm:property_testing_with_stable_partitions}}(k_*, \epsilon, \ell)}}$.
        It then samples $t$ different vertices from $G$ uniformly at random.
        $\mathcal{A}$ queries all edges from the sampled set, and checks whether a copy of $H$ as an induced
        subgraph can be found in it.
        If a copy of $H$ is found, then $\mathcal{A}$ accepts $G$.
        Otherwise, $\mathcal{A}$ rejects it.
        See \Cref{alg:h-freeness_tester} for a more detailed step to step description of $\mathcal{A}$.

        \begin{algorithm}[H]
            \caption{$\epsilon$-test $\mathcal{A}$ for deciding $H$-freeness for a given graph $H$ of size $\ell$}
            \label{alg:h-freeness_tester}
            \begin{algorithmic}[1]
                \Require a graph $H$ of size $\ell$, a natural number $k_*$ and a real number $\epsilon > 0$.
                \Require an oracle $\mathcal{O}$ accepting queries of whether two vertices of a graph $G$ are adjacent.
                    The graph $G$ has the non-$k_*$-order property, and only its size $n$ is known.
                \State $t \leftarrow \ell \log\parround{\frac{2}{3}} /
                    \log\parround{1 - \eta_{\ref{thm:property_testing_with_stable_partitions}}(k_*, \epsilon, \ell)}$ \Comment{
                        Compute sample size $t$}
                \If{$n < \ell$} \Comment{Check if $G$ is large enough to contain $H$}
                    \State $\mathcal{A}$ \emph{rejects} $G$ \label{line:G_smaller_then_H}
                \ElsIf{$n < t$} \Comment{Check if $G$ is small enough to query all edges}
                    \State query all pairs of vertices of $G$ to $\mathcal{O}$
                    \If{$\exists v_{i_1}, \dots, v_{i_\ell} \in G$ such that
                            $\parcurly{v_{i_1}, \dots, v_{i_\ell}}$ induces a copy of $H$ in $G$}
                        \State $\mathcal{A}$ \emph{accepts} $G$ \label{line:G_small_enough_found_H}
                    \Else
                        \State $\mathcal{A}$ \emph{rejects} $G$ \label{line:G_small_enough_not_found_H}
                    \EndIf
                \Else \label{line:random_sampling} \Comment{Sample $t$ vertices uniformly at random, without repetitions}
                    \State $S \leftarrow \emptyset$
                    \While{$i \leq t$}
                        \State $s_{i} \sim G$
                        \While{$s_i \in S$} \Comment{Repeat until a new vertex is sampled}
                            \State $s_{i} \sim G$
                        \EndWhile
                        \State $S \leftarrow S \cup \parcurly{s_i}$
                    \EndWhile
                    \State query all pairs of vertices of $S$ to $\mathcal{O}$
                    \If{$\exists v_1, \dots, v_\ell \in S$ such that
                            $\parcurly{v_1, \dots, v_\ell}$ induces a copy of $H$ in $G$}
                        \State $\mathcal{A}$ \emph{accepts} $G$ \label{line:found_H}
                    \Else
                        \State $\mathcal{A}$ \emph{rejects} $G$ \label{line:not_found_H}
                    \EndIf
                \EndIf
            \end{algorithmic}
        \end{algorithm}

        We now proceed to prove that, indeed, $\mathcal{A}$ is an $\epsilon$-test.
        If the input graph $G$ is $H$-free, then the algorithm returns $0$, either because the graph $G$ is too small to
        contain $H$ (\cref{line:G_smaller_then_H}) or because all attempts of finding $H$ as an induced subgraph of $G$
        failed (either \cref{line:G_small_enough_not_found_H} or \cref{line:not_found_H}).
        On the other hand, if $G$ is $\epsilon$-far from being $H$-free, \Cref{thm:property_testing_with_stable_partitions}
        ensures that $H$ is $\eta_{\ref{thm:property_testing_with_stable_partitions}}(k_*, \epsilon, \ell)$-abundant in $G$.
        Thus, checking $t_*$ times whether a random sample of $\ell$ vertices contains an
        induced copy of $H$, the probability of not finding any copy of $H$ is at most
        $\parround{1 - \eta_{\ref{thm:property_testing_with_stable_partitions}}(k_*, \epsilon, \ell)}^{t_*}$.
        By letting $t_* = \frac{\log\parround{\frac{2}{3}}}
            {\log\parround{1 - \eta_{\ref{thm:property_testing_with_stable_partitions}}(k_*, \epsilon, \ell)}}$
        the probability of finding at least one copy of $H$ is at least $\frac{2}{3}$.
        The total number of vertices included in the samples is at most (as there may be repetitions) $t_* \coloneqq t \cdot \ell$,
        and this probability is at most as high as simply querying all the edges within a sample of vertices of size
        $t_*$, and checking whether $H$ appears as an induced subgraph of $G$.
        For completeness, we also need to ensure that $n \geq t_*$.
        If $n < t_*$, then the algorithm simply queries all edges of $G$, checks whether $H$ appears as an induced subgraph
        of $G$ and reports accordingly (either \cref{line:G_small_enough_found_H} or \cref{line:G_small_enough_not_found_H}).

        The resulting query complexity of the algorithm $\mathcal{A}$ can bounded by
        \[
            q \leq {t_* \choose 2}
              \leq \Parround{ \frac{\log\parround{\frac{2}{3}}}
                   {\log\parround{1 - \eta_{\ref{thm:property_testing_with_stable_partitions}}(k_*, \epsilon, \ell)}}}^2
        \]

        \todo{Comment on optimization such as checking if copies of $H$ are found as soon as the sample is large enough and
            stopping early if so.}

        % up to here it was compiling with no problem.
        \newpage

    \section{Conclusion} \label{sec:conclusion}
    In this thesis, we delved into the powerful framework of Szemerédi's Regularity Lemma,
    focusing on a refined version for the class of stable graphs.
    Our primary contributions were threefold: we provided a detailed, self-contained combinatorial proof of the Stable
    Regularity Lemma from~\cite{regularity_lemmas_for_stable_graphs}, clarifying parameters and simplifying arguments;
    we developed a unified notational framework to bridge concepts from extremal graph theory, stability,
    and property testing; and, most significantly, we designed an efficient property testing algorithm for
    $H$-freeness specifically for stable graphs.

    Our motivation for this final contribution stemmed from a desire to exploit the most striking feature of the Stable
    Regularity Lemma: the complete absence of irregular pairs in its partitions.
    We hypothesized that this structural guarantee could be a powerful tool in property testing, and we found a
    direct application in testing for forbidden induced subgraphs ($H$-freeness).

    Interestingly, during the design of our tester, we observed that a similar argument could be constructed using the
    Ultra-Strong Regularity Lemma for graphs with bounded VC-dimension, as presented
    in~\cite{regularity_partitions_and_the_topology_of_graphons}.
    The stronger regularity conditions of this lemma also allow one to effectively \say{avoid} the issue of irregular
    pairs.
    This led to a crucial comparison of the bounds on the number of parts in the partitions.
    The Ultra-Strong lemma provides a bound of $(1/\epsilon)^{c \cdot d^2}$, where $d$ is the VC-dimension,
    while the Stable Regularity Lemma's bound is $(1/\epsilon)^{c \cdot 2^k}$, where $k$ is the stability parameter.
    As established in \Cref{sec:section_3}, stability is a stronger condition than bounded VC-dimension,
    and the parameters $k$ and $d$ are closely related.

    This comparison suggests that the bound for the Stable Regularity Lemma might not be optimal.
    The fact that a broader class of graphs (bounded VC-dimension) admits a partition with a polynomially better
    exponent raises the possibility that the bound for stable graphs could be improved to something akin to
    $(1/\epsilon)^{c \cdot k^2}$, a question also posed by~\cite{julia_wolf_private_comunication}.
    Despite the potentially suboptimal bound, the Stable Regularity Lemma remains a valuable tool.
    Its guarantee of having no irregular pairs whatsoever allows for a uniquely clean and straightforward
    proof of the testability of $H$-freeness in the context of stable graphs.

    Finally, this work opens several avenues for future research, which could not be included in this thesis due to time
    constraints.
    One promising direction is to leverage the lack of irregular pairs to test for $H_n$-freeness, where the forbidden
    subgraph $H_n$ grows in size with the input graph $G$.
    Another intriguing possibility is to move beyond simple freeness testing towards subgraph counting, using the clean
    structure of the stable regular partition to develop a tester that provides an interval estimate for the number of
    induced copies of a graph $H$.
        \newpage

    \printbibliography

    \appendix

\vfill\newpage \section{Other proofs} \label{sec:appendix_other_proofs}

    For completeness, here we leave secondary proofs we skipped in the thesis.

    \begin{proof}[Proof of \Cref{cor:k_order_propery_bounds_BAbs}]
        \begin{enumerate}
            \item First of all, notice that $B^+_{A,b} = A - B^-_{A,b}$, since by definition they are complementary.
                Thus, for any $b, b' \in G$, $B^+_{A,b} = B^+_{A,b'} \Leftrightarrow B^-_{A,b} = B^-_{A,b'}$.
                It follows that
                \[
                    \parstraight{\parcurly{B^-_{A,b} \mid b \in G}} =
                    \parstraight{\parcurly{B^+_{A,b} \mid b \in G}} \leq |A|^k,
                \]
                where the last inequality follows from \Cref{lem:k_order_property_bounds_BAbs}.
            \item Consider the following map:
                \begin{align*}
                    \pi: \parcurly{B^+_{A,b} \mid b \in G} & \longrightarrow \parcurly{\overline{B}_{A,b} \mid b \in G}. \\
                                                 B^+_{A,b} & \longmapsto \overline{B}_{A,b}
                \end{align*}
                We first prove that the map $\pi$ is well-defined.
                If $B^+_{A,b}$ and $B^+_{A,b'}$ are equal, then they have the same size, and thus the same truth value.
                Then,
                \begin{itemize}
                    \item if $t(A,b) = t(A,b') = 1$, we have that $\overline{B}_{A,b} = B^+_{A,b} = B^+_{A,b'} = \overline{B}_{A,b'}$.
                    \item if $t(A,b) = t(A,b') = 0$, we have that
                    $\overline{B}_{A,b} = B^-_{A,b} = A \setminus B^+_{A,b} = A \setminus B^+_{A,b'} = B^-_{A,b'} = \overline{B}_{A,b'}$.
                \end{itemize}
                which proves that the map is well-defined.
                The map $\pi$ is also surjective, since for each $b \in G$, and thus for each $\overline{B}_{A,b}$,
                the set $B^+_{A,b}$ is mapped to $\overline{B}_{A,b}$ by construction.
                Hence,
                \[
                    \parstraight{\parcurly{\overline{B}_{A,b} \mid b \in G}} \leq
                    \parstraight{\parcurly{B^+_{A,b} \mid b \in G}} \leq
                    \sum_{i \leq k} \binom{|A|}{i} \leq |A|^k.
                \]
                This concludes the proof.
                Notice that, actually, the map $\pi$ is a not necessarily a bijection, since (at most) two $b$'s with
                different truth value with respect to $A$ may induce the same set $\overline{B}_{A,b}$. \qedhere
        \end{enumerate}
    \end{proof}

    \begin{proof}[Proof of \Cref{lem:exceptions_bound_of_f_indivisible_sets}]
        Notice that, by the average condition of the pair $(A,B)$:
        \begin{itemize}
            \item there are at most $f(|A|)$ vertices of $A$ (hence in $A' \subseteq A$), say $S$, which are exceptional
                with respect to $B$, so the number of edges $(a,b) \in S \times B'$ which are exceptional is at most
                $|S| \cdot |B'|$, and
            \item for each $a \in A$ (hence in $A' \subseteq A$) not in $S$, there are at most $g(|B|)$ elements
                $b \in B$ such that $(a,b)$ does not satisfy the truth value $t(A,B)$, i.e. that are exceptional.
                Thus, we have at most $(a,b) \in (A' \setminus S) \times B'$ is at most $(|A'| - |S|) g(|B|)$.
        \end{itemize}
        The overall worse case in this scenario is when $S$ is maximum ($|S| = f(|A|)$), and thus we have at most
        $f(|A|) |B'| + (|A'| - f(|A|)) g(|B|)$ exceptional edges in $A' \times B'$, as $|B'| \geq g(|B|)$.
        Putting it all together:
        \[
            \begin{split}
                \frac{|\parcurly{(a,b) \in (A',B') \mid a R b \equiv \neg t(A,B)}|}{|A' \times B'|}
                    &\leq \frac{f(|A|) |B'| + (|A'| - f(|A|)) g(|B|)}{|A'| |B'|} \\
                    &= \frac{f(|A|)}{|A'|} + \frac{|A'| - f(|A|)}{|A'|} \frac{g(|B|)}{|B'|} \\
                    &\leq \frac{f(|A|)}{|A'|} + \frac{g(|B|)}{|B'|} \\
                    &\leq \frac{f(|A|)}{f(|A|) |A|^{\epsilon_1}} + \frac{g(|B|)}{g(|B|) |B|^{\zeta_1}} \\
                    &= \frac{1}{|A|^{\epsilon_1}} + \frac{1}{|B|^{\zeta_1}}.
            \end{split}
        \]
        This finishes the proof.
    \end{proof}

    \begin{proof}[Proof of \Cref{clm:floor_exponential_composition_bound}]
        We first note that for all natural number $n \geq 1$, and real values $x \geq 1$ and $\epsilon < 1$,
        we have that:
        \begin{equation} \label{eq:exp_bound}
            (x+n)^{\epsilon} \leq x^{\epsilon} + n
        \end{equation}
        and
        \begin{equation} \label{eq:floor_bound}
            \floor{x+n} \leq \floor{x} + n.
        \end{equation}
        We now prove the statement by induction on $k$.
        If $k=2$,
        \[
            f_{\epsilon_1 \epsilon_2} = \floor{x^{\epsilon_1 \epsilon_2}}
                \leq \floor{\floor{x^{\epsilon_1}+1}^{\epsilon_2}}
                \leq \floor{\floor{x^{\epsilon_1}}^{\epsilon_2}} + 1,
        \]
        where the second inequality uses~\eqref{eq:exp_bound} and~\eqref{eq:floor_bound}.
        If $k > 2$,
        \[
            f_{\epsilon_1 \epsilon_2 \dots \epsilon_k} (x)
                \leq f_{\epsilon_1 \epsilon_2} \circ f_{\epsilon_3} \circ \dots \circ f_{\epsilon_{k}} + k - 2
                \leq f_{\epsilon_1} \circ f_{\epsilon_2} \circ f_{\epsilon_3} \circ \dots \circ f_{\epsilon_{k}} + k - 1,
        \]
        where the first inequality uses I.H. for $k-1$, and the second inequality uses I.H. for $2$.
        This proves the statement.
    \end{proof}

    \begin{proof}[Proof of \Cref{lem:existance_of_excellent_subsets_fixed_size_choices}]
        Suppose the converse.
        We use this fact to build sets $\parcurly{b_\eta \mid \eta \in \parcurly{0,1}^{<k_{**}}}$ and
        $\parcurly{A_\eta \mid \eta \in \parcurly{0,1}^{\leq k_{**}}}$ on induction over $k<k_{**}$, where $k = |\eta|$,
        satisfying:
        \begin{enumerate}
            \item\label{itm:existance_of_excellent_subsets_fixed_size_choices.1} $A_{\Partriangle{\cdot}} \subseteq A$, with $|A|_{\Partriangle{\cdot}} = m_0$.
            \item\label{itm:existance_of_excellent_subsets_fixed_size_choices.2} $B_\eta$ is an $\zeta$-good set witnessing that $A_\eta$ is not
                $\parround{\frac{m_{k+1}}{m_{k}}, \zeta}$-excellent, for all $k < k_{**}$.
            \item\label{itm:existance_of_excellent_subsets_fixed_size_choices.3} $A_{\eta \frown \Partriangle{i}} = \parcurly{a \in A_\eta \mid t(a, B_\eta) \equiv i}$
                for all $i \in \parcurly{0,1}$ and $k < k_{**}$.
            \item\label{itm:existance_of_excellent_subsets_fixed_size_choices.4} $|A_{\eta}| = m_k$, for all $k \leq k_{**}$.
            \item\label{itm:existance_of_excellent_subsets_fixed_size_choices.6} $A_{\eta \frown \Partriangle{0}} \sqcup A_{\eta \frown \Partriangle{1}} \subseteq A_\eta$,
                for all $k < k_{**}$.
            \item\label{itm:existance_of_excellent_subsets_fixed_size_choices.7} $\overline{A_k} = \parcurly{A_\eta \mid \eta \in \parcurly{0,1}^k}$ is a partition of
                a subset of $A$, for all $k \leq k_{**}$.
        \end{enumerate}
        Notice that, by \dref{itm:existance_of_excellent_subsets_fixed_size_choices.1} and
        \dref{itm:existance_of_excellent_subsets_fixed_size_choices.4}, the size of $A_\eta$ is $m_k$,
        so by IH none of the sets $A_\eta$ is $\parround{\frac{m_{k+1}}{m_{k}}, \zeta}$-excellent.
        Then, $B_\eta$ in \dref{itm:existance_of_excellent_subsets_fixed_size_choices.2} is well-defined.
        Also, by $\zeta$-goodness of $B_\eta$, $t(a, B_\eta)$ in \dref{itm:existance_of_excellent_subsets_fixed_size_choices.3} is well-defined.
        Then, since $B_\eta$ is witnessing the non-$\parround{\frac{m_{k+1}}{m_{k}}, \zeta}$-excellence of $A_\eta$,
        we have that $|A_{\eta \frown \Partriangle{i}}| \geq \frac{m_{k+1}}{m_k} m_{k} = m_{k+1}$ for all
        $i \in \parcurly{0,1}$, satisfying \dref{itm:existance_of_excellent_subsets_fixed_size_choices.4}.
        Finally, by definition \dref{itm:existance_of_excellent_subsets_fixed_size_choices.3}, we have the disjoint union
        \dref{itm:existance_of_excellent_subsets_fixed_size_choices.6} which by itself
        ensures \dref{itm:existance_of_excellent_subsets_fixed_size_choices.7}.

        Now, our goal is to build two sequences $\parcurly{b_\eta \mid \eta \in \parcurly{0,1}^{<k_{**}}}$ and
        $\parcurly{a_\eta \mid \eta \in \parcurly{0,1}^{k_{**}}}$ to contradict the tree bound $k_{**}$.
        First of all, notice that, for $\eta \in \parcurly{0,1}^{k_{**}}$
        \[
            |A_\eta| = m_k \geq m_{k_{**}} \geq 1
        \]
        So, for each $\eta \in \parcurly{0,1}^{k_{**}}$, $A_\eta \neq \emptyset$ and we may choose an $a_\eta \in A_\eta$.
        Now, for $\nu \in \parcurly{0,1}^{<k_{**}}$ and $\eta \in \parcurly{0,1}^{k_{**}}$ such that $\nu \triangleleft \eta$, let
        \[
            U_{\nu,\eta} = \parcurly{b \in B_\nu \mid (a_\eta R b) \not\equiv t(a_\eta, B_\nu)}
        \]
        be the subset of elements of $B_\nu$ that do not relate with $a_\eta$ in the expected way.
        By $\zeta$-goodness of $B_\nu$, $|U_{\nu, \eta}| < \zeta |B_\nu|$, and thus for every $\nu \in \parcurly{0,1}^{<k_{**}}$,
        \[
            \parstraight{\bigcup\parcurly{ U_{\nu,\eta} \mid \nu \triangleleft \eta \in \parcurly{0,1}^{k_{**}}}} <
            2^{k_{**}} \zeta |B_\nu| \leq |B_\nu|
        \]
        We may choose $b_\nu \in B_\nu \setminus \bigcup\parcurly{U_{\nu,\eta} \mid \nu \triangleleft \eta \in \parcurly{0,1}^{k_{**}}}$,
        for all $\nu \in \parcurly{0,1}^{<k_{**}}$.
        Finally, the sequences $\Partriangle{a_\eta \mid \eta \in \parcurly{0,1}^{k_{**}}}$ and
        $\Partriangle{b_\nu \mid \nu \in \parcurly{0,1}^{<k_{**}}}$ satisfy that $\forall \eta, \nu$ such that
        $\nu \frown \Partriangle{i} \triangleleft \eta$, $a_\eta R b_\nu \equiv i$, which follows
        \dref{itm:existance_of_excellent_subsets_fixed_size_choices.3}.
        This contradicts \Cref{def:tree_bound} of tree bound $k_{**}$.
    \end{proof}

\vfill\newpage \section{Main changes} \label{sec:main_changes}
    This section of the appendix is dedicated at showing the main changes this thesis applies to the original results
    of~\cite{regularity_lemmas_for_stable_graphs} and~\cite{notes_on_the_stable_regularity_lemma}.

    \begin{itemize}
        % Section 2
        \item Definition 2.3 of the $k$-order property in~\cite{regularity_lemmas_for_stable_graphs} does not
            specify adjacency (or not) of vertices with the same index.
        % Section 4
        \item In order for the arguments of Section 4 to work, most results require that the function
            $f$ (of the $f$-indivisibility) satisfies $x \geq f(x)$, instead of the \emph{non-decreasing} condition
            given in Definition 4.2 in~\cite{regularity_lemmas_for_stable_graphs}, which is redundant.
        \item In order for the average condition to be satisfied, and thus being able to apply Claim 4.8 in the proof
            of Claim 4.10 in~\cite{regularity_lemmas_for_stable_graphs}, something like the extra condition provided
            by \Cref{rmk:sufficient_requirement_for_average_condition} needs to be added to the claim statement.
        \item Second to last inequality in the equation of $P_1$ at page 1569 of~\cite{regularity_lemmas_for_stable_graphs}
            is actually opposite (the $<$ should be a $>$).
            The same occurs, with last inequality of $P_2$ equation at the same page.
            This breaks the proof's argument, requiring extra conditions and some (non-trivial) changes in the argument.
            The most important change in the result is the extra condition $m_0 \geq n^\epsilon$ in
            \Cref{lem:bound_on_the_probability_of_a_subpair_having_no_exceptions}, which strongly reduces the
            interval of possible choices of parts size in the result, and needs to be carried until the end of the subsection.
        \item Condition $m_{**} > k_{**}$, which is persistent in results of Section 4
            in~\cite{regularity_lemmas_for_stable_graphs} can be relaxed into $m_{**} \geq 1$.
        \item Proof of Theorem 4.16 in~\cite{regularity_lemmas_for_stable_graphs} is unclear, even more when previous points
            are noted.
            \Cref{thm:existance_of_equitative_partition_with_perfect_pairs_but_with_bound_exceptional_pairs} provides
            a complete proof of a weaker (but coherent) version of the same result.
        \item Theorem 4.23 proof construction first finds an $\epsilon$-indivisible set, and then applies Claim 4.21
            to find a $c$-indivisible set.
            But Claim 4.21 itself does not require an $\epsilon$-indivisible set as input, as it is constructed in its
            own proof.
            Noticing this allows to fully rewrite the theorem for a stronger (and more interpretable) result
            (\Cref{thm:equitative_partition_high_regularity_parts_grow_with_n}).
        % Section 5
        \item Non-monotonicity\todo{To do...}

    \end{itemize}
    % Section 3
    Also, we note that\dots\todo{Section 3 argument does not work because...}

\vfill\newpage \section{Excellence is not monotonic.} \label{sec:excellence_is_not_monotonic}
    Here give more details on the counterexample to the monotonicity of the excellence property given in
    \Cref{fig:non-monotonic_example}.
    We see that this example is in fact the smallest bipartite graph of a family of counterexamples.
    Each element of the family can be described by the following adjacency matrix, defined by blocks:
    \[
        G_r = \left[
            \begin{array}{c|c}
                0 & H_r \\
                \hline
                H_r^T & 0 \\
            \end{array}
        \right]
        \text{, with }\quad
        H_r = \left[
            \begin{array}{c|c}
                0 & \mathbbm{1}_r - \mathbb{I}_r \\
                \hline
                \mathbb{J}_r & \mathbb{I}_r \\
            \end{array}
        \right]
    \]
    where $\mathds{1}_r$ is the $r \times r$ matrix of all $1$'s,
    $\mathbb{I}_r$ is the $r \times r$ diagonal matrix,
    and $\mathbb{J}_r$ is the $r \times r$ anti-diagonal matrix.
    Also, we use $H^T$ to refer to the transpose of $H$.

    By calling $A$ the set of the first (as indices) $2r$ vertices of $G_r$, and $B$ the last $2r$,
    we have the desired counterexample:
    $A \subseteq G$ is $\frac{1}{2r-1}$-excellent, but $B$ witnesses that $A$ is not $\frac{1}{2r-1}$-excellent.
    The example in \Cref{fig:non-monotonic_example} shows $G_r$ for $r=3$.

    A sufficient proof of this is a simple exhaustive check, and code for this precise purpose is provided with all
    the material of this thesis in a GitHub repository\footnote{See \url{https://github.com/SeverinoDaDalt/tfm_severino_da_dalt/}}.
    There are two main relevant scripts in the repository.
    One allows to check whether $G_r$ for a given value of $r$ is in fact $\frac{1}{2r-1}$-excellent and not
    $\frac{1}{2r-1}$-excellent.
    The other allows for an exhaustive search of \emph{possible} counterexamples under some given parameters, which is how this
    counterexamples were found in the first place.
    Read the documentation for more information on how to run the code.

\end{document}
%%%%%%%%%


